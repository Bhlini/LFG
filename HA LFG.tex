\documentclass[12pt,a4paper]{article}
%%% PACKAGES
%\usepackage{times} % Schriftart Times verwenden
\usepackage{graphicx} % support the \includegraphics command and options
\usepackage{booktabs} % for much better looking tables
\usepackage{array} % for better arrays (eg matrices) in maths
\usepackage{paralist} % very flexible & customisable lists (eg. enumerate/itemize, etc.)
\usepackage{verbatim} % adds environment for commenting out blocks of text & for better verbatim
\usepackage{subfig} % make it possible to include more than one captioned figure/table in a single float
\usepackage{colortbl} % enables to shade tables
\usepackage[style=authoryear]{biblatex}
\bibliography{quellen}
\usepackage[utf8]{inputenc}
\usepackage{url}
\usepackage{qtree}
\usepackage{amsmath}
\usepackage{amsfonts}
\usepackage{amssymb}
\usepackage[T1]{fontenc}  % stellt sicher, dass im PDF auch Umlaute gefunden werden
\usepackage{tgtermes}
\usepackage{pdfpages}
\usepackage{listings}
\usepackage{fixltx2e}
\usepackage[ngerman]{babel} % deutsche Begriffe (z.B. Inhaltsverzeichnis statt Contents)
\usepackage[german=quotes]{csquotes}
\renewcommand{\baselinestretch}{1.5} % Zeilenabstand
%\usepackage[onehalfspacing]{setspace} %Zeilenabstand
\usepackage{setspace}
%Seitenränder
\usepackage{geometry}
%\geometry{a4paper, top=2cm, left=3cm, right=3cm, bottom=2cm}
%Linenumbers
\usepackage[modulo]{lineno}
\usepackage{tabularx}
\usepackage{tablefootnote}
\usepackage{tikz}
\usetikzlibrary{tikzmark,positioning}
\usepackage{avm}
\DeclareBibliographyCategory{primary}
\DeclareBibliographyCategory{secondary}
\DeclareBibliographyCategory{online}
\addtocategory{primary}{lucil1, lucil2, original, seneca66}
\addtocategory{secondary}{hachmann1995, bartsch, becker1893sittlichen, cancik, inwood, edwards, motto, becker1893sittlichen}
\addtocategory{online}{philatinFLU, philatinAUDIS}
\defbibheading{primary}{\subsection*{Textausgaben und Kommentare}}
\defbibheading{secondary}{\subsection*{Sekundärliteratur}}
\defbibheading{online}{\subsection*{Online Ressourcen}}

\begin{document}
%%%%%%%%%%%%%%%%%%%%%%%%%%
% Deckblatt
% The title
\begin{titlepage}

\begin{center}


% Upper part of the page
\begin{minipage}{0.55\textwidth}
\begin{flushleft} \small
Ruprecht-Karls-Universität Heidelberg\\
Seminar für Klassische Philologie\\
Sommersemester 2013\\
Leitung: Dr. Kathrin Winter\\ 
Proseminar: Seneca, \textit{epistulae morales}
\end{flushleft}
\end{minipage}
\begin{minipage}{0.4\textwidth}
\begin{flushright} \large

\end{flushright}
\end{minipage}
\\[3.3cm]
\rule{\textwidth}{0.4pt}\\[0.4cm]

% Title

{\Large Bedeutung, Notwendigkeit und Konsequenzen \\ der Selbstgenügsamkeit} \linebreak {\large -- Eine Betrachtung anhand von Sen. \textit{epist.} 72,7-8}\\[0.2cm]

\rule{\textwidth}{0.4pt}\\[2.4cm]

% Author and supervisor
%\begin{minipage}{0.4\textwidth}
\begin{flushleft} \small
Natalia Bihler\\
Matrikelnummer: 2925340\\
6. Fachsemester (Gymnasiallehramt nach GymPO)\\
Latein und Englisch\\
Dammweg 1, 69123 Heidelberg\\
E-mail: Bihler@stud.uni-heidelberg.de
\end{flushleft}
%\end{minipage}


\vfill

% Bottom of the page
{\large 21. August 2014}

\end{center}

\end{titlepage}
%%%%%%%%%%%%%%%%%%%%%%%%%%
\setcounter{page}{2}
\begingroup
\flushbottom
\tableofcontents
\thispagestyle{empty}
%\newpage
\pagebreak
\endgroup
%\setcounter{page}{1}
% The introduction

\nocite{lucil1}
\nocite{lucil2} 
\nocite{original}
\nocite{seneca66} 
\nocite{hachmann1995} 
\nocite{bartsch}  
\nocite{philatinFLU} 
\nocite{becker1893sittlichen} 
\nocite{cancik} 
\nocite{inwood}
\nocite{edwards}
\nocite{motto} 
\nocite{becker1893sittlichen}
\nocite{philatinAUDIS}

% AVM Referenz http://nlp.stanford.edu/manning/tex/avm-doc.pdf
%https://github.com/Bhlini/LFG
%https://en.wikibooks.org/wiki/LaTeX/Linguistics#Syntactic_trees
%https://en.wikibooks.org/wiki/LaTeX/Labels_and_Cross-referencing
%http://nlp.stanford.edu/manning/tex/avm-doc.pdf
%http://nlp.stanford.edu/cmanning/tex/
%http://tex.stackexchange.com/questions/157131/problems-using-avm-package-for-lfg-structures
%http://latex-community.org/forum/viewtopic.php?f=12&t=11365
%http://www.essex.ac.uk/linguistics/external/clmt/latex4ling/avms/
%http://web.ift.uib.no/Teori/KURS/WRK/TeX/symALL.html Zeichen

\section{Einleitung}
LFG im Lateinischen anhang der Partizipialkonstruktionen...
\section{Einführung in Thematik und Terminologie}
\subsection{Partizipien}
Die Partizipien nehmen, wie bereits der Name impliziert, teil an den Eigenschaften des Nomens und des Verbums. Die Kongruenz mit dem Bezugswort in Kasus, Numerus und Genus und die Möglichkeit der Steigerung und Substantivierung spiegeln die nominalen, die Teilnahme an Aktionsart, Genus und Rektion des Verbums die verbalen Eigenschaften wider.\footnote{Vgl. LHS, S. 383, § 206.}
Im Lateinischen werden drei Partizipien verwendet: das Partizip Präsens Aktiv (PPA), das Partizip Perfekt Passiv (PPP) und das Partizip Futur Aktiv (PFA).
Wie alle Partizipialien bezeichnen die Partizipien jedoch nicht die Zeit an sich, sondern das zeitliche Verhältnis des Partizips zum \textit{verbum finitum}: Dabei kennzeichnet das PPA die Gleichzeitigkeit, das PPP die Vorzeitigkeit und das PPA die Nachzeitigkeit.\footnote{Vgl. KSt, S. 756, §136,3f.}
Des Weiteren haben PPA und PFA aktivische Bedeutung, das PPP passivische. In der Regel sind auch die Partizipien von Deponentien in der Bedeutung aktivisch.\footnote{NM, S. 708, § 496. In dieser Arbeit wird nur auf das klassische Latein Caesars und Ciceros Bezug genommen. Deshalb wird entgegen den üblichen wissenschaftlichen Konventionen auch der NM verwendet, der sich auf den Stil dieser beiden spezialisiert hat.} Daneben gibt es jedoch einige Partizipien Perfekt, die die Bedeutung eines PPA haben, wie beispielsweise \textit{confisus} oder \textit{diffisus}.\footnote{Footnote: Vgl. NM, S. 711, § 497.}

Partizipien bilden meist in Verbindung mit Substantiven spezifische Konstruktionen. 
Diese Partizipialkonstruktionen sind satzwertige Konstruktionen, in denen das Partizip dem Prädikat, das Bezugswort dem Subjekt eines Nebensatzes entspricht. Dies ist im Weiteren für die Funktionszuteilung der einzelnen Satzbestandteile von Bedeutung.
Im Folgenden sollen das rein attributive Partizip, das substantivierte Partizip, das Participium coniunctum (PC), der Ablativus absolutus (Abl. abs.), der Accusativus cum Participio (AcP) und das dominante Partizip näher betrachtet werden, um sie anschließend in das System der LFG einfügen zu können. Dabei sollen, ausgehend von Lexikoneinträgen und Syntaxregeln, sowohl c- als auch f-Strukturen zu den einzelnen Phänomenen entwickelt werden.
\subsection{Die Lexikalisch-Funktionale Grammatik}
\subsubsection{Allgemeines zu den Einschränkungen}
Um die syntaktischen Korrektheit der ausgegebenen Sätze zu gewährleisten, müssen für die verschiedenen grammatikalischen Konstruktionen zunächst spezifische Bedingungen festgelegt werden.
Da im Lateinischen im Gegensatz zu den modernen Sprachen die Wortstellung innerhalb eines Satzes nicht explizit festgelegt ist \textbf{(stimmt das als Grund?)},\footnote{Die gewöhnliche Wortstellung im Lateinischen ist zwar Subjekt – Objekt – Prädikat, jedoch wird diese, vor allem aus Gründen der Betonung und des Wohlklangs, nur selten streng eingehalten. Vgl. LHS S. 397, § 212.} muss der Großteil dieser Bedingungen nicht wie üblicherweise in den Syntaxregeln, sondern im Lexikoneintrag festgelegt werden.
%Sie sollen zunächst für das PC, den Abl. abs. und den AcP als allgemeine Einschränkungen für die jeweiligen Partizipien definiert werden.

Die in dieser Arbeit unter dem Titel ``Einschränkungen'' festgehaltenen Bedingungen sind jedoch nicht Teil der LFG, sondern dienen lediglich dem besseren Verständnis der lateinischen Grammatik, das für diese Arbeit unerlässlich ist. Diese Einschränkungen sollen zunächst allgemein anhand von Lexikoneinträgen eines Partizips x definiert werden. 

\subsubsection{Allgemeines zu den Lexikoneinträgen}

Neben diesen für die Partizipialkonstruktionen im Allgemeinen gültigen Einschränkungen finden sich in den Lexikoneinträgen der konkreten Partizipialformen Angaben zur Bestimmung der Wortform. Diese umfassen bei den Partizipien Kasus, Numerus, Genus, \textbf{Verbform} (,,MOOD''), d.h. hier stets Partizip (,,PART''), Zeitverhältnis (,,RELTENSE'', abgekürzt für ,,relative tense'') und Diathese (wobei das Attribut ,,PASSIVE'' entweder den Wert ,,+'' oder ,,-'' erhält).\footnote{Das Genus verbi, d.h. die rein morphologische Erscheinung in entweder aktiver oder passiver Form, ergibt sich aus der Grundform -- hier in Anlehnung an gängige lateinische Wörterbücher stets die erste Person Singular Präsens Indikativ -- des Prädikats des Partizips im Lexikoneintrag, wie z.B. \textit{mittor} statt \textit{mitto}. Durch diese Notierung stellen auch Deponentien kein Problem für die LFG dar, deren Diathese aktiv ist, während ihre morphologische Form im Passiv steht.} \\


*Der Subkategorisierungsrahmen -- gekennzeichnet durch $\langle$ $\rangle$ -- gibt diejenigen Argumente des Prädikats an, die von ihm gefordert werden.

*evt. Bsp.-Lex von PFA

Die konkreten Lexikoneinträge finden sich bei der Betrachtung der spezifischen Partizipialkonstruktionen.

%Im Folgenden sollen exemplarische Lexikoneinträge zu den Partizipien \textit{missum} (für das PC in Objektstellung), \textit{missi} (für das PC in Subjektstellung), \textit{victis} (für den Abl. abs.) und \textit{iaecentem} für den AcP aufgeführt werden.

%\subsection{Zeichen}

%$\theta$

%$\mid$

%$\neq$

%$\in$

%$\ni$

%$\vdash$

%$\subset$

%$\ast$

%$\neg$

\subsubsection{Allgemeines zu den Syntaxregeln}
*Funktion der Syntaxregeln; Grund, Vorteile; anders als bei früheren Ansätzen (welche?!) wurde viel auf die Lexikoneinträge ausgelagert, um die Syntaxregeln weitgehend überschaubar zu halten. Dennoch sind hier nur die für die Partizipialkonstruktionen relevanten Regeln angegeben, da alles Weitere den Rahmen dieser Arbeit sprengen würde. Dabei wurde im Hinblick auf den effizienzorientierten Ansatz der LFG auf weitgehende Anwendbarkeit der Regeln auf verschiedenste Partizipialkonstruktionen geachtet. \\

*Part trotz nominaler Eigenschaften als V, da die Partizipialkonstruktionen satzwertig sind, und im Lateinischen stets V der Kopf eines S ist und die 
Partizipien die übrigen Argumente der Konstruktion fordern.\\

*Part trotz nominaler Eigenschaften als V, da die Prädikate der Partizipien die übrigen Argumente der Partizipialkonstruktion fordern. Auch die Tatsache, dass Partizipialkonstruktionen satzwertig sind, spricht hierfür, da dies sozusagen einem S entspricht und im Lateinischen stets V der Kopf eines S ist.\\

%S  $\rightarrow$  \{V, V  XP AP, V AP XP, XP V AP, AP V XP, XP AP V, AP XP V\}\\
%S  $\rightarrow$  \{V , XP V, V XP\}\\
%XP $\rightarrow$  \{V , XP V, V XP\}\\

*allgemein für Latein
* Der Aspekt der Reihenfolge wird hier außer acht gelassen, da die erschöpfende Notation aller Alternativen den Rahmen dieser Arbeit bei Weitem übersteigen würde.
*Ausnahme Gen-Attribut (könnte man der Einfachheit halber hier als PP auffassen)

*VPs können nur Partizipialien (Partizipien und Gerundium und Gerundivum) und Infinitive sein! (wird hier so definiert)

*Da Adjektive, Adverbien und Pronominaladjektive (u.a. bei Snijder (\textbf{QUELLE} s. 7) )unter ``Determiner'' gefasst), Subjunktionen, Konjunktionen in dieser Arbeit nicht von Bedeutung sind, wird in den folgenden Syntaxregeln nicht darauf eingegangen. 

* Mehrere Vs bzw. Ns können in derselben übergeordneten Struktur nur vorkommen, wenn sie beigeordnet sind.

\begin{singlespace}
\begin{tabular}{ l  l  l  c  c  c  c }
GF & $\equiv$ & \{ARG-GF $\mid$ ADJ-GF\} \\
ARG-GF & $\equiv$ & \{SUBJ $\mid$ OBJ $\mid$\ OBJ\textsubscript{$\theta$} $\mid$ OBL\textsubscript{$\theta$} $\mid$ OBL\textsubscript{$\theta$} OBJ $\mid$ ADJ $\in$ OBJ\} \\
ADJ-GF & $\equiv$ & \{ADJ $\in$ $\mid$ XADJ $\in$\} \\
\end{tabular}\\
\end{singlespace}

%\textbf{Variante 1 - Disjunktivität beachtet (in Anlehnung an Snijder}
%\begin{singlespace}
%\begin{tabular}{ l  l  c  c  c  c  c  c  c}
%S & $\rightarrow$ & [\{V & $\mid$ & VP & $\mid$ & NP & $\mid$ & PP \}*] ,\\
 % & $\qquad$ & \textsuperscript{$\uparrow$ = $\downarrow$} & & \textsuperscript{($\uparrow$ARG-GF) = $\downarrow$} & & \textsuperscript{($\uparrow$ARG-GF) = $\downarrow$} & & \textsuperscript{($\uparrow$OBL$\theta$ $\mid$  ADJ-GF) = $\downarrow$} \\
  %$\qquad$ & $\qquad$ & [ (XP) & (\{ V & $\mid$ & NP \})& (XP)]* \\
  %& $\qquad$ & \textsuperscript{($\uparrow$ADJ-GF) = $\downarrow$} & \textsuperscript{$\uparrow$ = $\downarrow$} & & \textsuperscript{($\uparrow$ARG-GF) = $\downarrow$} & \textsuperscript{($\uparrow$ADJ-GF) = $\downarrow$} \\
 % XP & $\rightarrow$ & \{ NP, &  VP, & PP \}* \\
%\end{tabular}\\
%\newline
%\newline
%\end{singlespace}

\textbf{Variante 2 - Disjunktivität nicht beachtet}
\begin{singlespace}
\begin{tabular}{ l  l  c  c  c  c  c  c  c}
S & $\rightarrow$ & V, & XP $\mid$ & V \\
   & $\qquad$ & \textsuperscript{$\uparrow$ = $\downarrow$} & \textsuperscript{($\uparrow$ADJ-GF) = $\downarrow$ $\mid$ ($\uparrow$ARG-GF) = $\downarrow$} &  \textsuperscript{$\uparrow$ = $\downarrow$} \\
%XP & $\rightarrow$ & \{ NP & , & VP & , & PP \}* \\
XP & $\rightarrow$ & \{ NP $\mid$ &  VP $\mid$ & PP \}* \\
   & $\qquad$ & \textsuperscript{($\uparrow$ADJ-GF) = $\downarrow$ $\mid$ ($\uparrow$ARG-GF) = $\downarrow$} &\textsuperscript{($\uparrow$ADJ-GF) = $\downarrow$ $\mid$ ($\uparrow$ARG-GF) = $\downarrow$} & \textsuperscript{($\uparrow$ADJ-GF) = $\downarrow$ $\mid$ ($\uparrow$ARG-GF) = $\downarrow$} \\
   NP & $\rightarrow$ & N &  \{XP\}* $\mid$ & N \\
   & $\qquad$ & \textsuperscript{$\uparrow$ = $\downarrow$}  & \textsuperscript{($\uparrow$ADJ-GF) = $\downarrow$ $\mid$ ($\uparrow$ARG-GF) = $\downarrow$} & \textsuperscript{$\uparrow$ = $\downarrow$}\\
VP & $\rightarrow$ & V  & \{PP $\mid$     & NP\}* $\mid$ & V \\
   & $\qquad$ & \textsuperscript{$\uparrow$ = $\downarrow$} & \textsuperscript{($\uparrow$ADJ-GF) = $\downarrow$ $\mid$ ($\uparrow$ARG-GF) = $\downarrow$} & \textsuperscript{($\uparrow$ADJ-GF) = $\downarrow$ $\mid$ ($\uparrow$ARG-GF) = $\downarrow$} & \textsuperscript{$\uparrow$ = $\downarrow$}\\
PP & $\rightarrow$ & P &  \{VP $\mid$     & NP\}* \\
   & $\qquad$ & \textsuperscript{$\uparrow$ = $\downarrow$} & \textsuperscript{($\uparrow$ADJ-GF) = $\downarrow$ $\mid$ ($\uparrow$ARG-GF) = $\downarrow$} & \textsuperscript{($\uparrow$ADJ-GF) = $\downarrow$ $\mid$ ($\uparrow$ARG-GF) = $\downarrow$}\\
\end{tabular}\\
\newline
\newline
\newline
\newline
\end{singlespace}

* \textbf{In der Menge werden oder-Striche verwendet anstatt wie oft üblich Kommata, da das Komma in diesem Kontext zu einem "shuffle operator" umdefiniert wurde (vgl. Snijder) ERKLÄREN!}

Im Lateinischen ist in aller Regel V der Kopf von S.\footnote{Eine Ausnahme könnte eventuell der nominale Ablativus absolutus bilden, dessen Kopf nach logischer Betrachtung N sein müsste. Vgl. \textbf{GRAMMATIK} und vgl. \textbf{Falk, S. 64}.}
*keine feste Wortstellung / Platz des Prädikats --> nicht IP

\subsubsection{HÄUFIGE SYNTAXREGELN - keine Ahnung}
*Häufiger finden sich in dieser Arbeit die folgenden Auflösungen

\begin{singlespace}
\begin{tabular}{ l  l  c  c  c  c }
    NP & $\rightarrow$ & N \\
   & $\qquad$ & \textsuperscript{$\uparrow$ = $\downarrow$} \\
    VP & $\rightarrow$ & PP & V & \\
   & $\qquad$ & \textsuperscript{($\uparrow$OBL\textsubscript{$\theta$}) = $\downarrow$ } & \textsuperscript{$\uparrow$ = $\downarrow$} \\
   		 PP & $\rightarrow$ & P & NP \\
   & $\qquad$ & \textsuperscript{$\uparrow$ = $\downarrow$} & \textsuperscript{($\uparrow$OBJ) = $\downarrow$} \\
       NP & $\rightarrow$ & N & VP \\
   & $\qquad$ & \textsuperscript{$\uparrow$ = $\downarrow$} & \textsuperscript{$\downarrow$ $\in$ ($\uparrow$XADJ)} \\
		    VP & $\rightarrow$ & V & NP \\
   & $\qquad$ & \textsuperscript{$\uparrow$ = $\downarrow$} & \textsuperscript{($\uparrow$OBL\textsubscript{$\theta$}) = $\downarrow$ }  \\
		    VP & $\rightarrow$ & NP & V \\
   & $\qquad$ & \textsuperscript{($\uparrow$OBJ) = $\downarrow$} & \textsuperscript{$\uparrow$ = $\downarrow$} \\
		    PP & $\rightarrow$ & P & VP \\
   & $\qquad$ & \textsuperscript{$\uparrow$ = $\downarrow$} & \textsuperscript{($\uparrow$OBJ) = $\downarrow$} \\
	    VP & $\rightarrow$ &  NP& V \\
   & $\qquad$ & \textsuperscript{($\uparrow$SUBJ) = $\downarrow$} &\textsuperscript{$\uparrow$ = $\downarrow$} \\
   S\textsubscript{part} & $\rightarrow$ & NP & V'\\
   & \textsuperscript{$\qquad$} & \textsuperscript{($\uparrow$SUBJ) = $\downarrow$} & \textsuperscript{$\uparrow$ = $\downarrow$} \\

  S & $\rightarrow$ & NP\textsubscript{1} & VP & V\\
   & $\qquad$ & \textsuperscript{($\uparrow$OBJ) = $\downarrow$} & \textsuperscript{($\uparrow$XCOMP) = $\downarrow$} & \textsuperscript{$\uparrow$ = $\downarrow$} \\
    NP\textsubscript{1} & $\rightarrow$ & N \\
\end{tabular} 
\newline
\newline
\end{singlespace}


%S $\rightarrow$ NP \, VP \: XP\\

%S $\rightarrow$ NP \, VP \: V\\

%S\textsubscript{part} $\rightarrow$ NP \: V'\\

%S $\rightarrow$ NP \, VP \: V\\

%S $\rightarrow$ NP \, VP \: V\\

%($\uparrow$OBJ) = $\downarrow$\\

*evt. Bsp. - Syntaxregeln von objekt-PC

\subsubsection{Allgemeines zur c-Struktur}
\subsubsection{Allgemeines zur f-Struktur}

\section{das Participium coniunctum}
Partizipien können als Vertreter von Adverbialsätzen aufgefasst werden und stehen dabei für Temporal-, Kausal-, Modal-, Kondizional- und Konzessivsätze. Das Partizip ist hierbei mit seinem Bezugswort verbunden, welches in einem der fünf Kasus Bestandteil des Hauptsatzes und gleichzeitig Subjekt des Nebensatzes ist. Partizip und Bezugswort stimmen daher in Kasus, Numerus und Genus überein. Diese Partizipialkonstruktion bezeichnet man als PC, welches sowohl attributive als auch prädikative Funktion übernehmen kann\footnote{Vgl. KSt, S. 766, § 138,1 u. S. 771, § 138,5a; Vgl. NM, S. 715, § 500.} \\
* irgendwo mal erklären, dass wir ``Funktion'' einerseits normalsprachlich und andererseits als LFG-Terminus verwenden...?

* die P'-Knoten haben wir schon vor dem Gespräch mit Geiger fast immer mit Dreieck umgangen, die waren nur in den Syntaxregeln... vielleicht können wir einfach einmal am Anfang irgendwo anmerken, das erste Mal wenn das vorkommt eben, dass ne Präposition und ihr Objekt immer zusammenstehen müssen, und dass wir hier Postpositionen außer acht lassen wegen ihres deutlich geringeren Vorkommens und weil es nich relevant is im Moment, und dass das deswegen hier mit Dreieck abgekürzt wird... und dann können wir auf die Snijder verweisen

\subsection{allgemeine Vorüberlegungen zur Umsetzung in der LFG}
Die Konstruktion des PC erfüllt im Satz immer die syntaktische Funktion des XADJ: \\ 
($\uparrow$XADJ) = $\downarrow$ \\

\subsection{Einschränkungen}

Das Partizip muss in Kasus, Numerus und Genus mit seinem Bezugswort kongruent sei:\footnote{Vgl. KSt S. 771, § 138,5a.}\\
($\uparrow$SUBJ KNG) = ($\uparrow$KNG)\\
Dieses Bezugswort des Partizips ist eine grammatikalische Funktion der dem XADJ übergeordneten Struktur, und somit Element des Hauptsatzes:\footnote{Vgl. KSt S. 771, § 138,5a.}\\
($\uparrow$SUBJ) = ((XADJ$\uparrow$)GF) \\

\subsection{das objektabhängige Participium coniunctum}
Beispielsatz:\\
\textit{legatum in Galliam missum Caesar revocat.} \\
Lexikoneintrag wie folgt

\subsubsection{Lexikoneintrag}
\begin{singlespace}
\begin{tabular}{ l  l  l  l  } 
\textbf{missum}: & $[1]$ \:  ($\uparrow$PRED) & = & `mittor$\langle$SUBJ, OBL\textsubscript{GOAL}$\rangle$'\\
$\qquad$ & $[2]$ \:  ($\uparrow$SUBJ) & = & ((XADJ$\uparrow$)GF)\\
$\qquad$ & $[3]$ \:  ($\uparrow$SUBJ NUM) & = & sg \\
$\qquad$ & $[3.1]$ \:  \{(($\uparrow$ SUBJ GEN) & = & m \\ 
$\qquad$ & $[3.2]$ \:  ($\uparrow$SUBJ CASE) & = & acc ) $\mid$\\
$\qquad$ & $[3.3]$ \: (($\uparrow$SUBJ GEN) & = & n \\
$\qquad$ & $[3.4]$ \:  ($\uparrow$SUBJ CASE) & = & \{nom $\mid$ acc\} ) \}\\
$\qquad$ & $[4]$ \:  ($\uparrow$MOOD) & = & part\\
$\qquad$ & $[5]$ \:  ($\uparrow$PASSIVE) & = & + \\
$\qquad$ & $[6]$ \:  ($\uparrow$RELTENSE) & = & past \\
$\qquad$ & $[7]$ \:  ($\uparrow$NUM) & = & sg \\
$\qquad$ & $[8]$ \:  \{(($\uparrow$GEN) & = & m \\ 
$\qquad$ & $[8.1]$ \:  ($\uparrow$CASE) & = & acc ) $\mid$\\
$\qquad$ & $[8.2]$ \: (($\uparrow$GEN) & = & n \\
$\qquad$ & $[8.3]$ \:  ($\uparrow$CASE) & = & \{nom $\mid$ acc\} ) \}\\
\end{tabular}
\newline
\newline
\end{singlespace}

%((XADJ$\uparrow$)OBJ) = ($\uparrow$SUBJ) \\

\subsubsection{Syntaxregeln}
*Der Satz ist grammatikalisch korrekt, da die konkreten Syntaxregeln, die hier beispielhaft aufgeschlüsselt sind, sich aus den oben genannten allgemeinen Regeln ergeben.


\begin{singlespace}
\begin{tabular}{ l  l  c  c  c  c }
S & $\rightarrow$ & NP\textsubscript{1} & VP & NP & V\\
   & $\qquad$ & \textsuperscript{($\uparrow$OBJ) = $\downarrow$} & \textsuperscript{$\downarrow$ $\in$ ($\uparrow$XADJ)} & \textsuperscript{($\uparrow$SUBJ) = $\downarrow$} & \textsuperscript{$\uparrow$ = $\downarrow$} \\
    NP & $\rightarrow$ & N \\
   & $\qquad$ & \textsuperscript{$\uparrow$ = $\downarrow$} \\
    VP & $\rightarrow$ & PP & V & \\
   & $\qquad$ & \textsuperscript{($\uparrow$OBL\textsubscript{GOAL}) = $\downarrow$ } & \textsuperscript{$\uparrow$ = $\downarrow$} \\
   		 PP & $\rightarrow$ & P' \\
	& $\qquad$   & \textsuperscript{$\uparrow$ = $\downarrow$} \\
    		P' & $\rightarrow$ & P & NP\textsubscript{3} \\
   & $\qquad$ & \textsuperscript{$\uparrow$ = $\downarrow$} & \textsuperscript{($\uparrow$OBJ) = $\downarrow$} \\
\end{tabular}\\
\end{singlespace}


\subsubsection{c-Struktur}
\begin{singlespace}
\Tree [.S 
		[\qroof{legatum}.{NP\textsubscript{($\uparrow$OBJ)=$\downarrow$}} ] 
		[.VP{\textsubscript{$\downarrow$ $\in$ ($\uparrow$XADJ)}}
				[\qroof{in Galliam}.PP\textsubscript{($\uparrow$OBL\textsubscript{GOAL})=$\downarrow$} ]
				[.V\textsubscript{$\uparrow$=$\downarrow$} missum ]						
		] 
		[\qroof{Caesar}.NP\textsubscript{($\uparrow$SUBJ)=$\downarrow$} ]
		[.V\textsubscript{$\uparrow$=$\downarrow$} revocat ]	
	]
\end{singlespace}

\subsubsection{f-Struktur}
\begin{singlespace}
\begin{avm}

\[ PRED &  \rm ‘revoco \q<SUBJ, OBJ\q>’\\
SUBJ & \[PRED & `Caesar' \\
CASE & nom \\
NUM & sg \\
GEN & m \]\\
OBJ & \[ PRED & `legatus' \\
CASE & acc \\
NUM & sg \\
GEN & m \]\tikzmark{aim} \\
XADJ & \{ \[PRED &  \rm ‘mitto \q<SUBJ, OBL\textsubscript{GOAL}\q>’\\
MOOD & part \\
PASSIVE & + \\
RELTENSE & past \\
CASE & acc \\
NUM & sg \\
GEN & m \\
SUBJ &  \tikzmark{start} \\
OBL\textsubscript{GOAL} & \[``in scholam''\] \]\\
\} &            $\qquad$ \\
TENSE & present \\
NUM & sg \\
PERS & 3 \\
PASSIVE & - \\
MODE & ind \\
\]
\end{avm}
\end{singlespace}

\tikz[remember picture,overlay] 
    \draw[<-] (pic cs:aim) to[out=0,in=0,looseness=3.5]  (pic cs:start);

\newpage
\subsection{das subjektabhängige Participium coniunctum}
*blabla allg \\
* einziger Unterschied wirklich der, ob es von Subjekt oder Objekt des Hauptsatz-Prädikats abhängt

Beispielsatz:\\
\textit{milites in Galliam missi hostes vicerunt.} \\

\subsubsection{Lexikoneintrag}
\begin{singlespace}
\begin{tabular}{ l  l  l  l  } 
\textbf{missi}: & $[1]$ \:  ($\uparrow$PRED) & = & `mittor$\langle$SUBJ, OBL\textsubscript{GOAL}$\rangle$'\\
$\qquad$ & $[2]$ \:  ($\uparrow$SUBJ) & = & ((XADJ$\uparrow$)GF) \\
$\qquad$ & $[3]$ \:  \{(($\uparrow$SUBJ NUM) & = & pl \\ 
$\qquad$ & $[3.1]$ \:  ($\uparrow$SUBJ CASE) & = & nom \\
$\qquad$ & $[3.2]$ \:  ($\uparrow$SUBJ GEN) & = & m) $\mid$\\
$\qquad$ & $[3.3]$ \:  (($\uparrow$SUBJ NUM) & = & sg \\ 
$\qquad$ & $[3.4]$ \: ($\uparrow$SUBJ CASE) & = & gen \\
$\qquad$ & $[3.5]$ \:  ($\uparrow$SUBJ GEN) & = & \{m $\mid$ n\} ) \} \\
$\qquad$ & $[4]$ \:  ($\uparrow$MOOD) & = & part \\
$\qquad$ & $[5]$ \:  ($\uparrow$PASSIVE) & = & + \\
$\qquad$ & $[6]$ \: ($\uparrow$RELTENSE) & = & past \\
$\qquad$ & $[7]$ \:  \{(($\uparrow$NUM) & = & pl \\ 
$\qquad$ & $[7.1]$ \:  ($\uparrow$CASE) & = & nom \\
$\qquad$ & $[7.2]$ \:  ($\uparrow$GEN) & = & m) $\mid$\\
$\qquad$ & $[7.3]$ \:  (($\uparrow$NUM) & = & sg \\ 
$\qquad$ & $[7.4]$ \: ($\uparrow$CASE) & = & gen \\
$\qquad$ & $[7.5]$ \:  ($\uparrow$GEN) & = & \{m $\mid$ n\} ) \} \\
\end{tabular}
\end{singlespace}

\subsubsection{Syntaxregeln}
\begin{singlespace}
\begin{tabular}{ l  l  c  c  c  c }
S & $\rightarrow$ & NP\textsubscript{1} & VP & NP\textsubscript{2} & V\\
   & $\qquad$ & \textsuperscript{($\uparrow$SUBJ) = $\downarrow$} & \textsuperscript{$\downarrow$ $\in$ ($\uparrow$XADJ)} & \textsuperscript{($\uparrow$OBJ) = $\downarrow$} & \textsuperscript{$\uparrow$ = $\downarrow$} \\
    NP\textsubscript{1} & $\rightarrow$ & N \\
   & $\qquad$ & \textsuperscript{$\uparrow$ = $\downarrow$} \\
    VP & $\rightarrow$ & V' \\
   & $\qquad$ & \textsuperscript{$\uparrow$ = $\downarrow$} \\
  	  V' & $\rightarrow$ & PP & V & \\
   & $\qquad$ & \textsuperscript{($\uparrow$OBL\textsubscript{GOAL}) = $\downarrow$ } & \textsuperscript{$\uparrow$ = $\downarrow$} \\
   		 PP & $\rightarrow$ & P' \\
	& $\qquad$   & \textsuperscript{$\uparrow$ = $\downarrow$} \\
    		P' & $\rightarrow$ & P & NP\textsubscript{3} \\
   & $\qquad$ & \textsuperscript{$\uparrow$ = $\downarrow$} & \textsuperscript{($\uparrow$OBJ) = $\downarrow$} \\
 		   NP\textsubscript{3} & $\rightarrow$ & N \\
   & $\qquad$ & \textsuperscript{$\uparrow$ = $\downarrow$} \\
    NP\textsubscript{2} & $\rightarrow$ & N \\
   & $\qquad$ & \textsuperscript{$\uparrow$ = $\downarrow$} \\
\end{tabular}\\
\end{singlespace}

\subsubsection{c-Struktur}
\begin{singlespace}
\Tree [.S 
		[\qroof{milites}.{NP\textsubscript{($\uparrow$SUBJ)=$\downarrow$}} ] 
		[.VP{\textsubscript{$\downarrow$ $\in$ ($\uparrow$XADJ)}}
				[\qroof{in Galliam}.PP\textsubscript{($\uparrow$OBL\textsubscript{GOAL})=$\downarrow$} ]
				[.V\textsubscript{$\uparrow$=$\downarrow$} missi ]						
		] 
		[\qroof{hostes}.NP\textsubscript{($\uparrow$OBJ)=$\downarrow$} ]
		[.V\textsubscript{$\uparrow$=$\downarrow$} vicerunt ]	
	]
\end{singlespace}

\subsubsection{f-Struktur}
\begin{singlespace}
\begin{avm}
\[ PRED &  \rm ‘vinco \q<SUBJ, OBJ\q>’\\
SUBJ & \[ PRED & `miles' \\
CASE & nom \\
NUM & pl \\
GEN & m \]\tikzmark{meow} \\
XADJ & \{ \[PRED &  \rm ‘mitto \q<SUBJ, OBL\textsubscript{GOAL}\q>’\\
MOOD & part \\
PASSIVE & + \\
RELTENSE & past \\
CASE & nom \\
NUM & pl \\
GEN & m \\
SUBJ &  \tikzmark{objectmeow} \\
OBL\textsubscript{GOAL} & \[``in Galliam''\] \]\\
\} &            $\qquad$ \\
OBJ & \[``hostes'' \]\\
\]
\end{avm}
\tikz[remember picture,overlay] 
    \draw[<-] (pic cs:meow) to[out=0,in=0,looseness=3.5]  (pic cs:objectmeow);
\end{singlespace}

\subsection{Das rein attributive Participium Coniunctum}

Das rein attributive Partizip hat zum \textit{verbum finitum} keinerlei Beziehung, sondern charakterisiert nur sein Bezugswort; es ersetzt somit einen attributiven Gliedsatz.\footnote{Vgl. NM, S. 713, § 498.} \\


* da rein attributiv ist es abhängig von NP \\
* XADJ

Beispielsatz:\\
\textit{insulam obiectam portui tenuit.}

\subsubsection{Lexikoneintrag}
\begin{singlespace}
\begin{tabular}{ l  l  l  l  } 
\textbf{obiectam}: & $[1]$ \:  ($\uparrow$PRED) & = & `obicior$\langle$SUBJ, OBJ\textsubscript{LOC}$\rangle$'\\
$\qquad$ & $[2]$ \:  ($\uparrow$SUBJ) & = & ((XADJ$\uparrow$)GF)\\
$\qquad$ & $[2.1]$ \:  ($\uparrow$SUBJ NUM) & = & sg \\
$\qquad$ & $[2.2]$ \: ($\uparrow$SUBJ CASE) & = & acc \\
$\qquad$ & $[2.3]$ \: ($\uparrow$SUBJ GEN) & = & f \\
$\qquad$ & $[3]$ \: ($\uparrow$OBJ CASE) & = & dat \\
$\qquad$ & $[4]$ \:  ($\uparrow$MOOD) & = & part\\
$\qquad$ & $[5]$ \:  ($\uparrow$PASSIVE) & = & + \\
$\qquad$ & $[6]$ \:  ($\uparrow$RELTENSE) & = & past \\
$\qquad$ & $[7]$ \:  ($\uparrow$NUM) & = & sg \\
$\qquad$ & $[8]$ \: ($\uparrow$CASE) & = & acc \\
$\qquad$ & $[9]$ \: ($\uparrow$GEN) & = & f \\
\end{tabular}
\newline
\newline
\end{singlespace}

\subsubsection{Syntaxregeln}
\begin{singlespace}
\begin{tabular}{ l  l  c  c  c  c }
  S & $\rightarrow$ & NP\textsubscript{1} & V\\
   & $\qquad$ & \textsuperscript{($\uparrow$OBJ) = $\downarrow$} & \textsuperscript{$\uparrow$ = $\downarrow$} \\
    NP\textsubscript{1} & $\rightarrow$ & N' \\
   & $\qquad$ & \textsuperscript{$\uparrow$ = $\downarrow$} \\
       N' & $\rightarrow$ & N & VP \\
   & $\qquad$ & \textsuperscript{$\uparrow$ = $\downarrow$} & \textsuperscript{$\downarrow$ $\in$ ($\uparrow$XADJ)} \\
		    VP & $\rightarrow$ & V' \\
   & $\qquad$ & \textsuperscript{$\uparrow$ = $\downarrow$} \\
  				  V' & $\rightarrow$ & V & NP\textsubscript{2} \\
   & $\qquad$ & \textsuperscript{$\uparrow$ = $\downarrow$} & \textsuperscript{($\uparrow$OBL\textsubscript{LOC}) = $\downarrow$ }  \\
   					 NP\textsubscript{2} & $\rightarrow$ & N \\
   & $\qquad$ & \textsuperscript{$\uparrow$ = $\downarrow$} \\
\end{tabular} 
\end{singlespace}

\subsubsection{c-Struktur}
\begin{singlespace}
\Tree [.S 
		[.{NP\textsubscript{($\uparrow$OBJ)=$\downarrow$}} 
				[.N\textsubscript{$\uparrow$=$\downarrow$} insulam ]		
				[.VP{\textsubscript{$\downarrow$ $\in$ ($\uparrow$XADJ)}}
						[.V\textsubscript{$\uparrow$=$\downarrow$} obiectam ] 
						[\qroof{portui}.NP\textsubscript{($\uparrow$OBJ\textsubscript{DAT})=$\downarrow$} ]
				]
				]
		[.V\textsubscript{$\uparrow$=$\downarrow$} tenuit ]	
	]
\end{singlespace}

\subsubsection{f-Struktur}
\begin{singlespace}
\begin{avm}
\[ PRED &  \rm ‘teneo \q<SUBJ, OBJ\q>’\\
SUBJ & \[ PRED & 'pro' \\
PRON-TYPE & mis \] \\
OBJ & \[PRED & `insula' \\
CASE & acc \\
NUM & sg \\
GEN & f \]\tikzmark{a} \\
XADJ & \{ \[PRED &  \rm ‘obicio \q<SUBJ, OBJ\textsubscript{DAT}\q>’\\
MOOD & part \\
PASSIVE & + \\
RELTENSE & past \\
CASE & acc \\
NUM & sg \\
GEN & f \\
SUBJ &  \tikzmark{z} \\
OBJ\textsubscript{DAT} & \[PRED & `portus' \\
CASE & dat \\
NUM & sg \\
GEN & m \\
\] \]\\
\} &            $\qquad$ \\
\]
\end{avm}
\end{singlespace}

\tikz[remember picture,overlay] 
    \draw[<-] (pic cs:a) to[out=0,in=0,looseness=3.4]  (pic cs:z);


\newpage
\section{das substantivierte Partizip}

Da Partizipien einige Eigenschaften der Adjektive übernehmen, können sie wie diese substantiviert werden und die Rolle eines Substantives übernehmen. Der Neue Menge bezeichnet auch das substantivierte Partizip als rein attributiv. Da das Vorhandensein eines Bezugswortes für die LFG jedoch einen erheblichen Unterschied darstellt, wird das substantivierte Partizip in dieser Arbeit gesondert aufgeführt.\footnote{Vgl. NM, S. 713, § 498.} \\
\textbf{+ klassisch selten / weniger häufig als PC, Abl abs, AcP} \\
\textbf{+ kommt v.a. in bestimmten Kontexten vor, wie... ?}


\subsection{Vorüberlegungen zur Umsetzung in der LFG}
Die Umsetzung des substantivierten Partizips in die LFG-Struktur soll anhand des Beispielsatzes \textit{auxilium petentibus Caesar parcit} veranschaulicht werden.


\subsection{Variante 1: das Partizip als XADJ zum OBJ}
Folgt man dem Neuen Menge\footnote{Vgl. NM \textbf{§ ??}} und betrachtet das substantivierte Partizip (\textit{petentibus}) als Attribut zu einem sozusagen fehlenden Bezugswort -- in diesem Fall also etwa \textit{eis} oder \textit{viris} -- so würde die Partizipialkonstruktion in der Rolle eines XADJ zu diesem Bezugswort stehen; das Bezugswort selbst wäre dann das Objekt des Hauptsatzprädikats \textit{parcit}. Dieses fehlende Objekt wird in der c-Struktur unten durch ,,?'' bezeichnet. Da vom substantivierten Partizip petentibus in unserem Beispiel noch ein Nomen in Objektfunktion abhängt, spaltet sich die Partizipial-VP noch einmal in V und NP auf. Es ergeben sich folgende Syntaxregeln, c- und f-Strukturen:\\

\subsubsection{Einschränkungen}
\textbf{Variante 1: XADJ}:\\
Das Subjekt der untergeordneten Struktur ist das Objekt der dem XADJ übergeordneten Struktur (welches fehlt): \\
($\downarrow$SUBJ) = ((OBJ$\uparrow$)XADJ) \textbf{???} \\

\subsubsection{Lexikoneintrag}
Der Lexikoneintrag des Partizips lautet wie folgt:\footnote{Im Rahmen des Umfangs der Arbeit wird nur die für unseren Beispielsatz relevanten Argumente aufgezählt. Für andere mögliche Konstruktionen von \textit{petere} -- wie (SUBJ, OBJ, OBL\textsubscript{LOC})bzw. (SUBJ, OBJ, OBL\textsubscript{PURPOSE}) -- müssten eigene Lexikoneinträge erstellt werden.\textbf{(Vgl. RHH §119 und § 234)}.}
\begin{singlespace}
\begin{tabular}{ l  l  l  l  } 
\textbf{petentibus}: & $[1]$ \:  ($\uparrow$PRED) & = & `peto$\langle$SUBJ, OBJ$\rangle$' \\
$\qquad$ & $[2]$ \:  ($\uparrow$SUBJ) & = & ((XADJ$\uparrow$)GF)\\
$\qquad$ & $[3.1]$ \:  ($\uparrow$SUBJ CASE) & = & \{abl $\mid$ dat\} \\
$\qquad$ & $[3.2]$ \:  ($\uparrow$SUBJ NUM) & = & pl \\
$\qquad$ & $[3.3]$ \:  ($\uparrow$SUBJ GEN) & = & \{m $\mid$ n $\mid$ f\} \\
$\qquad$ & $[4]$ \:  ($\uparrow$OBJ CASE) & = & acc \\
$\qquad$ & $[5]$ \:  ($\uparrow$MOOD) & = & part\\
$\qquad$ & $[6]$ \:  ($\uparrow$PASSIVE) & = & - \\
$\qquad$ & $[7]$ \:  ($\uparrow$RELTENSE) & = & present \\ 
$\qquad$ & $[8]$ \:  ($\uparrow$CASE) & = & \{abl $\mid$ dat\} \\
$\qquad$ & $[9]$ \:  ($\uparrow$NUM) & = & pl \\
$\qquad$ & $[10]$ \:  ($\uparrow$GEN) & = & \{m $\mid$ n $\mid$ f\} \\
\end{tabular}
\newline
\newline
\end{singlespace}


\subsubsection{Syntaxregeln}
\begin{singlespace}
\begin{tabular}{ l  l  c  c  c  c }
  S & $\rightarrow$ & NP\textsubscript{1} & NP\textsubscript{2} & V \\
   & $\qquad$ & \textsuperscript{($\uparrow$OBJ) = $\downarrow$} & \textsuperscript{($\uparrow$SUBJ) = $\downarrow$} & \textsuperscript{$\uparrow$ = $\downarrow$} \\
		NP\textsubscript{1} & $\rightarrow$ & N' \\
   & $\qquad$ & \textsuperscript{$\uparrow$ = $\downarrow$} \\
  		  N' & $\rightarrow$ & N & VP \\
   & $\qquad$ & \textsuperscript{$\uparrow$ = $\downarrow$} & \textsuperscript{($\uparrow$XADJ) = $\downarrow$} \\		    
		    VP & $\rightarrow$ & V' \\
   & $\qquad$ & \textsuperscript{$\uparrow$ = $\downarrow$} \\
  				  V' & $\rightarrow$ & NP\textsubscript{3} & V \\
   & $\qquad$ & \textsuperscript{($\uparrow$OBJ) = $\downarrow$} & \textsuperscript{$\uparrow$ = $\downarrow$} \\
   					 NP\textsubscript{3} & $\rightarrow$ & N \\
   & $\qquad$ & \textsuperscript{$\uparrow$ = $\downarrow$} \\
    NP\textsubscript{2} & $\rightarrow$ & N \\
   & $\qquad$ & \textsuperscript{$\uparrow$ = $\downarrow$} \\
\end{tabular} 
\end{singlespace}

\subsubsection{c-Struktur}
\begin{singlespace}
\Tree [.S 
		[.NP{\textsubscript{$\downarrow$ = ($\uparrow$OBJ)}}
			[.N\textsubscript{$\uparrow$=$\downarrow$} \textit{?} ]
			[.VP\textsubscript{($\uparrow$XADJ)=$\downarrow$}  
				[\qroof{auxilium}.NP\textsubscript{($\uparrow$OBJ)=$\downarrow$} ]
				[.V\textsubscript{$\uparrow$=$\downarrow$} petentibus ] 				
			]
		]	
		[\qroof{Caesar}.NP\textsubscript{($\uparrow$SUBJ)=$\downarrow$} ]
		[.V\textsubscript{$\uparrow$=$\downarrow$} parcit ]	
	]
\end{singlespace}

\subsubsection{f-Struktur}
\begin{singlespace}
\begin{avm}
\[ PRED &  \rm ‘parco \q<SUBJ, OBJ\textsubscript{REC}\q>’\\
SUBJ & \[``Caesar'' \] \\
OBJ\textsubscript{REC} & \[PRED & `pro' \\
PRON-TYPE & mis \\
CASE & dat \\
NUM & pl \\
GEN & m \]\tikzmark{alpha} \\
XADJ & \[PRED &  \rm ‘peto \q<SUBJ, OBJ\q>’\\
MOOD & part \\
PASSIVE & - \\
RELTENSE & present \\
CASE & dat \\
NUM & pl \\
GEN & m \\
SUBJ &  \tikzmark{omega} \\
OBJ & \[PRED & `auxilium' \\
CASE & acc \\
NUM & sg \\
GEN & n \\
\] \]  &            $\qquad$ \\
\]
\end{avm}
\tikz[remember picture,overlay] 
    \draw[<-] (pic cs:alpha) to[out=0,in=0,looseness=2.5]  (pic cs:omega);
    
\end{singlespace}

\subsection{Variante 2: das Partizip als OBJ}
In der obigen Variante würde man sich also immer ein nur ausgelassenes Bezugswort des Partizips hinzudenken. Dies verkompliziert unserer Ansicht nach die Anlegenheit und bietet keinen Mehrwert. Da das Partizip substantiviert ist, und somit eben gerade keinem Bezugswort untergeordnet, haben wir uns für die folgende Variante entschieden, deren c-Struktur sichtbar unkomplizierter ist. Dabei ist das Partizip Kopf der Partizipialphrase VP und somit alleiniges Objekt des Hauptsatz-Prädikats \textit{parcit}. In diesem Fall überwiegen zwar die nominalen Eigenschaften des Partizips, die Bezeichnung `VP' wird jedoch um der Konsistenz willen beibehalten.


\subsubsection{Einschränkungen}
\subsubsection{Lexikoneintrag}
*wie oben, mit einzigem Unterschied ...

\begin{singlespace}
\begin{tabular}{ l  l  l  l  } 
\textbf{petentibus}: & $[1]$ \:  ($\uparrow$PRED) & = & `peto$\langle$SUBJ, OBJ$\rangle$' \\
$\qquad$ & $[2]$ \:  ($\uparrow$SUBJ PRED) & = & `pro' \\
$\qquad$ & $[2.1]$ \:  ($\uparrow$SUBJ PRON-TYPE) & = & missing \\
$\qquad$ & $[2.2]$ \:  ($\uparrow$SUBJ CASE) & = & \{abl $\mid$ dat\} \\
$\qquad$ & $[2.3]$ \:  ($\uparrow$SUBJ NUM) & = & pl \\
$\qquad$ & $[2.4]$ \:  ($\uparrow$SUBJ GEN) & = & \{m $\mid$ n $\mid$ f\} \\
$\qquad$ & $[3]$ \:  ($\uparrow$OBJ CASE) & = & acc \\
$\qquad$ & $[4]$ \:  ($\uparrow$MOOD) & = & part\\
$\qquad$ & $[5]$ \:  ($\uparrow$PASSIVE) & = & - \\
$\qquad$ & $[6]$ \:  ($\uparrow$RELTENSE) & = & present \\ 
$\qquad$ & $[7]$ \:  ($\uparrow$CASE) & = & \{abl $\mid$ dat\} \\
$\qquad$ & $[8]$ \:  ($\uparrow$NUM) & = & pl \\
$\qquad$ & $[9]$ \:  ($\uparrow$GEN) & = & \{m $\mid$ n $\mid$ f\} \\
\end{tabular}
\newline
\newline
\end{singlespace}


\subsubsection{Syntaxregeln}
Syntaxregeln sehen wie folgt aus: \\
\begin{singlespace}
\begin{tabular}{ l  l  c  c  c  c }
  S & $\rightarrow$ & VP & NP\textsubscript{1} & V\\
   & $\qquad$ & \textsuperscript{($\uparrow$OBJ\textsubscript{REC}) = $\downarrow$} & \textsuperscript{($\uparrow$SUBJ) = $\downarrow$} & \textsuperscript{$\uparrow$ = $\downarrow$} \\
		    VP & $\rightarrow$ & V' \\
   & $\qquad$ & \textsuperscript{$\uparrow$ = $\downarrow$} \\
  				  V' & $\rightarrow$ & NP\textsubscript{2} & V \\
   & $\qquad$ & \textsuperscript{($\uparrow$OBJ) = $\downarrow$} & \textsuperscript{$\uparrow$ = $\downarrow$} \\
   					 NP\textsubscript{2} & $\rightarrow$ & N \\
   & $\qquad$ & \textsuperscript{$\uparrow$ = $\downarrow$} \\
    NP\textsubscript{1} & $\rightarrow$ & N' \\
   & $\qquad$ & \textsuperscript{$\uparrow$ = $\downarrow$} \\
\end{tabular} 
\end{singlespace}

\subsubsection{c-Struktur}
\begin{singlespace}
\Tree [.S 
		[.VP{\textsubscript{($\uparrow$OBJ\textsubscript{REC}) = $\downarrow$}}
					[\qroof{auxilium}.NP\textsubscript{($\uparrow$OBJ)=$\downarrow$} ]
					[.V\textsubscript{$\uparrow$=$\downarrow$} petentibus ] 
		]
		[\qroof{Caesar}.NP\textsubscript{($\uparrow$SUBJ)=$\downarrow$} ]
		[.V\textsubscript{$\uparrow$=$\downarrow$} parcit ]	
	]
\end{singlespace}

\subsubsection{f-Struktur}
\begin{singlespace}
\begin{avm}
\[ PRED &  \rm ‘parco \q<SUBJ, OBJ\textsubscript{REC}\q>’\\
SUBJ & \[``Caesar'' \] \\
OBJ\textsubscript{REC} & \[PRED &  \rm ‘peto \q<SUBJ, OBJ\q>’\\
MOOD & part \\
PASSIVE & - \\
RELTENSE & present \\
CASE & dat \\
NUM & pl \\
GEN & m \\
SUBJ & \[PRED & `pro' \\
PRON-TYPE  & mis \] \\
OBJ & \[PRED & `auxilium' \\
CASE & acc \\
NUM & sg \\
GEN & n \] \\
\] \]
\end{avm}
\end{singlespace}

\newpage
\section{das dominante Partizip}
Beim sogenannten dominanten Partizip trägt nicht das Substantiv, sondern das in Kasus, Numerus und Genus übereinstimmenden Partizip die Hauptbedeutung; das Partizip ,dominiert` daher sozusagen sein Bezugswort. Aus diesem Grund wird das dominante Partizip im Deutschen in der Regel mit einem Verbalsubstantiv wiedergegeben, von dem das im Lateinischen regierende Substantiv als Genetiv abhängt. Meistens verwendet man das Partizip Perfekt Passiv als dominantes Partizip.\footnote{Vgl. NM, S. 717 f., § 502.}\\

\subsection{Version mit Präpositionalphrase}
\subsubsection{Vorüberlegungen zur Umsetzung in der LFG}
Der Lexikoneintrag zum Partizip der Konstruktion unterscheidet sich nicht wesentlich von den vorherigen.
\textbf{ (ich glaub wir brauchen hier echt nich nochmal nen Lexikoneintrag...)} \\
Das dominante Partizip soll zunächst am Beispielsatz \textit{ab urbe condita Roma viguit} betrachtet werden. Da der Restsatz \textit{Roma viguit} auch ohne die Partizipialkonstruktion Sinn ergibt, muss letztere wie beim Abl. abs. ein ADJ zum finiten Satz sein. Als nächstes ergibt sich aufgrund der Präposition \textit{ab} eine Präpositionalphrase, von der wiederum Partizip und Bezugswort abhängen.

%\subsubsection{Variante 1 -- Partizip als attributives XADJ}
Nun sieht das dominante Partizip \textit{condita} rein formal zunächst aus wie ein attributives Partizip zum Bezugswort \textit{urbe} \textbf{(?) +vgl NM}, weswegen man eine NP mit \textit{urbe} als Kopf konstruieren könnte (siehe Variante 1). Das Partizip wäre somit seinem Bezugswort untergeordnet. Da das Subjekt des Partizips aus der übergeordneten Struktur -- in diesem Fall von der NP mit Kopf \textit{urbe} -- bezieht, müsste das Partizip eine X-Rolle erhalten; da ein XCOMP zum Bezugswort -- in diesem Fall \textit{urbe} -- nicht zu rechtfertigen wäre \textbf{(??? weil es dann von urbe gefordert werden müsste? oder wieso eig?)}, bliebe für das Partizip -- hier \textit{condita} -- nur die Rolle des XADJ. Die zugehörigen c- und f-Strukturen sähen demnach wie folgt aus:

\textbf{c-Struktur}
\begin{singlespace}
\Tree [.S 
		[.PP{\textsubscript{$\downarrow$ $\in$ ($\uparrow$XADJ)}}
			[.P'\textsubscript{$\uparrow$=$\downarrow$} 
				[.P\textsubscript{$\uparrow$=$\downarrow$} ab ] 
				[.NP\textsubscript{($\uparrow$OBJ)=$\downarrow$}
					[.N'\textsubscript{$\uparrow$=$\downarrow$} 
						[.N\textsubscript{$\uparrow$=$\downarrow$} urbe ]
						[\qroof{condita}.VP\textsubscript{$\downarrow$ $\in$ ($\uparrow$ADJ)} ]
					] 
				]
			]				
		] 	
		[\qroof{Roma}.NP\textsubscript{($\uparrow$SUBJ)=$\downarrow$} ]
		[.V\textsubscript{$\uparrow$=$\downarrow$} viguit ]	
	]\\
\newline
\end{singlespace}

\textbf{f-Struktur}
\begin{singlespace}
\begin{avm}
\[ PRED &  \rm ‘vigeo \q<SUBJ\q>’\\
SUBJ & ``Roma'' \\
ADJ & \[ PRED &  \rm ‘ab \q<OBJ\q>’\\
OBJ & \[ PRED & `urbs' \tikzmark{begin} \\ 
CASE & abl \\
NUM & sg \\
GEN & f  \\
XADJ & \[PRED &  \rm ‘condo \q<SUBJ\q>’\\
MOOD & part \\
PASSIVE & + \\
RELTENSE & past \\
CASE & abl \\
NUM & sg \\ 
GEN & f  \\
SUBJ &  \tikzmark{end} \] &            $\qquad$ \\
\]  \\
\] \]
\end{avm}
\end{singlespace}

\tikz[remember picture,overlay] 
    \draw[<-] (pic cs:begin) to[out=0,in=0,looseness=2.4]  (pic cs:end);
    
\begin{singlespace}    
\begin{avm}
\[ PRED &  \rm ‘vigeo \q<SUBJ\q>’\\
SUBJ & ``Roma'' \\
ADJ & \[ PRED &  \rm ‘ab \q<OBJ\q>’\\
OBJ & \[ PRED & `urbs' \\ 
CASE & abl \\
NUM & sg \\
GEN & f  \\
XADJ & \[PRED &  \rm ‘condo \q<SUBJ\q>’\\
MOOD & part \\
PASSIVE & + \\
RELTENSE & past \\
CASE & abl \\
NUM & sg \\ 
GEN & f  \\
SUBJ &  \tikzmark{Ziel} \] \] \tikzmark{Start} & $\qquad$ & $\qquad$  \\
\] \\
\]
\end{avm}
\newline
\newline
\end{singlespace}

\tikz[remember picture,overlay] 
    \draw[<-] (pic cs:Start) to[out=10,in=0,looseness=2.4]  (pic cs:Ziel);

%\subsubsection{Variante 2 -- Partizip in tatsächlich dominanter Position}
Da Adjunkte jedoch nach Belieben weggelassen werden können, würde dies bedeuten, dass der Satz \textit{ab urbe Roma viguit} korrekt wäre. Das stimmt zwar formal -- ist jedoch semantisch sinnfrei. Eine semantisch sinnvollere Darstellung ergibt sich, wenn das Bezugswort vom Prädikat des Partizips gefordert wird; da das Partizip sein Bezugswort dominiert, sollte ihm in der LFG-Darstellung eine seinem Bezugswort übergeordnete Funktion zukommen. Somit würde die Partizipialkonstruktion von einer VP mit dem Kopf \textit{condita} abhängen; das Bezugsnomen \textit{urbe} wäre dann schlicht das Subjekt der Partizipialkonstruktion.

\subsubsection{Lexikoneintrag}
Es ergeben sich demnach folgende Syntaxregeln,  c- und f-Strukturen: 

\begin{singlespace}
\begin{tabular}{ l  l  l  l  } 
\textbf{condita}: & $[1]$ \:  ($\uparrow$PRED) & = & `condor$\langle$SUBJ$\rangle$'\\
%$\qquad$ & $[2]$ \:  ($\uparrow$SUBJ) & = & ((ADJ$\uparrow$)GF)\\
$\qquad$ & $[5]$ \:  \{(($\uparrow$SUBJ GEN) & = & f \\ 
$\qquad$ & $[5.1]$ \:  ($\uparrow$SUBJ NUM) & = & sg \\
$\qquad$ & $[5.2]$ \:  ($\uparrow$SUBJ CASE) & = & \{nom $\mid$ abl\} ) $\mid$\\
$\qquad$ & $[5.2]$ \: (($\uparrow$SUBJ GEN) & = & n \\
$\qquad$ & $[6.3]$ \:  ($\uparrow$SUBJ NUM) & = & pl \\
$\qquad$ & $[6.4]$ \:  ($\uparrow$SUBJ CASE) & = & \{nom $\mid$ acc\} ) \}\\
$\qquad$ & $[2]$ \:  ($\uparrow$MOOD) & = & part\\
$\qquad$ & $[3]$ \:  ($\uparrow$PASSIVE) & = & + \\
$\qquad$ & $[4]$ \:  ($\uparrow$RELTENSE) & = & past \\
$\qquad$ & $[5]$ \:  \{(($\uparrow$GEN) & = & f \\ 
$\qquad$ & $[5.1]$ \:  ($\uparrow$NUM) & = & sg \\
$\qquad$ & $[5.2]$ \:  ($\uparrow$CASE) & = & \{nom $\mid$ abl\} ) $\mid$\\
$\qquad$ & $[5.2]$ \: (($\uparrow$GEN) & = & n \\
$\qquad$ & $[6.3]$ \:  ($\uparrow$NUM) & = & pl \\
$\qquad$ & $[6.4]$ \:  ($\uparrow$CASE) & = & \{nom $\mid$ acc\} ) \}\\
\end{tabular}
\newline
\newline
\end{singlespace}

\subsubsection{Syntaxregeln}
\begin{singlespace}
\begin{tabular}{ l  l  c  c  c  c }
  S & $\rightarrow$ & PP & NP\textsubscript{1} & V\\
   & $\qquad$ & \textsuperscript{$\downarrow$ $\in$ ($\uparrow$ADJ)} & \textsuperscript{($\uparrow$SUBJ) = $\downarrow$} & \textsuperscript{$\uparrow$ = $\downarrow$} \\
		    PP & $\rightarrow$ & P' \\
   & $\qquad$ & \textsuperscript{$\uparrow$ = $\downarrow$} \\
  				  P' & $\rightarrow$ & P & VP \\
   & $\qquad$ & \textsuperscript{$\uparrow$ = $\downarrow$} & \textsuperscript{($\uparrow$OBJ) = $\downarrow$} \\
					    VP & $\rightarrow$ & V' \\
   & $\qquad$ & \textsuperscript{$\uparrow$ = $\downarrow$} \\
		  				  V' & $\rightarrow$ & V & NP\textsubscript{2} \\
   & $\qquad$ & \textsuperscript{$\uparrow$ = $\downarrow$} & \textsuperscript{($\uparrow$SUBJ) = $\downarrow$} \\
		   					 NP\textsubscript{2} & $\rightarrow$ & N \\
   & $\qquad$ & \textsuperscript{$\uparrow$ = $\downarrow$} \\
    NP\textsubscript{1} & $\rightarrow$ & N \\
   & $\qquad$ & \textsuperscript{$\uparrow$ = $\downarrow$} \\
\end{tabular} 
\newline
\end{singlespace}

\subsubsection{c-Struktur}
\begin{singlespace}
\Tree [.S 
		[.PP{\textsubscript{$\downarrow$ $\in$ ($\uparrow$ADJ)}}
			[.P\textsubscript{$\uparrow$=$\downarrow$} ab ] 
			[.VP\textsubscript{($\uparrow$OBJ)=$\downarrow$}
				[.V\textsubscript{$\uparrow$=$\downarrow$} condita ]
				[\qroof{urbe}.NP\textsubscript{($\uparrow$SUBJ) = $\downarrow$} ]
			]
			]				
		[\qroof{Roma}.NP\textsubscript{($\uparrow$SUBJ)=$\downarrow$} ]
		[.V\textsubscript{$\uparrow$=$\downarrow$} viguit ]	
	]\\
\newline
\end{singlespace}

\subsubsection{f-Struktur}
\begin{singlespace}
\begin{avm}
\[ PRED &  \rm ‘vigeo \q<SUBJ\q>’\\
SUBJ & ``Roma'' \\
ADJ & \[ PRED &  \rm ‘ab \q<OBJ\q>’\\
OBJ & \[ PRED &  \rm ‘condo \q<SUBJ\q>’\\
MOOD & part \\
PASSIVE & + \\
RELTENSE & past \\
CASE & abl \\
NUM & sg \\
GEN & f \\
SUBJ & \[PRED & `urbs' \\
CASE & abl \\
NUM & sg \\
GEN  & f \] \] \] \]
\end{avm}\\
\end{singlespace}

\subsection{Version ohne Präpositionalphrase}
\subsubsection{Vorüberlegungen zur Umsetzung in der LFG}
Nun war zu klassischen Zeiten jedoch die präpositionslose Variante des dominanten Partizips vorherrschend,\footnote{Vgl. LHS \textbf{§ ???}} weswegen auch hierzu ein Beispielsatz betrachtet werden soll: \textit{libertate amissa doleo.} Formal ist diese Konstruktion im Ablativ kaum vom Abl. abs. zu unterscheiden; der Satz könnte schließlich auch bedeuten: "Ich trauere wegen der verlorenen Freiheit". Korrekter, da näher an der lateinischen Bedeutung, wäre jedoch die Übersetzung: "Ich trauere wegen des Verlusts der Freiheit." Um diesem -- wenn hier auch semantisch geringen -- Unterschied gerecht zu werden, sollte auch hier in der LFG-Darstellung die Dominanz des Partizips über sein Bezugswort deutlich werden. Auch hier ist daher die gesamte Partizipialkonstruktion ein ADJ zum finiten Prädikat und das Bezugsnomen darin seinem Partizip unterstellt. Der Unterschied zu Variante 2 oben ergibt sich lediglich aus dem Fehlen der Präposition.
\textbf{neu}
Da lateinische Partizipialkonstruktionen jedoch stets als VP definiert werden, haben sie ohnehin V als Kopf; daher kann die besondere Dominanz des Partizips nach unserer Darstellungsweise nicht gesondert hervorgehoben werden. Dies wäre ein Argument, die Partizipialkonstruktionen als gesonderte Partizipialphrasen darzustellen. \textbf{(auch wegen der nominalen Eigenschaften der Partizipien) in Schlussfolgerung + Diese Darstellungsweise wurde in dieser Arbeit bereits in Anfängen versucht, nämlich bei der c-Struktur-Darstellung des Abl. abs. als S\textsubscript{part}.}

\subsubsection{Lexikoneintrag}
\begin{singlespace}
\begin{tabular}{ l  l  l  l  } 
\textbf{amissa}: & $[1]$ \:  ($\uparrow$PRED) & = & `amitto$\langle$SUBJ, OBJ, OBL\textsubscript{LOC}$\rangle$'\\
%$\qquad$ & $[2]$ \:  ($\uparrow$SUBJ) & = & ((XADJ$\uparrow$)OBJ)\\
$\qquad$ & $[3]$ \:  ($\uparrow$MOOD) & = & part\\
$\qquad$ & $[4]$ \:  ($\uparrow$PASSIVE) & = & + \\
$\qquad$ & $[5]$ \:  ($\uparrow$RELTENSE) & = & past \\
$\qquad$ & $[6]$ \:  \{(($\uparrow$GEN) & = & f \\ 
$\qquad$ & $[6.1]$ \:  ($\uparrow$NUM) & = & sg \\
$\qquad$ & $[6.2]$ \:  ($\uparrow$CASE) & = & \{nom $\mid$ abl\} ) $\mid$\\
$\qquad$ & $[6.2]$ \: (($\uparrow$GEN) & = & n \\
$\qquad$ & $[6.3]$ \:  ($\uparrow$NUM) & = & pl \\
$\qquad$ & $[6.4]$ \:  ($\uparrow$CASE) & = & \{nom $\mid$ acc\} ) \}\\
\end{tabular}
\newline
\newline
\end{singlespace}

"Der Unterschied zu Variante 2 oben ergibt sich lediglich aus dem Fehlen der Präposition." - Letzteres wird auch in den Syntaxregeln deutlich.

\subsubsection{Syntaxregeln}
\begin{singlespace}
\begin{tabular}{ l  l  c  c  c  c }
  S & $\rightarrow$ & VP & V\\
   & $\qquad$ & \textsuperscript{$\downarrow$ $\in$ ($\uparrow$ADJ)} & \textsuperscript{($\uparrow$SUBJ) = $\downarrow$} & \textsuperscript{$\uparrow$ = $\downarrow$} \\
	    VP & $\rightarrow$ & V' \\
   & $\qquad$ & \textsuperscript{$\uparrow$ = $\downarrow$} \\
			  V' & $\rightarrow$ & NP& V \\
   & $\qquad$ & \textsuperscript{($\uparrow$SUBJ) = $\downarrow$} &\textsuperscript{$\uparrow$ = $\downarrow$} \\
		   					 NP & $\rightarrow$ & N \\
   & $\qquad$ & \textsuperscript{$\uparrow$ = $\downarrow$} \\
\end{tabular} 
\newline
\end{singlespace}

\subsubsection{c-Struktur}
Daraus, sowie aus dem hier nicht extra aufgeführten Lexikoneintrag, gehen gemäß der obigen Überlegungen folgende c- und f-Strukturen hervor:

\begin{singlespace}
\Tree [.S 
		[.VP{\textsubscript{$\downarrow$ $\in$ ($\uparrow$ADJ)}}
				[\qroof{libertate}.NP\textsubscript{($\uparrow$SUBJ) = $\downarrow$} ]
				[.V\textsubscript{$\uparrow$=$\downarrow$} amissa ]
		]				 	
			[.V\textsubscript{$\uparrow$=$\downarrow$} doleo ]		
	]\\
\newline
\end{singlespace}

\subsubsection{f-Struktur}
\begin{singlespace}
\begin{avm}
\[ PRED &  \rm ‘doleo \q<SUBJ\q>’\\
SUBJ & \[PRED & `pro' \\
PRON-Type & mis\] \\
ADJ & \{ \[ PRED &  \rm ‘amitto \q<SUBJ\q>’\\
MOOD & part \\
PASSIVE & + \\
RELTENSE & past \\
CASE & abl \\
NUM & sg \\
GEN & f \\
SUBJ & \[PRED & `libertas' \\
CASE & abl \\
NUM & sg \\
GEN  & f \] \] \} \]
\end{avm}\\
\end{singlespace}


\section{Abl. abs.}
Wie beim PC vertritt auch die Partizipialkonstruktion des Ablativus absolutus einen Adverbialsatz, wobei das Bezugswort dem Subjekt, das Partizip dem Prädikat entspricht \textbf{(das kann eig weg wenn wir das in der Einführung lassen)}. Dabei wird das Bezugswort nicht vom Prädikat des finiten Satzes gefordert, und besitzt demnach keine eigene Satzgliedfunktion. Der Abl. abs. ist somit vom Rest des Satzes losgelöst, weswegen dem Abl. abs. die Satzgliedfunktion der freien Angabe zukommt. \textbf{Partizip und Bezugswort stehen immer im Ablativ. (doppelt - bei Neugliederung beachten} Aufgrund seiner Entsprechung mit dem Prädikat des zugrunde liegenden Satzes kann sein Partizip nur als prädikativ aufgefasst werden; dass es nicht in attributiver Funktion zu einem Nomen steht, wird auch daran deutlich, dass der Satz bei Wegfall des Partizips nicht mehr grammatikalisch korrekt wäre. Der Ablativ ist im Lateinischen für diese Konstruktion gewählt, da dieser Kasus bereits ohne Partizip adverbiale Verhältnisse, beispielsweise der Zeit, bezeichnet.\footnote{Vgl. KSt, S. 766, § 138,1 u. S. 771, § 138,5b; Vgl. NM, S. 718 f., § 503. Anstelle eines Partizips können auch bestimmte Nomina in den Ablativus absolutus treten. Auf dies kann im Rahmen des Umfangs dieser Arbeit, die sich auf Partizipialkonstruktionen konzentriert, nicht näher eingegangen werden. Vgl. NM, S. 720, § 504.} \\

\subsection{Vorüberlegungen zur Umsetzung in der LFG}
Beispielsatz: \\
\textit{barbaris in Gallia victis Caesar gaudet.} \\

Da der Restsatz auch ohne den Abl. abs. noch Sinn ergeben würde, 

\subsection{Einschränkungen}
Da der \textit{Ablativus absolutus} vom finiten Satz (s\textsubscript{fin}) losgelöst ist, steht er in der Funktion eines ADJ: \\
($\uparrow$ADJ) = $\downarrow$ \\
Auch beim Abl. abs. muss das Partizip in Kasus, Numerus und Genus mit seinem Bezugswort übereinstimmen:\footnote{Vgl. KSt S. 771, § 138,5a.}\\
($\uparrow$SUBJ KNG) = ($\uparrow$KNG)\\
Sowohl Partizip als auch Bezugswort stehen stets im Ablativ:\footnote{Vgl. KSt S. 771, § 138,5b.} \\
($\uparrow$CASE) = abl \\
($\uparrow$SUBJ CASE) = abl \\
Da das Bezugswort des Partizips im Abl. abs. keine Rolle im übergeordneten Satz spielen darf, ist es keine grammatikalische Funktion der dem XADJ übergeordneten Struktur. Der Abl. abs. ist daher vom finiten Satz losgelöst:\footnote{Vgl. KSt S. 771, § 138,5b.} \\
$\neg$ ($\uparrow$SUBJ) = ((ADJ$\uparrow$)GF) \\
Da sich diese Arbeit ausschließlich auf das klassische Latein Caesars und Ciceros bezieht, gilt für die folgenden Betrachtungen die Annahme, dass im Abl. abs. kein Partizip Futur Aktiv (PFA) verwendet wird.\footnote{Vgl. KSt. S. 760, § 136,4c oder NM S. 771, § 469.}\\
$\neg$ ($\uparrow$RELTENSE) = future \\

%($\uparrow$RELTENSE (ADJ)) $\neq$ future \\
%$\neg$ ($\downarrow$PRED) = ($\uparrow$GF PRED) \\

\subsection{Lexikoneintrag}
Obige, für den Abl. abs. gültige Einschränkungen können jedoch nicht im Lexikoneintrag der Partizipien festgehalten werden, da Partizipien im Ablativ auch in anderen Partizipialkonstruktionen vorkommen; ist ein Partizip wie \textit{victis} beispielsweise teil eines PC, ist sein Subjekt eine grammatikalische Funktion der der Partizipialkonstruktion übergeordneten Struktur.
\begin{singlespace}
\begin{tabular}{ l  l  l  l  } 
\textbf{victis}: & $[1]$ \:  ($\uparrow$PRED) & = & `vincor$\langle$SUBJ$\rangle$'\\
%$\qquad$ & $[2]$ \: $\neg$ ($\uparrow$SUBJ) & = & ((ADJ$\uparrow$)GF) \\
$\qquad$ & $[5]$ \: ($\uparrow$SUBJ CASE) & = & \{dat $\mid$ abl\} \\
$\qquad$ & $[6]$ \:  ($\uparrow$SUBJ NUM) & = & pl \\
$\qquad$ & $[7]$ \: ($\uparrow$SUBJ GEN) & = & \{m $\mid$ f $\mid$ n\} \\
$\qquad$ & $[2]$ \:  ($\uparrow$MOOD) & = & part\\
$\qquad$ & $[3]$ \: ($\uparrow$PASSIVE) & = & + \\
$\qquad$ & $[4]$ \: ($\uparrow$RELTENSE) & = & past \\
$\qquad$ & $[5]$ \: ($\uparrow$CASE) & = & \{dat $\mid$ abl\} \\
$\qquad$ & $[6]$ \:  ($\uparrow$NUM) & = & pl \\
$\qquad$ & $[7]$ \: ($\uparrow$GEN) & = & \{m $\mid$ f $\mid$ n\} \\
\end{tabular}
\end{singlespace}

\subsection{Syntaxregeln}
Somit muss die Losgelöstheit der Ablativus-absolutus-Konstruktion in den Syntaxregeln festgehalten werden. Dies geschieht, indem der Abl. abs. in einem gesonderten Satz, hier bezeichnet als S\textsubscript{part}, dargestellt wird und die Funktion eines Adjunkts erhält. Wir haben uns für diese Variante entschieden, da durch die bloße Bezeichnung als VP nicht zur Geltung kommen würde, dass der Abl. abs. nur durch einen adverbialen Gliedsatz ersetzt werden kann und sowohl sein Subjekt als auch sein Prädikat innerhalb desselben Knotens enthalten sind (und nicht wie beispielsweise beim PC das Subjekt aus der übergeordneten Struktur bezogen werden muss).\footnote{Vergleiche auch die diesbezüglichen Anmerkungen in der Schlussfolgerung dieser Arbeit.} Die oben genannten Syntaxregeln müssen daher erweitert werden: \\
\begin{singlespace}
\begin{tabular}{ l  l  c  c  c  c }
   S\textsubscript & $\rightarrow$ & S\textsubscript{part} & NP & V\\
   & $\qquad$ & \textsuperscript{ $\downarrow$ $\in$ ($\uparrow$ADJ)} & \textsuperscript{($\uparrow$SUBJ) = $\downarrow$} & \textsuperscript{$\uparrow$ = $\downarrow$} \\
   S\textsubscript{part} & $\rightarrow$ & NP & PP & V & \\
   & $\qquad$ &  \textsuperscript{($\uparrow$SUBJ) = $\downarrow$} &\textsuperscript{$\downarrow$ $\in$ ($\uparrow$ADJ)} & \textsuperscript{$\uparrow$ = $\downarrow$} \\
\end{tabular} 
\end{singlespace}

\subsection{c-Struktur}
\begin{singlespace}
\Tree [.S\textsubscript{fin} 
		[.S{\textsubscript{part} \textsubscript{($\downarrow$ $\in$ $\uparrow$ADJ)}}
			[\qroof{barbaris}.NP{\textsubscript{($\uparrow$SUBJ)=$\downarrow$}}	]
			[\qroof{in Gallia}.PP\textsubscript{($\downarrow$ $\in$ $\uparrow$ADJ)} ]
			[.V\textsubscript{$\uparrow$=$\downarrow$} victis ]
		]							
		[\qroof{Caesar}.{NP\textsubscript{($\uparrow$SUBJ)=$\downarrow$}} ] 
		[.V{\textsubscript{$\uparrow$=$\downarrow$}} gaudet ]
	]
\end{singlespace}

\subsection{f-Struktur}
\begin{singlespace}
\begin{avm}
\[ PRED &  \rm ‘gaudeo \q<SUBJ\q>’\\
SUBJ & \["Caesar" \]\\
ADJ & \{ \[PRED &  \rm ‘vincor \q<SUBJ\q>’\\
MOOD & part \\
PASSIVE & + \\
RELTENSE & past \\
CASE & abl \\
NUM & pl \\
GEN & m \\
SUBJ & \[PRED & `barbarus' \\
CASE & abl \\
NUM & pl \\
GEN & m \\ \] \\
ADJ & \{\[``in Gallia''\] \} \] \\
\}
\]
\end{avm}
\end{singlespace}


\section{der Accusativus cum Participio}
Bei den Verben der unmittelbaren sinnlichen Wahrnehmung, oft bei \textit{videre} und \textit{audire}, sowie bei den Verben des Darstellens und Einführens, besonders bei \textit{facere} und \textit{inducere}, steht die satzwertige Ergänzung oft in Verbindung mit einem Objekt \textbf{und dem Partizip Präsens Aktiv im Akkusativ (Lex-Eintrag, nicht doppeln)}. Man nennt diese Verbindung Accusativus cum Participio (AcP).\footnote{Vgl. KSt, S. 763, § 137,2a; Vgl. NM, S. 714, § 499.}\\
\textbf{+ prädikativ} \\
Der AcP ist von einem Verb der unmittelbaren sinnlichen Wahrnehmung oder von \textit{facere} bzw. \textit{inducere} im Sinne von ‚in einem Werk, in einem Drama darstellen, (auftreten) lassen‘ abhängig.\footnote{Vgl. NM S. 714, § 499. Vgl. auch KSt S. 763, § 137,2a.} \\ 
Verben der Wahrnehmung, wie sie im AcI und AcP vorkommen, erfordern im Lateinischen eine Ergänzung, die sowohl durch ein bloßes Nomen, als auch durch eine Partizipialkonstruktion ausgedrückt werden kann. Letztere verleiht dem Satzgefüge eine enorme Bedeutungserweiterung, da die Partizipialkonstruktion satzwertig ist.

%\textbf{beim KsT heißt es: Zweitens wird das Partizip in prädikativem Sinne zur Ergänzung eines Verbalbegriffes gebraucht. Dieser Fall tritt ein: a) "Beshreibung des 'normalen AcPs' b) bei den Verben habeo und teneo in Verbindung mit dem PPP, entweder allein oder mit einem Objekte, um einen aus einer vollendeten Handlung hervorgegangenen Zusatnd oder bleibenden Besitz zu bezeichnen}

%\textbf{NM: Das PPP findet sich zur Umschreibung des Perfekts zusammen mit finiten Verbformen von \textit{habere} bzw. \textit{tenere} und bezeichnet dann einen dauernden Zustand oder bleibenden Besitz.}

%\footnote{Vgl. KSt S. 763, § 137,2a. Vgl. auch NM S. 763, § 137. \textbf{Vgl. auch LHS S. 387-88 § 207 c; auch KSt S. 763, § 137,2b? tenere + habere mit PPP?}} \\


\subsection{Vorüberlegungen zur Umsetzung in der LFG}
* Beispielsatz: \\
\textit{militem in campo iacentem vidit.} \\
* da prädikativ: direkt von S abhängig, nicht z.B. von der NP \\
* muss auf jeden Fall entweder XADJ oder XCOMP sein, da das Subjekt zum Prädikat der Struktur vom Prädikat der darüberliegenden Struktur (d.h. vom finiten Verb, "vidit") gefordert wird // da das Prädikat der AcP-Konstruktion, d.h. das Partizip, sein Subjekt aus der übergeordneten Struktur bezieht. \\
* XCOMP oder XADJ?
	
	\textbf{*für XADJ spricht:} Restsatz ergibt auch so Sinn; analog zum PC;
	
	\textbf{*für XCOMP spricht:} semantisch großer Unterschied (andere Bedeutung als PC wegen Verben der Wahrnehmung, würde dem Sinn der Konstruktion sonst nicht gerecht werden); facere / inducere; analog zu AcI -> also haben wir uns dafür entschieden + wird also vom PRED gefordert\\
	
	Da das Partizip einer AcI- oder auch AcP-Konstrukion eine Ergänzung des Verbalbegriffs ist, nimmt es die Funktion des XCOMP an. Formal sind die AcI und AcP kaum auseinanderzuhalten; der Unterschied liegt in der Semantik: Während beim AcI der Inhalt der Verbalhandlung betont wird, liegt beim AcP der Nachdruck auf der sinnlichen Rezeption der Handlung oder eines Zustandes.\footnote:{Vgl. LHS S. 387, §207.} Diese Bedeutungsdifferenz kann jedoch nicht in der f-Struktur ausgedrückt werden.

\subsection{Einschränkungen}
Da das Partizip im AcP eine Ergänzung des Verbalbegriffs ist, erfüllt es im Satz stets die Funktion eines XCOMP: \\
($\uparrow$XCOMP) = $\downarrow$ \\
Das Partizip und sein Bezugswort stehen auch beim AcP im selben Kasus, Numerus und Genus:\footnote{Vgl. KSt S. 771, § 138,5a.}\\
($\uparrow$SUBJ KNG) = ($\uparrow$KNG)\\
Wie auch hier der Name der Konstruktion vermuten lässt, stehen beim AcP Partizip und Bezugswort im Akkusativ:\footnote{Vgl. KSt S. 763, § 137,2a.} \\
($\uparrow$CASE) = acc \\
($\uparrow$SUBJ CASE) = acc \\
Das Bezugswort des Partizips ist das Objekt der dem XCOMP übergeordneten Struktur: \\
	($\uparrow$SUBJ) = ((XCOMP$\uparrow$)OBJ) \\
%($\uparrow$XCOMP SUBJ) = ($\uparrow$OBJ) (?) \\

Das Partizip ist beim AcP meist ein PPA, selten ein PPP. \\
%($\uparrow$XCOMP RELTENSE) = present   \\
$\neg$ ($\uparrow$RELTENSE) = future \\
	

\subsection{Lexikoneintrag}

\begin{singlespace}
\begin{tabular}{ l  l  l  l  } 
\textbf{iacentem}: & $[1]$ \: ($\uparrow$PRED) & = & `iaceo$\langle$SUBJ, OBL\textsubscript{LOC}$\rangle$'\\
$\qquad$ & $[2]$ \:  ($\uparrow$SUBJ) & = & ((XADJ$\uparrow$)GF)\\
$\qquad$ & $[5]$ \: ($\uparrow$SUBJ CASE) & = & acc \\
$\qquad$ & $[6]$ \: ($\uparrow$SUBJ NUM) & = & sg \\
$\qquad$ & $[7]$ \: ($\uparrow$SUBJ GEN) & = & \{m $\mid$ f\} \\
$\qquad$ & $[2]$ \: ($\uparrow$MOOD) & = & part\\
$\qquad$ & $[3]$ \: ($\uparrow$PASSIVE) & = & - \\
$\qquad$ & $[4]$ \: ($\uparrow$RELTENSE) & = & present \\
$\qquad$ & $[5]$ \: ($\uparrow$CASE) & = & acc \\
$\qquad$ & $[6]$ \: ($\uparrow$NUM) & = & sg \\
$\qquad$ & $[7]$ \: ($\uparrow$GEN) & = & \{m $\mid$ f\} \\
\end{tabular}\\
\newline
Im Lexikoneintrag des Prädikats der dem AcP-XCOMP übergeordneten Struktur müsste, wie oben erwähnt, zunächst spezifiziert sein, dass es ein XCOMP zu sich nehmen kann, und im Folgenden die Bedingungen, die dieses XCOMP erfüllen muss:\\
\textbf{video}: $\langle$SUBJ, OBJ, XCOMP$\rangle$\\
($\uparrow$XCOMP SUBJ) = ($\uparrow$OBJ)\\
($\uparrow$OBJ CASE) = acc\\

Alternative:
video: $\langle$SUBJ, OBJ, COMP$\rangle$\\
($\uparrow$COMP SUBJ) = `pro'\\
($\uparrow$COMP SUBJ KNG) = ($\uparrow$OBJ KNG)\\

\end{singlespace}

\subsection{Syntaxregeln}

\begin{singlespace}
\renewcommand{\arraystretch}{1}  
\begin{tabular}{ l  l  c  c  c }
  S & $\rightarrow$ & NP\textsubscript{1} & VP & V\\
   & $\qquad$ & \textsuperscript{($\uparrow$OBJ) = $\downarrow$} & \textsuperscript{($\uparrow$XCOMP) = $\downarrow$} & \textsuperscript{$\uparrow$ = $\downarrow$} \\
    NP\textsubscript{1} & $\rightarrow$ & N \\
   & $\qquad$ & \textsuperscript{$\uparrow$ = $\downarrow$} \\
    VP & $\rightarrow$ & V' \\
   & $\qquad$ & \textsuperscript{$\uparrow$ = $\downarrow$} \\
    V' & $\rightarrow$ & PP & V & \\
   & $\qquad$ & \textsuperscript{($\uparrow$OBL\textsubscript{LOC}) = $\downarrow$ } & \textsuperscript{$\uparrow$ = $\downarrow$} \\
    PP & $\rightarrow$ & P' \\
	& $\qquad$   & \textsuperscript{$\uparrow$ = $\downarrow$} \\
    P' & $\rightarrow$ & P & NP\textsubscript{2} \\
   & $\qquad$ & \textsuperscript{$\uparrow$ = $\downarrow$} & \textsuperscript{($\uparrow$OBJ) = $\downarrow$} \\
    NP\textsubscript{2} & $\rightarrow$ & N \\
   & $\qquad$ & \textsuperscript{$\uparrow$ = $\downarrow$} \\
\end{tabular} 
\end{singlespace}
 
\subsection{c-Struktur}
\begin{singlespace}
\Tree [.S 
		[\qroof{militem}.{NP\textsubscript{($\uparrow$OBJ)=$\downarrow$}} ] 
		[.VP{\textsubscript{($\uparrow$XCOMP)=$\downarrow$}}
			[\qroof{in campo}.PP\textsubscript{($\uparrow$OBL\textsubscript{LOC})=$\downarrow$} ]
				[.V\textsubscript{$\uparrow$=$\downarrow$} iacentem ]
		] 
		[.V\textsubscript{$\uparrow$=$\downarrow$} vidit ]	
	]
\end{singlespace}

\subsection{f-Struktur}
\begin{singlespace}
\begin{avm}
\[ PRED &  \rm ‘video \q<SUBJ, OBJ, XCOMP\q>’\\
SUBJ & \[ PRED & `pro' \\
		PRON-TYPE & mis	\]\\
OBJ & \[ PRED & `miles' \\
CASE & acc \\
NUM & sg \\
GEN & m \]\tikzmark{topic} \\
XCOMP & \[PRED &  \rm ‘iaceo \q<SUBJ, OBL\textsubscript{LOC}\q>’\\
MOOD & part \\
PASSIVE & - \\
RELTENSE & present \\
CASE & acc \\
NUM & sg \\
GEN & m \\
SUBJ &  \tikzmark{object} \\
OBL\textsubscript{LOC} & \[``in campo''\] \]  &            $\qquad$\\
\]
\end{avm}
\end{singlespace}

\tikz[remember picture,overlay] 
    \draw[<-] (pic cs:topic) to[out=0,in=0,looseness=3]  (pic cs:object);


%\end{singlespace}
%\bibliographystyle{plain}
\pagebreak
\section*{Literaturverzeichnis}
\bibbycategory
\addcontentsline{toc}{section}{Literaturverzeichnis}
\end{document}
