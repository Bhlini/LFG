\documentclass[12pt,a4paper]{article}
%%% PACKAGES
%\usepackage{times} % Schriftart Times verwenden
\usepackage{graphicx} % support the \includegraphics command and options
\usepackage{booktabs} % for much better looking tables
\usepackage{array} % for better arrays (eg matrices) in maths
\usepackage{paralist} % very flexible & customisable lists (eg. enumerate/itemize, etc.)
\usepackage{verbatim} % adds environment for commenting out blocks of text & for better verbatim
\usepackage{subfig} % make it possible to include more than one captioned figure/table in a single float
\usepackage{colortbl} % enables to shade tables
\usepackage[style=authoryear]{biblatex}
\bibliography{quellen}
\usepackage[utf8]{inputenc}
\usepackage{url}
\usepackage{qtree}
\usepackage{amsmath}
\usepackage{amsfonts}
\usepackage{amssymb}
\usepackage[T1]{fontenc}  % stellt sicher, dass im PDF auch Umlaute gefunden werden
\usepackage{tgtermes}
\usepackage{pdfpages}
\usepackage{listings}
\usepackage{fixltx2e}
\usepackage[ngerman]{babel} % deutsche Begriffe (z.B. Inhaltsverzeichnis statt Contents)
\usepackage[german=quotes]{csquotes}
\renewcommand{\baselinestretch}{1.5} % Zeilenabstand
%\usepackage[onehalfspacing]{setspace} %Zeilenabstand
\usepackage{setspace}
%Seitenränder
\usepackage{geometry}
%\geometry{a4paper, top=2cm, left=3cm, right=3cm, bottom=2cm}
%Linenumbers
\usepackage[modulo]{lineno}
\usepackage{tabularx}
\usepackage{tablefootnote}
\usepackage{tikz}
\usetikzlibrary{tikzmark,positioning}
\usepackage{avm}

\begin{document}
%%%%%%%%%%%%%%%%%%%%%%%%%%
% Deckblatt
% The title
\begin{titlepage}

\begin{center}


% Upper part of the page
\begin{minipage}{0.55\textwidth}
\begin{flushleft} \small
Ruprecht-Karls-Universität Heidelberg\\
Seminar für Klassische Philologie\\
Sommersemester 2013\\
Leitung: Dr. Kathrin Winter\\ 
Proseminar: Seneca, \textit{epistulae morales}
\end{flushleft}
\end{minipage}
\begin{minipage}{0.4\textwidth}
\begin{flushright} \large

\end{flushright}
\end{minipage}
\\[3.3cm]
\rule{\textwidth}{0.4pt}\\[0.4cm]

% Title

{\Large Bedeutung, Notwendigkeit und Konsequenzen \\ der Selbstgenügsamkeit} \linebreak {\large -- Eine Betrachtung anhand von Sen. \textit{epist.} 72,7-8}\\[0.2cm]

\rule{\textwidth}{0.4pt}\\[2.4cm]

% Author and supervisor
%\begin{minipage}{0.4\textwidth}
\begin{flushleft} \small
Natalia Bihler\\
Matrikelnummer: 2925340\\
6. Fachsemester (Gymnasiallehramt nach GymPO)\\
Latein und Englisch\\
Dammweg 1, 69123 Heidelberg\\
E-mail: Bihler@stud.uni-heidelberg.de
\end{flushleft}
%\end{minipage}


\vfill

% Bottom of the page
{\large 21. August 2014}

\end{center}

\end{titlepage}
%%%%%%%%%%%%%%%%%%%%%%%%%%
\setcounter{page}{2}
\begingroup
\flushbottom
\begin{spacing}{1.25}
\tableofcontents
\end{spacing}
\thispagestyle{empty}
%\newpage
\pagebreak
\endgroup
%\setcounter{page}{1}
% The introduction

\nocite{Menge}
\nocite{LHS}
\nocite{KSt}
\nocite{Rohrer}
\nocite{Skript}
\nocite{Dal}
\nocite{Falk}
\nocite{Bresnan}
\nocite{Snijders}
\nocite{DAZ}


% AVM Referenz http://nlp.stanford.edu/manning/tex/avm-doc.pdf
%https://github.com/Bhlini/LFG
%https://en.wikibooks.org/wiki/LaTeX/Linguistics#Syntactic_trees
%https://en.wikibooks.org/wiki/LaTeX/Labels_and_Cross-referencing
%http://nlp.stanford.edu/manning/tex/avm-doc.pdf
%http://nlp.stanford.edu/cmanning/tex/
%http://tex.stackexchange.com/questions/157131/problems-using-avm-package-for-lfg-structures
%http://latex-community.org/forum/viewtopic.php?f=12&t=11365
%http://www.essex.ac.uk/linguistics/external/clmt/latex4ling/avms/
%http://web.ift.uib.no/Teori/KURS/WRK/TeX/symALL.html Zeichen

\section{Einleitung}
Diese Arbeit beschäftigt sich mit der Beschreibung lateinischer Partizipialkonstruktionen im System der lexikalisch-funktionalen Grammatik (LFG). Die LFG ist eine in den späten 1970er Jahren vor allem aus der Generativen Grammatik Noam Chomskys hervorgegangene Theorie zur Beschreibung der Syntax natürlicher Sprachen.\footnote{Vgl. \cite[4]{Skript}; vgl. auch \cite[1]{Dal}; \cite[3]{Bresnan}.} Neben der Generativen Grammatik stützt sich die LFG auch auf strukturalistische und funktionale Ansätze.\footnote{Vgl. \cite[3]{Bresnan}.} Zu ihren wichtigsten Begründern zählen Joan Bresnan und Ronald Kaplan.

Die Regeln der LFG sollen sowohl die Erzeugung einer unendlichen Menge grammatisch korrekter Sätze aus der endlichen Anzahl an Wörtern einer Sprache ermöglichen als auch ungrammatische Sätze als solche erkennen.\footnote{Vgl. \cite[18]{Skript}.} Daher ist die LFG auch als Grammatikformalismus für die Computerlinguistik interessant, wobei sie der automatisierten Prüfung von Sätzen hinsichtlich ihrer Grammatikalität sowie der Generation neuer grammatischer Sätze dienen soll.\footnote{Vgl. \cite[18]{Skript}; \cite[vii]{Bresnan}.}
 
Da sich die Forschung im Bereich der LFG hinsichtlich der lateinischen Sprache bislang noch in ihren Anfängen befindet, soll in dieser Arbeit die Darstellung der im Lateinischen sehr prävalenten Partizipialkonstruktionen im Konzept der LFG beleuchtet werden. Dabei ist die Kenntnis des grundlegenden Aufbaus lateinischer Partizipialkonstruktionen unabdingbar. Deshalb werden -- nach einer Einführung in Thematik und Terminologie der LFG -- diese verschiedenen Konstruktionen kurz erklärt, bevor sie anhand von Beispielsätzen in das Gerüst der LFG eingefügt werden. Dabei sollen, ausgehend von Syntaxregeln und den jeweiligen Lexikoneinträgen, sowohl c- als auch f-Strukturen zu den einzelnen Phänomenen entwickelt werden. Auch werden verschiedene Ansätze und diverse Schwierigkeiten bei der Umsetzung der verschiedenen Konstruktionen erläutert.

\section{Einführung in Thematik und Terminologie}
Zunächst soll in die Thematik und Terminologie sowohl der Partizipien als auch der Lexikalisch-Funktionalen Grammatik eingeführt werden. Unter letzterem Punkt sollen die Aspekte Syntaxregeln, c-Struktur, Lexikoneinträge, Redundanz- bzw. Default-Regeln, f-Struktur und Mapping zwischen c- und f-Struktur näher erläutert werden. Es gilt zu beachten, dass in dieser Arbeit nur auf das klassische Latein Caesars und Ciceros Bezug genommen wird. 

\subsection{Partizipien}
Die Partizipien nehmen, wie bereits der Name impliziert, teil an den Eigenschaften des Nomens und des Verbums. Die Kongruenz mit dem Bezugswort in Kasus, Numerus und Genus und die Möglichkeit der Steigerung und Substantivierung spiegeln die nominalen, die Teilnahme an Aktionsart, Genus und Rektion des Verbums die verbalen Eigenschaften wider.\footnote{Vgl. LHS, S. 383, §206.}
Im Lateinischen werden drei Partizipien verwendet: das Partizip Präsens Aktiv (PPA), das Partizip Perfekt Passiv (PPP) und das Partizip Futur Aktiv (PFA).
Wie alle Partizipialien bezeichnen die Partizipien jedoch nicht die Zeit an sich, sondern das zeitliche Verhältnis des Partizips zum \textit{verbum finitum}: Dabei kennzeichnet das PPA die Gleichzeitigkeit, das PPP die Vorzeitigkeit und das PPA die Nachzeitigkeit.\footnote{Vgl. KSt, S. 756, §136,3 f.}
Des Weiteren haben PPA und PFA aktivische Bedeutung, das PPP passivische. In der Regel sind auch die Partizipien von Deponentien in der Bedeutung aktivisch.\footnote{NM, S. 708, §496.} Daneben gibt es jedoch einige Partizipien Perfekt, die die Bedeutung eines PPA haben, wie beispielsweise \textit{confisus} oder \textit{diffisus}.\footnote{Vgl. NM, S. 711, §497.}

Partizipien bilden meist in Verbindung mit Substantiven spezifische satzwertige Konstruktionen, in denen das Partizip dem Prädikat, das Bezugswort dem Subjekt eines Nebensatzes entspricht. Dies wird im Weiteren für die Zuweisung der grammatikalischen Funktionen zu den einzelnen Satzbestandteilen von Bedeutung sein.
In dieser Arbeit sollen das Participium coniunctum (PC) -- zuerst in Abhängigkeit des Objekts, dann in Abhängigkeit des Subjekts des Satzes --, das rein attributive Partizip, das substantivierte Partizip, das dominante Partizip, der Ablativus absolutus (Abl. abs.) und der Accusativus cum Participio (AcP) näher betrachtet werden.

\subsection{Lexikalisch-Funktionale Grammatik}
Wie bereits Chomskys Generative Grammatik postuliert auch die LFG zwei Darstellungsebenen: Der oberflächlichen Ebene syntaktischen Aufbaus wird eine zweite abstraktere hinzugefügt. Von Chomskys Ansatz unterscheidet sich die LFG vor allem durch die parallele Darstellung dieser beiden Ebenen: die Oberflächendarstellung und die allgemeinere, funktionale Ebene der LFG entstehen nicht durch Transformationsprozesse und befinden sich in keinem Ableitung-Verhältnis.\footnote{Vgl. \cite[64]{Falk}; \cite[8]{Skript}; \cite[2; 4; 7]{Dal}; \cite[3-4]{Bresnan}; \cite[11; 13]{Rohrer}.} So soll ein höheres Maß an Generalität als in der frühen Transformationsgrammatik erreicht werden.\footnote{Vgl. \cite[1-3; 9]{Dal}. Weitere Unterschiede zu und Kritik an Chomskys Ansatz fassen \cite[11]{Rohrer} konzise zusammen.}

Die parallele Darstellung von Phrasen- und lexikalischer Struktur erlaubt, dass grammatikalische Zusammenhänge nicht nur durch die syntaktischen Beziehungen sondern auch durch morphologische Endungen ausgedrückt werden können.\footnote{Vgl. \cite[10; 14]{Bresnan}.} Dies macht die LFG besonders geeignet zur Darstellung nicht-konfigurationaler Sprachen, wie dem Lateinischen, in denen die grammatikalische Funktion eines Wortes durch dessen morphologische Form bestimmt ist, und nicht durch die Konfiguration der Konstituenten in der c-Struktur;\footnote{Vgl. \cite[19]{Rohrer}; \cite[65]{Dal}.} So ist im Lateinischen beispielsweise die Wortform \textit{canem} stets Akkusativ-Objekt, gleichgültig an welcher Position im Satz sie sich befindet. Die Bedeutung der lexikalischen und funktionalen Elemente spiegelt sich auch im Namen der LFG wieder.\footnote{Vgl.\cite[3]{Dal}.}
% Des Weiteren ist die LFG nicht-kompositionell.\footnote{Vgl. \cite[vii (preface)]{Bresnan}, ,,noncompositional''.}

Die Architektur der LFG basiert auf Einschränkungen, d.h. ungrammatische Sätze werden durch Regeln, die die Möglichkeiten der Satzbildung einschränken, ausgeschlossen.\footnote{Vgl. \cite[vii (preface)]{Bresnan}, ,,constraint-based architecture''.}
In der LFG wird ein kontextfreies Skelett, bestehend aus Syntaxregeln und der oberflächlichen Konstituenten-Struktur – im Folgenden c-Struktur genannt --  durch weitere einschränkende Regeln ergänzt, um die ,,generative Kraft der Grammatik'' zu erhöhen.\footnote{\cite[9]{Rohrer}.} Diese Einschränkungen werden im Lexikon einer Sprache, sowie durch funktionale Annotationen innerhalb der Syntaxregeln und c-Struktur ausgedrückt; weitere Bedingungen können über Redundanz- und Default-Regeln, die über dem Lexikon operieren, aufgestellt werden. All diese Regeln dienen dazu, die Erzeugung ungrammatischer funktionaler Strukturen -- f-Strukturen -- zu verhindern.

Da c- und f-Struktur jeden sprachlichen Ausdruck\footnote{Im Folgenden wird der Einfachheit halber nur von Sätzen die Rede sein.} gemeinsam beschreiben, sind beide für die korrekte Analyse eines Satzes notwendig.\footnote{Vgl. \cite[3]{Dal}; \cite[4]{Skript}.} Die beiden Strukturen repräsentieren unterschiedliche Aspekte linguistischer Organisation und stehen in einem Korrespondenzverhältnis zueinander.\footnote{Vgl. \cite[1]{Dal}.} Die c-Struktur ist die konkrete Darstellung hierarchischer Organisation von Wörtern und Phrasen, vergleichbar mit den Syntaxbäumen kontextfreier Grammatiken.\footnote{Vgl. \cite[7]{Dal}; \cite[13]{Rohrer}. Für eine Definition von ,,Phrase'', siehe \cite[5]{Bresnan}.} Die f-Struktur hingegen beschreibt auf abstrakter Ebene die funktionalen Beziehungen zwischen grammatikalischen Strukturen.\footnote{Vgl. \cite[7]{Dal}; \cite[4]{Skript}.} Über die Relationen der unterschiedlichen Regeln und Strukturen zueinander soll hier bereits kurz ein Überblick gegeben werden.\footnote{Diese Beschreibungen, wie auch alle folgenden hinsichtlich des Aufbauprozesses der Strukturen, beschränken sich auf die Analyse von Sätzen. Für die Generation neuer Sätze laufen diese Prozesse teilweise in anderer Reihenfolge ab.}

Zuerst werden die allgemeingültigen Syntaxregeln auf einen Satz angewandt. Darin sind auch die jeweils möglichen grammatikalischen Funktionen -- darunter können vorerst die Satzglieder verstanden werden -- verzeichnet. Aus der Anwendung der Syntaxregeln auf den Satz ergibt sich die c-Struktur, bzw. mehrere an dieser Stelle möglichen Varianten davon; die grammatikalischen Funktionen aus den Syntaxregeln werden an den jeweils passenden Knoten annotiert. Nun muss eine mit der c-Struktur korrespondierende f-Struktur erstellt werden. In der annotierten c-Struktur werden nun zusammengehörige Strukturen erfasst und die Knoten entsprechend bezeichnet; hierbei werden die funktionalen Annotationen sowie Informationen aus den Lexikoneinträgen und gegebenenfalls weiteren überlexikalischen Regeln miteinbezogen. In Gleichungen mit diesen Knoten-Bezeichnungen werden die c-Struktur-Knoten den f-Strukturen zugeordnet, in der Regel in einem viele-zu-eins-Verhältnis.\footnote{Vgl. \cite[64]{Falk}; \cite[9]{Skript}.} Die Minimallösung dieser funktionalen Gleichungen ist schließlich die f-Struktur.

Die einzelnen Strukturen sollen nun im Folgenden genauer betrachtet werden.

\subsubsection{Syntaxregeln}
Die Syntaxregeln -- oder Phrasenstrukturregeln -- sind der Startpunkt der Analyse eines Satzes in der LFG. Diese Regeln müssen für alle Sätze einer Sprache allgemein gültig sein.\footnote{Vgl. \cite[47]{Dal}.} Dies ist möglich, da in der LFG, anders als in früheren Ansätzen,\footnote{Ein Beispiel hierfür wäre eine kontextfreie Grammatik nach Chomsky.} viele für die Erzeugung der Satzstruktur notwendigen Bedingungen in das Lexikon ausgelagert werden. Durch die Sytaxregeln wird die grammatikalisch korrekte Verkettung von Wörtern zu Phrasen und Phrasen zu einem Satz schematisch gewährleistet. Die Darstellung erfolgt allerdings von den größeren Elementen hin zu den kleineren, d.h. vom Satz über die Phrasen bis hin zu den Wörtern. 

Generalisierungen konkreter Phrasen führen zu ihrer Klassifikation als beispielsweise Nominalphrasen oder Verbalphrasen, je nachdem, welcher Kategorie das lexikalische Element angehört, das die Phrase dominiert.\footnote{Vgl. \cite[47; 53; 58-9]{Dal}; \cite[15]{Rohrer}.} Die lexikalischen Kategorien, die für das Lateinische angenommen werden können, sind N(omen), V(erb), P(räposition), A(djektiv) und Adv(erb).\footnote{Vgl. \cite[46; 52]{Dal} für das Englische.}

Die Syntax des Lateinischen organisiert sich, wie bei einer Vielzahl weiterer Sprachen, um ein finites Verb.\footnote{Vgl. \cite[53]{Dal}. In Sprachen, in denen dieses finite Verbalelement stets in einer bestimmten Position im Satz vorkommt, wird seine dominierende Phrase als IP, für inflectional phrase, bezeichnet. I ist dabei eine funktionale, keine lexikalische, Kategorie. Eine andere funktionale Kategorie, C, für complementizer phrase, existiert zwar auch im Lateinischen, wird jedoch in dieser Arbeit aus Platzgründen außer Acht gelassen; vgl. \cite[46; 53; 63-5]{Dal}. Für Überlegungen bezüglich der Definition eines Konstituenten, siehe \cite[48-9]{Dal}.}

Da das Lateinische eine nicht-konfigurationale Sprache ist, deren Bestandteile weitgehend frei umgestellt werden können, wird der Kopf eines jeden Satzes im Lateinischen als Kategorie ,,S'' klassifiziert.\footnote{Die Kategorie S wird nach Leonard Bloomfield als ,,exozentrisch'' bezeichnet und den sogenannten endozentrischen Kategorien gegenübergestellt, welche lexikalische Köpfe besitzen; vgl. \cite[46]{Dal}.} Die Töchter von S sind ein Prädikat mit seinen Argumtenten, inklusive des Subjekts, und gegebenenfalls weitere, optionale Konstituenten.\footnote{Vgl. \cite[64-65]{Dal}.}

Folglich besteht S aus einer bestimmten Menge von Phrasen und lexikalischen Elementen. In den Syntaxregeln wird dieser Zusammenhang durch ,,kontextfreie Ersetzungsregeln der Form ,ersetze a durch b' (konventionell geschrieben: ,a $\rightarrow$ b')'' beschrieben.\footnote{\cite[18]{Rohrer}.}

Auch wenn die Syntaxregeln der LFG verglichen mit früheren Ansätzen weitgehend überschaubar sind, werden hier nur die für die Partizipialkonstruktionen relevanten Regeln angegeben, da alles Weitere den Rahmen dieser Arbeit sprengen würde:\footnote{Diese Regeln erheben keinen Anspruch auf Vollständigkeit. Da Adjektive, Adverbien und Pronominaladjektive (zu denen sich u.a. bei Snijders unter dem Begriff ,,Determiner'' weitere Informationen finden), sowie Nebensätze aller Art  in dieser Arbeit nicht von Bedeutung sind, wird in den folgenden Syntaxregeln und auch im Weiteren nicht darauf eingegangen. Dies betrifft natürlich nur lateinische Nebensätze: Während Partizipialkonstruktionen im Deutschen zwar in der Regel durch Gliedsätze wiedergegeben werden, sind sie im Lateinischen Teil des Hauptsatzes. Auch die formal schwierige Frage der Koordination kann hier nicht berücksichtigt werden; Überlegungen bezüglich der Subjunktionen und Konjunktionen des Deutschen in der LFG finden sich unter \cite[103-119]{Skript} bzw. \cite[120-136]{Skript}. Ebensowenig werden -- abgesehen von den Präpositionalphrasen -- Fälle beachtet, in denen auch im Lateinischen die Wortstellung eine Rolle spielt, wie beispielsweise bei der Anfangsstellung bei Befehlen und Aufforderungen oder beim Konzessiv; vgl. NM S. 577, §422. Schließlich werden auch die im Lateinischen häufig auftretenden diskontinuierlichen Phrasen aus Gründen der Relevanz für diese Arbeit außer acht gelassen; mehr Informationen hierzu finden sich bei \cite{Snijders}.} 
\\
\\
\begin{tabular}{ l  l  c  c  c  c  c  c  c}
S & $\rightarrow$ & [V & , & XP] & $\mid$ & V \\
%   & $\qquad$ & \textsuperscript{$\uparrow$ = $\downarrow$} & \textsuperscript{($\uparrow$ADJ-GF) = $\downarrow$ $\mid$ ($\uparrow$ARG-GF) = $\downarrow$} &  \textsuperscript{$\uparrow$ = $\downarrow$} \\
%      & $\qquad$ & \textsuperscript{($\uparrow$FIN)} & &  \textsuperscript{($\uparrow$FIN)}\\
%XP & $\rightarrow$ & \{ NP & , & VP & , & PP \}* \\
XP & $\rightarrow$ & \{ NP & $\mid$ &  VP & $\mid$ & PP \}* \\
%   & $\qquad$ & \textsuperscript{($\uparrow$ADJ-GF) = $\downarrow$ $\mid$ ($\uparrow$ARG-GF) = $\downarrow$} &\textsuperscript{($\uparrow$ADJ-GF) = $\downarrow$ $\mid$ ($\uparrow$ARG-GF) = $\downarrow$} & \textsuperscript{($\uparrow$ADJ-GF) = $\downarrow$ $\mid$ ($\uparrow$ARG-GF) = $\downarrow$} \\
   NP & $\rightarrow$ & [N & & XP] & $\mid$ & N \\
 %  & $\qquad$ & \textsuperscript{$\uparrow$ = $\downarrow$}  & \textsuperscript{($\uparrow$ADJ-GF) = $\downarrow$ $\mid$ ($\uparrow$ARG-GF) = $\downarrow$} & \textsuperscript{$\uparrow$ = $\downarrow$}\\
VP & $\rightarrow$ & [V & &  XP] & $\mid$ & V \\
 %  & $\qquad$ & \textsuperscript{$\uparrow$ = $\downarrow$} & \textsuperscript{($\uparrow$ADJ-GF) = $\downarrow$ $\mid$ ($\uparrow$ARG-GF) = $\downarrow$} & \textsuperscript{($\uparrow$ADJ-GF) = $\downarrow$ $\mid$ ($\uparrow$ARG-GF) = $\downarrow$} & \textsuperscript{$\uparrow$ = $\downarrow$}\\
   PP & $\rightarrow$ & [P & , & VP] & $\mid$  & [P & ,   & NP]
  % & $\qquad$ & \textsuperscript{$\uparrow$ = $\downarrow$} & \textsuperscript{($\uparrow$ADJ-GF) = $\downarrow$ $\mid$ ($\uparrow$ARG-GF) = $\downarrow$} & \textsuperscript{$\uparrow$ = $\downarrow$} & \textsuperscript{($\uparrow$ADJ-GF) = $\downarrow$ $\mid$ ($\uparrow$ARG-GF) = $\downarrow$}\\
%PP & $\rightarrow$ & P &  \{VP $\mid$     & NP\}* \\
 %  & $\qquad$ & \textsuperscript{$\uparrow$ = $\downarrow$} & \textsuperscript{($\uparrow$ADJ-GF) = $\downarrow$ $\mid$ ($\uparrow$ARG-GF) = $\downarrow$} & \textsuperscript{($\uparrow$ADJ-GF) = $\downarrow$ $\mid$ ($\uparrow$ARG-GF) = $\downarrow$}\\
\end{tabular} \\

Die erste Regel besagt, dass S entweder durch V und XP ersetzt wird oder nur durch V. Der vertikale Disjunktionsstrich $\mid$ denotiert dabei die ,entweder-oder'-Beziehung; die eckigen Klammern sollen lediglich zeigen, dass V und XP als Einheit zusammengefasst werden und den ersten Teil der Disjunktion bilden. Die Reihenfolge von V und XP wird durch das zwischen diesen beiden Elementen stehende Komma offen gelassen: Dieses Komma fungiert hier als sogenannter ,shuffle operator'. V bezeichnet ein finites Verbalelement, XP wird in der Zeile darunter definiert als eine Menge an Nominalphrasen (NP), Verbalphrasen (VP) und Präpositionalphrasen (PP). Der Asterisk ist ein Kleene-Stern, der die vorangehende Menge zu einer Kleenschen Hülle umdefiniert; in einer solchen Kleenschen Hülle können die einzelnen Elemente beliebig oft -- also zum Beispiel auch überhaupt nicht -- und in beliebiger Reihenfolge vorkommen.\footnote{In der Menge werden hier Disjunktionsstriche verwendet anstatt wie oft üblich Kommata, da das Komma in diesem Kontext als shuffle operator definiert wurde; vgl. \cite[26]{Skript}; \cite{Snijders}.} Zusammengenommen erlauben die Regeln der ersten und zweiten Zeile also, dass ein Satz beispielsweise aus einer Nominalphrase, einem finiten Verbalelement und einer weiteren Nominalphrase besteht. Die Tatsache, dass diese nur eine von vielen möglichen Auflösungen von S ist, wird der weitgehend freien Wortstellung des Lateinischen geschuldet.\footnote{Vgl. \cite[19]{Rohrer}.}

In den Zeilen 3 bis 5 werden die jeweiligen Expansionsmöglichkeiten von NP, VP und PP beschrieben. Eine Nominalphrase besteht demnach im Allgemeinen entweder aus einem lexikalischen Element der Kategorie N – d.h. aus einem Nomen -- und einer beliebigen, ungeordneten Menge an NPs, VPs und PPs, oder aber nur aus einem Nomen. Dasselbe gilt analog für Verbalphrasen. Die Notation der Expansionsmöglichkeiten einer Präpositionalphrase enthält keine Kleensche Hülle, da in einer Präpositionalphrase stets zumindest ein Element einer VP oder NP direkt nach (bzw. vor) einer Präposition (bzw. Postposition) stehen muss.

Zu beachten ist, dass Partizipialien – d.h. Partizipien, Infinitve und Gerundialien -- in dieser Arbeit, ebenso wie das finite Verb, als V bezeichnet werden. Dies macht insofern Sinn, als V im Lateinischen stets der Kopf von S ist,\footnote{Eine Ausnahme hiervon könnte eventuell der nominale Ablativus absolutus darstellen, dessen Kopf nach logischer Betrachtung N sein müsste -- vgl. NM S. 721, §504; \cite[64]{Falk}. Wenn man jedoch die Abl-abs-Konstruktion S\textsubscript{part} unterordnet, wie wir es hier tun, ergibt sich kein Problem hinsichtlich S an sich.} und die Prädikate der Partizipien die übrigen Argumente der Partizipialkonstruktion ebenso fordern wie das finite V die Argumente von S;\footnote{Allerdings werden die Partizipialkonstruktionen in den späteren Ausführungen -- abgesehen vom S\textsubscript{part} des Ablativus absolutus -- nicht explizit als S bezeichnet, um ihre Zugehörigkeit zum Rest des Satzes anzuzeigen.}

Allerdings sind die Syntaxregeln so wie hier dargestellt noch nicht komplett; es fehlen die Annotationen der grammatikalischen Funktionen, die ein Konstituent an der jeweiligen Position einnehmen kann. Eine der Stärken der LFG besteht darin, dass durch diese funktionalen Annotationen die Menge der Syntaxregeln stark reduziert wird.\footnote{Vgl. \cite[45]{Dal}.} In der LFG wird ein universelles Inventar an grammatikalischen Funktionen angenommen; dieses umfasst im Allgemeinen Subjekt (SUBJ), Objekt (OBJ), thematisch restringiertes Objekt (OBJ\textsubscript{$\theta$}), Oblique (OBL\textsubscript{$\theta$}), Komplement (COMP), ,offenes' Komplement (XCOMP), Adjunkt (ADJ) und ,offenes' Adjunkt (XADJ).\footnote{Vgl. \cite[9]{Dal}; \cite[58]{Falk}. Neben den grammatikalischen Funktionen gibt es auch Diskursfunktionen; vgl. \cite[28; 76-84; 94-101]{Skript}. Auf diese wird in dieser Arbeit nicht eingegangen.} Diese Funktionen sind im Allgemeinen nicht auf spezielle syntaktische Kategorien beschränkt und werden in verschiedenen Sprachen unterschiedlich realisiert.\footnote{Vgl. \cite[9-10]{Bresnan}.} Die genannten grammatikalischen Funktionen können auf verschiedene Art und Weise nach bestimmten Gemeinsamkeiten bzw. Unterschieden klassifiziert werden.\footnote{Vgl. \cite[56-8]{Falk}.} 

Ein wichtiger Unterschied besteht zwischen den vom Prädikat regierbaren Funktionen einerseits, und ADJ und XADJ andererseits.\footnote{Vgl. \cite[56]{Falk}.} Wie bei der Besprechung der Lexikoneinträge deutlich werden wird, fordert das Prädikat seine Argumente; daher werden die regierbaren Funktionen auch Argument-Funktionen (ARG-GF oder AF) genannt.\footnote{Vgl. \cite[28]{Skript}; \cite[58]{Falk}.} Da Adjunkte keinerlei syntaktische Wichtigkeit für die Grammatikalität eines Satzes besitzen, werden sie nicht vom Prädikat gefordert bzw. regiert; sie können als Adjunkt-Funktionen (ADJ-GF) bezeichnet werden.\footnote{Vgl. \cite[10-1]{Dal}; \cite[38]{Skript}.} Attributive Adjektive und Adverbien beispielsweise treten immer in der Funktion eines ADJ auf, Präpositionalphrasen häufig.\footnote{Vgl. \cite[38]{Skript}. Mehr Informationen zu Adjunkten finden sich beispielsweise in \cite[61-2]{Falk}.} Diese grundlegende Unterscheidung der grammatikalischen Funktionen in Argument- und Adjunkt-Funktionen können folgendermaßen formelhaft dargestellt werden:\footnote{Diese Darstellung wurde reflektiert übernommen aus \cite{Snijders}. GF steht für grammatikalische Funktion. Die Element-Zeichen werden bei der Betrachtung der f-Struktur erklärt.}

\begin{singlespace}
\begin{tabular}{ l  l  l  c  c  c  c }
GF & $\equiv$ & \{ARG-GF $\mid$ ADJ-GF\} \\
ARG-GF & $\equiv$ & \{SUBJ $\mid$ OBJ $\mid$\ OBJ\textsubscript{$\theta$} $\mid$ OBL\textsubscript{$\theta$} $\mid$ OBL\textsubscript{$\theta$} OBJ $\mid$ ADJ $\in$ OBJ\} \\
ADJ-GF & $\equiv$ & \{ADJ $\in$ $\mid$ XADJ $\in$\} \\
\end{tabular}\\
\end{singlespace}

Weiter ist den Funktionen COMP, XCOMP und XADJ eigen, dass sie stets satzwertig sind;\footnote{Vgl. \cite[24]{Dal}, ,,clausal functions''.} Ein ADJ kann, muss jedoch keine satzwertige Konstruktion sein.\footnote{Vgl. \cite[40]{Skript}.} Während ADJ und COMP geschlossene Funktionen sind, können XADJ und XCOMP insofern als ,offen' bezeichnet werden, als sie kein internes Subjekt enthalten; dies wird bei der Betrachtung der f-Struktur verständlicher werden.\footnote{Vgl. \cite[10; 14; 24]{Dal}; \cite[54]{Skript}.}

Ein weiterer Unterschied besteht zwischen den semantisch unrestringierten Funktionen auf der einen und OBJ\textsubscript{$\theta$} und OBL\textsubscript{$\theta$} auf der anderen Seite.\footnote{Vgl. \cite[10; 15-7]{Dal}.} Sprachen erlauben in aller Regel nur ein thematisch unrestringiertes Objekt, jedoch zusätzlich ein oder mehrere thematisch beschränkte.\footnote{Vgl. \cite[21]{Dal}.} Während OBJ im Lateinischen das direkte Objekt bezeichnet,\footnote{Der Kasus des direkten Objekts ist im Lateinischen vom Verb abhängig und muss im Lexikoneintrag des entsprechenden Verbs festgelegt werden; vgl. \cite[30]{Skript}.} wird OBJ\textsubscript{$\theta$} für das indirekte Objekt verwendet – beispielsweise ,tibi' in ,dono tibi librum', wobei ,librum' das direkte Objekt ist.\footnote{\cite[30]{Skript} bestätigt diese Zuteilung für das Deutsche; dort finden sich auch nähere Erklärungen hierzu. Welche Verben ein indirektes Objekt zu sich nehmen muss ebenfalls in den jeweiligen Lexikoneinträgen festgelegt werden; die Hinzunahme eines indirekten Objekts muss auch in den c-Struktur-Annotationen erlaubt sein.} Der Index $\theta$ wird in der konkreten Verwendung durch eine Abkürzung der semantischen Rolle ersetzt,\footnote{Vgl. \cite[32]{Skript}; \cite[21]{Rohrer}.} im obigen Beispiel also durch ,rec', für ,receiver'. Auch Oblique-Argumente zeigen ihre semantische Rolle stets an;\footnote{Vgl. \cite[26]{Dal}.} Präpositionalphrasen beispielsweise erfüllen häufig die Funktionen OBL\textsubscript{goal} oder OBL\textsubscript{loc}.

Unter Einbeziehung der grammatikalischen Funktionen sähen die ersten beiden Zeilen der oben aufgeführten Syntaxregeln etwa folgendermaßen aus:
%\textbf{Variante 1 - Disjunktivität beachtet (in Anlehnung an Snijder}
%\begin{singlespace}
%\begin{tabular}{ l  l  c  c  c  c  c  c  c}
%S & $\rightarrow$ & [\{V & $\mid$ & VP & $\mid$ & NP & $\mid$ & PP \}*] ,\\
 % & $\qquad$ & \textsuperscript{$\uparrow$ = $\downarrow$} & & \textsuperscript{($\uparrow$ARG-GF) = $\downarrow$} & & \textsuperscript{($\uparrow$ARG-GF) = $\downarrow$} & & \textsuperscript{($\uparrow$OBL$\theta$ $\mid$  ADJ-GF) = $\downarrow$} \\
  %$\qquad$ & $\qquad$ & [ (XP) & (\{ V & $\mid$ & NP \})& (XP)]* \\
  %& $\qquad$ & \textsuperscript{($\uparrow$ADJ-GF) = $\downarrow$} & \textsuperscript{$\uparrow$ = $\downarrow$} & & \textsuperscript{($\uparrow$ARG-GF) = $\downarrow$} & \textsuperscript{($\uparrow$ADJ-GF) = $\downarrow$} \\
 % XP & $\rightarrow$ & \{ NP, &  VP, & PP \}* \\
%\end{tabular}\\
%\newline
%\newline
%\end{singlespace}

\begin{singlespace}
\begin{tabular}{ l  l  c  c  c  c  c  c  c}
S & $\rightarrow$ & V, & XP $\mid$ & V \\
   & $\qquad$ & \textsuperscript{$\uparrow$ = $\downarrow$} & \textsuperscript{($\uparrow$ADJ-GF) = $\downarrow$ $\mid$ ($\uparrow$ARG-GF) = $\downarrow$} &  \textsuperscript{$\uparrow$ = $\downarrow$} \\
      & $\qquad$ & \textsuperscript{($\downarrow$FIN) = +} & &  \textsuperscript{($\downarrow$FIN)= +} \\
%XP & $\rightarrow$ & \{ NP & , & VP & , & PP \}* \\
XP & $\rightarrow$ & \{ NP $\mid$ &  VP $\mid$ & PP \}* \\
   & $\qquad$ & \textsuperscript{\{ ($\uparrow$SUBJ) = $\downarrow$} &\textsuperscript{\{ ($\uparrow$SUBJ) = $\downarrow$} & \textsuperscript{\{ ($\uparrow$ADJ) = $\downarrow$} \\
    & $\qquad$ & \textsuperscript{($\downarrow$CASE) = nom} & \textsuperscript{($\downarrow$FIN) = -} &  \\
    & $\qquad$ & \textsuperscript{$\mid$ ($\uparrow$OBJ) = $\downarrow$} & \textsuperscript{$\mid$ ($\uparrow$OBJ) = $\downarrow$} &  \textsuperscript{$\mid$ ($\uparrow$OBL\textsubscript{$\theta$}) = $\downarrow$ \}} \\
       & $\qquad$ & \textsuperscript{$\mid$ ($\uparrow$OBL\textsubscript{$\theta$}) = $\downarrow$} & \textsuperscript{$\mid$ ($\uparrow$OBL\textsubscript{$\theta$}) = $\downarrow$} & \\
         & $\qquad$ & \textsuperscript{$\mid$ ($\uparrow$ADJ) = $\downarrow$} & \textsuperscript{$\mid$ ($\uparrow$ADJ) = $\downarrow$} &  \\
           & $\qquad$ & \textsuperscript{$\mid$ ($\uparrow$COMP) = $\downarrow$ \}} & \textsuperscript{$\mid$ ($\uparrow$COMP) = $\downarrow$} &  \\
             & $\qquad$ & & \textsuperscript{$\mid$ ($\uparrow$XADJ) = $\downarrow$} &  \\
               & $\qquad$ & & \textsuperscript{$\mid$ ($\uparrow$XCOMP) = $\downarrow$ \} } &  \\
%   NP & $\rightarrow$ & N &  \{XP\}* $\mid$ & N \\
%   & $\qquad$ & \textsuperscript{$\uparrow$ = $\downarrow$}  & \textsuperscript{($\uparrow$ADJ-GF) = $\downarrow$ $\mid$ ($\uparrow$ARG-GF) = $\downarrow$} & \textsuperscript{$\uparrow$ = $\downarrow$}\\
%VP & $\rightarrow$ & V  & \{PP $\mid$     & NP\}* $\mid$ & V \\
%   & $\qquad$ & \textsuperscript{$\uparrow$ = $\downarrow$} & \textsuperscript{($\uparrow$ADJ-GF) = $\downarrow$ $\mid$ ($\uparrow$ARG-GF) = $\downarrow$} & \textsuperscript{($\uparrow$ADJ-GF) = $\downarrow$ $\mid$ ($\uparrow$ARG-GF) = $\downarrow$} & \textsuperscript{$\uparrow$ = $\downarrow$}\\
%   PP & $\rightarrow$ & P, & VP $\mid$  & P,   & NP \\
%   & $\qquad$ & \textsuperscript{$\uparrow$ = $\downarrow$} & \textsuperscript{($\uparrow$ADJ-GF) = $\downarrow$ $\mid$ ($\uparrow$ARG-GF) = $\downarrow$} & \textsuperscript{$\uparrow$ = $\downarrow$} & \textsuperscript{($\uparrow$ADJ-GF) = $\downarrow$ $\mid$ ($\uparrow$ARG-GF) = $\downarrow$}\\
%PP & $\rightarrow$ & P &  \{VP $\mid$     & NP\}* \\
 %  & $\qquad$ & \textsuperscript{$\uparrow$ = $\downarrow$} & \textsuperscript{($\uparrow$ADJ-GF) = $\downarrow$ $\mid$ ($\uparrow$ARG-GF) = $\downarrow$} & \textsuperscript{($\uparrow$ADJ-GF) = $\downarrow$ $\mid$ ($\uparrow$ARG-GF) = $\downarrow$}\\
\end{tabular}\\
\newline
\end{singlespace}

Diese funktionalen Annotationen zeigen beispielsweise an, welche grammatikalischen Funktionen eine NP, die direkt von S dominiert wird, einnehmen kann; sie besagen auch, dass ihr Kasus, wenn sie an dieser Position als Subjekt auftritt, Nominativ sein muss. Die Pfeile werden bei der Betrachtung der c-Struktur verständlich werden; auf die Annotation ($\downarrow$FIN) = + wird im Abschnitt ,,Redundanz- und Defaultregeln'' näher eingegangen werden. Diese Annotationen sind hier jedoch nur exemplarisch für die ersten zwei Zeilen der Syntaxregeln dargestellt, da die erschöpfende Notation aller möglichen grammatikalischen Funktionen für jeden Konstituenten den Rahmen dieser Arbeit bei Weitem überschreiten würde. So müsste z.B. XP in den obigen Regeln konkret in NP, VP und PP getrennt dargestellt werden, damit gezeigt werden könnte, dass eine NP, die direkt von S dominiert wird, andere grammatikalische Funktionen erfüllen kann als eine NP, die von einer anderen NP, VP oder PP dominiert wird; so kann beispielsweise das Subjekt einer PC-VP in allen Kasus vorkommen, nicht nur im Nominativ.\footnote{Eine NP, die direkt von S dominiert wird, kann z.b. SUBJ sein und im Nominativ stehen, während dies für eine NP, die von einer anderen NP dominiert wird, nicht möglich ist.}

\subsubsection{c-Struktur}
Aus den Syntaxregeln, und in der Regel unter Einbeziehung der Wortform und ihrer lexikalischen Kategorie aus dem Lexikon, kann die c-Struktur erzeugt werden.\footnote{Vgl. \cite[14]{Rohrer}; \cite[6]{Skript}.} In der LFG repräsentiert die c-Struktur die oberflächliche, konkrete Konfiguration der Satz-Konstituenten.\footnote{Vgl. \cite[47]{Dal}. Überlegungen über Kriterien für Konstituenten finden sich in \cite[48-9]{Dal}.} In ihr ist sowohl die hierarchische Dominanz der Konstituenten als auch die lineare Reihenfolge der lexikalischen Elemente sichtbar.\footnote{Vgl. \cite[7]{Dal}; \cite[13]{Rohrer}.} Diese Dominanzverhältnisse -- sowie, v.a. in konfigurationalen Sprachen, auch Präzedenzverhältnisse -- zwischen den Konstituenten werden bereits durch die Syntaxregeln festgelegt.\footnote{Vgl. \cite[19]{Rohrer}.} Jedes dort aufgeführte lexikalische Element und jede Phrase kann einen Konstituenten in der c-Struktur bezeichnen. Jeder c-Struktur-Knoten stellt einen Konstituenten dar.\footnote{Vgl. \cite[5]{Skript}.} 

Die lexikalischen Elemente befinden sich auf der untersten Ebene der c-Struktur -- sie bilden sozusagen die Blätter des Syntaxbaums.\footnote{Vgl. \cite[7]{Dal}. Dort können nur einzelne, vollständige Wörter stehen, d.h. keine Phrasen oder bloße Affixe.} Sie werden hier in Anlehnung an Rohrer und Schwarze als terminale Knoten bezeichnet.\footnote{Vgl. \cite[14; 61]{Rohrer}.} Über jedem terminalen Knoten steht direkt ein nicht-terminaler Knoten gleicher lexikalischer Kategorie;\footnote{Funktionale Kategorien -- vgl. \cite[46; 53; 63-4]{Dal} -- werden, wie oben erwähnt, in dieser Arbeit nicht berücksichtigt. Dass ein lexikalisch Element von mehreren c-Struktur-Knoten dominiert wird, ist ausgeschlossen; vgl. \cite[63]{Skript}.} das Wort \textit{Caesar} beispielsweise würde, da es ein Nomen ist, von einem Knoten der Kategorie N unmittelbar dominiert werden.

Oberhalb dieser Knoten befinden sich eine bis mehrere Ebenen nicht-terminaler Knoten der Konstituenten-Kategorien, also z.B. S, NP, VP, etc. Diese Konstituenten-Kategorien werden als Projektionen ihrer jeweiligen lexikalischen Kategorien angesehen; das Wort, das dabei die beherrschende Rolle spielt, wird als Kopf der entsprechenden Phrase bezeichnet.\footnote{Vgl. \cite[13; 15]{Rohrer}; \cite[64]{Dal}; \cite[5; 28]{Skript}.} So ist beispielsweise VP als Maximalphrase eine Projektion des Verbs V, wobei V der Kopf von VP ist.\footnote{S kann ebenfalls als Projektion des Verbs betrachtet werden, jedoch auf einer noch höheren Ebene als die VP; vgl. \cite[15]{Rohrer}.} 

Zwischen der lexikalischen Kategorie (beispielsweise V) und ihrer Maximalprojektion (VP) können sich Zwischenprojektionen befinden; diese werden, in Anlehnung an die X-Bar-Theorie, mit einem Strich pro Projektionsebene gekennzeichnet: V', V'', usw.\footnote{Vgl. \cite[15-6]{Rohrer}; \cite[56-7]{Dal}; \cite[5]{Skript}.} Aufgrund des Prinzips der Ökonomie des Ausdrucks (economy of expression) werden diese Knoten in der LFG jedoch nur dargestellt, wenn sie für die syntaktische Struktur von Bedeutung sind. Latein hat im Allgemeinen eine flache c-Struktur mit wenigen Zwischenprojektionsebenen, was seiner Nicht-Konfigurationalität geschuldet ist.\footnote{Vgl. \cite[46]{Rohrer}.} Nach der sogenannten ,,Chomsky-Adjunktion'' können zudem Maximalphrasen anderen Maximalphrasen untergeordnet werden.\footnote{Vgl. \cite[46; 57]{Dal}.}

Die grammatikalischen Funktionen aus den Syntaxregeln werden ebenfalls in die c-Struktur übertragen. Eine c-Struktur mit funktionalen Annotationen wird annotierte c-Struktur genannt.\footnote{Vgl. \cite[69]{Falk}; \cite[22]{Rohrer}.} Die Annotationen haben die Form funktionaler Schemata: verallgemeinert ($\uparrow$GF) = $\downarrow$.\footnote{Vgl. \cite[33]{Rohrer}; \cite[15-6]{Skript}.} Ein nach unten gerichteter Pfeil verweist, vereinfacht ausgedrückt, auf den Knoten, dem die funktionale Annotation zugeordnet ist. Ein nach oben gerichteter Pfeil bedeutet, dass die funktionale Information der dahinterstehenden Funktion an den übergeordneten Knoten ,weitergegeben' werden soll. Beim sogenannten trivialen Schema $\uparrow$ = $\downarrow$ werden alle funktionalen Informationen nach oben gegeben; Mutter- und Tochterknoten tragen also dieselbe funktionale Information. Dieses funktionale Schema findet sich immer dann, wenn ,,zwei Knoten in einem Projektionsverhältnis zueinander stehen'', also beispielsweise an jedem N, das direkt von einer NP dominiert wird, oder auch bei jedem V, das direkt unter S steht.\footnote{\cite[28]{Skript}; vgl. \cite[25; 33]{Rohrer}. Die genaue Bedeutung der Pfeile wird bei der Betrachtung des Mappings zwischen c- und F-Struktur kurz erklärt werden.} Durch die Annotationen werden den Konstituenten also ihre jeweiligen grammatikalischen Funktionen zugeordnet.\footnote{Vgl. \cite[28]{Skript}.}

Eine beispielhafte c-Struktur eines lateinischen Satzes ist zur Veranschaulichung hier dargestellt:

\begin{singlespace}
\Tree [.S 
%		[.{NP\textsubscript{($\uparrow$SUBJ)=$\downarrow$}}
%			[.N\textsubscript{$\uparrow$=$\downarrow$} Caesar ]
%		]							
		[.{NP\textsubscript{($\uparrow$OBJ)=$\downarrow$}}
			[.N\textsubscript{$\uparrow$=$\downarrow$} barbaros ]
		]							
		[.{PP\textsubscript{$\downarrow$ $\in$($\uparrow$ADJ)}}
			[.P'\textsubscript{$\uparrow$=$\downarrow$}						
				[.P\textsubscript{$\uparrow$=$\downarrow$} in ]
				[\qroof{Gallia}.{NP\textsubscript{($\uparrow$OBJ)=$\downarrow$}} ]
			]
        ] 	
        [.V\textsubscript{$\uparrow$=$\downarrow$} vicit ]						
	]
\end{singlespace}

\textit{barbaros in Gallia vicit.} \\

Die terminalen Knoten \textit{barbaros}, \textit{in}, \textit{Gallia} und \textit{vicit} repräsentieren die lexikalischen Elemente in ihrer linearen Reihenfolge. Darüber befinden sich, wie bei \textit{barbaros} ersichtlich wird, Knoten, die mit der lexikalischen Kategorie des darunter stehenden lexikalischen Elements bezeichnet werden. So werden \textit{barbaros} von N, \textit{in} von P, \textit{Gallia} von N und \textit{vicit} von V dominiert. Für eine einfachere und übersichtlichere Darstellung können diese Knoten in einem Dreieck zusammengefasst dargestellt werden, wie bei \textit{Gallia} ersichtlich wird. Zwischen der lexikalischen Kategorie P und ihrer Maximalprojektion PP findet sich die Zwischenprojektion P'. Trotz des Prinzips der Ökonomie des Ausdrucks wird bei einer PP diese Zwischenprojektionsebene gekennzeichnet,  da in einer Präpositionalphrase auf eine Präposition direkt wenigstens ein Teil einer VP oder NP folgen muss.\footnote{Vgl. \cite{Snijders}.} So werden P und NP von der Zwischenprojektion P', P' wiederum von seiner Maximalprojektion PP dominiert. NP, PP und V werden letztlich von S dominiert.

Daneben erhalten die Knoten funktionale Annotationen. Dabei ist V durch das trivile Schema $\uparrow$ = $\downarrow$ sozusagen als Kopf von S gekennzeichnet. Die Annotationen zu den Knoten der Konstituenten-Kategorien geben deren konkrete grammatikalischen Funktionen nach oben: So stellen die NP, die \textit{barbaros} enthält, das Objekt von S, die PP das Adjunkt von S, und die NP, die \textit{Gallia} beinhaltet, das Objekt von P dar.

\subsubsection{Lexikoneinträge}
Wie oben erwähnt fließen beim Aufbau der c-Struktur eines konkreten Satzes Informationen aus dem Lexikon einer Sprache mit ein. Des Weiteren werden die Lexikoneinträgen -- gemeinsam mit den funktionalen Annotationen der c-Struktur  -- benötigt, um die f-Struktur aufzubauen.\footnote{Vgl. \cite[63]{Skript}.} Da im Lateinischen im Gegensatz zu den modernen Sprachen die Wortstellung innerhalb eines Satzes nicht explizit festgelegt ist,\footnote{Die gewöhnliche Wortstellung im Lateinischen ist zwar Subjekt – Objekt – Prädikat, jedoch wird diese, vor allem aus Gründen der Betonung und des Wohlklangs, nur selten streng eingehalten. Vgl. LHS, S. 397, §212.} muss der Großteil dieser Bedingungen nicht wie üblicherweise in den Syntaxregeln, sondern in den Lexikoneinträgen festgelegt werden.\footnote{Vgl. \cite[6]{Bresnan}.} Das Lexikon der LFG listet, anders als das anderer Grammatiktheorien, nicht nur Ausnahmen auf, sondern stellt einen grundlegenden Bestandteil der Theorie dar, der für die Analyse bzw. Erzeugung eines jeden Satzes vonnöten ist.\footnote{Vgl. \cite[3]{Dal}.} Jede einzelne Flexionsform eines Wortes erhält ihren eigenen Lexikoneintrag.\footnote{Tatsächlich werden im Lexikon auch die systematischen Beziehungen zwischen den lexikalischen Elementen durch Regeln zur morphologischen Umformung festgehalten (vgl. \cite[3]{Dal}). Vor allem die regelmäßigen Formen werden im Normalfall von einem Computerprogramm erzeugt (vgl. \cite[15]{Rohrer}). Bei \cite[63-76]{Skript} und \cite[20-21]{Rohrer} finden sich weitere Erklärungen zu diesen lexikalischen Regeln. Im Rahmen dieser Arbeit werden lediglich einige Beispiel-Lexikoneinträge dargestellt. Daher wird auch nicht auf andere Verwendungsmöglichkeiten der Partizipien eingegangen, wie beispielsweise in Verbindung mit einer Form von \textit{esse}.}

% ``Zugleich unterliegen die möglichen natürlichen Sprachen gewissen Einschränkungen was die mögliche Verkettung von Wörtern zu Sätzen betrifft." (Skript S. 4)

Jeder Lexikoneintrag beinhaltet also die schriftliche bzw. lautliche Form des Wortes -- die gewöhnlich mit ,,PRED'', für Prädikat, bezeichnet wird --, seine lexikalische Kategorie sowie diverse funktionale Spezifikationen.\footnote{Vgl. \cite[27; 33]{Rohrer}; \cite[16]{Skript}.} Die lexikalische Kategorie wird für die Zuordnung von Lexemen zu ihren möglichen terminalen c-Struktur-Knoten benötigt.\footnote{Vgl. \cite[63]{Skript}.} Die funktionalen Bestimmungen finden sich in der späteren f-Struktur wieder.

Das Prädikat eines jeden Wortes im Lexikon fordert bestimmte syntaktische Argumente; diese entsprechen in der Valenzgrammatik der Menge der Ergänzungen, die ein Verb zu sich nehmen kann. Sie werden im sogenannten Subkategorisierungsrahmen -- gekennzeichnet durch $\langle$ $\rangle$ -- aufgeführt.\footnote{Vgl. \cite[7]{Dal}; \cite[70]{Skript}; \cite[27]{Rohrer}.} In diesem Subkategorisierungsrahmen können sämtliche grammatikalische Funktionen auftreten; da Adjunkte keine regierbaren Funktionen sind, tauchen sie in den Lexikoneinträgen nicht auf.\footnote{Vgl. \cite[27]{Rohrer}.} Die geforderten Argumente tauchen, sofern ihr Wert definiert werden muss, im Lexikoneintrag des fordernden Prädikats als Attribute -- stets auf der linken Seite aufgelistet -- auf. Jedes Attribut muss einen Wert erhalten, der es spezifiziert -- die Werte finden sich immer auf der rechten Seite.\footnote{Nähere Erklärungen zu Attribut-Wert-Paaren finden sich bei der Betrachtung der f-Struktur.} Ferner werden weitere Bestimmungen der Wortform durch zusätzliche Attribut-Wert-Paare definiert.\footnote{Die Komplemente, die ein spezifisches Prädikat zu sich nehmen kann, müssen je nach Konstituenten-Typ klassifiziert werden; so macht es einen Unterschied, ob ein Verb ein verbales Kompliment (VCOMP) oder eines, das einen Satz ersetzt (SCOMP), zu sich nehmen kann; vgl. \cite[22]{Rohrer}.}

Bei der Erstellung der f-Struktur werden, im Lateinischen beginnend beim Verb, die Prädikate mitsamt all ihrer Funktionsbestimmungen in die f-Struktur übertragen.\footnote{Vgl. \cite[28]{Rohrer}.} Die Lexikoneinträge liefern somit einerseits große Teile des Inhalts der Strukturen über die Definition von Attribut-Wert-Paaren und schränken andererseits die als grammatisch geltenden Sätze ein.\footnote{Vgl. \cite[63]{Skript}.}  Taucht eine vom Prädikat geforderte grammatikalische Funktion nämlich nicht in der f-Struktur auf, so ist die Struktur unvollständig.\footnote{Vgl. \cite[28]{Rohrer}.} Letzteres wird bei der Besprechung der Besprechung der Wohlgeformtheits-Bedingungen der f-Struktur in Abschnitt 2.2.5 dieser Arbeit deutlicher werden.
%Die Lexikoneinträge spezifizieren jedoch nicht die Verbindung konkreter Worte mit ihren grammatikalischen Funktionen; dies geschieht in der c-Struktur. 

Welche Attribute zur näheren Bestimmung einer syntaktischen Struktur benötigt werden, wird nach den linguistischen Erfordernissen einer Sprache bestimmt.\footnote{Vgl. \cite[8]{Skript}.} Bei den Partizipien umfassen die nötigen Angaben hinsichtlich der konkreten Wortform Kasus, Numerus, Genus, Verbform (,,MOOD'')\footnote{Hierbei leitet sich ,,MOOD'' genaugenommen von ,,Modus'' her. Obwohl unter Modus in der Regel die Unterscheidung Indikativ - Konjunktiv verstanden wird, kann diese Bezeichnung hier auch im Zusammenhang mit Partizipien verwendet werden, da sich die Eigenschaften `Partizip' und `Indikativ' bzw. `Konjunktiv' gegenseitig ausschließen.}, d.h. hier stets Partizip (,,PART''), Zeitverhältnis (,,RELTENSE'', abgekürzt für ,,relative tense'') und Diathese (wobei das Attribut ,,PASSIVE'' entweder den Wert ,,+'' oder ,,-'' erhält).\footnote{Das Genus verbi, d.h. die rein morphologische Erscheinung in entweder aktiver oder passiver Form, ergibt sich aus der Grundform -- hier in Anlehnung an gängige lateinische Wörterbücher stets die erste Person Singular Präsens Indikativ -- des Prädikats des Partizips im Lexikoneintrag, wie z.B. \textit{mittor} statt \textit{mitto}. Durch diese Notierung stellen auch Deponentien kein Problem für die LFG dar, deren Diathese aktiv ist, während ihre morphologische Form im Passiv steht.}

%Liste der üblicherweise angenommenen f-strukturellen Merkmale/Argumente zusammen mit den (möglichen) Werten dieser Merkmale. (Dalrymple, S. 27-8)
%das Merkmal Kasus (``CASE'') beispielsweise ist mit bestimmten grammatikalischen Funktionen verbunden (--> so kann ein Subjekt nur den Wert `` Nominativ'' (``nom'') für das Merkmal CASE tragen). Merkmale wie TENSE spezifizieren die morphologische Form eines Arguments. (gekürzt zitiert) (Dalrymple, S. 27)
% (Grund dafür, dass die Subkategorisierung im funktionalen Bereich anstatt auf der Ebene der Oberflächenstruktur stattfindet: Dalrymple, S. 29.
%(Chomsky’s Syntaxbäume + Unterschiede in der LFG) (Dalrymple, S. 47)

\begin{singlespace}
\begin{tabular}{ l  l  l  l  } 
\textbf{moritura}: & V \\
$\qquad$ & $[1]$ \:  ($\uparrow$PRED) & = & `morior$\langle$SUBJ$\rangle$'\\
%$\qquad$ & $[2]$ \:  ($\uparrow$SUBJ) & = & \{((XADJ$\uparrow$)GF) $\mid$ ((XCOMP$\uparrow$)GF)\} \textbf{?} \\
$\qquad$ & $[2]$ \:  \{(($\uparrow$SUBJ GEN) & = & f \\ 
$\qquad$ & $[2.1]$ \:  ($\uparrow$SUBJ NUM) & = & sg \\
$\qquad$ & $[2.2]$ \:  ($\uparrow$SUBJ CASE) & = & \{nom $\mid$ abl\} $\mid$\\
$\qquad$ & $[2.3]$ \: (($\uparrow$SUBJ GEN) & = & n \\
$\qquad$ & $[2.4]$ \:  ($\uparrow$SUBJ NUM) & = & pl \\
$\qquad$ & $[2.5]$ \:  ($\uparrow$SUBJ CASE) & = & \{nom $\mid$ acc\} ) \}\\
$\qquad$ & $[3]$ \:  ($\uparrow$MOOD) & = & part\\
$\qquad$ & $[4]$ \:  ($\uparrow$FIN) & = & - \\
$\qquad$ & $[5]$ \:  ($\uparrow$PASSIVE) & = & - \\
$\qquad$ & $[6]$ \:  ($\uparrow$RELTENSE) & = & future \\
$\qquad$ & $[7]$ \:  \{(($\uparrow$GEN) & = & f \\ 
$\qquad$ & $[7.1]$ \:  ($\uparrow$NUM) & = & sg \\
$\qquad$ & $[7.2]$ \:  ($\uparrow$CASE) & = & \{nom $\mid$ abl\} $\mid$\\
$\qquad$ & $[7.3]$ \: (($\uparrow$GEN) & = & n \\
$\qquad$ & $[7.4]$ \:  ($\uparrow$NUM) & = & pl \\
$\qquad$ & $[7.5]$ \:  ($\uparrow$CASE) & = & \{nom $\mid$ acc\} ) \}\\
\end{tabular}
\newline
\newline
\end{singlespace}

Die hier aufgeführten Eigenschaften eines Lexikoneintrages sollen durch die Beschreibung des konkreten Lexikoneintrags des Partizips \textit{moritura} deutlich gemacht werden. Die konkreten Lexikoneinträge finden sich bei der Betrachtung der spezifischen Partizipialkonstruktionen. [1] Dem Prädikat \textit{moritura} ist die grammatikalische Funktion des Subjekts durch den Subkategorisierungsrahmen als Argument zugeordnet. Dies veranschaulicht, dass das Verb morior ein Subjekt bei sich hat, jedoch -- aus semantisch leicht versändlichen Gründen -- keine weiteren Argumente wie beispielsweise ein Objekt.
[2 \& 6] Das geforderte Subjekt muss mit dem Partizip in Kasus, Numerus und Genus übereinstimmen. [2] Aus diesem Grund kann dieses Subjekt entweder im Nominativ oder Ablativ Singular Femininum oder im Nominativ oder Akkusativ Plural Neutrum stehen. Die Disjunktion wird dabei im Lexikoneintrag mit geschweiften Mengenklammern eingefasst und die einzelnen Glieder werden durch vertikale Striche voneinander getrennt.

[3-6] Durch das Auflisten weiterer Attribute und deren Werte wird die Verbform genau bestimmt: Dabei gibt der Wert ,,part'' des Attributs MOOD an, dass es sich bei moritura um ein Partizip handelt. Da Partizipien keine finiten Verbformen sind, erhält das Attribut ,,FIN'', für finit, den Wert - (negativ). Das Attribut PASSIVE erhält in diesem Fall ebenfalls den Wert - , da die Diathese jedes Deponens aktiv ist,\footnote{Im speziellen Fall des PFA ist auch das morphologische Genus verbi aktiv.} und das Attribut RELTENSE erhält den Wert ,,future''. Diese drei Attribut-Wert-Paare definieren \textit{moritura} somit als PFA. Die Attribute CASE, NUM und GEN komplettieren den Lexikoneintrag: Diese Werte geben an, dass die Verbform entweder im Nominativ oder Ablativ Singular Femininum oder im Nominativ oder Akkusativ Plural Neutrum stehen darf. Somit wird auch die Kasus-Numerus-Genus-Kongruenz zwischen Subjekt und Verbform erfüllt.

\subsubsection{Redundanz- bzw. Default-Regeln}
Um die syntaktische Korrektheit der ausgegebenen Sätze zu gewährleisten, können zusätzlich zu den Lexikoneinträgen weitere Bedingungen festgelegt werden. Dies geschieht durch sogenannte Redundanz- bzw. Default-Regeln, die über dem Lexikon operieren\footnote{Vgl. \cite[23-4]{Rohrer}.} Hier werden diejenigen aufgelistet, die für die vorliegenden Phänomene relevant sind. Auch wenn diese Regeln nicht unbedingt notwendig sind, so können sie unseres Erachtens nach die Erzeugung der c-Struktur aus den Syntaxregeln beschleunigen, die durch die Klassifzierung der Partizipien als V erschwert
wurde.

Anstatt dieser Redundanz- bzw. Default-Regeln könnte man auch zusätzliche Bedingungen den Syntaxregeln anfügen, wie Bresnan und Kaplan beschreiben\footnote{Bresnan und Kaplan 1982, S. 210; zitiert nach \cite[54]{Rohrer}.} und wie in Abschnitt 2.2.1 dieser Arbeit bereits exemplarisch gezeigt wurde. Aufgrund der leichteren Verständlichkeit für den Leser erfolgt die Darstellung hier durch Redundanz- bzw. Default-Regeln. Auf eine formelhafte Darstellung wird an dieser Stelle verzichtet.

Allgemein gilt, dass ein V, das direkt von S dominiert wird, immer eine
finite Verbform ist; das Attribut FIN muss als Wert +, das Attribut MOOD den Wert Indikativ, Konjunktiv oder Imperativ erhalten.
Der Kasus des Subjekts dieses finiten V ist immer Nominativ; zu beachten ist hier, dass dies beim Bezugswort eines Partizips, d.h. seinem Subjekt, nicht der der Fall ist.
Ferner kann ein V, das von einer VP dominiert wird, kein finites Verb sein; es ist stets entweder ein Partizip, ein Infinitv oder ein Gerundial. Somit muss als Wert von FIN -, als Wert von MOOD part, inf, gerundium  oder gerundivum eingesetzt sein.
Wenn V von S-part dominiert wird, muss es ein Partizip sein.

%\footnote{Auf weitere Funktionsspezifikationen neben denen der Partizipien bzw. Partizipialkonstruktionen kann im Rahmen dieser Arbeit nicht eingegangen werden. Sie werden für das Deutsche beispielsweise in Skript, S. 48-51 behandelt.}

Weitere allgemeine Regeln sollen nun anhand von Lexikoneinträgen eines Partizips x definiert werden.
%Sie beziehen sich auf alle Partizipial-VPs, die nicht direkt von S dominiert werden.
Die grammatikalische Funktion jedes Partizips in einer Partizipialkonstruktion, welches nicht Teil einer finiten Verbform ist, ist immer entweder ein XADJ, ein ADJ oder ein XCOMP der übergeordneten grammatikalischen Funktion:\footnote{Ausgenommen sind hier das dominante Partizip in Abhängigkeit von einer Präposition und je nach Umsetzung das substantivierte Partizip.} \\
($\uparrow$GF) = ($\uparrow$XADJ) $\mid$ ($\uparrow$ADJ) $\mid$ ($\uparrow$XCOMP)

Da das Bezugswort eines PC in jedem Fall eine grammatikalische Funktion der übergeordneten Struktur ist,\footnote{Vgl. KSt S. 771, § 138,5a.} nimmt das PC die grammatikalische Funktion XADJ an. Partizipien in allen Kasus können in einer PC-Konstruktion auftreten, auch wenn Genitiv und Ablativ hierbei seltener vorkommen als die übrigen Kasus. Dies bedeutet, dass das Partizip in einer XADJ-Funktion in jedem Kasus vorkommen kann. Selbiges gilt auch für das substantivierte Partizip -- sofern sein Subjekt als logisch vorhanden angenommen wird.

Der Abl. abs. jedoch ist vom Restsatz semantisch und syntaktisch losgelöst, weswegen sein Bezugswort keine grammatikalische Funktion der übergeordneten Struktur (in aller Regel S) sein darf. Dem Partizip eines Abl. abs. kommt daher unumgänglich die Funktion des ADJ zu. Somit kann geschlussfolgert werden, dass ein Partizip, welches im Ablativ steht, neben der Funktion eines XADJ auch die Funktion eines ADJ annehmen kann. Auch das dominante Partizip nimmt aufgrund seiner syntaktischen Eigenständigkeit die Funktion eines ADJ an. Daher kann ein ADJ zudem in allen anderen Kasus, ausgenommen des Nominativs, vorkommen.

Ein AcP kann, wie weiter unten in Abschnitt 7.1 deutlich wird, am besten als XCOMP zur übergeordneten grammatikalischen Funktion beschrieben werden. Partizip und Bezugswort stehen im AcP im Akkusativ. Daraus folgt, dass ein Partizip im Akkusativ sowohl die Funktion eines XADJ als auch die eines XCOMP annehmen kann.\footnote{Vgl. \cite[48]{Skript}.} Diese lexozentrische Funktionsassoziierung kann durch Konditionale ausgedrückt werden:\footnote{Vgl. \cite[48]{Skript}.} \\
%($\downarrow$XADJ CASE) = \{nom $\mid$ gen $\mid$ acc $\mid$ dat $\mid$ abl\} (weglassen) \\
%($\downarrow$ADJ CASE) = abl \\
%($\downarrow$XCOMP CASE) = acc \\
%($\downarrow$CASE) = \{nom $\mid$ gen $\mid$ dat\} $\Rightarrow$ ($\uparrow$XADJ)= $\downarrow$ \\
%$\mid$ ($\downarrow$CASE) = gen $\Rightarrow$ ($\uparrow$XADJ)= $\downarrow$ \\
%$\mid$ ($\downarrow$CASE) = dat $\Rightarrow$ ($\uparrow$XADJ)= $\downarrow$ \\
%$\mid$ ($\downarrow$CASE) = acc $\Rightarrow$ \{($\uparrow$XADJ) $\mid$ ($\uparrow$XCOMP)\}= $\downarrow$ \\
%$\mid$ ($\downarrow$CASE) = abl $\Rightarrow$ \{($\uparrow$XADJ) $\mid$ ($\uparrow$ADJ)\}= $\downarrow$ \\
($\downarrow$CASE) = nom $\Rightarrow$ ($\uparrow$XADJ)= $\downarrow$ \\
($\downarrow$CASE) = \{gen $\mid$ dat\} $\Rightarrow$ \{($\uparrow$XADJ) $\mid$ ($\uparrow$ADJ)\}= $\downarrow$ \\
$\mid$ ($\downarrow$CASE) = acc $\Rightarrow$ \{($\uparrow$XADJ) $\mid$ ($\uparrow$XCOMP) $\mid$ ($\uparrow$ADJ)\}= $\downarrow$ \\
$\mid$ ($\downarrow$CASE) = abl $\Rightarrow$ \{($\uparrow$XADJ) $\mid$ ($\uparrow$ADJ)\}= $\downarrow$ 

%PC:
Das Partizip muss in Kasus, Numerus und Genus mit seinem Bezugswort, d.h. mit seinem Subjekt, kongruent sein:\footnote{Vgl. KSt S. 771, §138,5a. Das substantivierte Partizip stellt, wenn sein Bezugswort als tatsächlich fehlend verstanden wird -- wie in Abschnitt 4.2 diskutiert --, einen Ausnahmefall dar. Da jedoch keine existenzielle Forderung an das Vorhandensein des Subjekts gestellt wird, werden hierfür keine gesonderten Regeln benötigt.}\\ 
($\uparrow$KNG) = ($\uparrow$SUBJ KNG)\footnote{Diese Darstellung ist verkürzend, um eine weitere disjunktive Menge von Kasus-, Numerus- und Genus- Attributen zu vermeiden; vgl. \cite[49]{Skript}.}

Wenn das Partizip ein XADJ der übergeordneten grammatikalischen Funktion ist, ist sein Bezugswort eine grammatikalische Funktion dieser dem XADJ übergeordneten Struktur: \\
($\uparrow$SUBJ XADJ) = ((XADJ$\uparrow$)GF)

Wenn das Partizip ein XCOMP der übergeordneten grammatikalischen Funktion ist, ist sein Bezugswort das Objekt dieser dem XCOMP übergeordneten Struktur: \\
($\uparrow$SUBJ XCOMP) = ((XCOMP$\uparrow$)OBJ) 

Wenn das Partizip ein ADJ der übergeordneten grammatikalischen Funktion ist, ist sein Bezugswort keine grammatikalische Funktion der dem ADJ übergeordneten Struktur: \\
($\uparrow$SUBJ ADJ) = ((ADJ$\uparrow$)GF)

Da sich diese Arbeit ausschließlich auf das klassische Latein Caesars und Ciceros bezieht, gilt für die folgenden Betrachtungen die Annahme, dass im Abl. abs. kein Partizip Futur Aktiv (PFA) verwendet wird.\footnote{Vgl. KSt. S. 760, §136,4c oder NM S. 771, §469.}\\
%$\neg$ ($\uparrow$(RELTENSE ADJ) = future \\
($\uparrow$RELTENSE) = future $\Rightarrow$ ($\uparrow$GF) $\neq$ ($\uparrow$ADJ)

Das Partizip ist im AcP meist ein PPA, selten ein PPP. Mit Sicherheit kann daher gesagt werden, dass kein PFA in einem Partizip mit XCOMP-Funktion auftauchen kann: \\
%($\uparrow$XCOMP RELTENSE) = present   \\
($\uparrow$RELTENSE) = future $\Rightarrow$ ($\uparrow$GF) $\neq$ ($\uparrow$XCOMP)

\subsubsection{f-Struktur}
Unter Einbeziehung der Informationen aus den Lexikoneinträgen, der annotierten c-Struktur, und gegebenenfalls den Redundanz- und Default-Regeln kann schließlich die f-Struktur aufgebaut werden.\footnote{Vgl. \cite[13; 23]{Rohrer}; \cite[14]{Skript}.} Auch wenn Informationen aus der c-Struktur in die f-Struktur einfließen, ist keine der beiden Strukturen in der anderen vorhanden; beide existieren parallel und stellen unterschiedliche Aspekte des Aufbaus eines Satzes dar.\footnote{Vgl. \cite[26-7; 35]{Rohrer}; \cite[8]{Skript}.}

Die f-Struktur bildet dabei die abstrakte, funktionale Organisation ab;\footnote{Vgl. \cite[7]{Dal}.} diese wird als weitgehend universell angesehen.\footnote{Vgl. \cite[7; 9]{Bresnan}.} Die f-Struktur wird nicht durch einen Baumgraphen, sondern durch eine Attribut-Wert-Matrix dargestellt.\footnote{Vgl. \cite[55]{Falk}; \cite[7]{Skript}.} Darin sind die Attribute stets atomar; die Werte jedoch können sowohl durch einzelne Merkmale (features) oder semantischen Formen als auch durch eine weitere f-Struktur gebildet werden.\footnote{Vgl. \cite[55]{Falk}; \cite[13]{Rohrer}; \cite[8]{Skript}.} Die inneren f-Strukturen wie auch die übergeordnete als Ganzes werden jeweils in eckige Klammern gefasst.

Welche Attribute benötigt werden, wird durch die Lexikoneinträge der lexikalischen Terminalsymbole der c-Struktur bestimmt. \footnote{Vgl. \cite[13; 23]{Rohrer}.} Jede f-Struktur muss mindestens ein Prädikat enthalten. Den Wert dieses Prädikats bildet stets eine semantische Form, die durch einfache Anführungszeichen gekennzeichnet wird.\footnote{Vgl. \cite[8]{Skript}.} Mit dem Prädikat gehen auch seine übrigen Informationen aus dem Lexikoneintrag in die f-Struktur ein, d.h. seine Subkategorisierungserfordernisse sowie die übrigen Attribut-Wert-Paare, die zur näheren Bestimmung seiner Form nötig sind, wie beispielsweise Tempus und Modus.\footnote{Vgl. \cite[23; 28-9]{Rohrer}; \cite[7; 9]{Skript}; \cite[7]{Dal}. Des Weiteren können funktionale Gleichungen in f-Strukturen enthalten sein; vgl. \cite[21]{Rohrer}. In den Lexikoneinträgen dieser Arbeit wurde auf derartige Funktionsgleichungen verzichtet, da das Wichtigste bereits in den Redundanz- und Default-Regeln aufgeführt wurde.} 

Sämtliche im Subkategorisierungsrahmen geforderten grammatikalischen Funktionen, also zum Beispiel Subjekt und Objekt, müssen demnach als Attribute in der f-Struktur des Hauptsatzprädikats vorkommen. Ist dies nicht der Fall, ist die f-Struktur nicht komplett; dies verletzt die Completeness Condition, eine der drei Wohlgeformtheitsbedingungen, die an die f-Struktur gestellt werden.\footnote{Vgl. \cite[58-9]{Falk}; \cite[28]{Rohrer}; \cite[19-20]{Skript}.} Da der entstehende Satz ungrammatisch wäre, würde an diesem Punkt der Aufbau der f-Struktur abgebrochen werden. Eine Erweiterung der Vollständigkeitsbedingung besagt, dass jede grammatikalische Funktion ein Prädikat-Merkmal enthalten muss.\footnote{Vgl. \cite[61]{Falk}.}

Ebenso wäre ein Satz ungrammatisch, wenn in seiner f-Struktur regierbare grammatikalische Funktionen als Attribute auftauchen, die nicht vom Prädikat gefordert werden. Eine f-Struktur mit überzähligen regierbaren Funktionen gilt als inkohärent, da die Coherence Condition verletzt wird.\footnote{Vgl. \cite[59-62]{Falk}; \cite[29; 39]{Rohrer}; \cite[20]{Skript}.} Ausgenommen hiervon sind eben die nicht-regierbaren Adjunkte. Von ihnen können beliebig viele als Attribute einer f-Struktur erscheinen, was in der f-Struktur durch die geschweifte Mengenklammer symbolisiert wird.\footnote{Vgl. \cite[61; 72]{Falk}; \cite[12]{Dal}; \cite[28]{Rohrer}; \cite[38-40]{Skript}. Hierdurch wird vermieden, dass die Uniqueness Condition verletzt wird, da ohne die Mengen Klammern ein Attribut ADJ mehrere Werte haben müsste.} ,,Ein einzelnes Adjunkt bildet dann ein Element dieser Menge'', was in den funktionalen Schemata der Syntaxregeln und c-Struktur durch das Element-Zeichen $\in$ angezeigt wird.\footnote{ \cite[39]{Skript}.}

Die Werte der grammatikalischen Funktionen sind weitere f-Strukturen;\footnote{Vgl. \cite[35]{Rohrer}.} diese enthalten wiederum die Prädikate der ihnen zugeordneten Lexeme samt deren Subkategorisierungserfordernissen, wenn vorhanden, sowie die weiteren für die Bestimmung der Wortform nötigen Attribut-Wert-Paare -- analog zur äußeren f-Struktur. Die Vollständig"-keits- und Kohärenzbedingungen gelten auch für diese inneren f-Strukturen, die von der gesamten f-Struktur umschlossen werden. Die gesamte f-Struktur ist nur dann vollständig und kohärent, wenn alle in ihr enthaltenen f-Strukturen diese Bedingungen lokal erfüllen.\footnote{Vgl. \cite[60]{Falk}; \cite[19-21]{Skript}.}
Die Attribut-Wert-Paare können in beliebiger Reihenfolge auftreten.\footnote{Vgl. \cite[35]{Rohrer}.}

Zwar ist es möglich, dass verschiedene Attribute den gleichen Wert erhalten, jedoch darf ein Attribut nicht mehrere Werte haben; letzteres wird durch die Uniqueness Condition, oder Konsistenz-Bedingung ausgeschlossen.\footnote{Vgl. \cite[62]{Falk}; \cite[29]{Rohrer}; \cite[18-9]{Skript}. Die Uniqueness Condition ist auch bei der Unifikation von Bedeutung: Die Attribut-Wert-Paare zweier f-Strukturen können zu einer f-Struktur verschmolzen werden, sofern sowohl die Attribute als auch die Werte die exakt gleiche Spezifikation haben; sind die Werte nicht gleich, können sie nicht zusammengeführt werden und müssten nebeneinander existieren, was jedoch eben durch die Uniqueness Condition untersagt wird; vgl. \cite[68]{Falk}; \cite[37]{Rohrer}; \cite[18-9]{Skript}.} Wenn beispielsweise das Attribut CASE innerhalb derselben f-Struktur sowohl den Wert nom als auch den Wert gen erhielte, gälte die f-Struktur als inkonsistent und ihr Aufbau würde nicht vollendet werden.\footnote{Vgl. \cite[29; 35]{Rohrer}. Es sollte beachtet werden, dass die Uniqueness-Bedingung durch Disjunktion nicht verletzt wird, da der Disjunktionsstrich gebietet, dass nur eines der aufgeführten Glieder ausgewählt werden darf.} 

Alle drei genannten Wohlgeformtheitsbedingungen gelten universell für alle Sprachen.\footnote{Vgl. \cite[21]{Skript}.} Sie alle dienen dazu, sicherzustellen, dass alle Teile des Satzes zusammenpassen und das ungrammatische Sätze ausgeschlossen werden.\footnote{Vgl. \cite[58; 62]{Falk}; \cite[29]{Rohrer}.} Dies wird nämlich allein durch die c-Struktur nicht gewährleistet; der gesamte Bereich der Kongruenz beispielsweise wird erst beim Aufbau der f-Struktur bzw. bei der Prüfung der Wohlgeformtheitsbedingungen erfasst.\footnote{Vgl. \cite[24]{Rohrer}; \cite[18]{Skript}.}

Die offenen Funktionen XCOMP und XADJ, deren Subjekt von einem Argument außerhalb ihrer eigenen f-Struktur kontrolliert wird, benötigen eine besondere Darstellung in der f-Struktur:\footnote{Vgl. \cite[10; 14]{Dal}.} Ein Pfeil, beginnend beim Subjekt der offenen Funktion, verweist auf die grammatikalische Funktion einer übergeordneten f-Struktur, welche die Kontrolle ausübt.\footnote{Vgl. \cite[54-5]{Skript}; \cite[40]{Rohrer}.} Da die kontrollierende Funktion vom Prädikat eines Werks regiert wird, muss im Lexikoneintrag dieses Verbs spezifiziert sein, dass es eine solche Kontrolle über das Subjekt eines seiner Argumente ausüben kann. Ein Beispiel hierfür findet sich in dieser Arbeit bei der Betrachtung des Lexikoneintrags der AcP-Konstruktion).\footnote{Vgl. \cite[54-5]{Skript}; \cite[30; 40]{Rohrer}. Auf eine andere Art der Kontrolle, die anaphorische Kontrolle, wird in dieser Arbeit nicht eingegangen.}

Die f-Struktur des einfachen Beispielsatzes von vorhin, \textit{Caesar barbaros in Gallia vicit}, wird hier zum besseren Verständnis kurz aufgeführt: \\

\begin{singlespace}
\begin{avm}

\[ PRED &  \rm ‘vinco \q<SUBJ, OBJ\q>'\\
SUBJ & \[PRED & `pro' \\
PRON-TYPE & mis \] \\
OBJ & \[ PRED & `barbarus' \\
CASE & acc \\
NUM & pl \\
GEN & m \] \\
ADJ & \{ \[PRED &  \rm ‘in \q<OBJ\q>'\\
OBJ & \[PRED & ‘schola' \\
CASE & acc \\
NUM & sg \\
GEN & f \] \\
\] \} \\
TENSE & past \\
NUM & sg \\
PERS & 3 \\
PASSIVE & - \\
MOOD & ind \\
\]
\end{avm}
\newline
\end{singlespace}

Die vorliegende f-Struktur zeigt, dass der Satz das Prädikat \textit{vinco} enthält, welches das Vorkommen zweier Grammatischer Funktionen, nämlich eines Subjekts und eines Objekts verlangt. Da \textit{vicit} das finite Verb des Satzes ist, welches im Lateinischen stets Kopf des Satzes ist, bildet sein Prädikat die äußerste f-Struktur. Person, Numerus und Tempus von \textit{vicit} werden durch die Attribute  PERS, NUM und TENSE festgelegt, deren Werte die 3. Person Singular Perfekt definieren. Die Verneinung des Attributs PASSIVE und die Angabe des Attributs MOOD mit dem Wert Indikativ vervollständigen die Bestimmung: 3. Person Singular Indikativ Perfekt aktiv. Es ist anzumerken, dass die f-Struktur keinerlei kategoriale Bestimmungen beinhaltet; auch wenn das Prädikat in diesem Satz ein lexikalisches Element der Kategorie Verb gebildet, ist dies für den Gehalt der f-Struktur irrelevant.\footnote{Vgl. \cite[7]{Skript}.}

Die Werte der geforderten Argumente Subjekt und Objekt sind wiederum f-Strukturen, die die Prädikate der ihnen zugeordneten Lexeme und deren eventuelle Subkatekorisierungserfordernisse enthalten. So wird das PRED des Subjekts durch den Wert pro, der PRON-TYPE durch den Wert mis definiert; dies drückt aus, dass das Subjekt der übergeordneten f-Struktur ein Pronomen ist, das jedoch im Satz nicht als konkretes lexikalisches Element auftritt. Diese Besonderheit ist dem Lateinischen geschuldet, in dem Pronomina bereits in den finiten Verbformen ausgedrückt werden und nicht unbedingt, wie beispielsweise im Deutschen, gesondert genannt werden müssen. Das Objekt erhält als PRED den Wert \textit{barbarus}. Zusätzlich wird es durch weitere Attribut-Wert-Paare als Akkusativ Plural maskulin bestimmt. 

Neben den geforderten Argumenten können in der f-Struktur des Prädikats die nicht-regierbaren Adjunkte definiert werden. Da von diesen beliebig viele hinzutreten können, wird der Wert des hier vorliegenden Adjunkts, der erneut eine f-Struktur ist, in einer Mengenklammer angegeben. In dieser gibt \textit{in} den Wert des Attributs PRED an. Zusätzlich fordert \textit{in} ein Objekt, dessen PRED \textit{schola} in einer untergeordneten f-Struktur durch seine Attribut-Wert-Paare als Akkusativ Singular feminin definiert wird. Im Folgenden werden einzelne untergeordnete f-Strukturen, deren Attribut-Wert-Paare entweder klar oder für den zu erklärenden Punkt nicht von Bedeutung sind, aus Gründen der Übersichtlichkeit durch doppelte Anführungszeichen (``...'') abgekürzt. Dies würde bei dem eben beschriebenen Adjunkt folgendermaßen aussehen: ADJ [“in scholam”].\footnote{Vgl. Falk, S. 59.}
Da sowohl Vollständigkeit, Kohärenz und Unifikation gewährleistet sind, gilt der in der f-Struktur dargestellte Satz als wohlgeformt und grammatisch.

\subsubsection{Mapping zwischen c- und f-Struktur}
Zwischen bestimmten Teilen der c-Struktur und bestimmten Teilen der f-Struktur besteht ein Korrespondenzverhältnis.\footnote{Vgl. \cite[8]{Skript}.} Diese Abbildung wird als Mapping bezeichnet. Falk beschreibt Mapping als das Herz der deskriptiven Kraft der LFG, da es die Beziehung zwischen oberflächlichen syntaktischen Elementen und den Merkmalen, die sie repräsentieren, schafft.\footnote{Vgl. auch \cite[62; 68]{Falk}.} In der Tat vervollständigt das Mapping das mathematische Modell der LFG und stellt dessen formale Korrektheit her. Für eine computerbasierte Umsetzung des Grammatikmodells ist das Mapping unabdingbar. Im Rahmen dieser Arbeit kann hier jedoch nur eine sehr oberflächliche Erklärung dargeboten werden.\footnote{Das Mapping wird in Anbetracht des Umfangs dieser Arbeit nicht konkret am Beispiel der Partizipialkonstruktionen vollzogen.}

%\textbf{drinlassen? C- und f-Strukturen drücken unterschiedliche strukturelle Eigenschaften eines Satzes aus und sind völlig unterschiedlich aufgebaut;\footnote{Vgl. \cite[8]{Skript}.} 
Aus der f-Struktur ist der Aufbau der c-Struktur nicht direkt erkennbar; die Korrespondenz-Beziehung zwischen c-Struktur-Knoten\footnote{An dieser Stelle sind mit dem Begriff c-Struktur-Knoten lediglich nicht-terminale Knoten nach obiger Definition gemeint. Die Lexeme fließen als semantische Form der Prädikate in die f-Struktur ein.} und Teilen einer f-Struktur ist daher komplexer.\footnote{Vgl. auch \cite[55]{Falk}.} Sie ,,wird durch eine Funktion $\phi$ etabliert, die jeden C-Struktur-Knoten auf eine F-Struktur abbildet.''\footnote{\cite[8]{Skript}.} Dies geschieht über die funktionalen Deskription, auch f-Beschreibung genannt.\footnote{Vgl. \cite[63-4]{Falk}; \cite[34]{Rohrer}; \cite[17]{Skript}.} Hierzu werden alle korrespondierenden Paare sowohl in der c- als auch in der f-Struktur durch die Variablen f\textsubscript{1}, f\textsubscript{2}, ... f\textsubscript{n} gekennzeichnet.\footnote{Vgl. auch \cite[65]{Falk}.} 

Nun finden beispielsweise eine Präpositionalphrase und der ihr untergeordnete Knoten P ihre Entsprechung in einer einzigen f-Struktur; daher können die Variablen, die die Knoten PP und P denotieren, gleichgesetzt werden. Ferner kann die NP, die dieser PP untergeordnet ist, mit dem Objekt dieser PP identifiziert werden. So ergeben sich funktionale Gleichungen für sämtliche c-Struktur-Knoten; (f\textsubscript{x} OBJ) = f\textsubscript{y} beispielsweise ist zu lesen als: ,,Es gibt eine F-Struktur f\textsubscript{x} und in dieser F-Struktur gibt es ein Attribut-Wert-Paar [OBJ f\textsubscript{y}].\footnote{\cite[12]{Skript}.}
Alle funktionalen Gleichungen einer (gesamten) f-Struktur wird f-Beschreibung genannt.\footnote{Vgl. \cite[66-8]{Falk}.} Die minimale Lösung der Gleichungen in der f-Beschreibung ist schließlich die f-Struktur.\footnote{Vgl. \cite[68]{Falk}; \cite[17]{Skript}.}

Die funktionalen Gleichungen werden außerdem an den betreffenden c-Struktur-Knoten notiert, wo sie die lokalen Beziehungen zwischen Mutter- und Tochter-Knoten ausdrücken.\footnote{Vgl. \cite[69]{Falk}. Vgl auch \cite[12]{Skript}: ,,Der Ort der Annotation ist zweckmäßigerweise der Knoten, dessen F-Struktur den Wert im Attribut-Wert-Paar liefert.''} Zur besseren Übersichtlichkeit werden die Variablen durch Metavariablen ersetzt; diese Metavariablen sind genau die Pfeile, die wir in den funktionalen Schemata kennengelernt haben -- ein Pfeil nach oben verweist auf den Mutter-Knoten, nach unten auf den Tochter-Knoten.\footnote{Vgl. \cite[69]{Falk}; \cite[15-6]{Skript}.}


Das Mapping kann zudem die Erzeugung der f-Struktur aus der annotierten c-Struktur unterstützen. Hierbei wird von den c-Struktur-Annotationen ausgegangen, deren Metavariablen -- die Pfeile -- als Variablen der Form f\textsubscript{1}, f\textsubscript{2}, ... f\textsubscript{n} instantiiert werden. So wird gewährleistet, dass die Angaben aus der c-Struktur an die passenden Stellen in der f-Struktur eingebaut werden.\footnote{Vgl. \cite[34]{Rohrer} und \cite[8; 10-11; 14; 17; 19; 28; 54]{Skript}.} 

\section{Participium coniunctum}
Partizipien können als Vertreter von Adverbialsätzen aufgefasst werden und stehen dabei für Temporal-, Kausal-, Modal-, Kondizional- und Konzessivsätze. Das Partizip ist hierbei mit seinem Bezugswort verbunden, mit dem es daher in Kasus, Numerus und Genus übereinstimmt. Wenn das Bezugswort des Partizips Bestandteil des Hauptsatzes und gleichzeitig Subjekt des Nebensatzes ist, bezeichnet man diese Konstruktion als Participium coniunctum. Partizip und Bezugswort können dabei in einem der fünf Kasus  auftreten. Das Partizip kann sowohl attributiv als auch prädikativ verwendet werden.\footnote{Vgl. KSt, S. 766, § 138,1 u. S. 771, § 138,5a; Vgl. NM, S. 715, § 500.} \\

\subsection{Vorüberlegungen zur Umsetzung in der LFG}
Das Partizip erhält in der Konstruktion eines PC immer die syntaktische Funktion eines XADJ, da sein Bezugswort eine grammatikalische Funktion der übergeordneten Struktur ist. Aufgrund der prädikativen Verwendungsmöglichkeit des Partizips hängt die Partizipialkonstruktion direkt von S ab. Da das Partizip jedoch nicht vom Prädikat gefordert wird, kann es nicht als XCOMP klassifiziert werden.
Im Übrigen gelten die in den Redundanz- und Default-Regeln genannten Bedingungen der Kongruenz zwischen Partizip und Bezugswort.

\subsection{Objektabhängiges Participium coniunctum}
Beim objektabhänigen PC bezieht sich das Partizip auf das Objekt des Hauptsatzes. In diesem Fall stehen Bezugswort und Partizip im Akkusativ. \\
Lexikoneintrag, c- und f-Struktur sollen an folgendem Beispielsatz erarbeitet werden: \\
\textit{legatum in Galliam missum Caesar revocat.}
\subsubsection{Lexikoneintrag}
\begin{singlespace}
\begin{tabular}{ l  l  l  l  } 
\textbf{missum}: & \: V \\
$\qquad$ & \:  ($\uparrow$PRED) & = & `mittor$\langle$SUBJ, OBL\textsubscript{GOAL}$\rangle$'\\
%$\qquad$ & $[2]$ \:  ($\uparrow$SUBJ) & = & \{((XADJ$\uparrow$)OBJ) $\mid$ ((XCOMP$\uparrow$)OBJ)\} \\
$\qquad$ & \:  ($\uparrow$SUBJ NUM) & =  & sg \\
$\qquad$ & \:  \{(($\uparrow$ SUBJ GEN) & = & m \\ 
$\qquad$ & \: \: \: ($\uparrow$SUBJ CASE) & = & acc ) $\mid$\\
$\qquad$ & \: \: (($\uparrow$SUBJ GEN) & = & n \\
$\qquad$ & \: \: \: ($\uparrow$SUBJ CASE) & = & \{nom $\mid$ acc\} ) \}\\
$\qquad$ & \:  ($\uparrow$MOOD) & = & part\\
$\qquad$ & \:  ($\uparrow$FIN) & = & - \\
$\qquad$ & \:  ($\uparrow$PASSIVE) & = & + \\
$\qquad$ & \:  ($\uparrow$RELTENSE) & = & past \\
$\qquad$ & \:  ($\uparrow$NUM) & = & sg \\
$\qquad$ & \:  \{(($\uparrow$GEN) & = & m \\ 
$\qquad$ & \: \: \: ($\uparrow$CASE) & = & acc ) $\mid$\\
$\qquad$ & \: \: (($\uparrow$GEN) & = & n \\
$\qquad$ & \: \: \: ($\uparrow$CASE) & = & \{nom $\mid$ acc\} ) \}\\
\end{tabular}
\newline
\newline
\end{singlespace}

%((XADJ$\uparrow$)OBJ) = ($\uparrow$SUBJ) \\

\subsubsection{Syntaxregeln}
Um zu demonstrieren, wie die Syntaxregeln auf einen konkreten Satz angewandt werden, sollen sie hier für den obigen Beispielsatz exemplarisch aufgeschlüsselt werden: \\

\begin{singlespace}
\begin{tabular}{ l  l  c  c  c  c }
S & $\rightarrow$ & NP & VP & NP & V\\
   & $\qquad$ & \textsuperscript{($\uparrow$OBJ) = $\downarrow$} & \textsuperscript{$\downarrow$ $\in$ ($\uparrow$XADJ)} & \textsuperscript{($\uparrow$SUBJ) = $\downarrow$} & \textsuperscript{$\uparrow$ = $\downarrow$} \\
    NP & $\rightarrow$ & N \\
   & $\qquad$ & \textsuperscript{$\uparrow$ = $\downarrow$} \\
    VP & $\rightarrow$ & PP & V & \\
   & $\qquad$ & \textsuperscript{($\uparrow$OBL\textsubscript{GOAL}) = $\downarrow$ } & \textsuperscript{$\uparrow$ = $\downarrow$} \\
   		 PP & $\rightarrow$ & P' \\
	& $\qquad$   & \textsuperscript{$\uparrow$ = $\downarrow$} \\
    		P' & $\rightarrow$ & P & NP \\
   & $\qquad$ & \textsuperscript{$\uparrow$ = $\downarrow$} & \textsuperscript{($\uparrow$OBJ) = $\downarrow$} \\
\end{tabular}\\
\end{singlespace}


\subsubsection{c-Struktur}
\begin{singlespace}
\Tree [.S 
		[\qroof{legatum}.{NP\textsubscript{($\uparrow$OBJ)=$\downarrow$}} ] 
		[.VP{\textsubscript{$\downarrow$ $\in$ ($\uparrow$XADJ)}}
				[.PP\textsubscript{($\uparrow$OBL\textsubscript{GOAL})=$\downarrow$} 
					[.P'\textsubscript{$\uparrow$=$\downarrow$} 
						[.P\textsubscript{$\uparrow$=$\downarrow$} in ]
						[\qroof{Galliam}.{NP\textsubscript{($\uparrow$OBJ)=$\downarrow$}} ]
					]
				]		 
				[.V\textsubscript{$\uparrow$=$\downarrow$} missum ]						
		] 
		[\qroof{Caesar}.NP\textsubscript{($\uparrow$SUBJ)=$\downarrow$} ]
		[.V\textsubscript{$\uparrow$=$\downarrow$} revocat ]	
	]
\end{singlespace}

\subsubsection{f-Struktur}
In dieser f-Struktur werden auch Attribut-Wert-Paare zur Spezifikation der finiten Verbform aufgelistet; im weiteren Verlauf wird hierauf verzichtet werden. Dennoch werden auch hier aus Gründen der Relevanz und des Platzes nicht sämtliche Attribut-Wert-Paare aufgeführt.\footnote{Beispielsweise können für Nomina die Attribute Menschlichkeit, Eigenname oder Zählbarkeit angeführt werden.}
\begin{singlespace}
\begin{avm}

\[ PRED &  \rm ‘revoco \q<SUBJ, OBJ\q>’\\
SUBJ & \[PRED & `Caesar' \\
CASE & nom \\
NUM & sg \\
GEN & m \]\\
OBJ & \[ PRED & `legatus' \\
CASE & acc \\
NUM & sg \\
GEN & m \]\tikzmark{aim} \\
XADJ & \{ \[PRED &  \rm ‘mittor \q<SUBJ, OBL\textsubscript{GOAL}\q>’\\
MOOD & part \\
FIN & - \\
PASSIVE & + \\
RELTENSE & past \\
CASE & acc \\
NUM & sg \\
GEN & m \\
SUBJ &  \tikzmark{start} \\
OBL\textsubscript{GOAL} & \[``in scholam''\] \]\\
\} &            $\qquad$ \\
TENSE & present \\
NUM & sg \\
PERS & 3 \\
PASSIVE & - \\
MOOD & ind \\
FIN & + \\
\]
\end{avm}
\end{singlespace}

\tikz[remember picture,overlay] 
    \draw[<-] (pic cs:aim) to[out=0,in=0,looseness=3.5]  (pic cs:start);

\subsection{Subjektabhängiges Participium coniunctum}
Wie der Name impliziert, bestimmt das Partizip beim subjektabhängigen PC das Subjekt des Hauptsatzes näher, weshalb beide den Nominativ als Kasus einnehmen. Die Umsetzung dieser Variante des PC in den Formalismus der LFG unterscheidet sich nur in der grammatikalischen Funktion des Bezugsworts. Sie wird an einem Beispielsatz veranschaulicht:
\textit{milites in Galliam missi hostes vicerunt.}

\subsubsection{Lexikoneintrag}
\begin{singlespace}
\begin{tabular}{ l  l  l  l  } 
\textbf{missi}: & \: V \\
$\qquad$ & \:  ($\uparrow$PRED) & = & `mittor$\langle$SUBJ, OBL\textsubscript{GOAL}$\rangle$'\\
%$\qquad$ & $[2]$ \:  ($\uparrow$SUBJ) & = & ((XADJ$\uparrow$)GF) \\
$\qquad$ & \:  \{(($\uparrow$SUBJ NUM) & = & pl \\ 
$\qquad$ & \: \: \: ($\uparrow$SUBJ CASE) & = & nom \\
$\qquad$ & \: \: \: ($\uparrow$SUBJ GEN) & = & m) $\mid$\\
$\qquad$ & \: \: (($\uparrow$SUBJ NUM) & = & sg \\ 
$\qquad$ & \: \: \: ($\uparrow$SUBJ CASE) & = & gen \\
$\qquad$ & \: \: \: ($\uparrow$SUBJ GEN) & = & \{m $\mid$ n\} ) \} \\
$\qquad$ & \:  ($\uparrow$MOOD) & = & part \\
$\qquad$ & \:  ($\uparrow$FIN) & = & - \\
$\qquad$ & \:  ($\uparrow$PASSIVE) & = & + \\
$\qquad$ & \: ($\uparrow$RELTENSE) & = & past \\
$\qquad$ & \:  \{(($\uparrow$NUM) & = & pl \\ 
$\qquad$ & \: \: \: ($\uparrow$CASE) & = & nom \\
$\qquad$ & \: \: \: ($\uparrow$GEN) & = & m) $\mid$\\
$\qquad$ & \: \: (($\uparrow$NUM) & = & sg \\ 
$\qquad$ & \: \: \: ($\uparrow$CASE) & = & gen \\
$\qquad$ & \: \: \: ($\uparrow$GEN) & = & \{m $\mid$ n\} ) \} \\
\end{tabular}
\end{singlespace}

%\subsubsection{Syntaxregeln}
%\begin{singlespace}
%\begin{tabular}{ l  l  c  c  c  c }
%S & $\rightarrow$ & NP\textsubscript{1} & VP & NP\textsubscript{2} & V\\
 %  & $\qquad$ & \textsuperscript{($\uparrow$SUBJ) = $\downarrow$} & \textsuperscript{$\downarrow$ $\in$ ($\uparrow$XADJ)} & \textsuperscript{($\uparrow$OBJ) = $\downarrow$} & \textsuperscript{$\uparrow$ = $\downarrow$} \\
  %  NP\textsubscript{1} & $\rightarrow$ & N \\
   %& $\qquad$ & \textsuperscript{$\uparrow$ = $\downarrow$} \\
    %VP & $\rightarrow$ & V' \\
%   & $\qquad$ & \textsuperscript{$\uparrow$ = $\downarrow$} \\
 % 	  V' & $\rightarrow$ & PP & V & \\
  % & $\qquad$ & \textsuperscript{($\uparrow$OBL\textsubscript{GOAL}) = $\downarrow$ } & \textsuperscript{$\uparrow$ = $\downarrow$} \\
   %		 PP & $\rightarrow$ & P' \\
	%& $\qquad$   & \textsuperscript{$\uparrow$ = $\downarrow$} \\
    %		P' & $\rightarrow$ & P & NP\textsubscript{3} \\
%   & $\qquad$ & \textsuperscript{$\uparrow$ = $\downarrow$} & \textsuperscript{($\uparrow$OBJ) = $\downarrow$} \\
 %		   NP\textsubscript{3} & $\rightarrow$ & N \\
  % & $\qquad$ & \textsuperscript{$\uparrow$ = $\downarrow$} \\
   % NP\textsubscript{2} & $\rightarrow$ & N \\
%   & $\qquad$ & \textsuperscript{$\uparrow$ = $\downarrow$} \\
%\end{tabular}\\
%\end{singlespace}

\subsubsection{c-Struktur}
\begin{singlespace}
\Tree [.S 
		[\qroof{milites}.{NP\textsubscript{($\uparrow$SUBJ)=$\downarrow$}} ] 
		[.VP{\textsubscript{$\downarrow$ $\in$ ($\uparrow$XADJ)}}
			[.PP\textsubscript{($\uparrow$OBL\textsubscript{GOAL})=$\downarrow$} 
				[.P'\textsubscript{$\uparrow$=$\downarrow$} 
					[.P\textsubscript{$\uparrow$=$\downarrow$} in ]
					[\qroof{Galliam}.{NP\textsubscript{($\uparrow$OBJ)=$\downarrow$}} ]
				]
			]	
			[.V\textsubscript{$\uparrow$=$\downarrow$} missi ]						
		] 
		[\qroof{hostes}.NP\textsubscript{($\uparrow$OBJ)=$\downarrow$} ]
		[.V\textsubscript{$\uparrow$=$\downarrow$} vicerunt ]	
	]
\end{singlespace}

\subsubsection{f-Struktur}
\begin{singlespace}
\begin{avm}
\[ PRED &  \rm ‘vinco \q<SUBJ, OBJ\q>’\\
SUBJ & \[ PRED & `miles' \\
CASE & nom \\
NUM & pl \\
GEN & m \]\tikzmark{meow} \\
XADJ & \{ \[PRED &  \rm ‘mittor \q<SUBJ, OBL\textsubscript{GOAL}\q>’\\
MOOD & part \\
PASSIVE & + \\
RELTENSE & past \\
CASE & nom \\
NUM & pl \\
GEN & m \\
SUBJ &  \tikzmark{objectmeow} \\
OBL\textsubscript{GOAL} & \[``in Galliam''\] \]\\
\} &            $\qquad$ \\
OBJ & \[``hostes'' \]\\
\]
\end{avm}
\tikz[remember picture,overlay] 
    \draw[<-] (pic cs:meow) to[out=0,in=0,looseness=3.5]  (pic cs:objectmeow);
\end{singlespace}

\subsection{Rein attributives Participium coniunctum}
Das rein attributive Partizip hat zum \textit{verbum finitum} keinerlei semantische Beziehung, sondern charakterisiert nur sein Bezugswort; es ersetzt somit einen attributiven Relativsatz.\footnote{Vgl. NM, S. 713, §498.} Daher ist es in der Darstellung der c-Struktur von der NP seines Bezugswortes abhängig. Anhand des folgenden Beispielsatzes soll die Einordnung in den Rahmen der LFG vorgenommen werden:
\textit{insulam obiectam portui tenuit.}

\subsubsection{Lexikoneintrag}
\begin{singlespace}
\begin{tabular}{ l  l  l  l  } 
\textbf{obiectam}: &  \: V \\
$\qquad$ & \:  ($\uparrow$PRED) & = & `obicior$\langle$SUBJ, OBJ\textsubscript{LOC}$\rangle$'\\
%$\qquad$ & $[2]$ \:  ($\uparrow$SUBJ) & = & \{((XADJ$\uparrow$)GF) $\mid$ ((XCOMP$\uparrow$)GF)\} \\
$\qquad$ & \:  ($\uparrow$SUBJ NUM) & = & sg \\
$\qquad$ & \: ($\uparrow$SUBJ CASE) & = & acc \\
$\qquad$ & \: ($\uparrow$SUBJ GEN) & = & f \\
$\qquad$ & \: ($\uparrow$OBJ CASE) & = & dat \\
$\qquad$ & \:  ($\uparrow$MOOD) & = & part\\
$\qquad$ & \:  ($\uparrow$FIN) & = & - \\
$\qquad$ & \:  ($\uparrow$PASSIVE) & = & + \\
$\qquad$ & \:  ($\uparrow$RELTENSE) & = & past \\
$\qquad$ & \:  ($\uparrow$NUM) & = & sg \\
$\qquad$ & \: ($\uparrow$CASE) & = & acc \\
$\qquad$ & \: ($\uparrow$GEN) & = & f \\
\end{tabular}
\newline
\end{singlespace}

%\subsubsection{Syntaxregeln}
%\begin{singlespace}
%\begin{tabular}{ l  l  c  c  c  c }
%  S & $\rightarrow$ & NP\textsubscript{1} & V\\
 %  & $\qquad$ & \textsuperscript{($\uparrow$OBJ) = $\downarrow$} & \textsuperscript{$\uparrow$ = $\downarrow$} \\
  %  NP\textsubscript{1} & $\rightarrow$ & N' \\
   %& $\qquad$ & \textsuperscript{$\uparrow$ = $\downarrow$} \\
    %   N' & $\rightarrow$ & N & VP \\
%   & $\qquad$ & \textsuperscript{$\uparrow$ = $\downarrow$} & \textsuperscript{$\downarrow$ $\in$ ($\uparrow$XADJ)} \\
%		    VP & $\rightarrow$ & V' \\
 %  & $\qquad$ & \textsuperscript{$\uparrow$ = $\downarrow$} \\
  %				  V' & $\rightarrow$ & V & NP\textsubscript{2} \\
   %& $\qquad$ & \textsuperscript{$\uparrow$ = $\downarrow$} & \textsuperscript{($\uparrow$OBL\textsubscript{LOC}) = $\downarrow$ }  \\
   	%				 NP\textsubscript{2} & $\rightarrow$ & N \\
%   & $\qquad$ & \textsuperscript{$\uparrow$ = $\downarrow$} \\
%\end{tabular} 
%\end{singlespace}

\subsubsection{c-Struktur}
\begin{singlespace}
\Tree [.S 
		[.{NP\textsubscript{($\uparrow$OBJ)=$\downarrow$}} 
				[.N\textsubscript{$\uparrow$=$\downarrow$} insulam ]		
				[.VP{\textsubscript{$\downarrow$ $\in$ ($\uparrow$XADJ)}}
						[.V\textsubscript{$\uparrow$=$\downarrow$} obiectam ] 
						[\qroof{portui}.NP\textsubscript{($\uparrow$OBJ\textsubscript{LOC})=$\downarrow$} ]
				]
				]
		[.V\textsubscript{$\uparrow$=$\downarrow$} tenuit ]	
	]
\end{singlespace}

\subsubsection{f-Struktur}
\begin{singlespace}
\begin{avm}
\[ PRED &  \rm ‘teneo \q<SUBJ, OBJ\q>’\\
SUBJ & \[ PRED & 'pro' \\
PRON-TYPE & mis \] \\
OBJ & \[PRED & `insula' \\
CASE & acc \\
NUM & sg \\
GEN & f \]\tikzmark{a} \\
XADJ & \{ \[PRED &  \rm ‘obicior \q<SUBJ, OBJ\textsubscript{DAT}\q>’\\
MOOD & part \\
PASSIVE & + \\
RELTENSE & past \\
CASE & acc \\
NUM & sg \\
GEN & f \\
SUBJ &  \tikzmark{z} \\
OBJ\textsubscript{DAT} & \[PRED & `portus' \\
CASE & dat \\
NUM & sg \\
GEN & m \\
\] \]\\
\} &            $\qquad$ \\
\]
\end{avm}
\end{singlespace}

\tikz[remember picture,overlay] 
    \draw[<-] (pic cs:a) to[out=0,in=0,looseness=3.4]  (pic cs:z);


\newpage
\section{Substantiviertes Partizip}
Da Partizipien einige Eigenschaften der Adjektive übernehmen, können sie wie diese substantiviert werden.

\subsection{Vorüberlegungen zur Umsetzung in der LFG}
Das Vorhanden- bzw. Nichtvorhandensein eines Bezugswortes stellt in der LFG einen erheblichen Unterschied dar. Deswegen wird die Umsetzung des substantivierten Partizips in der LFG anhand zweier Varianten -- erstere ohne Annahme eines fehlenden Bezugswortes, letztere unter Berücksichtigung eines fehlenden Bezugswortes -- erarbeitet.
Die Umsetzung des substantivierten Partizips in die LFG-Struktur soll anhand des Beispielsatzes \textit{auxilium petentibus Caesar parcit} veranschaulicht werden.

\subsection{Variante 1: Partizip ohne Bezugswort}
Da das substantivierte Partizip die Rolle eines Substantivs übernimmt, kann es analog zu anderen NPs grammatikalische Funktionen wie beispielsweise SUBJ oder OBJ annehmen. Daher ist es keinem Bezugswort untergeordnet und somit direkt von S abhängig. Dabei ist das Partizip Kopf der Partizipialphrase VP und somit alleiniges Objekt des Hauptsatz-Prädikats \textit{parcit}. In diesem Fall überwiegen zwar die nominalen Eigenschaften des Partizips, die Bezeichnung VP wird jedoch um der Konsistenz willen beibehalten. Ein Vorteil dieser Variante besteht in der sichtbar unkomplizierteren c-Struktur.

%\subsubsection{Einschränkungen}
\subsubsection{Lexikoneintrag}
\begin{singlespace}
\begin{tabular}{ l  l  l  l  } 
\textbf{petentibus}: & \: V \\
$\qquad$ & \:  ($\uparrow$PRED) & = & `peto$\langle$SUBJ, OBJ$\rangle$' \\
$\qquad$ & \:  ($\uparrow$SUBJ PRED) & = & `pro' \\
$\qquad$ & \:  ($\uparrow$SUBJ PRON-TYPE) & = & missing \\
$\qquad$ & \:  ($\uparrow$SUBJ CASE) & = & \{abl $\mid$ dat\} \\
$\qquad$ & \:  ($\uparrow$SUBJ NUM) & = & pl \\
$\qquad$ & \:  ($\uparrow$SUBJ GEN) & = & \{m $\mid$ n $\mid$ f\} \\
$\qquad$ & \:  ($\uparrow$OBJ CASE) & = & acc \\
$\qquad$ & \:  ($\uparrow$MOOD) & = & part\\
$\qquad$ & \:  ($\uparrow$FIN) & = & - \\
$\qquad$ & \:  ($\uparrow$PASSIVE) & = & - \\
$\qquad$ & \:  ($\uparrow$RELTENSE) & = & present \\ 
$\qquad$ & \:  ($\uparrow$CASE) & = & \{abl $\mid$ dat\} \\
$\qquad$ & \:  ($\uparrow$NUM) & = & pl \\
$\qquad$ & \:  ($\uparrow$GEN) & = & \{m $\mid$ n $\mid$ f\} \\
\end{tabular}
\newline
\end{singlespace}

%\subsubsection{Syntaxregeln}
%Syntaxregeln sehen wie folgt aus: \\
%\begin{singlespace}
%\begin{tabular}{ l  l  c  c  c  c }
 % S & $\rightarrow$ & VP & NP\textsubscript{1} & V\\
  % & $\qquad$ & \textsuperscript{($\uparrow$OBJ\textsubscript{REC}) = $\downarrow$} & \textsuperscript{($\uparrow$SUBJ) = $\downarrow$} & \textsuperscript{$\uparrow$ = $\downarrow$} \\
%		    VP & $\rightarrow$ & V' \\
 %  & $\qquad$ & \textsuperscript{$\uparrow$ = $\downarrow$} \\
  %				  V' & $\rightarrow$ & NP\textsubscript{2} & V \\
   %& $\qquad$ & \textsuperscript{($\uparrow$OBJ) = $\downarrow$} & \textsuperscript{$\uparrow$ = $\downarrow$} \\
   	%				 NP\textsubscript{2} & $\rightarrow$ & N \\
%   & $\qquad$ & \textsuperscript{$\uparrow$ = $\downarrow$} \\
 %   NP\textsubscript{1} & $\rightarrow$ & N' \\
  % & $\qquad$ & \textsuperscript{$\uparrow$ = $\downarrow$} \\
%\end{tabular} 
%\end{singlespace}

\subsubsection{c-Struktur}
\begin{singlespace}
\Tree [.S 
		[.VP{\textsubscript{($\uparrow$OBJ) = $\downarrow$}}
					[\qroof{auxilium}.NP\textsubscript{($\uparrow$OBJ)=$\downarrow$} ]
					[.V\textsubscript{$\uparrow$=$\downarrow$} petentibus ] 
		]
		[\qroof{Caesar}.NP\textsubscript{($\uparrow$SUBJ)=$\downarrow$} ]
		[.V\textsubscript{$\uparrow$=$\downarrow$} parcit ]	
	]
\end{singlespace}

\subsubsection{f-Struktur}
\begin{singlespace}
\begin{avm}
\[ PRED &  \rm ‘parco \q<SUBJ, OBJ\textsubscript{REC}\q>’\\
SUBJ & \[``Caesar'' \] \\
OBJ\textsubscript{REC} & \[PRED &  \rm ‘peto \q<SUBJ, OBJ\q>’\\
MOOD & part \\
PASSIVE & - \\
RELTENSE & present \\
CASE & dat \\
NUM & pl \\
GEN & m \\
SUBJ & \[PRED & `pro' \\
PRON-TYPE  & mis \] \\
OBJ & \[PRED & `auxilium' \\
CASE & acc \\
NUM & sg \\
GEN & n \] \\
\] \]
\end{avm}
\end{singlespace}

\subsection{Variante 2: Partizip mit angenommenem Bezugswort}
%In der obigen Variante würde folglich ein ausgelassenes Bezugswort des Partizips angenommen werden.
Das Hauptroblem bei der erstgenannten Variante ergibt sich dadurch, dass durch das komplett fehlende Bezugswort bei der Implementierung/Formulierung der Redundanz- und Default-Regeln Probleme auftreten. Des Weiteren erfüllen die Partizipialkonstruktionen in der Regel die grammatikalischen Funktionen eines XADJ, XCOMP oder ADJ; die Klassifikation als OBJ in der obigen Überlegung würde daher eine Ausnahme darstellen.

Folgt man dem NM und betrachtet die Verwendung des substantivierten Partizips als rein attributiv,\footnote{Vgl. NM, S. 713, §498.} muss man das Partizip (\textit{petentibus}) als Attribut zu einem sozusagen fehlenden Bezugswort betrachten -- in diesem Fall also etwa \textit{eis} oder \textit{viris}. So steht die Partizipialkonstruktion in Analogie zum rein attributiven PC in der Rolle eines XADJ in Abhängigkeit von einer NP; das Bezugswort selbst ist dann in unserem Beispiesatz das Objekt des Hauptsatzprädikats \textit{parcit}. Dieses als lexikalisches Element fehlende Objekt wird in der c-Struktur mit \textit{mis} (für ,,missing'', ,,fehlend'') bezeichnet.  
Die Einordnung als XADJ entspricht zudem dem in den allgemeinen Regeln aufgestellten Konzept.
%\subsubsection{Einschränkungen}
%\textbf{Variante 1: XADJ}:\\
%Das Subjekt der untergeordneten Struktur ist das Objekt der dem XADJ übergeordneten Struktur (welches fehlt): \\
%($\downarrow$SUBJ) = ((OBJ$\uparrow$)XADJ) \textbf{???} \\

\subsubsection{Lexikoneintrag}
Der Lexikoneintrag des Partizips lautet wie folgt:\footnote{Im Rahmen des Umfangs der Arbeit wird nur die für unseren Beispielsatz relevanten Argumente aufgezählt. Für andere mögliche Konstruktionen von \textit{petere} -- wie (SUBJ, OBJ, OBL\textsubscript{LOC})bzw. (SUBJ, OBJ, OBL\textsubscript{PURPOSE}) -- müssten eigene Lexikoneinträge erstellt werden.}
\begin{singlespace}
\begin{tabular}{ l  l  l  l  } 
\textbf{petentibus}: & \: V \\
$\qquad$ & \:  ($\uparrow$PRED) & = & `peto$\langle$SUBJ, OBJ$\rangle$' \\
%$\qquad$ & $[2]$ \:  ($\uparrow$SUBJ) & = & \{((XADJ$\uparrow$)GF) $\mid$ ((ADJ$\uparrow$)GF)\} \\
$\qquad$ & \:  ($\uparrow$SUBJ CASE) & = & \{abl $\mid$ dat\} \\
$\qquad$ & \:  ($\uparrow$SUBJ NUM) & = & pl \\
$\qquad$ & \:  ($\uparrow$SUBJ GEN) & = & \{m $\mid$ n $\mid$ f\} \\
$\qquad$ & \:  ($\uparrow$OBJ CASE) & = & acc \\
$\qquad$ & \:  ($\uparrow$MOOD) & = & part\\
$\qquad$ & \:  ($\uparrow$FIN) & = & - \\
$\qquad$ & \:  ($\uparrow$PASSIVE) & = & - \\
$\qquad$ & \:  ($\uparrow$RELTENSE) & = & present \\ 
$\qquad$ & \:  ($\uparrow$CASE) & = & \{abl $\mid$ dat\} \\
$\qquad$ & \:  ($\uparrow$NUM) & = & pl \\
$\qquad$ & \:  ($\uparrow$GEN) & = & \{m $\mid$ n $\mid$ f\} \\
\end{tabular}
\newline
\end{singlespace}

Da im Lateinischen die Verben die Kasus ihrer Objekte bestimmen, muss im Lexikoneintrag des Prädikats der übergeordneten Struktur festgelegt sein, dass sein Objekt im Dativ steht:\footnote{Vgl. \cite[48]{Skript}.} \\
\begin{singlespace}
\begin{tabular}{ l  l  l  l  } 
\textbf{parcit}: & \: V \\
$\qquad$ & \:  ($\uparrow$PRED) & = & `parco$\langle$SUBJ, OBJ$\rangle$' \\
$\qquad$ & $\qquad$ & . \\
$\qquad$ & $\qquad$ & . \\
$\qquad$ & $\qquad$ & . \\
$\qquad$ & \:  ($\uparrow$FIN) & = & + \\
$\qquad$ & \:  ($\uparrow$OBJ CASE) & = & dat \\
\end{tabular}
\newline
\end{singlespace}

%\subsubsection{Syntaxregeln}
%\begin{singlespace}
%\begin{tabular}{ l  l  c  c  c  c }
%  S & $\rightarrow$ & NP\textsubscript{1} & NP\textsubscript{2} & V \\
 %  & $\qquad$ & \textsuperscript{($\uparrow$OBJ) = $\downarrow$} & \textsuperscript{($\uparrow$SUBJ) = $\downarrow$} & \textsuperscript{$\uparrow$ = $\downarrow$} \\
%		NP\textsubscript{1} & $\rightarrow$ & N' \\
 %  & $\qquad$ & \textsuperscript{$\uparrow$ = $\downarrow$} \\
  %		  N' & $\rightarrow$ & N & VP \\
   %& $\qquad$ & \textsuperscript{$\uparrow$ = $\downarrow$} & \textsuperscript{($\uparrow$XADJ) = $\downarrow$} \\		    
	%	    VP & $\rightarrow$ & V' \\
%   & $\qquad$ & \textsuperscript{$\uparrow$ = $\downarrow$} \\
 % 				  V' & $\rightarrow$ & NP\textsubscript{3} & V \\
  % & $\qquad$ & \textsuperscript{($\uparrow$OBJ) = $\downarrow$} & \textsuperscript{$\uparrow$ = $\downarrow$} \\
   %					 NP\textsubscript{3} & $\rightarrow$ & N \\
%   & $\qquad$ & \textsuperscript{$\uparrow$ = $\downarrow$} \\
 %   NP\textsubscript{2} & $\rightarrow$ & N \\
  % & $\qquad$ & \textsuperscript{$\uparrow$ = $\downarrow$} \\
%\end{tabular} 
%\end{singlespace}

\subsubsection{c-Struktur}
\begin{singlespace}
\Tree [.S 
		[.NP{\textsubscript{($\uparrow$OBJ)=$\downarrow$}}
			[.N\textsubscript{$\uparrow$=$\downarrow$} \textit{mis} ]
			[.VP\textsubscript{$\downarrow$ $\in$ ($\uparrow$XADJ)}  
				[\qroof{auxilium}.NP\textsubscript{($\uparrow$OBJ)=$\downarrow$} ]
				[.V\textsubscript{$\uparrow$=$\downarrow$} petentibus ] 				
			]
		]	
		[\qroof{Caesar}.NP\textsubscript{($\uparrow$SUBJ)=$\downarrow$} ]
		[.V\textsubscript{$\uparrow$=$\downarrow$} parcit ]	
	]
\end{singlespace}

\subsubsection{f-Struktur}
\begin{singlespace}
\begin{avm}
\[ PRED &  \rm ‘parco \q<SUBJ, OBJ\q>’\\
SUBJ & \[``Caesar'' \] \\
OBJ & \[PRED & `pro' \\
PRON-TYPE & mis \\
CASE & dat \\
NUM & pl \\
GEN & m \]\tikzmark{alpha} \\
XADJ & \{ \[PRED &  \rm ‘peto \q<SUBJ, OBJ\q>’\\
MOOD & part \\
PASSIVE & - \\
RELTENSE & present \\
CASE & dat \\
NUM & pl \\
GEN & m \\
SUBJ &  \tikzmark{omega} \\
OBJ & \[PRED & `auxilium' \\
CASE & acc \\
NUM & sg \\
GEN & n \\
\] \] \} &            $\qquad$ \\
\]
\end{avm}
\tikz[remember picture,overlay] 
    \draw[<-] (pic cs:alpha) to[out=0,in=0,looseness=2.5]  (pic cs:omega);
    
\end{singlespace}

\section{Dominantes Partizip}
Beim sogenannten dominanten Partizip trägt nicht das Substantiv, sondern das in Kasus, Numerus und Genus übereinstimmenden Partizip die Hauptbedeutung; das Partizip ,dominiert' daher sein Bezugswort.
%Beim sogenannten dominanten Partizip ,dominiert` das Partizip sein Bezugswort, weswegen der Hauptfokus auf dem Partizip liegt.
Aus diesem Grund wird das dominante Partizip im Deutschen in der Regel mit einem Verbalsubstantiv wiedergegeben, von dem das im Lateinischen regierende Substantiv als Genitiv abhängt. Meistens verwendet man das Partizip Perfekt Passiv als dominantes Partizip.\footnote{Vgl. NM, S. 717 f., §502.}\\

\subsection{Vorüberlegungen zur Umsetzung in der LFG}
%Der Lexikoneintrag zum Partizip der Konstruktion unterscheidet sich nicht wesentlich von den vorherigen.
%\textbf{ (ich glaub wir brauchen hier echt nich nochmal nen Lexikoneintrag...)} \\
Das dominante Partizip soll zunächst einmal in Abhängigkeit von einer Präpositionalphrase anhand des Beispielsatzes \textit{ab urbe condita Roma viguit}, danach ohne derartige Abhängigkeit am Beispielsatz \textit{libertate amissa doleo} betrachtet werden. Da der Restsatz \textit{Roma viguit} bzw. \textit{doleo} auch ohne die Partizipialkonstruktion Sinn ergibt, muss letztere wie beim Abl. abs. ein ADJ zum finiten Satz sein.

\subsection{Version 1: Partizip in Abhängigkeit einer Präpositionalphrase}
%\subsubsection{Variante 1 -- Partizip als attributives XADJ}
Nun sieht das dominante Partizip \textit{condita} rein formal aus wie ein attributives Partizip zum Bezugswort \textit{urbe},\footnote{Vgl. NM S. 717, §502.} weswegen eine NP mit \textit{urbe} als Kopf anzunehmen wäre. Das Partizip wäre somit seinem Bezugswort untergeordnet. Es würde dann, analog zum attributiven PC, als XADJ klassifiziert werden. Aufgrund des Vorhandenseins der Präposition hängt die gesamte Konstruktion in diesem Fall von einer PP ab. Da das Prädikat \textit{viguit} neben dem Subjekt keine weiteren Argumente fordert, wird die PP als ADJ eingestuft. Die zugehörigen c- und f-Strukturen sähen demnach wie folgt aus:

\begin{singlespace}
\Tree [.S 
		[.PP{\textsubscript{$\downarrow$ $\in$ ($\uparrow$ADJ)}}
			[.P'\textsubscript{$\uparrow$=$\downarrow$} 
				[.P\textsubscript{$\uparrow$=$\downarrow$} ab ] 
				[.NP\textsubscript{($\uparrow$OBJ)=$\downarrow$}
					[.N'\textsubscript{$\uparrow$=$\downarrow$} 
						[.N\textsubscript{$\uparrow$=$\downarrow$} urbe ]
						[\qroof{condita}.VP\textsubscript{$\downarrow$ $\in$ ($\uparrow$XADJ)} ]
					] 
				]
			]				
		] 	
		[\qroof{Roma}.NP\textsubscript{($\uparrow$SUBJ)=$\downarrow$} ]
		[.V\textsubscript{$\uparrow$=$\downarrow$} viguit ]	
	]\\
\newline
\end{singlespace}
    
\begin{singlespace}    
\begin{avm}
\[ PRED &  \rm ‘vigeo \q<SUBJ\q>’\\
SUBJ & ``Roma'' \\
ADJ & \[ PRED &  \rm ‘ab \q<OBJ\q>’\\
OBJ & \[ PRED & `urbs' \\ 
CASE & abl \\
NUM & sg \\
GEN & f  \\
XADJ & \[PRED &  \rm ‘condo \q<SUBJ\q>’\\
MOOD & part \\
PASSIVE & + \\
RELTENSE & past \\
CASE & abl \\
NUM & sg \\ 
GEN & f  \\
SUBJ &  \tikzmark{Dawson} \] \] \tikzmark{Kimya} & $\qquad$ & $\qquad$  \\
\] \\
\]
\tikz[remember picture,overlay] 
    \draw[<-] (pic cs:Kimya) to[out=10,in=0,looseness=2.4]  (pic cs:Dawson);
\end{avm}
\newline
\newline
\end{singlespace}

%\subsubsection{Variante 2 -- Partizip in tatsächlich dominanter Position}
Da Adjunkte jedoch nach Belieben weggelassen werden können, würde dies bedeuten, dass auch das als XADJ klassifizierte Partizip fehlen könnte; somit wäre der Satz \textit{ab urbe Roma viguit} grammatikalisch korrekt. Das stimmt zwar formal -- ist jedoch semantisch sinnfrei. Eine semantisch sinnvollere Darstellung ergibt sich, wenn das Bezugswort vom Prädikat des Partizips gefordert wird; da das Partizip sein Bezugswort dominiert, sollte ihm in der LFG-Darstellung eine seinem Bezugswort übergeordnete Funktion zukommen. Somit würde die Partizipialkonstruktion von einer VP mit dem Kopf \textit{condita} abhängen; das Bezugsnomen \textit{urbe} wäre dann schlicht das Subjekt der Partizipialkonstruktion.
Aus diesen Gründen müssen die zuvor aufgestellten Strukturen korrigiert werden. Zunächst soll hierzu kurz der Lexikoneintrag des Partizips dargeboten werden.

\subsubsection{Lexikoneintrag}

\begin{singlespace}
\begin{tabular}{ l  l  l  l  } 
\textbf{condita}: & \: V \\
$\qquad$ & \:  ($\uparrow$PRED) & = & `condor$\langle$SUBJ$\rangle$'\\
%$\qquad$ & $[2]$ \:  ($\uparrow$SUBJ) & = & ((ADJ$\uparrow$)GF)\\
$\qquad$ & \:  \{(($\uparrow$SUBJ GEN) & = & f \\ 
$\qquad$ & \: \: \: ($\uparrow$SUBJ NUM) & = & sg \\
$\qquad$ & \: \: \: ($\uparrow$SUBJ CASE) & = & \{nom $\mid$ abl\} ) $\mid$\\
$\qquad$ & \: \: (($\uparrow$SUBJ GEN) & = & n \\
$\qquad$ & \: \: \: ($\uparrow$SUBJ NUM) & = & pl \\
$\qquad$ & \: \: \: ($\uparrow$SUBJ CASE) & = & \{nom $\mid$ acc\} ) \}\\
$\qquad$ & \:  ($\uparrow$MOOD) & = & part\\
$\qquad$ & \:  ($\uparrow$FIN) & = & - \\
$\qquad$ & \:  ($\uparrow$PASSIVE) & = & + \\
$\qquad$ & \:  ($\uparrow$RELTENSE) & = & past \\
$\qquad$ & \:  \{(($\uparrow$GEN) & = & f \\ 
$\qquad$ & \: \: \: ($\uparrow$NUM) & = & sg \\
$\qquad$ & \: \: \: ($\uparrow$CASE) & = & \{nom $\mid$ abl\} ) $\mid$\\
$\qquad$ & \: \: (($\uparrow$GEN) & = & n \\
$\qquad$ & \: \: \: ($\uparrow$NUM) & = & pl \\
$\qquad$ & \: \: \: ($\uparrow$CASE) & = & \{nom $\mid$ acc\} ) \}\\
\end{tabular}
\newline
\end{singlespace}

%\subsubsection{Syntaxregeln}
%\begin{singlespace}
%\begin{tabular}{ l  l  c  c  c  c }
 % S & $\rightarrow$ & PP & NP\textsubscript{1} & V\\
  % & $\qquad$ & \textsuperscript{$\downarrow$ $\in$ ($\uparrow$ADJ)} & \textsuperscript{($\uparrow$SUBJ) = $\downarrow$} & \textsuperscript{$\uparrow$ = $\downarrow$} \\
%		    PP & $\rightarrow$ & P' \\
 %  & $\qquad$ & \textsuperscript{$\uparrow$ = $\downarrow$} \\
  %				  P' & $\rightarrow$ & P & VP \\
   %& $\qquad$ & \textsuperscript{$\uparrow$ = $\downarrow$} & \textsuperscript{($\uparrow$OBJ) = $\downarrow$} \\
	%				    VP & $\rightarrow$ & V' \\
%   & $\qquad$ & \textsuperscript{$\uparrow$ = $\downarrow$} \\
%		  				  V' & $\rightarrow$ & V & NP\textsubscript{2} \\
 %  & $\qquad$ & \textsuperscript{$\uparrow$ = $\downarrow$} & \textsuperscript{($\uparrow$SUBJ) = $\downarrow$} \\
	%	   					 NP\textsubscript{2} & $\rightarrow$ & N \\
   %& $\qquad$ & \textsuperscript{$\uparrow$ = $\downarrow$} \\
    %NP\textsubscript{1} & $\rightarrow$ & N \\
   %& $\qquad$ & \textsuperscript{$\uparrow$ = $\downarrow$} \\
%\end{tabular} 
%\newline
%\end{singlespace}

\subsubsection{c-Struktur}
\begin{singlespace}
\Tree [.S 
		[.PP{\textsubscript{$\downarrow$ $\in$ ($\uparrow$ADJ)}}
			[.P\textsubscript{$\uparrow$=$\downarrow$} ab ] 
			[.VP\textsubscript{($\uparrow$OBJ)=$\downarrow$}
				[.V\textsubscript{$\uparrow$=$\downarrow$} condita ]
				[\qroof{urbe}.NP\textsubscript{($\uparrow$SUBJ) = $\downarrow$} ]
			]
			]				
		[\qroof{Roma}.NP\textsubscript{($\uparrow$SUBJ)=$\downarrow$} ]
		[.V\textsubscript{$\uparrow$=$\downarrow$} viguit ]	
	]\\
\newline
\end{singlespace}

\subsubsection{f-Struktur}
\begin{singlespace}
\begin{avm}
\[ PRED &  \rm ‘vigeo \q<SUBJ\q>’\\
SUBJ & ``Roma'' \\
ADJ & \{ \[ PRED &  \rm ‘ab \q<OBJ\q>’\\
OBJ & \[ PRED &  \rm ‘condor \q<SUBJ\q>’\\
MOOD & part \\
PASSIVE & + \\
RELTENSE & past \\
CASE & abl \\
NUM & sg \\
GEN & f \\
SUBJ & \[PRED & `urbs' \\
CASE & abl \\
NUM & sg \\
GEN  & f \] \] \] \} \]
\end{avm}\\
\end{singlespace}

\subsection{Version 2: Partizip ohne Abhängigkeit}
Nun breitete sich in klassischer Zeit jedoch die präpositionslose Variante des dominanten Partizips besonders aus,\footnote{Vgl. LHS S. 393, §210.} weswegen auch hierzu ein Beispielsatz betrachtet werden soll: \textit{libertate amissa doleo.} Formal ist diese Konstruktion im Ablativ kaum vom Abl. abs. zu unterscheiden; die Differenzt liegt lediglich darin, dass die Dominanz des Partizips in der vorliegenden Konstruktion stärker hervorgehoben wird. Um diesem -- wenn auch semantisch geringen -- Unterschied gerecht zu werden, sollte auch in der LFG-Darstellung die Dominanz des Partizips über sein Bezugswort deutlich werden. 
Da lateinische Partizipialkonstruktionen jedoch ohnehin stets V als Kopf tragen,\footnote{Dies ist auch beim S\textsubscript{part} der Abl. abs.-Konstruktion der Fall, wie in den Abschnitten 6.1 und 6.3 dieser Arbeit deutlich wird.} kann die besondere Dominanz des Partizips nach unserer Darstellungsweise nicht hervorgehoben werden. 
%Dies wäre ein Argument, die Partizipialkonstruktionen als gesonderte Partizipialphrasen darzustellen. \textbf{(auch wegen der nominalen Eigenschaften der Partizipien) in Schlussfolgerung + Diese Darstellungsweise wurde in dieser Arbeit bereits in Anfängen versucht, nämlich bei der c-Struktur-Darstellung des Abl. abs. als S\textsubscript{part}.}
Auch hier ist, ebenso wie in der obigen Variante, das Bezugsnomen seinem Partizip unterstellt und die gesamte Partizipialkonstruktion in der Funktion eines ADJ. Da hier keine Präposition vorhanden ist, und die Partizipialkonstruktion syntaktisch und semantisch vom finiten Verb losgelöst ist, hängt das ADJ direkt von S ab.

%\subsubsection{Syntaxregeln}
%\begin{singlespace}
%\begin{tabular}{ l  l  c  c  c  c }
%  S & $\rightarrow$ & VP & V\\
 %  & $\qquad$ & \textsuperscript{$\downarrow$ $\in$ ($\uparrow$ADJ)} & \textsuperscript{($\uparrow$SUBJ) = $\downarrow$} & \textsuperscript{$\uparrow$ = $\downarrow$} \\
%	    VP & $\rightarrow$ & V' \\
 %  & $\qquad$ & \textsuperscript{$\uparrow$ = $\downarrow$} \\
	%		  V' & $\rightarrow$ & NP& V \\
  % & $\qquad$ & \textsuperscript{($\uparrow$SUBJ) = $\downarrow$} &\textsuperscript{$\uparrow$ = $\downarrow$} \\
	%	   					 NP & $\rightarrow$ & N \\
   %& $\qquad$ & \textsuperscript{$\uparrow$ = $\downarrow$} \\
%\end{tabular} 
%\newline
%\end{singlespace}

\subsubsection{Lexikoneintrag}
\begin{singlespace}
\begin{tabular}{ l  l  l  l  } 
\textbf{amissa}: & \: V \\
$\qquad$ & \:  ($\uparrow$PRED) & = & `amittor$\langle$SUBJ$\rangle$'\\
%$\qquad$ & $[2]$ \:  ($\uparrow$SUBJ) & = & ((ADJ$\uparrow$)GF)\\
$\qquad$ & \:  \{(($\uparrow$SUBJ GEN) & = & f \\ 
$\qquad$ & \: \: \: ($\uparrow$SUBJ NUM) & = & sg \\
$\qquad$ & \: \: \: ($\uparrow$SUBJ CASE) & = & \{nom $\mid$ abl\} ) $\mid$\\
$\qquad$ & \: \: (($\uparrow$SUBJ GEN) & = & n \\
$\qquad$ & \: \: \: ($\uparrow$SUBJ NUM) & = & pl \\
$\qquad$ & \: \: \: ($\uparrow$SUBJ CASE) & = & \{nom $\mid$ acc\} ) \}\\
$\qquad$ & \:  ($\uparrow$MOOD) & = & part\\
$\qquad$ & \:  ($\uparrow$FIN) & = & - \\
$\qquad$ & \:  ($\uparrow$PASSIVE) & = & + \\
$\qquad$ & \:  ($\uparrow$RELTENSE) & = & past \\
$\qquad$ & \:  \{(($\uparrow$GEN) & = & f \\ 
$\qquad$ & \: \: \: ($\uparrow$NUM) & = & sg \\
$\qquad$ & \: \: \: ($\uparrow$CASE) & = & \{nom $\mid$ abl\} ) $\mid$\\
$\qquad$ & \: \: (($\uparrow$GEN) & = & n \\
$\qquad$ & \: \: \: ($\uparrow$NUM) & = & pl \\
$\qquad$ & \: \: \: ($\uparrow$CASE) & = & \{nom $\mid$ acc\} ) \}\\
\end{tabular}
\newline
\end{singlespace}

\subsubsection{c-Struktur}
\begin{singlespace}
\Tree [.S 
		[.VP{\textsubscript{$\downarrow$ $\in$ ($\uparrow$ADJ)}}
				[\qroof{libertate}.NP\textsubscript{($\uparrow$SUBJ) = $\downarrow$} ]
				[.V\textsubscript{$\uparrow$=$\downarrow$} amissa ]
		]				 	
			[.V\textsubscript{$\uparrow$=$\downarrow$} doleo ]		
	]\\
\newline
\end{singlespace}

\subsubsection{f-Struktur}
\begin{singlespace}
\begin{avm}
\[ PRED &  \rm ‘doleo \q<SUBJ\q>’\\
SUBJ & \[PRED & `pro' \\
PRON-Type & mis\] \\
ADJ & \{ \[ PRED &  \rm ‘amittor \q<SUBJ\q>’\\
MOOD & part \\
PASSIVE & + \\
RELTENSE & past \\
CASE & abl \\
NUM & sg \\
GEN & f \\
SUBJ & \[PRED & `libertas' \\
CASE & abl \\
NUM & sg \\
GEN  & f \] \] \} \]
\end{avm}\\
\end{singlespace}


\section{Ablativus absolutus}
Wie beim PC vertritt auch die Partizipialkonstruktion des Ablativus absolutus einen Adverbialsatz, wobei das Bezugswort dem Subjekt, das Partizip dem Prädikat entspricht. Dabei wird das Bezugswort nicht vom Prädikat des finiten Satzes gefordert, und besitzt demnach keine eigene Satzgliedfunktion. Der Abl. abs. ist somit vom Rest des Satzes losgelöst, weswegen er stets eine freie Angabe darstellt. Partizip und Bezugswort stehen dabei immer im Ablativ. Da der Abl. abs. einen Adverbialsatz ersetzt, ist sein Partizip prädikativ verwendet; dass es nicht als Attribut zu einem Nomen steht, wird zudem daran deutlich, dass der Satz bei Wegfall des Partizips grammatikalisch nicht mehr korrekt wäre. Der Ablativ ist im Lateinischen für diese Konstruktion gewählt, da dieser Kasus bereits ohne Partizip adverbiale Verhältnisse, beispielsweise der Zeit, bezeichnet.\footnote{Vgl. KSt, S. 766, §138,1; S. 771, §138,5b; NM, S. 718 f., §503. Anstelle eines Partizips können auch bestimmte Nomina in den Ablativus absolutus treten. Auf dies kann im Rahmen des Umfangs dieser Arbeit, die sich auf Partizipialkonstruktionen konzentriert, nicht näher eingegangen werden; vgl. NM, S. 721, §504.} \\

\subsection{Vorüberlegungen zur Umsetzung in der LFG}
Aufgrund der syntaktischen Losgelöstheit der Partizipialkonstruktion vom Rest des Satzes nimmt die gesamte Konstruktion die Funktion eines ADJ an. Im Gegensatz zu den vorherigen Partizipialkonstruktionen wird der Abl. abs. nicht als VP, sondern als S\textsubscript{part} benannt.\footnote{Es ist zu beachten, dass der Abl. abs. trotz der Bezeichnung als S\textsubscript{part} kein Nebensatz im eigentlichen Sinne ist. Der Zusatz ,part' soll dies kennzeichnen.}
Wir haben uns für diese Variante entschieden, da durch die bloße Bezeichnung als VP nicht zur Geltung kommen würde, dass der Abl. abs. aufgrund der ausschließlich prädikativen Verwendung seines Partizips nur durch einen adverbialen Gliedsatz ersetzt werden kann und sowohl sein Subjekt als auch sein Prädikat innerhalb desselben Knotens enthalten sind -- und nicht wie beispielsweise beim PC das Subjekt aus der übergeordneten Struktur bezogen werden muss; sowohl das Partizip als V als auch das Bezugswort als NP sind S\textsubscript{part} untergeordnet. Zudem wird durch diese Bezeichnung die vollkommene syntaktische und semantische Losgelöstheit sowie die Unterscheidung zum dominanten Partizip deutlicher.
Diese Überlegungen sollen am Beispiel des Satzes \textit{barbaris in Gallia victis Caesar gaudet} in die Praxis umgesetzt werden.

%\subsection{Einschränkungen}
%Da der Restsatz (S\textsubscript{fin}) auch ohne den Abl. abs. noch Sinn ergeben würde,  steht er in der Funktion eines ADJ: \\
%($\uparrow$ADJ) = $\downarrow$ \\
%Auch beim Abl. abs. muss das Partizip in Kasus, Numerus und Genus mit seinem Bezugswort übereinstimmen:\footnote{Vgl. KSt S. 771, § 138,5a.}\\
%($\uparrow$SUBJ KNG) = ($\uparrow$KNG)\\
%Sowohl Partizip als auch Bezugswort stehen stets im Ablativ:\footnote{Vgl. KSt S. 771, § 138,5b.} \\
%($\uparrow$CASE) = abl \\
%($\uparrow$SUBJ CASE) = abl \\
%Da das Bezugswort des Partizips im Abl. abs. keine Rolle im übergeordneten Satz spielen darf, ist es keine grammatikalische Funktion der dem XADJ übergeordneten Struktur. Der Abl. abs. ist daher vom finiten Satz losgelöst:\footnote{Vgl. KSt S. 771, § 138,5b.} \\
%$\neg$ ($\uparrow$SUBJ) = ((ADJ$\uparrow$)GF) \\
%Da sich diese Arbeit ausschließlich auf das klassische Latein Caesars und Ciceros bezieht, gilt für die folgenden Betrachtungen die Annahme, dass im Abl. abs. kein Partizip Futur Aktiv (PFA) verwendet wird.\footnote{Vgl. KSt. S. 760, § 136,4c oder NM S. 771, § 469.}\\
%$\neg$ ($\uparrow$RELTENSE) = future \\

%($\uparrow$RELTENSE (ADJ)) $\neq$ future \\
%$\neg$ ($\downarrow$PRED) = ($\uparrow$GF PRED) \\

\subsection{Lexikoneintrag}
%Obige, für den Abl. abs. gültige Einschränkungen können jedoch nicht im Lexikoneintrag der Partizipien festgehalten werden, da Partizipien im Ablativ auch in anderen Partizipialkonstruktionen vorkommen; ist ein Partizip wie \textit{victis} beispielsweise teil eines PC, ist sein Subjekt eine grammatikalische Funktion der der Partizipialkonstruktion übergeordneten Struktur.
\begin{singlespace}
\begin{tabular}{ l  l  l  l  } 
\textbf{victis}: & \: V \\
$\qquad$ & \:  ($\uparrow$PRED) & = & `vincor$\langle$SUBJ$\rangle$'\\
%$\qquad$ & $[2]$ \: $\neg$ ($\uparrow$SUBJ) & = & \{((XADJ$\uparrow$)GF) $\mid$ ((ADJ$\uparrow$)GF)\} \\
$\qquad$ & \: ($\uparrow$SUBJ CASE) & = & \{dat $\mid$ abl\} \\
$\qquad$ & \:  ($\uparrow$SUBJ NUM) & = & pl \\
$\qquad$ & \: ($\uparrow$SUBJ GEN) & = & \{m $\mid$ f $\mid$ n\} \\
$\qquad$ & \:  ($\uparrow$MOOD) & = & part\\
$\qquad$ & \:  ($\uparrow$FIN) & = & - \\
$\qquad$ & \: ($\uparrow$PASSIVE) & = & + \\
$\qquad$ & \: ($\uparrow$RELTENSE) & = & past \\
$\qquad$ & \: ($\uparrow$CASE) & = & \{dat $\mid$ abl\} \\
$\qquad$ & \:  ($\uparrow$NUM) & = & pl \\
$\qquad$ & \: ($\uparrow$GEN) & = & \{m $\mid$ f $\mid$ n\} \\
\end{tabular}
\end{singlespace}

\subsection{Syntaxregeln}
Die Losgelöstheit der Ablativus-absolutus-Konstruktion muss in den Syntaxregeln festgehalten werden. Die anfangs genannten Syntaxregeln müssen daher erweitert werden: \\
\begin{singlespace}
\begin{tabular}{ l  l  c  c  c  c }
   S\textsubscript & $\rightarrow$ & S\textsubscript{part} & NP & V\\
   & $\qquad$ & \textsuperscript{ $\downarrow$ $\in$ ($\uparrow$ADJ)} & \textsuperscript{($\uparrow$SUBJ) = $\downarrow$} & \textsuperscript{$\uparrow$ = $\downarrow$} \\
   S\textsubscript{part} & $\rightarrow$ & NP & PP & V & \\
   & $\qquad$ &  \textsuperscript{($\uparrow$SUBJ) = $\downarrow$} &\textsuperscript{$\downarrow$ $\in$ ($\uparrow$ADJ)} & \textsuperscript{$\uparrow$ = $\downarrow$} \\
\end{tabular} 
\end{singlespace}

\subsection{c-Struktur}
\begin{singlespace}
\Tree [.S
		[.S{\textsubscript{part} \textsubscript{$\downarrow$ $\in$ ($\uparrow$ADJ)}}
			[\qroof{barbaris}.NP{\textsubscript{($\uparrow$SUBJ)=$\downarrow$}}	]
			[.PP\textsubscript{$\downarrow$ $\in$ ($\uparrow$ADJ)} 
					[.P'\textsubscript{$\uparrow$=$\downarrow$} 
						[.P\textsubscript{$\uparrow$=$\downarrow$} in ]
						[\qroof{Gallia}.{NP\textsubscript{($\uparrow$OBJ)=$\downarrow$}} ]
					]
				]		 
			[.V\textsubscript{$\uparrow$=$\downarrow$} victis ]
		]							
		[\qroof{Caesar}.{NP\textsubscript{($\uparrow$SUBJ)=$\downarrow$}} ] 
		[.V{\textsubscript{$\uparrow$=$\downarrow$}} gaudet ]
	]
\end{singlespace}

\subsection{f-Struktur}
\begin{singlespace}
\begin{avm}
\[ PRED &  \rm ‘gaudeo \q<SUBJ\q>’\\
SUBJ & \[``Caesar'' \]\\
ADJ & \{ \[PRED &  \rm ‘vincor \q<SUBJ\q>’\\
MOOD & part \\
PASSIVE & + \\
RELTENSE & past \\
CASE & abl \\
NUM & pl \\
GEN & m \\
SUBJ & \[PRED & `barbarus' \\
CASE & abl \\
NUM & pl \\
GEN & m \\ \] \\
ADJ & \{\[``in Gallia''\] \} \] \\
\}
\]
\end{avm}
\end{singlespace}

\section{Accusativus cum Participio}
Bei den Verben der unmittelbaren sinnlichen Wahrnehmung, oft bei \textit{videre} und \textit{audire}, sowie bei den Verben des Darstellens und Einführens, besonders bei \textit{facere} und \textit{inducere} -- im Sinne von ,in einem Werk, in einem Drama darstellen, (auftreten) lassen' -- steht die Partizipialkonstruktion oft in Verbindung mit einem Objekt und dem PPA im Akkusativ. Man nennt diese Verbindung Accusativus cum Participio (AcP). Das Partizip wird dabei in prädikativem Sinn verwendet.\footnote{Vgl. KSt, S. 763, §137; NM, S. 714, §499.}\\


%Der AcP ist von einem Verb der unmittelbaren sinnlichen Wahrnehmung oder von \textit{facere} bzw. \textit{inducere} im Sinne von ‚in einem Werk, in einem Drama darstellen, (auftreten) lassen' abhängig.\footnote{Vgl. NM S. 714, § 499. Vgl. auch KSt S. 763, § 137,2a.} \\ 

%\textbf{beim KsT heißt es: Zweitens wird das Partizip in prädikativem Sinne zur Ergänzung eines Verbalbegriffes gebraucht. Dieser Fall tritt ein: a) "Beshreibung des 'normalen AcPs' b) bei den Verben habeo und teneo in Verbindung mit dem PPP, entweder allein oder mit einem Objekte, um einen aus einer vollendeten Handlung hervorgegangenen Zusatnd oder bleibenden Besitz zu bezeichnen}

%\textbf{NM: Das PPP findet sich zur Umschreibung des Perfekts zusammen mit finiten Verbformen von \textit{habere} bzw. \textit{tenere} und bezeichnet dann einen dauernden Zustand oder bleibenden Besitz.}

%\footnote{Vgl. KSt S. 763, § 137,2a. Vgl. auch NM S. 763, § 137. \textbf{Vgl. auch LHS S. 387-88 § 207 c; auch KSt S. 763, § 137,2b? tenere + habere mit PPP?}} \\


\subsection{Vorüberlegungen zur Umsetzung in der LFG}
Da das Partizip der AcP-Konstruktion sein Subjekt aus der übergeordneten Struktur bezieht, muss es die Funktion entweder eines XADJ oder XCOMP annehmen. Für die Klassifikation als XADJ spricht, dass der Restsatz auch ohne das Partizip, analog zum PC, Sinn ergibt und grammatikalisch korrekt ist. Für die Einordnung als XCOMP hingegen spricht zum einen, dass Verben der Wahrnehmung, wie sie im AcI und AcP vorkommen, im Lateinischen eine Ergänzung erfordern; diese kann zwar auch durch ein bloßes Nomen, aber auch durch eine Partizipialkonstruktion ausgedrückt werden. Da das Partizip dann als Ergänzung vom Prädikat gefordert wird, muss es die Funktion eines XCOMP annehmen. Formal sind AcI- und AcP-Konstruktionen daher kaum auseinanderzuhalten;\footnote{\cite[53]{Skript} klassifiziert AcI-Konstruktionen im Deutschen ebenfalls als XCOMP.} der Unterschied liegt in der Semantik. Während beim AcI der Inhalt der Verbalhandlung betont wird, liegt beim AcP der Nachdruck auf der sinnlichen Rezeption einer Handlung oder eines Zustandes.\footnote{Vgl. LHS S. 387, §207.} Diese Bedeutungsdifferenz kann jedoch im Rahmen der LFG nicht ausgedrückt werden. Besonders bei den Verben des Darstellens und Einführens, wie \textit{facere} und \textit{inducere}, wird deutlich, dass die Ergänzung durch die AcP-Konstruktion vom Prädikat gefordert wird.

Da das Partizip beim AcP prädikativ verwendet wird, und vor allem da es vom Hauptsatzprädikat gefordert wird, ist es direkt von S abhängig. Die vorangegangenen Überlegungen sollen nun an folgendem Beispielsatz veranschaulicht werden: \textit{militem in campo iacentem vidit.}

%\subsection{Einschränkungen}
%Da das Partizip im AcP eine Ergänzung des Verbalbegriffs ist, erfüllt es im Satz stets die Funktion eines XCOMP: \\
%($\uparrow$XCOMP) = $\downarrow$ \\
%Das Partizip und sein Bezugswort stehen auch beim AcP im selben Kasus, Numerus und Genus:\footnote{Vgl. KSt S. 771, § 138,5a.}\\
%($\uparrow$SUBJ KNG) = ($\uparrow$KNG)\\
%Wie auch hier der Name der Konstruktion vermuten lässt, stehen beim AcP Partizip und Bezugswort im Akkusativ:\footnote{Vgl. KSt S. 763, § 137,2a.} \\
%($\uparrow$CASE) = acc \\
%($\uparrow$SUBJ CASE) = acc \\
%Das Bezugswort des Partizips ist das Objekt der dem XCOMP übergeordneten Struktur: (kann ja nur OBJ sein wegen Akk) \\
%	($\uparrow$SUBJ) = ((XCOMP$\uparrow$)OBJ) \\
%($\uparrow$XCOMP SUBJ) = ($\uparrow$OBJ) (?) \\

%Das Partizip ist beim AcP meist ein PPA, selten ein PPP. \\
%($\uparrow$XCOMP RELTENSE) = present   \\
%$\neg$ ($\uparrow$RELTENSE) = future \\
	

\subsection{Lexikoneintrag}

\begin{singlespace}
\begin{tabular}{ l  l  l  l  } 
\textbf{iacentem}: & \: V \\ 
$\qquad$ & \: ($\uparrow$PRED) & = & `iaceo$\langle$SUBJ, OBL\textsubscript{LOC}$\rangle$'\\
%$\qquad$ & $[2]$ \:  ($\uparrow$SUBJ) & = & \{((XADJ$\uparrow$)GF) $\mid$ ((XCOMP$\uparrow$)GF)\} \\
$\qquad$ & \: ($\uparrow$SUBJ CASE) & = & acc \\
$\qquad$ & \: ($\uparrow$SUBJ NUM) & = & sg \\
$\qquad$ & \: ($\uparrow$SUBJ GEN) & = & \{m $\mid$ f\} \\
$\qquad$ & \: ($\uparrow$MOOD) & = & part\\
$\qquad$ & \:  ($\uparrow$FIN) & = & - \\
$\qquad$ & \: ($\uparrow$PASSIVE) & = & - \\
$\qquad$ & \: ($\uparrow$RELTENSE) & = & present \\
$\qquad$ & \: ($\uparrow$CASE) & = & acc \\
$\qquad$ & \: ($\uparrow$NUM) & = & sg \\
$\qquad$ & \: ($\uparrow$GEN) & = & \{m $\mid$ f\} \\

\end{tabular}\\
\newline
\end{singlespace}
Im Lexikoneintrag des Prädikats der dem XCOMP übergeordneten Struktur -- hier \textit{vidit} -- müsste, wie oben erwähnt, zunächst spezifiziert sein, dass es ein XCOMP zu sich nehmen kann. Zudem müssen dort auch die Bedingungen, die dieses XCOMP erfüllen muss, aufgelistet werden.\footnote{Vgl. \cite[56]{Skript}.} \\
\begin{singlespace}
\begin{tabular}{ l  l  l  l  } 
\textbf{vidit}: & \: V \\
$\qquad$ & \: ($\uparrow$PRED) & = & `video$\langle$SUBJ, OBJ, XCOMP$\rangle$'\\
$\qquad$ & $\qquad$ & . \\
$\qquad$ & $\qquad$ & . \\
$\qquad$ & $\qquad$ & . \\
$\qquad$ & \:  ($\uparrow$FIN) & = & + \\
$\qquad$ & \:  ($\uparrow$SUBJ XCOMP) & = & ($\uparrow$OBJ)\\
$\qquad$ & \: ($\uparrow$XCOMP CASE) & = & acc \\
$\qquad$ & \: ($\uparrow$OBJ CASE) & = & acc \\
\end{tabular}\\
\newline
\end{singlespace}

%Alternative:
%video: $\langle$SUBJ, OBJ, COMP$\rangle$\\
%($\uparrow$COMP SUBJ) = `pro'\\
%($\uparrow$COMP SUBJ KNG) = ($\uparrow$OBJ KNG)\\


%\subsection{Syntaxregeln}

%\begin{singlespace}
%\renewcommand{\arraystretch}{1}  
%\begin{tabular}{ l  l  c  c  c }
 % S & $\rightarrow$ & NP\textsubscript{1} & VP & V\\
  % & $\qquad$ & \textsuperscript{($\uparrow$OBJ) = $\downarrow$} & \textsuperscript{($\uparrow$XCOMP) = $\downarrow$} & \textsuperscript{$\uparrow$ = $\downarrow$} \\
   % NP\textsubscript{1} & $\rightarrow$ & N \\
   %& $\qquad$ & \textsuperscript{$\uparrow$ = $\downarrow$} \\
    %VP & $\rightarrow$ & V' \\
   %& $\qquad$ & \textsuperscript{$\uparrow$ = $\downarrow$} \\
    %V' & $\rightarrow$ & PP & V & \\
   %& $\qquad$ & \textsuperscript{($\uparrow$OBL\textsubscript{LOC}) = $\downarrow$ } & \textsuperscript{$\uparrow$ = $\downarrow$} \\
    %PP & $\rightarrow$ & P' \\
	%& $\qquad$   & \textsuperscript{$\uparrow$ = $\downarrow$} \\
    %P' & $\rightarrow$ & P & NP\textsubscript{2} \\
   %& $\qquad$ & \textsuperscript{$\uparrow$ = $\downarrow$} & \textsuperscript{($\uparrow$OBJ) = $\downarrow$} \\
   % NP\textsubscript{2} & $\rightarrow$ & N \\
   %& $\qquad$ & \textsuperscript{$\uparrow$ = $\downarrow$} \\
%\end{tabular} 
%\end{singlespace}
 
\subsection{c-Struktur}
\begin{singlespace}
\Tree [.S 
		[\qroof{militem}.{NP\textsubscript{($\uparrow$OBJ)=$\downarrow$}} ] 
		[.VP{\textsubscript{($\uparrow$XCOMP)=$\downarrow$}}
			[.PP\textsubscript{($\uparrow$OBL\textsubscript{LOC})=$\downarrow$} 
					[.P'\textsubscript{$\uparrow$=$\downarrow$} 
						[.P\textsubscript{$\uparrow$=$\downarrow$} in ]
						[\qroof{campo}.{NP\textsubscript{($\uparrow$OBJ)=$\downarrow$}} ]
					]
				]	
			[.V\textsubscript{$\uparrow$=$\downarrow$} iacentem ]
		] 
		[.V\textsubscript{$\uparrow$=$\downarrow$} vidit ]	
	]
\end{singlespace}

\subsection{f-Struktur}
\begin{singlespace}
\begin{avm}
\[ PRED &  \rm ‘video \q<SUBJ, OBJ, XCOMP\q>’\\
SUBJ & \[ PRED & `pro' \\
		PRON-TYPE & mis	\]\\
OBJ & \[ PRED & `miles' \\
CASE & acc \\
NUM & sg \\
GEN & m \]\tikzmark{topic} \\
XCOMP & \[PRED &  \rm ‘iaceo \q<SUBJ, OBL\textsubscript{LOC}\q>’\\
MOOD & part \\
PASSIVE & - \\
RELTENSE & present \\
CASE & acc \\
NUM & sg \\
GEN & m \\
SUBJ &  \tikzmark{object} \\
OBL\textsubscript{LOC} & \[``in campo''\] \]  &            $\qquad$\\
\]
\end{avm}
\end{singlespace}

\tikz[remember picture,overlay] 
    \draw[<-] (pic cs:topic) to[out=0,in=0,looseness=3]  (pic cs:object);

\section{Fazit}
In dieser Arbeit wurden Möglichkeiten aufgezeigt, die Partizipialkonstruktionen des Lateinischen in den Formalismus der LFG einzubinden. Auffallend hierbei war, dass die Partizipialkonstruktionen fast ausschließlich die grammatikalischen Funktionen ADJ, XADJ oder XCOMP annehmen. Zudem wurde die generelle Bezeichnung der Partizipien als V erarbeitet. Jedoch wurde festgestellt, dass semantische Unterscheidungen –  wie beispielsweise zwischen rein attributivem und eigentlichem PC, die beide als XADJ klassifiziert werden, oder zwischen Abl. abs. und dominantem Partizip, die als ADJ klassifiziert werden, – im Rahmen der LFG nicht ausgedrückt werden können. Obwohl hierzu bereits erste Überlegungen und Vorschläge, wie das Einführen eines S\textsubscript{part} beim Abl. abs., eingebracht wurden, kann dies zukünftig weiter erarbeitet werden. Es wäre zu überlegen, ob das Einführen einer lexikalischen Kategorie Part(izipien) für das Lateinische, in dem Partizipialkonstruktionen sehr häufig verwendet werden, sinnvoll wäre. Diese Bezeichnung würde auch die nominalen Eigenschaften der Partizipien in den Fokus rücken, was beispielsweise für das dominante Partizip von Vorteil sein könnte.

Auch die Redundanz- und Default-Regeln, bzw. die funktionalen Annotationen der Syntaxregeln, könnten hierdurch vereinfacht werden: ein V würde dann stets das finite Verb darstellen und müsste nicht als solches gesondert beschrieben werden, während ein Part stets infinit wäre und nicht in Verbalphrasen, die direkt von S dominiert werden, vorkommen könnte.\footnote{Die übrigen Partizipialien, d.h. Gerundialien und Infinitive, müssten dann jedoch ebenfalls gesondert von V klassifiziert werden, um eine tatsächliche Vereinfachung der Regeln zu erreichen. Dies stellt einen Vorbehalt dar und ist einer der Gründe, weswegen Partizipien in dieser Arbeit weiterhin als V klassifiziert wurden. Eine Kompromisslösung könnte eventuell darin bestehen, dass sämtliche Partizipialien unter die Kategorie Part gefasst werden.} Zudem könnten, wenn Partizipien als Part klassifiziert würden, Verbformen wie das Perfekt Passiv in den Morphologieregeln als PPP in Verbindung mit \textit{esse} dargestellt werden, was eher der üblichen Konzeptualisierung der Perfekt-Passiv-Bildung entspricht und auch in Schulgrammatiken so gelehrt wird. 

Da die LFG versucht, menschliche Sprachverarbeitungsprozesse nachzubilden, ist das letztgenannte Argument von nicht geringer Bedeutung. Auch die oben angedachte Klassfikation der Partizipialien als gesonderte Verbformen in den Partizipialkonstruktionen bringt derartige psycholinguistische Vorteile mit sich.\footnote{Vgl. \cite[12; 60]{Rohrer}.} Dies bietet wiederum nützliche Erkenntnisse hinsichtlich des Spracherwerbs, für den sich die LFG ohnehin als linguistische Grundlage gut eignet.\footnote{Vgl. \cite[21]{DAZ}.} 

Dies zeigt, dass die LFG neben ihrer leichten Umsetzbarkeit in Form eines Parsers zu Automatisierungszwecken auch zur Abbildung von Spracherwerbs- bzw. Sprachverarbeitungsprozessen genutzt werden kann. Diese Arbeit leistet dazu einen kleinen Beitrag, indem sie Konzepte zur Einordnung der lateinischen Partizipialkonstruktionen in die LFG entwickelt hat.

%\end{singlespace}
%\bibliographystyle{plain}
\pagebreak
%\section*{Literaturverzeichnis}
\addcontentsline{toc}{section}{Literaturverzeichnis}
\printbibliography
\end{document}
