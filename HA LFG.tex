\documentclass[12pt,a4paper]{article}
%%% PACKAGES
%\usepackage{times} % Schriftart Times verwenden
\usepackage{graphicx} % support the \includegraphics command and options
\usepackage{booktabs} % for much better looking tables
\usepackage{array} % for better arrays (eg matrices) in maths
\usepackage{paralist} % very flexible & customisable lists (eg. enumerate/itemize, etc.)
\usepackage{verbatim} % adds environment for commenting out blocks of text & for better verbatim
\usepackage{subfig} % make it possible to include more than one captioned figure/table in a single float
\usepackage{colortbl} % enables to shade tables
\usepackage[style=authoryear]{biblatex}
\bibliography{quellen}
\usepackage[utf8]{inputenc}
\usepackage{url}
\usepackage{qtree}
\usepackage{amsmath}
\usepackage{amsfonts}
\usepackage{amssymb}
\usepackage[T1]{fontenc}  % stellt sicher, dass im PDF auch Umlaute gefunden werden
\usepackage{tgtermes}
\usepackage{pdfpages}
\usepackage{listings}
\usepackage{fixltx2e}
\usepackage[ngerman]{babel} % deutsche Begriffe (z.B. Inhaltsverzeichnis statt Contents)
\usepackage[german=quotes]{csquotes}
\renewcommand{\baselinestretch}{1.5} % Zeilenabstand
%\usepackage[onehalfspacing]{setspace} %Zeilenabstand
\usepackage{setspace}
%Seitenränder
\usepackage{geometry}
%\geometry{a4paper, top=2cm, left=3cm, right=3cm, bottom=2cm}
%Linenumbers
\usepackage[modulo]{lineno}
\usepackage{tabularx}
\usepackage{tablefootnote}
\usepackage{tikz}
\usetikzlibrary{tikzmark,positioning}
\usepackage{avm}

\begin{document}
%%%%%%%%%%%%%%%%%%%%%%%%%%
% Deckblatt
% The title
\begin{titlepage}

\begin{center}


% Upper part of the page
\begin{minipage}{0.55\textwidth}
\begin{flushleft} \small
Ruprecht-Karls-Universität Heidelberg\\
Seminar für Klassische Philologie\\
Sommersemester 2013\\
Leitung: Dr. Kathrin Winter\\ 
Proseminar: Seneca, \textit{epistulae morales}
\end{flushleft}
\end{minipage}
\begin{minipage}{0.4\textwidth}
\begin{flushright} \large

\end{flushright}
\end{minipage}
\\[3.3cm]
\rule{\textwidth}{0.4pt}\\[0.4cm]

% Title

{\Large Bedeutung, Notwendigkeit und Konsequenzen \\ der Selbstgenügsamkeit} \linebreak {\large -- Eine Betrachtung anhand von Sen. \textit{epist.} 72,7-8}\\[0.2cm]

\rule{\textwidth}{0.4pt}\\[2.4cm]

% Author and supervisor
%\begin{minipage}{0.4\textwidth}
\begin{flushleft} \small
Natalia Bihler\\
Matrikelnummer: 2925340\\
6. Fachsemester (Gymnasiallehramt nach GymPO)\\
Latein und Englisch\\
Dammweg 1, 69123 Heidelberg\\
E-mail: Bihler@stud.uni-heidelberg.de
\end{flushleft}
%\end{minipage}


\vfill

% Bottom of the page
{\large 21. August 2014}

\end{center}

\end{titlepage}
%%%%%%%%%%%%%%%%%%%%%%%%%%
\setcounter{page}{2}
\begingroup
\flushbottom
\tableofcontents
\thispagestyle{empty}
%\newpage
\pagebreak
\endgroup
%\setcounter{page}{1}
% The introduction

\nocite{Menge}
\nocite{LHS}
\nocite{KSt}
\nocite{Rohrer}
\nocite{Skript}
\nocite{Dal}
\nocite{Falk}
\nocite{Bresnan}
\nocite{Snijders}


% AVM Referenz http://nlp.stanford.edu/manning/tex/avm-doc.pdf
%https://github.com/Bhlini/LFG
%https://en.wikibooks.org/wiki/LaTeX/Linguistics#Syntactic_trees
%https://en.wikibooks.org/wiki/LaTeX/Labels_and_Cross-referencing
%http://nlp.stanford.edu/manning/tex/avm-doc.pdf
%http://nlp.stanford.edu/cmanning/tex/
%http://tex.stackexchange.com/questions/157131/problems-using-avm-package-for-lfg-structures
%http://latex-community.org/forum/viewtopic.php?f=12&t=11365
%http://www.essex.ac.uk/linguistics/external/clmt/latex4ling/avms/
%http://web.ift.uib.no/Teori/KURS/WRK/TeX/symALL.html Zeichen

\section{Einleitung}
Diese Arbeit beschäftigt sich mit der Beschreibung lateinischer Partizipialkonstruktionen im System der lexikalisch-funktionalen Grammatik (LFG). Die LFG ist eine in den späten 1970er Jahren vor allem aus der Generativen Grammatik Noam Chomskys hervorgegangene Theorie zur Beschreibung der Syntax natürlicher Sprachen.\footnote{Vgl. \cite[4]{Skript}. Vgl. auch \cite[1]{Dal}.} Zu ihren wichtigsten Begründern zählen Joan Bresnan und Ronald Kaplan, wobei auch Mary Dalrymple, ... und ... Falk bedeutsame Beiträge zu ihrer Weiterentwicklung geleistet haben.
% nicht zu vergessen unsere Arbeit, hö.
%Wie andere generative Theorien versucht sie darzustellen, wie aus der endlichen Anzahl an Wörtern einer Sprache eine unendliche Menge -- grammatisch korrekter -- Sätze erzeugt werden kann. 
Ihre Regeln sollen sowohl die Erzeugung einer unendliche Menge grammatisch korrekter Sätze aus der endlichen Anzahl an Wörtern einer Sprache erfassen als auch ungrammatische Sätze als solche erkennen.\footnote{Vgl. Skript S. 18.} Daher ist die LFG auch als Grammatikformalismus für die Computerlinguistik interessant, wobei sie der automatisierten Prüfung von Sätzen hinsichtlich ihrer Grammatikalität sowie der Generation neuer grammatischer Sätze dienen soll.\footnote{Vgl. Skript, S. 4.}

Da sich die Forschung im Bereich der LFG hinsichtlich der lateinischen Sprache bislang noch in ihren Anfängen befindet, soll in dieser Arbeit die Darstellung der im Lateinischen sehr prävalenten Partizipialkonstruktionen im Konzept der LFG beleuchtet werden. Dabei ist die Kenntnis des grundlegenden Aufbaus lateinischer Partizipialkonstruktionen unabdingbar. Deshalb werden -- nach einer Einführung in Thematik und Terminologie der LFG -- diese verschiedenen Konstruktionen kurz erklärt, bevor sie anhand von Beispielsätzen in das Gerüst der LFG eingefügt werden. Hierbei werden auch verschiedene Ansätzen und diverse Schwierigkeiten bei der Umsetzung erläutert.

\section{Einführung in Thematik und Terminologie}
\subsection{Partizipien}
Die Partizipien nehmen, wie bereits der Name impliziert, teil an den Eigenschaften des Nomens und des Verbums. Die Kongruenz mit dem Bezugswort in Kasus, Numerus und Genus und die Möglichkeit der Steigerung und Substantivierung spiegeln die nominalen, die Teilnahme an Aktionsart, Genus und Rektion des Verbums die verbalen Eigenschaften wider.\footnote{Vgl. LHS, S. 383, § 206.}
Im Lateinischen werden drei Partizipien verwendet: das Partizip Präsens Aktiv (PPA), das Partizip Perfekt Passiv (PPP) und das Partizip Futur Aktiv (PFA).
Wie alle Partizipialien bezeichnen die Partizipien jedoch nicht die Zeit an sich, sondern das zeitliche Verhältnis des Partizips zum \textit{verbum finitum}: Dabei kennzeichnet das PPA die Gleichzeitigkeit, das PPP die Vorzeitigkeit und das PPA die Nachzeitigkeit.\footnote{Vgl. KSt, S. 756, §136,3f.}
Des Weiteren haben PPA und PFA aktivische Bedeutung, das PPP passivische. In der Regel sind auch die Partizipien von Deponentien in der Bedeutung aktivisch.\footnote{NM, S. 708, § 496. In dieser Arbeit wird nur auf das klassische Latein Caesars und Ciceros Bezug genommen. Deshalb wird entgegen den üblichen wissenschaftlichen Konventionen auch der NM verwendet, der sich auf den Stil dieser beiden spezialisiert hat.} Daneben gibt es jedoch einige Partizipien Perfekt, die die Bedeutung eines PPA haben, wie beispielsweise \textit{confisus} oder \textit{diffisus}.\footnote{Footnote: Vgl. NM, S. 711, § 497.}

Partizipien bilden meist in Verbindung mit Substantiven spezifische satzwertige Konstruktionen, in denen das Partizip dem Prädikat, das Bezugswort dem Subjekt eines Nebensatzes entspricht. Dies ist im Weiteren für die Funktionszuteilung der einzelnen Satzbestandteile von Bedeutung.
Im Folgenden sollen das rein attributive Partizip, das substantivierte Partizip, das Participium coniunctum (PC), der Ablativus absolutus (Abl. abs.), der Accusativus cum Participio (AcP) und das dominante Partizip näher betrachtet werden, um sie anschließend in das System der LFG einfügen zu können. Dabei sollen, ausgehend von Lexikoneinträgen und Syntaxregeln, sowohl c- als auch f-Strukturen zu den einzelnen Phänomenen entwickelt werden.
\subsection{Die Lexikalisch-Funktionale Grammatik}
Die LFG stellt zusammen mit einer Reihe weiterer Sprachtheorien\footnote{NOCH NACHGUCKEN ODER EVT WEGLASSEN} eine Weiterentwicklung von Chomskys Generativer Grammatik dar, insofern dass sie der oberflächlichen Ebene syntaktischen Aufbaus eine zweite, abstraktere, hinzufügt. Da von dieser abstrakten, sogenannten funktionalen Ebene übersprachliche Universalität angenommen wird, soll diese als grundlegendes Element der LFG der mangelnden Generalität der frühen Transformationsgrammatik etwas entgegensetzen.\footnote{Vgl. \cite[1-3; 9]{Dal}. Weitere Unterschiede zu und Kritik an Chomskys Ansatz fasst \cite[11]{Rohrer} konzise zusammen; vgl. auch \cite[1; 3]{Dal}.} In der LFG wird ein kontextfreies Skelett, bestehend aus Syntaxregeln und der oberflächlichen Konstituenten-Struktur – im Folgenden c-Struktur genannt --  durch weitere einschränkende Regeln ergänzt, was die ,,generative Kraft der Grammatik‘‘ erhöht.\footnote{\cite[9]{Rohrer}.} Diese Einschränkungen werden im Lexikon einer Sprache, sowie durch funktionale Annotationen innerhalb der Syntaxregeln und c-Struktur ausgedrückt; weitere Bedingungen können über Redundanz- und Default-Regeln, die über dem Lexikon operieren, aufgestellt werden. 
All diese Regeln dienen dazu, die Erzeugung ungrammatischer funktionaler Strukturen -- f-Strukturen -- auszuschließen. Diese f-Strukturen stellen im System der LFG die zweite Ebene syntaktischer Repräsentation dar. C- und f-Struktur entstehen nicht durch Transformationsprozesse und befinden sich in keinem Ableitung-Verhältnis; sie existieren vielmehr parallel und beschreiben gemeinsam jeden sprachlichen Ausdruck.\footnote{Vgl. \cite[64]{Falk}; \cite[8]{Skript}; \cite[2; 4; 7]{Dal}; \cite[11; 13]{Rohrer}. Im Folgenden wird der Einfachheit halber nur von Sätzen die Rede sein.} Die Bedeutung der lexikalischen und funktionalen Elemente spiegelt sich auch im Namen der Theorie wieder.\footnote{Vgl.\cite[3]{Dal}.}

Beide syntaktischen Strukturen -- c- und f-Struktur -- sind notwendig für die korrekte Analyse eines Satzes und stehen in einem Korrespondenzverhältnis zueinander, wie im Weiteren deutlich werden wird.\footnote{Vgl. \cite[3]{Dal}; \cite[4]{Skript}.} Die beiden Strukturen repräsentieren unterschiedliche Aspekte linguistischer Organisation.\footnote{Vgl. \cite[1]{Dal}.} Die c-Struktur ist die konkrete Darstellung hierarchischer Organisation von Wörtern und Phrasen, vergleichbar mit den Syntaxbäumen kontextfreier Grammatiken.\footnote{Vgl. \cite[7]{Dal}; \cite[13]{Rohrer}.} Die f-Struktur hingegen beschreibt auf abstrakter Ebene die funktionellen Beziehungen zwischen grammatikalischen Strukturen -- vorerst können hierunter Satzglieder verstanden werden.\footnote{Vgl. \cite[7]{Dal}; \cite[4]{Skript}.} Über die Relationen der unterschiedlichen Regeln und Strukturen zueinander soll hier bereits kurz ein Überblick anhand der Analyse eines Satzes gegeben werden.\footnote{Diese Beschreibungen, wie auch alle folgenden hinsichtlich des Aufbauprozesses der Strukturen, beschränken sich auf die Analyse von Sätzen. Für die Generation neuer Sätze laufen diese Prozesse teilweise in anderer Reihenfolge ab.}

Zuerst werden die allgemeingültigen Syntaxregeln auf einen Satz angewandt. Darin sind auch die je nach Position / Abhängigkeit möglichen grammatikalischen Funktionen -- so z.B. Satzglieder -- verzeichnet.\footnote{So kann eine NP, die direkt von S dominiert wird, beispielsweise SUBJ sein und im Nominativ stehen, während dies für eine NP, die von einer anderen NP dominiert wird, nicht möglich ist.} Aus der Anwendung der Syntaxregeln auf den Satz ergibt sich die c-Struktur, bzw. mehrere an dieser Stelle möglichen Varianten davon. Die grammatikalischen Funktionen aus den Syntaxregeln werden an den jeweils passenden Knoten annotiert. Nun muss eine mit der c-Struktur korrespondierende f-Struktur erstellt werden. In der annotierten c-Struktur werden nun zusammengehörige Strukturen erfasst und die Knoten entsprechend bezeichnet; hierbei werden die funktionalen Annotationen sowie Informationen aus den Lexikoneinträgen und gegebenenfalls weiteren überlexikalischen Regeln miteinbezogen. In Gleichungen mit diesen Knoten-Bezeichnungen, genannt funktionale Beschreibungen, werden die c-Struktur-Knoten den f-Strukturen in einem viele-zu-eins-Verhältnis zugeordnet.\footnote{Vgl. \cite[9]{Skript}.} Die Minimallösung dieser funktionalen Gleichungen ist schließlich die f-Struktur.
Die einzelnen Strukturen sollen nun im Folgenden genauer betrachtet werden.

\subsubsection{Allgemeines zu den Syntaxregeln}
Die Syntaxregeln -- oder Phrasenstrukturregeln -- sind der Startpunkt der Analyse eines Satzes in der LFG. Diese Regeln müssen für alle Sätze einer Sprache allgemein gültig sein.\footnote{Vgl. \cite[47]{Dal}.} Dies ist möglich, da in der LFG, anders als in früheren Ansätzen,\footnote{Ein Beispiel hierfür wäre eine kontextfreie Grammatik nach Chomsky.} viele für die Erzeugung der Satzstruktur notwendigen Bedingungen in das Lexikon ausgelagert werden. Durch die Sytaxregeln wird die grammatikalisch korrekte Verkettung von Wörtern zu Phrasen und Phrasen zu einem Satz schematisch gewährleistet. Die Darstellung erfolgt allerdings von den größeren Elementen hin zu den kleineren, d.h. vom Satz über die Phrasen bis hin zu den Wörtern. 

,,Generalisierungen konkreter Phrasen führen zu ihrer Klassifikation als beispielsweise Nominalphrasen, Verbalphrasen, je nachdem, welches lexikalische Element die Phrase dominiert.\footnote{\cite[47]{Dal}; \cite[53; 58-9]{Dal}; vgl. \cite[15]{Rohrer}.} Die lexikalischen Kategorien, die für das Lateinische angenommen werden können, sind N(omen), V(erb), P(räposition), A(djektiv) und Adv(erb).\footnote{Vgl. \cite[52]{Dal}.}

Die Syntax des Lateinischen organisiert sich, wie bei einer Vielzahl weiterer Sprachen, organisiert sich um ein finites Verb.\footnote{Vgl. \cite[53]{Dal}. Im Lateinischen ist in aller Regel V der Kopf von S. Eine Ausnahme könnte eventuell der nominale Ablativus absolutus bilden, dessen Kopf nach logischer Betrachtung N sein müsste -- vgl. \cite[???]{Menge}? und \cite[64]{Falk}. Wenn man jedoch die Abl-abs-Konstruktion S\textsubscript{part} unterordnet (?), wie wir es hier tun, ergibt sich kein Problem hinsichtlich S an sich.}
Das Lateinische ist eine nicht-konfigurationale Sprache: Es lässt eine weitgehend freie Wortstellung zu, da die grammatikalische Funktion eines Wortes durch dessen morphologische Form bestimmt ist, und nicht durch die Konfiguration der Konstituenten in der c-Struktur.\footnote{Vgl. \cite[19]{Rohrer}; \cite[65]{Dal}. In Sprachen, in denen dieses finite Verbalelement stets in einer bestimmten Position im Satz vorkommt, wird seine dominierende Phrase als IP, für inflectional phrase, bezeichnet. I ist dabei eine funktionale, keine lexikalische, Kategorie. Eine andere funktionelle Kategorie, C, für complementizer phrase, existiert zwar auch im Lateinischen, wird jedoch in dieser Arbeit aus Platzgründen außer Acht gelassen. Vgl. \cite[53; 65]{Dal}.} So ist im Lateinischen die Wortform \textit{canem} stets Akkusativ-Objekt, gleichgültig an welcher Position im Satz es sich befindet. Aufgrund dieser Gegebenheiten ist der Kopf eines jeden Satzes im Lateinischen die Kategorie ,,S‘‘.\footnote{Die Kategorie S wird nach Bloomfield (first name?) als ,,exozentrisch‘‘ bezeichnet, der sie den sogenannten endozentrischen Kategorien gegenüberstellt, die lexikalische Köpfe besitzen; vgl. \cite[64]{Dal}.} Die Klassifikation als S erlaubt die freie Umstellung der Töchter von S. Diese Töchter sind ein Prädikat mit seinen Argumtenten, inklusive des Subjekts und gegebenenfalls optionale Konstituenten.\footnote{Vgl. \cite[64-65]{Dal}.}

Folglich besteht S aus einer bestimmten Menge von Phrasen und lexikalischen Elementen. In den Syntaxregeln wird dieser Zusammenhang durch ,,kontextfreie Ersetzungsregeln der Form ,ersetze a durch b‘ (konventionell geschrieben: ,a $\rightarrow$ b‘)‘‘ beschrieben.\footnote{Siehe \cite[18]{Rohrer}.}

Auch wenn die Syntaxregeln der LFG verglichen mit früheren Ansätzen weitgehend überschaubar sind, werden hier nur die für die Partizipialkonstruktionen relevanten Regeln angegeben, da alles Weitere den Rahmen dieser Arbeit sprengen würde:\footnote{Da Adjektive, Adverbien und Pronominaladjektive (zu denen sich u.a. bei Snijder unter dem Begriff ,,Determiner‘‘ weitere Informationen finden (S. 7)), sowie Nebensätze aller Art  -- und damit auch Subjunktionen und Konjunktionen -- in dieser Arbeit nicht von Bedeutung sind, wird in den folgenden Syntaxregeln und auch im Weiteren nicht darauf eingegangen. Letzteres bezieht sich auf lateinische Nebensätze: Während Partizipialkonstruktionen im Deutschen zwar in der Regel durch Gliedsätze wiedergegeben werden, sind sie im Lateinischen Teil des Hauptsatzes. Überlegungen bezüglich der Subjunktionen im Deutschen finden sich unter \cite[103-119]{Skript}, zu den Konjunktionen im Deutschen unter \cite[120-136]{Skript}.} 

%\textbf{Wir schreiben die Regeln erstmal ohne Grammatikfunktionen auf, sagen dann in nem Satz, dass die Grammatikfunktionen hier auch dazugeschrieben werden sollten (+ gewisse weitere Beschränkungen u.U.), da eben eine NP unter ner NP nicht deren Subjekt sein kann (plus vielleicht noch 1-2 Beispiele um zu zeigen, dass wir's kapiert haben xD) und machen dann die ersten zwei Zeilen von den Regeln beispielhaft mit den Anmerkungen, einfach dass man sieht, aha, so könnte das dann aussehen (die hab ich schon gemacht, also wäre das kein extra Aufwand)}

%\begin{singlespace}
\begin{tabular}{ l  l  c  c  c  c  c  c  c}
S & $\rightarrow$ & [V & , & XP] & $\mid$ & V \\
%   & $\qquad$ & \textsuperscript{$\uparrow$ = $\downarrow$} & \textsuperscript{($\uparrow$ADJ-GF) = $\downarrow$ $\mid$ ($\uparrow$ARG-GF) = $\downarrow$} &  \textsuperscript{$\uparrow$ = $\downarrow$} \\
%      & $\qquad$ & \textsuperscript{($\uparrow$FIN)} & &  \textsuperscript{($\uparrow$FIN)}\\
%XP & $\rightarrow$ & \{ NP & , & VP & , & PP \}* \\
XP & $\rightarrow$ & \{ NP & $\mid$ &  VP & $\mid$ & PP \}* \\
%   & $\qquad$ & \textsuperscript{($\uparrow$ADJ-GF) = $\downarrow$ $\mid$ ($\uparrow$ARG-GF) = $\downarrow$} &\textsuperscript{($\uparrow$ADJ-GF) = $\downarrow$ $\mid$ ($\uparrow$ARG-GF) = $\downarrow$} & \textsuperscript{($\uparrow$ADJ-GF) = $\downarrow$ $\mid$ ($\uparrow$ARG-GF) = $\downarrow$} \\
   NP & $\rightarrow$ & [N & & XP] & $\mid$ & N \\
 %  & $\qquad$ & \textsuperscript{$\uparrow$ = $\downarrow$}  & \textsuperscript{($\uparrow$ADJ-GF) = $\downarrow$ $\mid$ ($\uparrow$ARG-GF) = $\downarrow$} & \textsuperscript{$\uparrow$ = $\downarrow$}\\
VP & $\rightarrow$ & [V & &  XP] & $\mid$ & V \\
 %  & $\qquad$ & \textsuperscript{$\uparrow$ = $\downarrow$} & \textsuperscript{($\uparrow$ADJ-GF) = $\downarrow$ $\mid$ ($\uparrow$ARG-GF) = $\downarrow$} & \textsuperscript{($\uparrow$ADJ-GF) = $\downarrow$ $\mid$ ($\uparrow$ARG-GF) = $\downarrow$} & \textsuperscript{$\uparrow$ = $\downarrow$}\\
   PP & $\rightarrow$ & [P & , & VP] & $\mid$  & [P & ,   & NP] \\
  % & $\qquad$ & \textsuperscript{$\uparrow$ = $\downarrow$} & \textsuperscript{($\uparrow$ADJ-GF) = $\downarrow$ $\mid$ ($\uparrow$ARG-GF) = $\downarrow$} & \textsuperscript{$\uparrow$ = $\downarrow$} & \textsuperscript{($\uparrow$ADJ-GF) = $\downarrow$ $\mid$ ($\uparrow$ARG-GF) = $\downarrow$}\\
%PP & $\rightarrow$ & P &  \{VP $\mid$     & NP\}* \\
 %  & $\qquad$ & \textsuperscript{$\uparrow$ = $\downarrow$} & \textsuperscript{($\uparrow$ADJ-GF) = $\downarrow$ $\mid$ ($\uparrow$ARG-GF) = $\downarrow$} & \textsuperscript{($\uparrow$ADJ-GF) = $\downarrow$ $\mid$ ($\uparrow$ARG-GF) = $\downarrow$}\\
\end{tabular}\\
\newline
%\end{singlespace}

%\begin{tabular}{ l  l  c  c  c  c  c  c  c}
%S & $\rightarrow$ & [V , XP] & $\mid$ & V \\
%XP & $\rightarrow$ & \{ NP & $\mid$ &  VP & $\mid$ & PP \}* \\
%   NP & $\rightarrow$ & [N \{XP\}*] & $\mid$ & N \\
%VP & $\rightarrow$ & [V \{PP & $\mid$ & NP\}*] & $\mid$ & V \\
%   PP & $\rightarrow$ & [P, VP] & $\mid$ & [P, NP] \\
%\end{tabular}\\
%\newline

%\begin{tabular}{ l  l  c  c  c  c  c  c  c}
%S & $\rightarrow$ & [V , XP] $\mid$ V \\
%XP & $\rightarrow$ & \{ NP $\mid$  VP $\mid$ PP \}* \\
%   NP & $\rightarrow$ & [N \{XP\}*] $\mid$ N \\
%VP & $\rightarrow$ & [V \{PP $\mid$ NP\}*] $\mid$ V \\
%   PP & $\rightarrow$ & [P, VP] $\mid$ [P, NP] \\
%\end{tabular}\\
%\newline

Die erste Regel besagt, dass S entweder durch V und XP ersetzt wird oder nur durch V. Der vertikale Disjunktionsstrich $\mid$ denotiert dabei die ,entweder-oder‘-Beziehung, die eckigen Klammern sollen lediglich zeigen, dass V und XP als Einheit zusammengefasst werden und den ersten Teil der Disjunktion bilden. Die Reihenfolge von V und XP wird durch das zwischen diesen beiden Elementen stehende Komma offen gelassen; dieses Komma fungiert hier als sogenannter ,shuffle operator‘. Stünde das Komma nicht dort, müsste XP stets V folgen. V bezeichnet ein finites Verbalelement, XP wird in der Zeile darunter definiert als eine Menge an Nominalphrasen (NP), Verbalphrasen (VP) und Präpositionalphrasen (PP). Die geschweiften Klammern in Zeile zwei umschließen eine Menge. Der Asterisk ist ein Kleene-Stern, der die vorangehende Menge zu einer Kleenschen Hülle um definiert; in einer solchen Kleenschen Hülle können die einzelnen Elemente beliebig oft -- also zum Beispiel auch überhaupt nicht -- und in beliebiger Reihenfolge vorkommen.\footnote{In der Menge werden hier Disjunktionsstriche verwendet anstatt wie oft üblich Kommata, da das Komma in diesem Kontext als shuffle operator definiert wurde; vgl. \cite[26]{Skript};\cite[7??14??]{Snijders}.} Zusammengenommen erlauben die Regeln der ersten und zweiten Zeile also, dass ein Satz beispielsweise aus einer Nominalphrase, einem finiten Verbalelement V und einer weiteren Nominalphrase besteht. Die Tatsache, dass diese nur eine von vielen möglichen Auflösungen von S ist, wird wiederum der weitgehend freien Wortstellung des Lateinischen geschuldet.\footnote{Vgl. \cite[19]{Rohrer}.}

In den Zeilen 3 bis 5 werden die jeweiligen Expansionsmöglichkeiten von NP, VP und PP beschrieben. Eine Nominalphrase besteht also im Allgemeinen entweder aus einem lexikalischen Element der Kategorie N – d.h. aus einem Nomen -- und einer beliebigen, ungeordneten Menge an NPs, VPs und PPs, oder aber nur aus einem Nomen.  Die Notation der Expansionsmöglichkeiten einer präpositionale Phrase enthält keine Kleensche Hülle, da in einer präpositionale Phrase stets zumindest ein Element einer VP oder NP direkt nach (bzw. vor) einer Präposition (bzw. Postposition) stehen muss.\footnote{Diese Regeln erheben keinen Anspruch auf Vollständigkeit. So werden beispielsweise die im Lateinischen häufig auftretenden diskontinuierlichen Phrasen aus Gründen der Relevanz für diese Arbeit außer acht gelassen; mehr Informationen hierzu finden sich bei \cite{Snijders}. Ebenso werden, abgesehen von den präpositionale Phrasen, Fälle, in denen auch im Lateinischen die Wortstellung eine Rolle spielt, wie beispielsweise beim konzessiven Konjunktiv \textbf{(?) und ?? (Geiger Dokumente)}, nicht beachtet. Auch die formal schwierige Frage der Koordination kann hier nicht berücksichtigt werden.}

Zu beachten ist, dass Partizipialien – d.h. Partizipien, Infinitve und Gerundialien -- in dieser Arbeit als V bezeichnet werden. Dies macht insofern Sinn, dass die Prädikate der Partizipien die übrigen Argumente der Partizipialkonstruktion fordern. Im Lateinischen ist V stets der Kopf von S; da die Partizipialkonstruktionen im Lateinischen satzwertig sind, ist die Bezeichnung der Partizipien als V angemessen.\footnote{Allerdings werden in den späteren Ausführungen die Partizipialkonstruktionen -- abgesehen vom Ablativus absolutus -- nicht explizit als S bezeichnet, um ihre Zugehörigkeit zum Rest des Satzes anzuzeigen.}\\
\textbf{WAS WAREN DIE GRÜNDE DAGEGEN?! einfachere Syntaxregeln, nominale Eigenschaften der Partizipien (durch Bezeichnung nicht gekennzeichnet)}

Allerdings sind die Syntaxregeln so wie hier dargestellt noch nicht komplett; es fehlen die Annotationen der grammatikalischen Funktionen, die ein Konstituent an der jeweiligen Position einnehmen kann. Eine der Stärken der LFG besteht darin, dass durch diese funktionalen Annotationen die Menge der Syntaxregeln stark reduziert wird. In der LFG wird ein universelles Inventar an grammatikalischen Funktionen angenommen; dieses umfasst im Allgemeinen Subjekt (SUBJ), Objekt (OBJ), thematisch restringiertes Objekt (OBJ\textsubscript{$\theta$}), Oblique (OBL\textsubscript{$\theta$}), Komplement (COMP) und Adjunkt (ADJ).\footnote{Vgl. \cite[9]{Dal}. Neben den grammatikalischen Funktionen gibt es auch Diskursfunktionen; vgl. \cite[28; 76-84; 94-101]{Skript}. Auf diese wird in dieser Arbeit nicht eingegangen.} Diese Funktionen können auf verschiedene Art und Weise nach bestimmten Gemeinsamkeiten bzw. Unterschieden klassifiziert werden.\footnote{Vgl. \cite[56-8]{Falk}.} 

Ein wichtiger Unterschied besteht zwischen den vom Prädikat regierbaren Funktionen einerseits, und ADJ und XADJ andererseits. \footnote{Vgl. \cite[56]{Falk}.} Wie bei der Besprechung der Lexikoneinträge deutlich werden wird, fordert das Prädikat seine Argumente; daher werden die regierbaren Funktionen auch Argument-Funktionen (ARG-GF oder AF) genannt.\footnote{Vgl. \cite[28]{Skript}; \cite[58]{Falk}.} Da Adjunkte keinerlei syntaktische Wichtigkeit für die Grammatikalität eines Satzes besitzen, werden sie nicht vom Prädikat gefordert; sie können als Adjunkt-Funktionen (ADJ-GF) bezeichnet werden.\footnote{Vgl. \cite[10-1]{Dal}; \cite[38]{Skript}.} Attributive Adjektive und Adverbien beispielsweise treten immer in der Funktion eines ADJ auf, Präpositionalphrasen häufig.\footnote{Vgl. \cite[38]{Skript}. Mehr Informationen zu Adjunkten finden sich beispielsweise in \cite[61-2]{Falk}.} Diese grundlegende Unterscheidung der grammatikalischen Funktionen in Argument- und Adjunkt-Funktionen können folgendermaßen formelhaft dargestellt werden:\footnote{Die Element-Zeichen werden bei der Betrachtung der f-Struktur erklärt.}

\begin{singlespace}
\begin{tabular}{ l  l  l  c  c  c  c }
GF & $\equiv$ & \{ARG-GF $\mid$ ADJ-GF\} \\
ARG-GF & $\equiv$ & \{SUBJ $\mid$ OBJ $\mid$\ OBJ\textsubscript{$\theta$} $\mid$ OBL\textsubscript{$\theta$} $\mid$ OBL\textsubscript{$\theta$} OBJ $\mid$ ADJ $\in$ OBJ\} \\
ADJ-GF & $\equiv$ & \{ADJ $\in$ $\mid$ XADJ $\in$\} \\
\end{tabular}\\
\end{singlespace}

Weiter ist den Funktionen COMP, XCOMP und XADJ eigen, dass sie stets satzwertige Funktionen sind;\footnote{Vgl. \cite[24]{Dal}, ,,clausal functions‘‘.} Adjunkte können, müssen jedoch nicht satzwertige Konstruktionen sein. \footnote{vgl. \cite[40]{Skript}.} Während ADJ und COMP geschlossene Funktionen sind, können XADJ und XCOMP insofern ,offen‘ genannt werden, da sie kein internes Subjekt enthalten; dies wird bei der Betrachtung der f-Struktur verständlicher werden.\footnote{Vgl. \cite[10; 14; 24]{Dal}.}

Ein weiterer Unterschied besteht zwischen den semantisch unrestringierten Funktionen (\textbf{nochmal nachgucken bei Dalrymple welche das alles sind, S. 10; 15-17!}) einerseits und OBJ\textsubscript{$\theta$} und OBL\textsubscript{$\theta$} andererseits. Sprachen erlauben in aller Regel nur ein thematisch unrestringiertes Objekt, jedoch ein oder mehrere thematisch beschränkte.\footnote{Vgl. \cite[21]{Dal}.} Während OBJ im Lateinischen das direkte Objekt bezeichnet,\footnote{Der Kasus des direkten Objekts ist im Lateinischen vom Verb abhängig und muss im Lexikoneintrag des entsprechenden Verbs festgelegt werden. Vgl. \cite[30]{Skript}.} wird OBJ\textsubscript{$\theta$} für das indirekte Objekt verwendet – beispielsweise ,tibi‘ in ,dono tibi librum‘, wobei ,librum‘ das direkte Objekt ist (??).\footnote{\cite[30]{Skript} bestätigt diese Zuteilung für das Deutsche. Dort finden sich auch nähere Erklärungen hierzu. Welche Verben ein indirektes Objekt zu sich nehmen muss ebenfalls in den jeweiligen Lexikoneinträgen festgelegt werden; die Hinzunahme eines indirekten Objekts muss auch in den c-Struktur-Annotationen erlaubt sein.}  Der Index $\theta$ wird in der konkreten Verwendung durch eine Abkürzung der semantischen Rolle ersetzt,\footnote{Vgl. \cite[32]{Skript}; \cite[21]{Rohrer}.} im obigen Beispiel also durch ,rec‘, für ,receiver‘. Oblique-Argumente zeigen ihre semantische Rolle stets an;\footnote{Vgl. \cite[26]{Dal}.} Präpositionale Phrasen beispielsweise erfüllen häufig die Funktionen OBL\textsubscript{goal} oder OBL\textsubscript{loc}.

Unter Einbeziehung der grammatikalischen Funktionen sähen die Syntaxregeln etwa folgendermaßen aus:
%\textbf{Variante 1 - Disjunktivität beachtet (in Anlehnung an Snijder}
%\begin{singlespace}
%\begin{tabular}{ l  l  c  c  c  c  c  c  c}
%S & $\rightarrow$ & [\{V & $\mid$ & VP & $\mid$ & NP & $\mid$ & PP \}*] ,\\
 % & $\qquad$ & \textsuperscript{$\uparrow$ = $\downarrow$} & & \textsuperscript{($\uparrow$ARG-GF) = $\downarrow$} & & \textsuperscript{($\uparrow$ARG-GF) = $\downarrow$} & & \textsuperscript{($\uparrow$OBL$\theta$ $\mid$  ADJ-GF) = $\downarrow$} \\
  %$\qquad$ & $\qquad$ & [ (XP) & (\{ V & $\mid$ & NP \})& (XP)]* \\
  %& $\qquad$ & \textsuperscript{($\uparrow$ADJ-GF) = $\downarrow$} & \textsuperscript{$\uparrow$ = $\downarrow$} & & \textsuperscript{($\uparrow$ARG-GF) = $\downarrow$} & \textsuperscript{($\uparrow$ADJ-GF) = $\downarrow$} \\
 % XP & $\rightarrow$ & \{ NP, &  VP, & PP \}* \\
%\end{tabular}\\
%\newline
%\newline
%\end{singlespace}

\textbf{mit zusätzlichen Beschränkungen}
\begin{singlespace}
\begin{tabular}{ l  l  c  c  c  c  c  c  c}
S & $\rightarrow$ & V, & XP $\mid$ & V \\
   & $\qquad$ & \textsuperscript{$\uparrow$ = $\downarrow$} & \textsuperscript{($\uparrow$ADJ-GF) = $\downarrow$ $\mid$ ($\uparrow$ARG-GF) = $\downarrow$} &  \textsuperscript{$\uparrow$ = $\downarrow$} \\
      & $\qquad$ & \textsuperscript{($\downarrow$FIN) = +} & &  \textsuperscript{($\downarrow$FIN)= +} \\
%XP & $\rightarrow$ & \{ NP & , & VP & , & PP \}* \\
XP & $\rightarrow$ & \{ NP $\mid$ &  VP $\mid$ & PP \}* \\
   & $\qquad$ & \textsuperscript{\{ ($\uparrow$SUBJ) = $\downarrow$} &\textsuperscript{\{ ($\uparrow$SUBJ) = $\downarrow$} & \textsuperscript{\{ ($\uparrow$ADJ) = $\downarrow$} \\
    & $\qquad$ & \textsuperscript{($\downarrow$CASE) = nom} & \textsuperscript{($\downarrow$FIN) = -} &  \\
    & $\qquad$ & \textsuperscript{$\mid$ ($\uparrow$OBJ) = $\downarrow$} & \textsuperscript{$\mid$ ($\uparrow$OBJ) = $\downarrow$} &  \textsuperscript{$\mid$ ($\uparrow$OBL\textsubscript{$\theta$}) = $\downarrow$ \}} \\
       & $\qquad$ & \textsuperscript{$\mid$ ($\uparrow$OBL\textsubscript{$\theta$}) = $\downarrow$} & \textsuperscript{$\mid$ ($\uparrow$OBL\textsubscript{$\theta$}) = $\downarrow$} & \\
         & $\qquad$ & \textsuperscript{$\mid$ ($\uparrow$ADJ) = $\downarrow$} & \textsuperscript{$\mid$ ($\uparrow$ADJ) = $\downarrow$} &  \\
           & $\qquad$ & \textsuperscript{$\mid$ ($\uparrow$COMP) = $\downarrow$ \}} & \textsuperscript{$\mid$ ($\uparrow$COMP) = $\downarrow$} &  \\
             & $\qquad$ & & \textsuperscript{$\mid$ ($\uparrow$XADJ) = $\downarrow$} &  \\
               & $\qquad$ & & \textsuperscript{$\mid$ ($\uparrow$XCOMP) = $\downarrow$ \} } &  \\
%   NP & $\rightarrow$ & N &  \{XP\}* $\mid$ & N \\
%   & $\qquad$ & \textsuperscript{$\uparrow$ = $\downarrow$}  & \textsuperscript{($\uparrow$ADJ-GF) = $\downarrow$ $\mid$ ($\uparrow$ARG-GF) = $\downarrow$} & \textsuperscript{$\uparrow$ = $\downarrow$}\\
%VP & $\rightarrow$ & V  & \{PP $\mid$     & NP\}* $\mid$ & V \\
%   & $\qquad$ & \textsuperscript{$\uparrow$ = $\downarrow$} & \textsuperscript{($\uparrow$ADJ-GF) = $\downarrow$ $\mid$ ($\uparrow$ARG-GF) = $\downarrow$} & \textsuperscript{($\uparrow$ADJ-GF) = $\downarrow$ $\mid$ ($\uparrow$ARG-GF) = $\downarrow$} & \textsuperscript{$\uparrow$ = $\downarrow$}\\
%   PP & $\rightarrow$ & P, & VP $\mid$  & P,   & NP \\
%   & $\qquad$ & \textsuperscript{$\uparrow$ = $\downarrow$} & \textsuperscript{($\uparrow$ADJ-GF) = $\downarrow$ $\mid$ ($\uparrow$ARG-GF) = $\downarrow$} & \textsuperscript{$\uparrow$ = $\downarrow$} & \textsuperscript{($\uparrow$ADJ-GF) = $\downarrow$ $\mid$ ($\uparrow$ARG-GF) = $\downarrow$}\\
%PP & $\rightarrow$ & P &  \{VP $\mid$     & NP\}* \\
 %  & $\qquad$ & \textsuperscript{$\uparrow$ = $\downarrow$} & \textsuperscript{($\uparrow$ADJ-GF) = $\downarrow$ $\mid$ ($\uparrow$ARG-GF) = $\downarrow$} & \textsuperscript{($\uparrow$ADJ-GF) = $\downarrow$ $\mid$ ($\uparrow$ARG-GF) = $\downarrow$}\\
\end{tabular}\\
\newline
\end{singlespace}

Diese funktionalen Annotationen zeigen beispielsweise an, welche grammatikalischen Funktionen eine NP, die direkt von S dominiert wird, einnehmen kann; sie besagen auch, dass ihr Kasus, wenn sie an dieser Position als Subjekt auftritt, Nominativ sein muss. Die Pfeile werden bei der Betrachtung der c-Struktur verständlich werden; auf die Annotation ($\downarrow$FIN) = + wird im Abschnitt ,,Redundanz- und Defaultregeln'' näher eingegangen werden. Diese Annotationen sind hier jedoch nur exemplarisch dargestellt, da die erschöpfende Notation aller möglichen grammatikalischen Funktionen für jeden Konstituenten den Rahmen dieser Arbeit bei Weitem überschreiten würde. So müsste z.B. die XP in den obigen Regeln konkret in NP, VP und PP getrennt dargestellt werden, damit gezeigt werden könnte, dass eine NP, die direkt von S dominiert wird, andere grammatikalische Funktionen erfüllen kann als eine NP, die von einer anderen NP, VP oder PP dominiert wird; so kann beispielsweise das Subjekt einer PC-VP in allen Kasus vorkommen, nicht nur im Nominativ.\footnote{Vgl. \cite[48]{Skript},,Die erstere Verfahrensweise, bei der die morphologischen Eigenschaften der Nominalphrasen Bezugspunkt der GF-Bestimmung sind, wird als lexozentrische Funktionsassoziierung bezeichnet, sie ist namentlich in Bresnan (2001) ausgeführt. Von lexozentrische Funktionsassoziierung wird weiterhin auch dann gesprochen, wenn die Grammatische Funktion durch morphologische Kongruenzmerkmale bestimmt wird. Beide Varianten können miteinander kombiniert sein, wie das im Deutschen mit Rücksicht auf das Subjekt der Fall ist. ... Wenn wir nun die Funktion aus dem Kasus folgern sollen, dann können wir dies durch ein Konditional ausdrücken''}

%\textbf{Variante 2 - alt}
%\begin{singlespace}
%\begin{tabular}{ l  l  c  c  c  c  c  c  c}
%S & $\rightarrow$ & V, & XP $\mid$ & V \\
%   & $\qquad$ & \textsuperscript{$\uparrow$ = $\downarrow$} & \textsuperscript{($\uparrow$ADJ-GF) = $\downarrow$ $\mid$ ($\uparrow$ARG-GF) = $\downarrow$} &  \textsuperscript{$\uparrow$ = $\downarrow$} \\
%      & $\qquad$ & \textsuperscript{($\uparrow$FIN)} & &  \textsuperscript{($\uparrow$FIN)}\\
%XP & $\rightarrow$ & \{ NP & , & VP & , & PP \}* \\
%XP & $\rightarrow$ & \{ NP $\mid$ &  VP $\mid$ & PP \}* \\
%   & $\qquad$ & \textsuperscript{($\uparrow$ADJ-GF) = $\downarrow$ $\mid$ ($\uparrow$ARG-GF) = $\downarrow$} &\textsuperscript{($\uparrow$ADJ-GF) = $\downarrow$ $\mid$ ($\uparrow$ARG-GF) = $\downarrow$} & \textsuperscript{($\uparrow$ADJ-GF) = $\downarrow$ $\mid$ ($\uparrow$ARG-GF) = $\downarrow$} \\
%   NP & $\rightarrow$ & N &  \{XP\}* $\mid$ & N \\
%   & $\qquad$ & \textsuperscript{$\uparrow$ = $\downarrow$}  & \textsuperscript{($\uparrow$ADJ-GF) = $\downarrow$ $\mid$ ($\uparrow$ARG-GF) = $\downarrow$} & \textsuperscript{$\uparrow$ = $\downarrow$}\\
%VP & $\rightarrow$ & V  & \{PP $\mid$     & NP\}* $\mid$ & V \\
%   & $\qquad$ & \textsuperscript{$\uparrow$ = $\downarrow$} & \textsuperscript{($\uparrow$ADJ-GF) = $\downarrow$ $\mid$ ($\uparrow$ARG-GF) = $\downarrow$} & \textsuperscript{($\uparrow$ADJ-GF) = $\downarrow$ $\mid$ ($\uparrow$ARG-GF) = $\downarrow$} & \textsuperscript{$\uparrow$ = $\downarrow$}\\
%   PP & $\rightarrow$ & P, & VP $\mid$  & P,   & NP \\
%   & $\qquad$ & \textsuperscript{$\uparrow$ = $\downarrow$} & \textsuperscript{($\uparrow$ADJ-GF) = $\downarrow$ $\mid$ ($\uparrow$ARG-GF) = $\downarrow$} & \textsuperscript{$\uparrow$ = $\downarrow$} & \textsuperscript{($\uparrow$ADJ-GF) = $\downarrow$ $\mid$ ($\uparrow$ARG-GF) = $\downarrow$}\\
%PP & $\rightarrow$ & P &  \{VP $\mid$     & NP\}* \\
 %  & $\qquad$ & \textsuperscript{$\uparrow$ = $\downarrow$} & \textsuperscript{($\uparrow$ADJ-GF) = $\downarrow$ $\mid$ ($\uparrow$ARG-GF) = $\downarrow$} & \textsuperscript{($\uparrow$ADJ-GF) = $\downarrow$ $\mid$ ($\uparrow$ARG-GF) = $\downarrow$}\\
%\end{tabular}\\
%\end{singlespace}



%\subsubsection{HÄUFIGE SYNTAXREGELN - evt lieber ein Beispiel beim obj-Part?}
%*Häufiger finden sich in dieser Arbeit die folgenden Auflösungen

%\begin{singlespace}
%\begin{tabular}{ l  l  c  c  c  c }
%    NP & $\rightarrow$ & N \\
%   & $\qquad$ & \textsuperscript{$\uparrow$ = $\downarrow$} \\
 %   VP & $\rightarrow$ & PP & V & \\
  % & $\qquad$ & \textsuperscript{($\uparrow$OBL\textsubscript{$\theta$}) = $\downarrow$ } & \textsuperscript{$\uparrow$ = $\downarrow$} \\
   %		 PP & $\rightarrow$ & P & NP \\
%   & $\qquad$ & \textsuperscript{$\uparrow$ = $\downarrow$} & \textsuperscript{($\uparrow$OBJ) = $\downarrow$} \\
  %     NP & $\rightarrow$ & N & VP \\
  % & $\qquad$ & \textsuperscript{$\uparrow$ = $\downarrow$} & \textsuperscript{$\downarrow$ $\in$ ($\uparrow$XADJ)} \\
	%	    VP & $\rightarrow$ & V & NP \\
%   & $\qquad$ & \textsuperscript{$\uparrow$ = $\downarrow$} & \textsuperscript{($\uparrow$OBL\textsubscript{$\theta$}) = $\downarrow$ }  \\
%		    VP & $\rightarrow$ & NP & V \\
 %  & $\qquad$ & \textsuperscript{($\uparrow$OBJ) = $\downarrow$} & \textsuperscript{$\uparrow$ = $\downarrow$} \\
	%	    PP & $\rightarrow$ & P & VP \\
%   & $\qquad$ & \textsuperscript{$\uparrow$ = $\downarrow$} & \textsuperscript{($\uparrow$OBJ) = $\downarrow$} \\
%	    VP & $\rightarrow$ &  NP& V \\
 %  & $\qquad$ & \textsuperscript{($\uparrow$SUBJ) = $\downarrow$} &\textsuperscript{$\uparrow$ = $\downarrow$} \\
  % S\textsubscript{part} & $\rightarrow$ & NP & V'\\
  % & \textsuperscript{$\qquad$} & \textsuperscript{($\uparrow$SUBJ) = $\downarrow$} & \textsuperscript{$\uparrow$ = $\downarrow$} \\

  %S & $\rightarrow$ & NP\textsubscript{1} & VP & V\\
   %& $\qquad$ & \textsuperscript{($\uparrow$OBJ) = $\downarrow$} & \textsuperscript{($\uparrow$XCOMP) = $\downarrow$} & \textsuperscript{$\uparrow$ = $\downarrow$} \\
   % NP\textsubscript{1} & $\rightarrow$ & N \\
%\end{tabular} 
%\newline
%\end{singlespace}


%S $\rightarrow$ NP \, VP \: XP\\

%S $\rightarrow$ NP \, VP \: V\\

%S\textsubscript{part} $\rightarrow$ NP \: V'\\

%S $\rightarrow$ NP \, VP \: V\\

%S $\rightarrow$ NP \, VP \: V\\

%($\uparrow$OBJ) = $\downarrow$\\

\subsubsection{Allgemeines zur c-Struktur}

Bei der Erzeugung der c-Struktur dienen die Syntaxregeln dazu, Dominanzverhältnisse -- und oftmals, jedoch nicht hier, auch Präzedenzverhältnisse -- zwischen den Konstituenten herzustellen.\footnote{Vgl. \cite[19]{Rohrer}.} Jedes hier aufgeführte lexikalische Element und jede Phrase kann einen Konstituenten in der c-Struktur bezeichnen. 

Aus den Syntaxregeln (und dem Lexikon???) kann die c-Struktur erzeugt werden. (vgl.: ,,Die Baumstruktur (1) kann durch eine Grammatik erzeugt werden, die aus einer Menge von Ersetzungsregeln und einem Lexikon besteht‘‘)\footnote{Vgl. \cite[6]{Skript}.} In der LFG repräsentiert die c-Struktur primär die oberflächliche, konkrete Konfiguration der Satz-Konstituenten.\footnote{Vgl. \cite[47]{Dal}. Überlegungen über Kriterien für Konstituenten finden sich in \cite[48-9]{Dal}.} In ihr ist sowohl die hierarchische Dominanz der Konstituenten als auch die lineare Reihenfolge der lexikalischen Elemente sichtbar.\footnote{Vgl. \cite[7]{Dal}; \cite[13]{Rohrer}.} Diese lexikalischen Elemente befinden sich auf der untersten Ebene der c-Struktur -- sie bilden sozusagen die Blätter des Syntaxbaums.\footnote{Vgl. \cite[7]{Dal}. Hier können nur einzelne, vollständige Wörter stehen, d.h. keine Phrasen oder Affixe.} \textbf{(wird im Skript anders gesehen:,,gehen durch Ersetzung eines terminalen Knotens gleicher Kategorie in die C-Struktur ein‘‘\footnote{Vgl. \cite[63]{Skript}.}} Direkt über ihnen steht ein terminaler c-Struktur-Knoten gleicher lexikalischer (?) Kategorie.\footnote{das ein lexikalisch Element von mehreren c-Struktur-Knoten dominiert wird, ist ausgeschlossen; vgl. \cite[63]{Skript}.}

Oberhalb der terminalen Knoten, die aus lexikalischen\footnote{Es existieren auch funktionale Kategorien mit funktionalen Köpfen; auf diese wird in dieser Arbeit jedoch nicht eingegangen.} Kategorien bestehen, finden sich eine bis mehrere Ebenen nicht-terminaler Knoten, welche aus den Konstituenten-Kategorien – S, NP, VP, etc. -- bestehen. Diese Konstituenten-Kategorien werden als Projektionen der lexikalischen Kategorien angesehen; das Wort, das dabei die beherrschende Rolle spielt \textbf{(semantisch oder syntatktisch??)} wird als Kopf dieses Konstituenten bezeichnet.\footnote{Vgl. \cite[13; 15]{Rohrer}.} So ist beispielsweise eine VP also eine Projektion des Verbs (V), wobei V der Kopf der Phrase ist. S kann ebenfalls als Projektion des Verbs betrachtet werden, jedoch auf einer noch höheren Ebene als die VP.\footnote{Vgl. \cite[15]{Rohrer}.}

-->	Zwischenprojektionen

\textbf{überprüfen Rohrer S. 15: terminale Symbole und so, Fußnote! : ,,Die K - Strukturen sind Baumgraphen mit beschrifteten Knoten. Die Symbole, durch die die Knoten benannt werden, zerfallen in zwei Klassen, nämlich die terminalen und die nicht - terminalen Symbole. Die terminalen Symbole sind die Wortformen, gegebenenfalls auch die Morpheme einer natürlichen Sprache. Sie sind im Lexikon aufgelistet. Die nicht-terminalen Symbole zerfallen wiederum in zwei Klassen, die lexikalischen Kategorien und die Konstituentenkategorien. Die lexikalischen Kategorien klassifizieren unmittelbar die Terminalsymbole. Sie entsprechen weitgehend den herkömmlichen Wortarten. folgende lexikalischen Kategorien brauchen: Nomen (N), Verb (V), (Rohrer S. 14) Präposition (P), Adjektiv (A), Adverb (ADV)''(Rohrer S. 15) }


\begin{singlespace}
\Tree [.S 
		[\qroof{Caesar}.{NP\textsubscript{($\uparrow$SUBJ)=$\downarrow$}} ] 
		[\qroof{barbaros}.{NP\textsubscript{($\uparrow$OBJ)=$\downarrow$}} ]
		[.VP{\textsubscript{$\uparrow$=$\downarrow$}}
			[\qroof{in Gallia}.PP\textsubscript{$\downarrow$ $\in$($\uparrow$ADJ)} ]
			[.V\textsubscript{$\uparrow$=$\downarrow$} vicit ]						
		] 	
	]
\end{singlespace}

\textit{Caesar barbaros in Gallia vicit.}

\subsubsection{Allgemeines zu den Lexikoneinträgen}

\textbf{BEI LEXIKONEINTRÄGEN UND EINSCHRÄNKUNGEN WEGEN PFEILEN GUCKEN!!! (vgl. Rohrer S. 54) --> muss manchmal nach unten zeigen!}

Neben den funktionalen Annotationen der c-Struktur werden Informationen aus den Lexikoneinträgen benötigt, um die f-Struktur aufzubauen.\footnote{Vgl. \cite[63]{Skript}.} Da im Lateinischen im Gegensatz zu den modernen Sprachen die Wortstellung innerhalb eines Satzes nicht explizit festgelegt ist,\footnote{Die gewöhnliche Wortstellung im Lateinischen ist zwar Subjekt – Objekt – Prädikat, jedoch wird diese, vor allem aus Gründen der Betonung und des Wohlklangs, nur selten streng eingehalten. Vgl. \cite[397 §212]{LHS} (S. 397, § 212).} muss der Großteil dieser Bedingungen nicht wie üblicherweise in den Syntaxregeln, sondern im Lexikoneintrag festgelegt werden. Das Lexikon der LFG listet, anders als das anderer Grammatiktheorien, nicht nur Ausnahmen auf, sondern stellt einen grundlegenden Bestandteil der Theorie dar, der für die Analyse bzw. Erzeugung eines jeden Satzes vonnöten ist.\footnote{Vgl. \cite[3]{Dal}.} Jede einzelne Flexionsform eines Wortes erhält ihren eigenen Lexikoneintrag.\footnote{Tatsächlich werden im Lexikon auch die systematischen Beziehungen zwischen den lexikalischen Elementen/Lexemen durch Regeln zur morphologischen Umformung festgehalten (vgl. \cite[3]{Dal}). Vor allem die regelmäßigen Formen werden im Normalfall von einem Computerprogramm erzeugt (vgl. \cite[15]{Rohrer}). Bei \cite[63-76]{Skript} und \cite[20-21]{Rohrer} finden sich weitere Erklärungen zu diesen lexikalischen Regeln. Im Rahmen dieser Arbeit werden lediglich einige Beispiel-Lexikoneinträge dargestellt.}


% ``Zugleich unterliegen die möglichen natürlichen Sprachen gewissen Einschränkungen was die mögliche Verkettung von Wörtern zu Sätzen betrifft." (Skript S. 4)

%Sie sollen zunächst für das PC, den Abl. abs. und den AcP als allgemeine Einschränkungen für die jeweiligen Partizipien definiert werden.

Jeder Lexikoneintrag beinhaltet also die schriftliche bzw. lautliche Form des Wortes -- die gewöhnlich mit ,,PRED‘‘, für Prädikat, bezeichnet wird --, seine lexikalische Kategorie sowie diverse funktionale Spezifikationen/Schemata (oben erwähnen!)\footnote{Vgl. \cite[27; 33]{Rohrer}; \cite[16]{Skript}.} Die lexikalische Kategorie wird für die Zuordnung von Lexemen zu ihren möglichen terminalen c-Struktur-Knoten benötigt.\footnote{Vgl. \cite[63]{Skript}.} Die funktionalen Bestimmungen finden sich in der späteren f-Struktur wieder. Das Prädikat eines jeden Wortes im Lexikon fordert bestimmte syntaktische Argumente; diese entsprechen in der Valenzgrammatik der Menge der Ergänzungen, die ein Verb zu sich nehmen kann. Sie werden im sogenannten Subkategorisierungsrahmen -- gekennzeichnet durch $\langle$ $\rangle$ -- aufgeführt.\footnote{Vgl. \cite[7]{Dal}; \cite[70]{Skript}; \cite[27]{Rohrer}.} In diesem Subkategorisierungsrahmen können sämtliche grammatikalische Funktionen auftreten; da Adjunkte keine regierbaren Funktionen sind, tauchen sie in den Lexikoneinträgen nicht auf.\footnote{Vgl. \cite[27]{Rohrer}.} Die geforderten Argumente tauchen, sofern ihr Wert definiert werden muss, im Lexikoneintrag des fordernden Prädikats auf. Ferner werden weitere Bestimmungen der Wortform durch zusätzliche Attribut-Wert-Paare definiert.

Bei der Erstellung der f-Struktur werden, im Lateinischen beginnend beim Verb, die Prädikate mitsamt all ihrer Funktionsbestimmungen in die f-Struktur übertragen.\footnote{Vgl. \cite[28]{Rohrer}.} Die Lexikoneinträge liefern somit einerseits große Teile des Inhalts der Strukturen über die Definition von Attribut-Wert-Paaren und schränken andererseits die als grammatisch geltenden Sätze ein.\footnote{Vgl. \cite[63]{Skript}.}  Taucht eine vom Prädikat geforderte grammatikalische Funktion nämlich nicht in der f-Struktur auf, so ist die Struktur unvollständig.\footnote{Vgl. \cite[28]{Rohrer}.} Letzteres wird bei der Besprechung der Prüfung der Wohlgeformtheits-Bedingungen der f-Struktur deutlicher werden (Abschnitt?).
%\textit{Die Lexikoneinträge spezifizieren jedoch nicht die Verbindung konkreter Worte mit ihren grammatikalischen Funktionen; dies geschieht in der c-Struktur. (?? ,,Was die Lexikoneinträge nicht liefern, ist die Verbindung mit den grammatischen Funktionen (SUBJ, OBJ, OBJ2). Dies muss über die K-Strukturen gesteuert werden. Diese müssen durch funktionale Beschreibungen der Form ($\uparrow$SUBJ)=$\downarrow$ angereichert werden (funktionale Annotationen). Diese funktionalen Annotationen werden über die Syntaxregeln eingeführt.‘‘ (GN, Folie 11) )}

Bei den Partizipien umfassen die nötigen Angaben hinsichtlich der konkreten Wortform Kasus, Numerus, Genus, Verbform (,,MOOD'')\footnote{Hierbei leitet sich ,,MOOD'' genaugenommen von ,,Modus'' her. Obwohl unter Modus in der Regel die Unterscheidung Indikativ - Konjunktiv verstanden wird, kann diese Bezeichnung hier 	auch im Zusammenhang mit Partizipien verwendet werden, da sich die Eigenschaften `Partizip' und `Indikativ' bzw. `Konjunktiv' gegenseitig ausschließen.}, d.h. hier stets Partizip (,,PART''), Zeitverhältnis (,,RELTENSE'', abgekürzt für ,,relative tense'') und Diathese (wobei das Attribut ,,PASSIVE'' entweder den Wert ,,+'' oder ,,-'' erhält).\footnote{Das Genus verbi, d.h. die rein morphologische Erscheinung in entweder aktiver oder passiver Form, ergibt sich aus der Grundform -- hier in Anlehnung an gängige lateinische Wörterbücher stets die erste Person Singular Präsens Indikativ -- des Prädikats des Partizips im Lexikoneintrag, wie z.B. \textit{mittor} statt \textit{mitto}. Durch diese Notierung stellen auch Deponentien kein Problem für die LFG dar, deren Diathese aktiv ist, während ihre morphologische Form im Passiv steht.}

%Liste der üblicherweise angenommenen f-strukturellen Merkmale/Argumente zusammen mit den (möglichen) Werten dieser Merkmale. (Dalrymple, S. 27-8)
%das Merkmal Kasus (``CASE‘‘) beispielsweise ist mit bestimmten grammatikalischen Funktionen verbunden (--> so kann ein Subjekt nur den Wert `` Nominativ‘‘ (``nom‘‘) für das Merkmal CASE tragen). Merkmale wie TENSE spezifizieren die morphologische Form eines Arguments. (gekürzt zitiert) (Dalrymple, S. 27)
% (Grund dafür, dass die Subkategorisierung im funktionalen Bereich anstatt auf der Ebene der Oberflächenstruktur stattfindet: Dalrymple, S. 29.
%(Chomsky’s Syntaxbäume + Unterschiede in der LFG) (Dalrymple, S. 47)

\begin{singlespace}
\begin{tabular}{ l  l  l  l  } 
\textbf{moritura}: & V \\
$\qquad$ & $[1]$ \:  ($\uparrow$PRED) & = & `morior$\langle$SUBJ$\rangle$'\\
$\qquad$ & $[2]$ \:  ($\uparrow$SUBJ) & = & \{((XADJ$\uparrow$)GF) $\mid$ ((XCOMP$\uparrow$)GF)\} \textbf{?} \\
$\qquad$ & $[3]$ \:  \{(($\uparrow$SUBJ GEN) & = & f \\ 
$\qquad$ & $[3.1]$ \:  ($\uparrow$SUBJ NUM) & = & sg \\
$\qquad$ & $[3.2]$ \:  ($\uparrow$SUBJ CASE) & = & \{nom $\mid$ abl\} $\mid$\\
$\qquad$ & $[3.3]$ \: (($\uparrow$SUBJ GEN) & = & n \\
$\qquad$ & $[3.4]$ \:  ($\uparrow$SUBJ NUM) & = & pl \\
$\qquad$ & $[3.5]$ \:  ($\uparrow$SUBJ CASE) & = & \{nom $\mid$ acc\} ) \}\\
$\qquad$ & $[4]$ \:  ($\uparrow$MOOD) & = & part\\
$\qquad$ & $[4]$ \:  ($\uparrow$FIN) & = & - \\
$\qquad$ & $[5]$ \:  ($\uparrow$PASSIVE) & = & - \\
$\qquad$ & $[6]$ \:  ($\uparrow$RELTENSE) & = & future \\
$\qquad$ & $[7]$ \:  \{(($\uparrow$GEN) & = & f \\ 
$\qquad$ & $[7.1]$ \:  ($\uparrow$NUM) & = & sg \\
$\qquad$ & $[7.2]$ \:  ($\uparrow$CASE) & = & \{nom $\mid$ abl\} $\mid$\\
$\qquad$ & $[7.3]$ \: (($\uparrow$GEN) & = & n \\
$\qquad$ & $[7.4]$ \:  ($\uparrow$NUM) & = & pl \\
$\qquad$ & $[7.5]$ \:  ($\uparrow$CASE) & = & \{nom $\mid$ acc\} ) \}\\
\end{tabular}
\newline
\newline
\end{singlespace}



Die hier aufgeführten Eigenschaften eines Lexikoneintrages sollen durch die Beschreibung des konkreten Lexikoneintrags des Partizips \textit{moritura} deutlich gemacht werden. Die konkreten Lexikoneinträge finden sich bei der Betrachtung der spezifischen Partizipialkonstruktionen. Dem Prädikat \textit{moritura} ist die grammatikalische Funktion des Subjekts durch den Subkategorisierungsrahmen als Argument zugeordnet. Dies veranschaulicht, dass das Verb morior ein Subjekt bei sich hat, jedoch -- aus semantisch leicht versändlichen Gründen -- keine weiteren Argumente wie beispielsweise ein Objekt.
[3] Das geforderte Subjekt muss mit dem Partizip in Kasus, Numerus und Genus übereinstimmen. Aus diesem Grund kann dieses Subjekt entweder im Nominativ oder Ablativ Singular Femininum oder im Nominativ oder Akkusativ Plural Neutrum stehen. Die Disjunktion wird dabei im Lexikoneintrag mit geschweiften Mengenklammern eingefasst und die einzelnen Glieder werden durch vertikale Striche voneinander getrennt.\footnote{Die Uniqueness-Bedingung wird nicht verletzt, da der vertikale Strich versinnbildlicht, dass nur eines der aufgeführten Glieder ausgewählt werden kann.} Damit zeigt die finite Verbform Beschränkungen für die mögliche Verkettung mit einer Subjekt-NP durch die Beschränkung derer CASE-, NUM- und GEN-Werte an.
[4-7] Durch das Auflisten weiterer Attribute und deren Werte wird die Verbform genau bestimmt: Dabei gibt der Wert ,,part‘‘ des Attributs MOOD an, dass es sich bei moritura um ein Partizip handelt. Da Partizipien keine finiten Verbformen sind, erhält das Attribut ,,FIN'', für finit, den Wert - (negativ). Das Attribut PASSIVE erhält in diesem Fall ebenfalls den Wert - , da die Diathese jedes Deponens aktiv ist,\footnote{Im speziellen Fall des PFA ist auch das morphologische Genus verbi aktiv.} und das Attribut RELTENSE erhält den Wert ,,future‘‘. Diese drei Attribut-Wert-Paare definieren \textit{moritura} somit als PFA. Die Attribute CASE, NUM und GEN komplettieren den Lexikoneintrag: Diese Werte geben an, dass die Verbform entweder im Nominativ oder Ablativ Singular Femininum oder im Nominativ oder Akkusativ Plural Neutrum stehen darf. Somit wird auch die Kasus-Numerus-Genus-Kongruenz zwischen Subjekt und Verbform erfüllt.


•	subject condition: jedes verbale Prädikat muss ein Subjekt haben (nicht sicher, ob das für alle Sprachen gilt) (Dalrymple, S. 19); für Latein ja eigentlich schon? 

- andere Verwendungsmöglichkeiten der Partizipien einschließen können... man könnte vielleicht sagen, dasdas hier für alle Partizipien gilt, die nicht gemeinsam mit einer finiten Verbform verwendet werden oder so? Würde das überhaupt reichen? Wäre das ansonsten evt. wieder ein Argument, die Partizipien als Part und nicht als V anzusehen? (also die Partzipien, die in finiten Verbformen verwendet werden, wie mortuus est oder so,wären ja weiterhin V dann). 

o	Die Komplemente werden ebenfalls subklassifiziert, und zwar nach dem Typ der ihnen entsprechenden Konstituenten. So kann man unterscheiden zwischen nominalen (17), adjektivischen (18), präpositionalen (19), verbalen (20) und satzhaften (21) Komplementen. (Rohrer S. 22) (s. Lex)

%Im Folgenden sollen exemplarische Lexikoneinträge zu den Partizipien \textit{missum} (für das PC in Objektstellung), \textit{missi} (für das PC in Subjektstellung), \textit{victis} (für den Abl. abs.) und \textit{iaecentem} für den AcP aufgeführt werden.

%\subsection{Zeichen}

%$\theta$

%$\mid$

%$\neq$

%$\in$

%$\ni$

%$\vdash$

%$\subset$

%$\ast$

%$\neg$

\subsubsection{Redundanz- bzw. Default-Regeln}
\textbf{BEI LEXIKONEINTRÄGEN UND EINSCHRÄNKUNGEN WEGEN PFEILEN GUCKEN!!! (vgl. Rohrer S. 54) --> muss manchmal nach unten zeigen!}

Um die syntaktische Korrektheit der ausgegebenen Sätze zu gewährleisten, können zusätzlich zu den Lexikoneinträgen weitere Bedingungen festgelegt werden. Dies geschieht durch sogenannte Redundanz- bzw. Default-Regeln, die über dem Lexikon operieren\footnote{Vgl. \cite[23-4]{Rohrer}.} Hier werden diejenigen aufgelistet, die für die vorliegenden Phänomene relevant sind. Auch wenn diese Regeln nicht unbedingt notwendig sind, so können sie unseres Erachtens nach die Erzeugung der c-Struktur aus den Syntaxregeln beschleunigen, die durch die Klassifzierung der Partizipien als V erschwert
wurde.

Anstatt dieser Redundanz- bzw. Default-Regeln könnte man auch zusätzliche Bedingungen den Syntaxregeln anfügen, wie Bresnan und Kaplan beschreiben\footnote{Bresnan und Kaplan 1982, S. 210; zitiert nach \cite[54]{Rohrer}.} und wie in Abschnitt ?? bereits exemplarisch gezeigt wurde. Aufgrund der leichteren Verständlichkeit für den Leser erfolgt die Darstellung hier durch Redundanz- bzw. Default-Regeln.

%Die in dieser Arbeit unter dem Titel ``Einschränkungen'' festgehaltenen Bedingungen sind jedoch nicht Teil der LFG, sondern dienen lediglich dem besseren Verständnis der lateinischen Grammatik, das für diese Arbeit unerlässlich ist.

* subst. Part. schwierig, da das kein BZW braucht (bei Variante OBJ); der Regel sowas hinzufügen wie ,,oder SUBJ = pro und PRON-TYPE mis"... ?? \\
%--> wenn keine existenzielle Gleichung vorhanden ist, die spezifiziert, dass ein Attribut ,,SUBJ'' vorhanden sein muss...
 Attribut SUBJ hat den Wert missing... \\


- ein V, das direkt von S (nicht S-part) dominiert wird, muss immer eine
finite Verbform sein (d.h. muss als Wert von "MOOD" Indikativ, Konjunktiv oder
 Imperativ haben)
- SUBJ CASE dieses finiten V ist immer Nominativ (was eben beim "Subjekt"
(Bezugsnomen) eines Partizips nicht der Fall sein muss)
- ein V, das von einer VP dominiert wird, kann kein finites Verb sein //
muss ein Partizip, Infinitv oder Gerund sein (d.h. muss als Wert von "MOOD"
"part", "inf" oder "gerund" bzw. "gerundiv" haben)
- ein V, das von S-part dominiert wird, muss ein Partizip sein (d.h. muss
als Wert von "MOOD" "part" haben)

dass ein V, das direkt von S dominiert wird, stets ein finites Verbalelement sein muss; das V einer

•	*VPs können nur Partizipialien (Partizipien und Gerundium und Gerundivum) und Infinitive sein! (wird hier so definiert)


\textbf{Konditional-Gleichungen / lexozentrische Funktionsassoziierung /GF-Spezifikation} 

%--> ($\downarrow$CASE) = nom $\Rightarrow$ ($\uparrow$SUBJ) = $\downarrow$ bedeutet, dass WENN der Kasus einer NP Nominativ ist, diese NP die GF des SUBJ erhält.
%--> im Bezug auf das finite Verb würde das bei uns gehen, oder? beim Partizip-Verb ja eben nich
%ABER: wir können diese ,,Konditionale'' bei den Einschränkungen verwenden, glaub ich!
\footnote{Auf weitere Funktionsspezifikationen neben denen der Partizipien bzw. Partizipialkonstruktionen kann im Rahmen dieser Arbeit nicht eingegangen werden. Sie werden für das Deutsche beispielsweise in Skript, S. 48-51 behandelt.}

%Die grammatikalische Funktion einer Partizipialkonstruktion -- und daher des Partizips, welches Kopf jeder Partizipialkonstruktion ist -- ist immer entweder ein XADJ, ein ADJ oder ein XCOMP: \\
Diese Regeln sollen zunächst allgemein anhand von Lexikoneinträgen eines Partizips x definiert werden. Sie beziehen sich daher auf alle Partizipial-VPs, die nicht direkt von S dominiert werden.
Die grammatikalische Funktion jedes Partizips in einer Partizipialkonstruktion,\footnote{Ausgenommen sind hier das dominante Partizip in Abhängigkeit von einer Präposition und je nach Umsetzung das substantivierte Partizip.} welches nicht Teil einer finiten Verbform ist, ist immer entweder ein XADJ, ein ADJ oder ein XCOMP der übergeordneten grammatikalischen Funktion: \\
($\uparrow$GF) = ($\uparrow$XADJ) $\mid$ ($\uparrow$ADJ) $\mid$ ($\uparrow$XCOMP) \\ 
%($\uparrow$GF) = ($\uparrow$XADJ) $\mid$ ((ADJ$\uparrow$)GF) $\mid$ ((XCOMP$\uparrow$)GF) \textbf{???} \\ 

%(ich denk da wir diese Regeln ja immer im Zusammenhang mit Partizipien formulieren, müssen wir nich überall noch ,,part'' einfügen, das ist ja im Endeffekt eh ne Implementierungssache; würde ich so eben auch in ner Fußnote vermerken, um auf der sicheren Seite zu sein...)

--> \footnote{Zwei Besonderheiten / Ausnahmen / Einschränkungen... dieser Regel werden je beim substantivierten und beim dominanten Partizip diskutiert.}

--> subst Part ist wenn tatsächlich substantiviert keine Part-KONSTRUKTION

--> dom Part in Abhängigkeit von Präp ist Ausnahmefall


%PC:
Das Partizip muss in Kasus, Numerus und Genus mit seinem Bezugswort kongruent sei:\footnote{Vgl. KSt S. 771, § 138,5a.}\\
($\uparrow$KNG) = ($\uparrow$SUBJ KNG)\footnote{Diese Darstellung ist verkürzend, um eine weitere disjunktive Menge (?) von Kasus-, Numerus- und Genus- Attributen zu vermeiden, wie sie ansatzweise dargestellt ist in Skript, S. 49 (? kA ob das dasselbe ist...)} \\
bzw. ($\downarrow$KNG) = ($\uparrow$SUBJ KNG) (Pfeile angepasst nach Skript, S. 49, nicht sicher welches stimmt)\\
\textbf{--> aber nur, wenn es ein SUBJ (BZW) gibt! --> einfach keine existenzielle Gleichung machen, sodass es nicht existieren MUSS?} \\

--> PFEILE?!?? was passt sich an was an, Part an BZW?

Da das Bezugswort eines PC in jedem Fall eine grammatikalische Funktion der übergeordneten Struktur ist,\footnote{Vgl. KSt S. 771, § 138,5a.} nimmt das PC die grammatikalische Funktion XADJ an. Das Subjekt eines Partizips in XADJ-Funktion ist somit eine grammatikalische Funktion der übergeordneten grammatikalischen Funktion (in der Regel S): \\
($\uparrow$XADJ SUBJ) = ((GF$\uparrow$)GF) \\
Partizipien in allen Kasus können in einer PC-Konstruktion auftreten, auch wenn Genitiv und Ablativ hierbei seltener vorkommen als die übrigen Kasus. Dies bedeutet, dass das Partizip in einer XADJ-Funktion in jedem Kasus vorkommen kann.

Der Abl. abs. jedoch ist vom Restsatz semantisch und syntaktisch losgelöst, weswegen sein Bezugswort keine grammatikalische Funktion der übergeordneten Struktur (in aller Regel S) sein darf. Dem Partizip eines Abl. abs. kommt daher unumgänglich die Funktion des ADJ zu. Somit kann geschlussfolgert werden, dass ein Partizip, welches im Ablativ steht, neben der Funktion eines XADJ auch die Funktion eines ADJ annehmen kann. \\

Ein AcP kann, wie weiter unten (Abschnitt?) deutlich wird, am besten als XCOMP zur übergeordneten grammatikalischen Funktion beschrieben werden. Partizip und Bezugswort stehen im AcP im Akkusativ. Daraus folgt, dass ein Partizip im Akkusativ sowohl die Funktion eines XADJ als auch die eines XCOMP annehmen kann.\footnote{Vgl. \cite[48]{Skript}.} Diese lexozentrische Funktionsassoziierung kann durch Konditionale ausgedrückt werden.\footnote{Vgl. \cite[48]{Skript}.} \\
%($\downarrow$XADJ CASE) = \{nom $\mid$ gen $\mid$ acc $\mid$ dat $\mid$ abl\} (weglassen) \\
%($\downarrow$ADJ CASE) = abl \\
%($\downarrow$XCOMP CASE) = acc \\
($\downarrow$CASE) = \{nom $\mid$ gen $\mid$ dat\} $\Rightarrow$ ($\uparrow$XADJ)= $\downarrow$ \\
%$\mid$ ($\downarrow$CASE) = gen $\Rightarrow$ ($\uparrow$XADJ)= $\downarrow$ \\
%$\mid$ ($\downarrow$CASE) = dat $\Rightarrow$ ($\uparrow$XADJ)= $\downarrow$ \\
$\mid$ ($\downarrow$CASE) = acc $\Rightarrow$ \{($\uparrow$XADJ) $\mid$ ($\uparrow$XCOMP)\}= $\downarrow$ \\
$\mid$ ($\downarrow$CASE) = abl $\Rightarrow$ \{($\uparrow$XADJ) $\mid$ ($\uparrow$ADJ)\}= $\downarrow$ \\

%ABL. ABS.:
%($\downarrow$ADJ CASE) = abl \\
%Dasselbe trifft auch auf sein Bezugswort zu: \\
%($\downarrow$SUBJ ADJ CASE) = abl \\
%\textbf{oder:} \\
%($\uparrow$CASE) = abl $\Rightarrow$ ($\uparrow$SUBJ CASE) = abl \textbf{?}\\

Wenn das Partizip ein XADJ der übergeordneten grammatikalischen Funktion ist, ist sein Bezugswort eine grammatikalische Funktion dieser dem XADJ übergeordneten Struktur: \\
($\uparrow$SUBJ XADJ) = ((XADJ$\uparrow$)GF) \\

Wenn das Partizip ein XCOMP der übergeordneten grammatikalischen Funktion ist, ist sein Bezugswort das Objekt dieser dem XCOMP übergeordneten Struktur: \\
($\uparrow$SUBJ XCOMP) = ((XCOMP$\uparrow$)OBJ) \\
%\textbf{oder:} \\ ($\uparrow$(GF)V) = ($\uparrow$XCOMP) $\Rightarrow$ ($\uparrow$SUBJ XCOMP) = ((XCOMP$\uparrow$)OBJ) \textbf{?} \\

Wenn das Partizip ein ADJ der übergeordneten grammatikalischen Funktion ist, ist sein Bezugswort keine grammatikalische Funktion der dem ADJ übergeordneten Struktur: \\
($\uparrow$SUBJ ADJ) = ((ADJ$\uparrow$)GF) \\

Da sich diese Arbeit ausschließlich auf das klassische Latein Caesars und Ciceros bezieht, gilt für die folgenden Betrachtungen die Annahme, dass im Abl. abs. kein Partizip Futur Aktiv (PFA) verwendet wird.\footnote{Vgl. KSt. S. 760, § 136,4c oder NM S. 771, § 469.}\\
%$\neg$ ($\uparrow$(RELTENSE ADJ) = future \\
($\uparrow$RELTENSE) = future $\Rightarrow$ ($\uparrow$GF) $\neq$ ($\uparrow$ADJ)\\

Das Partizip ist im AcP meist ein PPA, selten ein PPP. Mit Sicherheit kann daher gesagt werden, dass kein PFA in einem Partizip mit XCOMP-Funktion auftauchen kann: \\
%($\uparrow$XCOMP RELTENSE) = present   \\
($\uparrow$RELTENSE) = future $\Rightarrow$ ($\uparrow$GF) $\neq$ ($\uparrow$XCOMP)\\

(SUBST PART: Variante 1: XADJ:) \\
Falls kein Subjekt erkennbar vorhanden ist, wie beim substantivierten Partizip, ist das ,fehlende' / logische Subjekt ein XADJ zur übergeordneten grammatikalischen Funktion, welche ebenfalls fehlt: \\
($\uparrow$SUBJ\textsubscript{mis}) = ((GF\textsubscript{mis}$\uparrow$)XADJ) \textbf{???} \\
\textbf{oder:} \\
($\downarrow$SUBJ) = mis $\Rightarrow$ ($\uparrow$SUBJ) = ((GF\textsubscript{mis}$\uparrow$)XADJ) \textbf{?}\\

%AcP

%($\downarrow$XCOMP CASE) = acc \\
%($\downarrow$XCOMP SUBJ CASE) = acc \\


\subsubsection{Allgemeines zur f-Struktur}

\section{das Participium coniunctum}
Partizipien können als Vertreter von Adverbialsätzen aufgefasst werden und stehen dabei für Temporal-, Kausal-, Modal-, Kondizional- und Konzessivsätze. Das Partizip ist hierbei mit seinem Bezugswort verbunden, welches in einem der fünf Kasus Bestandteil des Hauptsatzes und gleichzeitig Subjekt des Nebensatzes ist. Partizip und Bezugswort stimmen daher in Kasus, Numerus und Genus überein. Diese Partizipialkonstruktion bezeichnet man als PC, welches sowohl attributiv als auch prädikativ verwendet werden kann.\footnote{Vgl. KSt, S. 766, § 138,1 u. S. 771, § 138,5a; Vgl. NM, S. 715, § 500.} \\
* irgendwo mal erklären, dass wir ``Funktion'' einerseits normalsprachlich und andererseits als LFG-Terminus verwenden...?

* die P'-Knoten haben wir schon vor dem Gespräch mit Geiger fast immer mit Dreieck umgangen, die waren nur in den Syntaxregeln... vielleicht können wir einfach einmal am Anfang irgendwo anmerken, das erste Mal wenn das vorkommt eben, dass ne Präposition und ihr Objekt immer zusammenstehen müssen, und dass wir hier Postpositionen außer acht lassen wegen ihres deutlich geringeren Vorkommens und weil es nich relevant is im Moment, und dass das deswegen hier mit Dreieck abgekürzt wird... und dann können wir auf die Snijder verweisen

* "je konfigurationaler die Sprache, desto mehr nicht-maximale Projektionsknoten (X'-Knoten) sind erforderlich"... wiesoooo??? (hat Geiger gesagt)

\subsection{allgemeine Vorüberlegungen zur Umsetzung in der LFG}
Die Konstruktion des PC erfüllt im Satz immer die syntaktische Funktion des XADJ: \\ 
($\uparrow$XADJ) = $\downarrow$ \\

%\subsection{Einschränkungen}
%Das Partizip muss in Kasus, Numerus und Genus mit seinem Bezugswort kongruent sei:\footnote{Vgl. KSt S. 771, § 138,5a.}\\
%($\uparrow$SUBJ KNG) = ($\uparrow$KNG)\\
%Dieses Bezugswort des Partizips ist eine grammatikalische Funktion der dem XADJ übergeordneten Struktur, und somit Element des Hauptsatzes:\footnote{Vgl. KSt S. 771, § 138,5a.}\\
%($\uparrow$SUBJ) = ((XADJ$\uparrow$)GF) \\

\subsection{das objektabhängige Participium coniunctum}
Beim objektabhänigen PC bezieht sich das Partizip auf das Objekt des Hauptsatzes und stimmt mit diesem in Kasus, Numerus und Genus überein. \\

Beispielsatz:\\
\textit{legatum in Galliam missum Caesar revocat.} \\
Lexikoneintrag wie folgt

\subsubsection{Lexikoneintrag}
\begin{singlespace}
\begin{tabular}{ l  l  l  l  } 
\textbf{missum}: & V \\
$\qquad$ & $[1]$ \:  ($\uparrow$PRED) & = & `mittor$\langle$SUBJ, OBL\textsubscript{GOAL}$\rangle$'\\
$\qquad$ & $[2]$ \:  ($\uparrow$SUBJ) & = & \{((XADJ$\uparrow$)OBJ) $\mid$ ((XCOMP$\uparrow$)GF)\} \\
$\qquad$ & $[3]$ \:  ($\uparrow$SUBJ NUM) & =\textsubscript{c} & sg \\
$\qquad$ & $[3.1]$ \:  \{(($\uparrow$ SUBJ GEN) & =\textsubscript{c} & m \\ 
$\qquad$ & $[3.2]$ \:  ($\uparrow$SUBJ CASE) & =\textsubscript{c} & acc ) $\mid$\\
$\qquad$ & $[3.3]$ \: (($\uparrow$SUBJ GEN) & =\textsubscript{c} & n \\
$\qquad$ & $[3.4]$ \:  ($\uparrow$SUBJ CASE) & =\textsubscript{c} & \{nom $\mid$ acc\} ) \}\\
$\qquad$ & $[4]$ \:  ($\uparrow$MOOD) & = & part\\
$\qquad$ & $[4]$ \:  ($\uparrow$FIN) & = & - \\
$\qquad$ & $[5]$ \:  ($\uparrow$PASSIVE) & = & + \\
$\qquad$ & $[6]$ \:  ($\uparrow$RELTENSE) & = & past \\
$\qquad$ & $[7]$ \:  ($\uparrow$NUM) & = & sg \\
$\qquad$ & $[8]$ \:  \{(($\uparrow$GEN) & = & m \\ 
$\qquad$ & $[8.1]$ \:  ($\uparrow$CASE) & = & acc ) $\mid$\\
$\qquad$ & $[8.2]$ \: (($\uparrow$GEN) & = & n \\
$\qquad$ & $[8.3]$ \:  ($\uparrow$CASE) & = & \{nom $\mid$ acc\} ) \}\\
\end{tabular}
\newline
\newline
\end{singlespace}

%((XADJ$\uparrow$)OBJ) = ($\uparrow$SUBJ) \\

\subsubsection{Syntaxregeln}
*Der Satz ist grammatikalisch korrekt, da die konkreten Syntaxregeln, die hier beispielhaft aufgeschlüsselt sind, sich aus den oben genannten allgemeinen Regeln ergeben.


\begin{singlespace}
\begin{tabular}{ l  l  c  c  c  c }
S & $\rightarrow$ & NP\textsubscript{1} & VP & NP & V\\
   & $\qquad$ & \textsuperscript{($\uparrow$OBJ) = $\downarrow$} & \textsuperscript{$\downarrow$ $\in$ ($\uparrow$XADJ)} & \textsuperscript{($\uparrow$SUBJ) = $\downarrow$} & \textsuperscript{$\uparrow$ = $\downarrow$} \\
    NP & $\rightarrow$ & N \\
   & $\qquad$ & \textsuperscript{$\uparrow$ = $\downarrow$} \\
    VP & $\rightarrow$ & PP & V & \\
   & $\qquad$ & \textsuperscript{($\uparrow$OBL\textsubscript{GOAL}) = $\downarrow$ } & \textsuperscript{$\uparrow$ = $\downarrow$} \\
   		 PP & $\rightarrow$ & P' \\
	& $\qquad$   & \textsuperscript{$\uparrow$ = $\downarrow$} \\
    		P' & $\rightarrow$ & P & NP\textsubscript{3} \\
   & $\qquad$ & \textsuperscript{$\uparrow$ = $\downarrow$} & \textsuperscript{($\uparrow$OBJ) = $\downarrow$} \\
\end{tabular}\\
\end{singlespace}


\subsubsection{c-Struktur}
\begin{singlespace}
\Tree [.S 
		[\qroof{legatum}.{NP\textsubscript{($\uparrow$OBJ)=$\downarrow$}} ] 
		[.VP{\textsubscript{$\downarrow$ $\in$ ($\uparrow$XADJ)}}
				[\qroof{in Galliam}.PP\textsubscript{($\uparrow$OBL\textsubscript{GOAL})=$\downarrow$} ]
				[.V\textsubscript{$\uparrow$=$\downarrow$} missum ]						
		] 
		[\qroof{Caesar}.NP\textsubscript{($\uparrow$SUBJ)=$\downarrow$} ]
		[.V\textsubscript{$\uparrow$=$\downarrow$} revocat ]	
	]
\end{singlespace}

\subsubsection{f-Struktur}
\begin{singlespace}
\begin{avm}

\[ PRED &  \rm ‘revoco \q<SUBJ, OBJ\q>’\\
SUBJ & \[PRED & `Caesar' \\
CASE & nom \\
NUM & sg \\
GEN & m \]\\
OBJ & \[ PRED & `legatus' \\
CASE & acc \\
NUM & sg \\
GEN & m \]\tikzmark{aim} \\
XADJ & \{ \[PRED &  \rm ‘mitto \q<SUBJ, OBL\textsubscript{GOAL}\q>’\\
MOOD & part \\
PASSIVE & + \\
RELTENSE & past \\
CASE & acc \\
NUM & sg \\
GEN & m \\
SUBJ &  \tikzmark{start} \\
OBL\textsubscript{GOAL} & \[``in scholam''\] \]\\
\} &            $\qquad$ \\
TENSE & present \\
NUM & sg \\
PERS & 3 \\
PASSIVE & - \\
MODE & ind \\
\]
\end{avm}
\end{singlespace}

\tikz[remember picture,overlay] 
    \draw[<-] (pic cs:aim) to[out=0,in=0,looseness=3.5]  (pic cs:start);

Die doppelten Anführungszeichen (``'') dienen der Abkürzung einer f-Struktur, deren Attribut-Wert-Paare entweder klar oder für den zu erklärenden Punkt nicht von Bedeutung sind.\footnote{Vgl. Falk, S. 59. Einem verarbeitenden (Parser?) Programm müssten selbstverständlich alle Attribut-Wert-Paare bekannt sein; diese Abkürzung soll lediglich dem Leser zugute kommen.}

\newpage
\subsection{das subjektabhängige Participium coniunctum}
Wie der Name impliziert, bestimmt das Partizip beim subjektabhängigen PC das Subjekt des Hauptsatzes näher und kongruiert mit diesem in Kasus, Numerus und Genus. \\

Beispielsatz:\\
\textit{milites in Galliam missi hostes vicerunt.} \\

\subsubsection{Lexikoneintrag}
\begin{singlespace}
\begin{tabular}{ l  l  l  l  } 
\textbf{missi}: & V \\
$\qquad$ & $[1]$ \:  ($\uparrow$PRED) & = & `mittor$\langle$SUBJ, OBL\textsubscript{GOAL}$\rangle$'\\
$\qquad$ & $[2]$ \:  ($\uparrow$SUBJ) & = & ((XADJ$\uparrow$)GF) \\
$\qquad$ & $[3]$ \:  \{(($\uparrow$SUBJ NUM) & = & pl \\ 
$\qquad$ & $[3.1]$ \:  ($\uparrow$SUBJ CASE) & = & nom \\
$\qquad$ & $[3.2]$ \:  ($\uparrow$SUBJ GEN) & = & m) $\mid$\\
$\qquad$ & $[3.3]$ \:  (($\uparrow$SUBJ NUM) & = & sg \\ 
$\qquad$ & $[3.4]$ \: ($\uparrow$SUBJ CASE) & = & gen \\
$\qquad$ & $[3.5]$ \:  ($\uparrow$SUBJ GEN) & = & \{m $\mid$ n\} ) \} \\
$\qquad$ & $[4]$ \:  ($\uparrow$MOOD) & = & part \\
$\qquad$ & $[4]$ \:  ($\uparrow$FIN) & = & - \\
$\qquad$ & $[5]$ \:  ($\uparrow$PASSIVE) & = & + \\
$\qquad$ & $[6]$ \: ($\uparrow$RELTENSE) & = & past \\
$\qquad$ & $[7]$ \:  \{(($\uparrow$NUM) & = & pl \\ 
$\qquad$ & $[7.1]$ \:  ($\uparrow$CASE) & = & nom \\
$\qquad$ & $[7.2]$ \:  ($\uparrow$GEN) & = & m) $\mid$\\
$\qquad$ & $[7.3]$ \:  (($\uparrow$NUM) & = & sg \\ 
$\qquad$ & $[7.4]$ \: ($\uparrow$CASE) & = & gen \\
$\qquad$ & $[7.5]$ \:  ($\uparrow$GEN) & = & \{m $\mid$ n\} ) \} \\
\end{tabular}
\end{singlespace}

%\subsubsection{Syntaxregeln}
%\begin{singlespace}
%\begin{tabular}{ l  l  c  c  c  c }
%S & $\rightarrow$ & NP\textsubscript{1} & VP & NP\textsubscript{2} & V\\
 %  & $\qquad$ & \textsuperscript{($\uparrow$SUBJ) = $\downarrow$} & \textsuperscript{$\downarrow$ $\in$ ($\uparrow$XADJ)} & \textsuperscript{($\uparrow$OBJ) = $\downarrow$} & \textsuperscript{$\uparrow$ = $\downarrow$} \\
  %  NP\textsubscript{1} & $\rightarrow$ & N \\
   %& $\qquad$ & \textsuperscript{$\uparrow$ = $\downarrow$} \\
    %VP & $\rightarrow$ & V' \\
%   & $\qquad$ & \textsuperscript{$\uparrow$ = $\downarrow$} \\
 % 	  V' & $\rightarrow$ & PP & V & \\
  % & $\qquad$ & \textsuperscript{($\uparrow$OBL\textsubscript{GOAL}) = $\downarrow$ } & \textsuperscript{$\uparrow$ = $\downarrow$} \\
   %		 PP & $\rightarrow$ & P' \\
	%& $\qquad$   & \textsuperscript{$\uparrow$ = $\downarrow$} \\
    %		P' & $\rightarrow$ & P & NP\textsubscript{3} \\
%   & $\qquad$ & \textsuperscript{$\uparrow$ = $\downarrow$} & \textsuperscript{($\uparrow$OBJ) = $\downarrow$} \\
 %		   NP\textsubscript{3} & $\rightarrow$ & N \\
  % & $\qquad$ & \textsuperscript{$\uparrow$ = $\downarrow$} \\
   % NP\textsubscript{2} & $\rightarrow$ & N \\
%   & $\qquad$ & \textsuperscript{$\uparrow$ = $\downarrow$} \\
%\end{tabular}\\
%\end{singlespace}

\subsubsection{c-Struktur}
\begin{singlespace}
\Tree [.S 
		[\qroof{milites}.{NP\textsubscript{($\uparrow$SUBJ)=$\downarrow$}} ] 
		[.VP{\textsubscript{$\downarrow$ $\in$ ($\uparrow$XADJ)}}
				[\qroof{in Galliam}.PP\textsubscript{($\uparrow$OBL\textsubscript{GOAL})=$\downarrow$} ]
				[.V\textsubscript{$\uparrow$=$\downarrow$} missi ]						
		] 
		[\qroof{hostes}.NP\textsubscript{($\uparrow$OBJ)=$\downarrow$} ]
		[.V\textsubscript{$\uparrow$=$\downarrow$} vicerunt ]	
	]
\end{singlespace}

\subsubsection{f-Struktur}
\begin{singlespace}
\begin{avm}
\[ PRED &  \rm ‘vinco \q<SUBJ, OBJ\q>’\\
SUBJ & \[ PRED & `miles' \\
CASE & nom \\
NUM & pl \\
GEN & m \]\tikzmark{meow} \\
XADJ & \{ \[PRED &  \rm ‘mitto \q<SUBJ, OBL\textsubscript{GOAL}\q>’\\
MOOD & part \\
PASSIVE & + \\
RELTENSE & past \\
CASE & nom \\
NUM & pl \\
GEN & m \\
SUBJ &  \tikzmark{objectmeow} \\
OBL\textsubscript{GOAL} & \[``in Galliam''\] \]\\
\} &            $\qquad$ \\
OBJ & \[``hostes'' \]\\
\]
\end{avm}
\tikz[remember picture,overlay] 
    \draw[<-] (pic cs:meow) to[out=0,in=0,looseness=3.5]  (pic cs:objectmeow);
\end{singlespace}

\subsection{Das rein attributive Participium Coniunctum}

Das rein attributive Partizip hat zum \textit{verbum finitum} keinerlei Beziehung, sondern charakterisiert nur sein Bezugswort; es ersetzt somit einen attributiven Relativsatz.\footnote{Vgl. NM, S. 713, § 498.} \\


* da rein attributiv ist es abhängig von NP \\
* XADJ

Beispielsatz:\\
\textit{insulam obiectam portui tenuit.}

\subsubsection{Lexikoneintrag}
\begin{singlespace}
\begin{tabular}{ l  l  l  l  } 
\textbf{obiectam}: &  & V \\
$\qquad$ & $[1]$ \:  ($\uparrow$PRED) & = & `obicior$\langle$SUBJ, OBJ\textsubscript{LOC}$\rangle$'\\
$\qquad$ & $[2]$ \:  ($\uparrow$SUBJ) & = & \{((XADJ$\uparrow$)GF) $\mid$ ((XCOMP$\uparrow$)GF)\} \\
$\qquad$ & $[2.1]$ \:  ($\uparrow$SUBJ NUM) & = & sg \\
$\qquad$ & $[2.2]$ \: ($\uparrow$SUBJ CASE) & = & acc \\
$\qquad$ & $[2.3]$ \: ($\uparrow$SUBJ GEN) & = & f \\
$\qquad$ & $[3]$ \: ($\uparrow$OBJ CASE) & = & dat \\
$\qquad$ & $[4]$ \:  ($\uparrow$MOOD) & = & part\\
$\qquad$ & $[4]$ \:  ($\uparrow$FIN) & = & - \\
$\qquad$ & $[5]$ \:  ($\uparrow$PASSIVE) & = & + \\
$\qquad$ & $[6]$ \:  ($\uparrow$RELTENSE) & = & past \\
$\qquad$ & $[7]$ \:  ($\uparrow$NUM) & = & sg \\
$\qquad$ & $[8]$ \: ($\uparrow$CASE) & = & acc \\
$\qquad$ & $[9]$ \: ($\uparrow$GEN) & = & f \\
\end{tabular}
\newline
\newline
\end{singlespace}

%\subsubsection{Syntaxregeln}
%\begin{singlespace}
%\begin{tabular}{ l  l  c  c  c  c }
%  S & $\rightarrow$ & NP\textsubscript{1} & V\\
 %  & $\qquad$ & \textsuperscript{($\uparrow$OBJ) = $\downarrow$} & \textsuperscript{$\uparrow$ = $\downarrow$} \\
  %  NP\textsubscript{1} & $\rightarrow$ & N' \\
   %& $\qquad$ & \textsuperscript{$\uparrow$ = $\downarrow$} \\
    %   N' & $\rightarrow$ & N & VP \\
%   & $\qquad$ & \textsuperscript{$\uparrow$ = $\downarrow$} & \textsuperscript{$\downarrow$ $\in$ ($\uparrow$XADJ)} \\
%		    VP & $\rightarrow$ & V' \\
 %  & $\qquad$ & \textsuperscript{$\uparrow$ = $\downarrow$} \\
  %				  V' & $\rightarrow$ & V & NP\textsubscript{2} \\
   %& $\qquad$ & \textsuperscript{$\uparrow$ = $\downarrow$} & \textsuperscript{($\uparrow$OBL\textsubscript{LOC}) = $\downarrow$ }  \\
   	%				 NP\textsubscript{2} & $\rightarrow$ & N \\
%   & $\qquad$ & \textsuperscript{$\uparrow$ = $\downarrow$} \\
%\end{tabular} 
%\end{singlespace}

\subsubsection{c-Struktur}
\begin{singlespace}
\Tree [.S 
		[.{NP\textsubscript{($\uparrow$OBJ)=$\downarrow$}} 
				[.N\textsubscript{$\uparrow$=$\downarrow$} insulam ]		
				[.VP{\textsubscript{$\downarrow$ $\in$ ($\uparrow$XADJ)}}
						[.V\textsubscript{$\uparrow$=$\downarrow$} obiectam ] 
						[\qroof{portui}.NP\textsubscript{($\uparrow$OBJ\textsubscript{LOC})=$\downarrow$} ]
				]
				]
		[.V\textsubscript{$\uparrow$=$\downarrow$} tenuit ]	
	]
\end{singlespace}

\subsubsection{f-Struktur}
\begin{singlespace}
\begin{avm}
\[ PRED &  \rm ‘teneo \q<SUBJ, OBJ\q>’\\
SUBJ & \[ PRED & 'pro' \\
PRON-TYPE & mis \] \\
OBJ & \[PRED & `insula' \\
CASE & acc \\
NUM & sg \\
GEN & f \]\tikzmark{a} \\
XADJ & \{ \[PRED &  \rm ‘obicio \q<SUBJ, OBJ\textsubscript{DAT}\q>’\\
MOOD & part \\
PASSIVE & + \\
RELTENSE & past \\
CASE & acc \\
NUM & sg \\
GEN & f \\
SUBJ &  \tikzmark{z} \\
OBJ\textsubscript{DAT} & \[PRED & `portus' \\
CASE & dat \\
NUM & sg \\
GEN & m \\
\] \]\\
\} &            $\qquad$ \\
\]
\end{avm}
\end{singlespace}

\tikz[remember picture,overlay] 
    \draw[<-] (pic cs:a) to[out=0,in=0,looseness=3.4]  (pic cs:z);


\newpage
\section{das substantivierte Partizip}

Da Partizipien einige Eigenschaften der Adjektive übernehmen,  können sie wie diese substantiviert werden und die Rolle eines Substantives übernehmen. NM bezeichnet die Verwendung des substantivierten Partizips als rein attributiv, nimmt also das Fehlen eines Bezugswortes an. Da jedoch das Vorhanden- bzw. Nichtvorhandensein eines Bezugswortes in der LFG einen erheblichen Unterschied darstellt, wird die Umsetzung des substantivierten Partizips in der LFG anhand zweier Varianten –  erste unter Berücksichtigung eines fehlenden Bezugswortes, letztere ohne Annahme eines fehlenden Bezugswortes –  erarbeitet.\footnote{Vgl. NM, S. 713, § 498.} \\
\textbf{+ klassisch selten / weniger häufig als PC, Abl abs, AcP} \\
\textbf{+ kommt v.a. in bestimmten Kontexten vor, wie... ?}


\subsection{Vorüberlegungen zur Umsetzung in der LFG}
Die Umsetzung des substantivierten Partizips in die LFG-Struktur soll anhand des Beispielsatzes \textit{auxilium petentibus Caesar parcit} veranschaulicht werden.



\subsection{Variante 1: das Partizip als OBJ}
In der obigen Variante würde folglich ein ausgelassenes Bezugswort des Partizips angenommen werden. Dies ist jedoch weder notwendig noch bietet es einen Mehrwert. Da das Partizip substantiviert ist, und somit eben gerade keinem Bezugswort untergeordnet, haben wir uns für die folgende Variante entschieden, deren c-Struktur sichtbar unkomplizierter ist. Dabei ist das Partizip Kopf der Partizipialphrase VP und somit alleiniges Objekt des Hauptsatz-Prädikats \textit{parcit}. In diesem Fall überwiegen zwar die nominalen Eigenschaften des Partizips, die Bezeichnung `VP' wird jedoch um der Konsistenz willen beibehalten.


%\subsubsection{Einschränkungen}
\subsubsection{Lexikoneintrag}
*wie oben, mit einzigem Unterschied ...

\begin{singlespace}
\begin{tabular}{ l  l  l  l  } 
\textbf{petentibus}: & V \\
$\qquad$ & $[1]$ \:  ($\uparrow$PRED) & = & `peto$\langle$SUBJ, OBJ$\rangle$' \\
$\qquad$ & $[2]$ \:  ($\uparrow$SUBJ PRED) & = & `pro' \\
$\qquad$ & $[2.1]$ \:  ($\uparrow$SUBJ PRON-TYPE) & = & missing \\
$\qquad$ & $[2.2]$ \:  ($\uparrow$SUBJ CASE) & = & \{abl $\mid$ dat\} \\
$\qquad$ & $[2.3]$ \:  ($\uparrow$SUBJ NUM) & = & pl \\
$\qquad$ & $[2.4]$ \:  ($\uparrow$SUBJ GEN) & = & \{m $\mid$ n $\mid$ f\} \\
$\qquad$ & $[3]$ \:  ($\uparrow$OBJ CASE) & = & acc \\
$\qquad$ & $[4]$ \:  ($\uparrow$MOOD) & = & part\\
$\qquad$ & $[4]$ \:  ($\uparrow$FIN) & = & - \\
$\qquad$ & $[5]$ \:  ($\uparrow$PASSIVE) & = & - \\
$\qquad$ & $[6]$ \:  ($\uparrow$RELTENSE) & = & present \\ 
$\qquad$ & $[7]$ \:  ($\uparrow$CASE) & = & \{abl $\mid$ dat\} \\
$\qquad$ & $[8]$ \:  ($\uparrow$NUM) & = & pl \\
$\qquad$ & $[9]$ \:  ($\uparrow$GEN) & = & \{m $\mid$ n $\mid$ f\} \\
\end{tabular}
\newline
\newline
\end{singlespace}


%\subsubsection{Syntaxregeln}
%Syntaxregeln sehen wie folgt aus: \\
%\begin{singlespace}
%\begin{tabular}{ l  l  c  c  c  c }
 % S & $\rightarrow$ & VP & NP\textsubscript{1} & V\\
  % & $\qquad$ & \textsuperscript{($\uparrow$OBJ\textsubscript{REC}) = $\downarrow$} & \textsuperscript{($\uparrow$SUBJ) = $\downarrow$} & \textsuperscript{$\uparrow$ = $\downarrow$} \\
%		    VP & $\rightarrow$ & V' \\
 %  & $\qquad$ & \textsuperscript{$\uparrow$ = $\downarrow$} \\
  %				  V' & $\rightarrow$ & NP\textsubscript{2} & V \\
   %& $\qquad$ & \textsuperscript{($\uparrow$OBJ) = $\downarrow$} & \textsuperscript{$\uparrow$ = $\downarrow$} \\
   	%				 NP\textsubscript{2} & $\rightarrow$ & N \\
%   & $\qquad$ & \textsuperscript{$\uparrow$ = $\downarrow$} \\
 %   NP\textsubscript{1} & $\rightarrow$ & N' \\
  % & $\qquad$ & \textsuperscript{$\uparrow$ = $\downarrow$} \\
%\end{tabular} 
%\end{singlespace}

\subsubsection{c-Struktur}
\begin{singlespace}
\Tree [.S 
		[.VP{\textsubscript{($\uparrow$OBJ) = $\downarrow$}}
					[\qroof{auxilium}.NP\textsubscript{($\uparrow$OBJ)=$\downarrow$} ]
					[.V\textsubscript{$\uparrow$=$\downarrow$} petentibus ] 
		]
		[\qroof{Caesar}.NP\textsubscript{($\uparrow$SUBJ)=$\downarrow$} ]
		[.V\textsubscript{$\uparrow$=$\downarrow$} parcit ]	
	]
\end{singlespace}

\subsubsection{f-Struktur}
\begin{singlespace}
\begin{avm}
\[ PRED &  \rm ‘parco \q<SUBJ, OBJ\textsubscript{REC}\q>’\\
SUBJ & \[``Caesar'' \] \\
OBJ\textsubscript{REC} & \[PRED &  \rm ‘peto \q<SUBJ, OBJ\q>’\\
MOOD & part \\
PASSIVE & - \\
RELTENSE & present \\
CASE & dat \\
NUM & pl \\
GEN & m \\
SUBJ & \[PRED & `pro' \\
PRON-TYPE  & mis \] \\
OBJ & \[PRED & `auxilium' \\
CASE & acc \\
NUM & sg \\
GEN & n \] \\
\] \]
\end{avm}
\end{singlespace}

\subsection{Variante 2: das Partizip als XADJ zum OBJ}
Das Hauptroblem bei der erstgenannten Variante ergibt sich dadurch, dass durch das komplett fehlende Bezugswort bei der Implementierung/Formulierung der Redundanz- und Default-Regeln Probleme auftreten. Des Weiteren erfüllen die Partizipialkonstruktionen in der Regel die grammatikalischen Funktionen eines XADJ, XCOMP oder ADJ; die Klassifikation als OBJ in der obigen Überlegung würde daher eine Ausnahme darstellen. Daher entspricht die Einordnung als XADJ dem aufgestellten Konzept.

Folgt man dem Neuen Menge\footnote{Vgl. NM \textbf{§ ??}} und betrachtet das substantivierte Partizip (\textit{petentibus}) als Attribut zu einem sozusagen fehlenden Bezugswort -- in diesem Fall also etwa \textit{eis} oder \textit{viris} -- so würde die Partizipialkonstruktion in der Rolle eines XADJ zu diesem Bezugswort stehen; das Bezugswort selbst wäre dann das Objekt des Hauptsatzprädikats \textit{parcit}. Dieses fehlende Objekt wird in der c-Struktur unten durch ,,mis'' (für ,,missing'', ,,fehlend'') bezeichnet. Da vom substantivierten Partizip petentibus in unserem Beispiel noch ein Nomen in Objektfunktion abhängt, spaltet sich die Partizipial-VP noch einmal in V und NP auf. \textbf{(eher zu PC, da das früher erwähnt wird)}

%\subsubsection{Einschränkungen}
%\textbf{Variante 1: XADJ}:\\
%Das Subjekt der untergeordneten Struktur ist das Objekt der dem XADJ übergeordneten Struktur (welches fehlt): \\
%($\downarrow$SUBJ) = ((OBJ$\uparrow$)XADJ) \textbf{???} \\

\subsubsection{Lexikoneintrag}
Der Lexikoneintrag des Partizips lautet wie folgt:\footnote{Im Rahmen des Umfangs der Arbeit wird nur die für unseren Beispielsatz relevanten Argumente aufgezählt. Für andere mögliche Konstruktionen von \textit{petere} -- wie (SUBJ, OBJ, OBL\textsubscript{LOC})bzw. (SUBJ, OBJ, OBL\textsubscript{PURPOSE}) -- müssten eigene Lexikoneinträge erstellt werden.\textbf{(Vgl. RHH §119 und § 234)}.}
\begin{singlespace}
\begin{tabular}{ l  l  l  l  } 
\textbf{petentibus}: & V \\
$\qquad$ & $[1]$ \:  ($\uparrow$PRED) & = & `peto$\langle$SUBJ, OBJ$\rangle$' \\
$\qquad$ & $[2]$ \:  ($\uparrow$SUBJ) & = & \{((XADJ$\uparrow$)GF) $\mid$ ((ADJ$\uparrow$)GF)\} \\
$\qquad$ & $[3.1]$ \:  ($\uparrow$SUBJ CASE) & = & \{abl $\mid$ dat\} \\
$\qquad$ & $[3.2]$ \:  ($\uparrow$SUBJ NUM) & = & pl \\
$\qquad$ & $[3.3]$ \:  ($\uparrow$SUBJ GEN) & = & \{m $\mid$ n $\mid$ f\} \\
$\qquad$ & $[4]$ \:  ($\uparrow$OBJ CASE) & = & acc \\
$\qquad$ & $[5]$ \:  ($\uparrow$MOOD) & = & part\\
$\qquad$ & $[4]$ \:  ($\uparrow$FIN) & = & - \\
$\qquad$ & $[6]$ \:  ($\uparrow$PASSIVE) & = & - \\
$\qquad$ & $[7]$ \:  ($\uparrow$RELTENSE) & = & present \\ 
$\qquad$ & $[8]$ \:  ($\uparrow$CASE) & = & \{abl $\mid$ dat\} \\
$\qquad$ & $[9]$ \:  ($\uparrow$NUM) & = & pl \\
$\qquad$ & $[10]$ \:  ($\uparrow$GEN) & = & \{m $\mid$ n $\mid$ f\} \\
\end{tabular}
\newline
\newline
\end{singlespace}

Da im Lateinischen die Verben die Kasus ihrer Objekte bestimmen, muss im Lexikoneintrag des Prädikats der übergeordneten Struktur festgelegt sein, dass sein Objekt im Dativ steht: \\
\begin{singlespace}
\begin{tabular}{ l  l  l  l  } 
\textbf{parcit}: & V \\
$\qquad$ & $[1]$ \:  ($\uparrow$PRED) & = & `parco$\langle$SUBJ, OBJ$\rangle$' \\
$\qquad$ & $\qquad$ & . \\
$\qquad$ & $\qquad$ & . \\
$\qquad$ & $\qquad$ & . \\
$\qquad$ & $[4]$ \:  ($\uparrow$FIN) & = & + \\
$\qquad$ & $[2]$ \:  ($\uparrow$OBJ CASE) & = & dat \\
\end{tabular}
\newline
\newline
\end{singlespace}
--> vgl. Skript, S. 48: ,,man geht vom Verb aus und formuliert eine Restriktion
des Kasus für die jeweils regierten Grammatischen Funktionen.''


%\subsubsection{Syntaxregeln}
%\begin{singlespace}
%\begin{tabular}{ l  l  c  c  c  c }
%  S & $\rightarrow$ & NP\textsubscript{1} & NP\textsubscript{2} & V \\
 %  & $\qquad$ & \textsuperscript{($\uparrow$OBJ) = $\downarrow$} & \textsuperscript{($\uparrow$SUBJ) = $\downarrow$} & \textsuperscript{$\uparrow$ = $\downarrow$} \\
%		NP\textsubscript{1} & $\rightarrow$ & N' \\
 %  & $\qquad$ & \textsuperscript{$\uparrow$ = $\downarrow$} \\
  %		  N' & $\rightarrow$ & N & VP \\
   %& $\qquad$ & \textsuperscript{$\uparrow$ = $\downarrow$} & \textsuperscript{($\uparrow$XADJ) = $\downarrow$} \\		    
	%	    VP & $\rightarrow$ & V' \\
%   & $\qquad$ & \textsuperscript{$\uparrow$ = $\downarrow$} \\
 % 				  V' & $\rightarrow$ & NP\textsubscript{3} & V \\
  % & $\qquad$ & \textsuperscript{($\uparrow$OBJ) = $\downarrow$} & \textsuperscript{$\uparrow$ = $\downarrow$} \\
   %					 NP\textsubscript{3} & $\rightarrow$ & N \\
%   & $\qquad$ & \textsuperscript{$\uparrow$ = $\downarrow$} \\
 %   NP\textsubscript{2} & $\rightarrow$ & N \\
  % & $\qquad$ & \textsuperscript{$\uparrow$ = $\downarrow$} \\
%\end{tabular} 
%\end{singlespace}

\subsubsection{c-Struktur}
Es ergeben sich folgende c- und f-Strukturen:\\

\begin{singlespace}
\Tree [.S 
		[.NP{\textsubscript{($\uparrow$OBJ)=$\downarrow$}}
			[.N\textsubscript{$\uparrow$=$\downarrow$} \textit{mis} ]
			[.VP\textsubscript{$\downarrow$ $\in$ ($\uparrow$XADJ)}  
				[\qroof{auxilium}.NP\textsubscript{($\uparrow$OBJ)=$\downarrow$} ]
				[.V\textsubscript{$\uparrow$=$\downarrow$} petentibus ] 				
			]
		]	
		[\qroof{Caesar}.NP\textsubscript{($\uparrow$SUBJ)=$\downarrow$} ]
		[.V\textsubscript{$\uparrow$=$\downarrow$} parcit ]	
	]
\end{singlespace}

\subsubsection{f-Struktur}
\begin{singlespace}
\begin{avm}
\[ PRED &  \rm ‘parco \q<SUBJ, OBJ\q>’\\
SUBJ & \[``Caesar'' \] \\
OBJ & \[PRED & `pro' \\
PRON-TYPE & mis \\
CASE & dat \\
NUM & pl \\
GEN & m \]\tikzmark{alpha} \\
XADJ & \{ \[PRED &  \rm ‘peto \q<SUBJ, OBJ\q>’\\
MOOD & part \\
PASSIVE & - \\
RELTENSE & present \\
CASE & dat \\
NUM & pl \\
GEN & m \\
SUBJ &  \tikzmark{omega} \\
OBJ & \[PRED & `auxilium' \\
CASE & acc \\
NUM & sg \\
GEN & n \\
\] \] \} &            $\qquad$ \\
\]
\end{avm}
\tikz[remember picture,overlay] 
    \draw[<-] (pic cs:alpha) to[out=0,in=0,looseness=2.5]  (pic cs:omega);
    
\end{singlespace}

\newpage
\section{das dominante Partizip}
Beim sogenannten dominanten Partizip trägt nicht das Substantiv, sondern das in Kasus, Numerus und Genus übereinstimmenden Partizip die Hauptbedeutung; das Partizip ,dominiert` daher sein Bezugswort.
%Beim sogenannten dominanten Partizip ,dominiert` das Partizip sein Bezugswort, weswegen der Hauptfokus auf dem Partizip liegt.
Aus diesem Grund wird das dominante Partizip im Deutschen in der Regel mit einem Verbalsubstantiv wiedergegeben, von dem das im Lateinischen regierende Substantiv als Genetiv abhängt. Meistens verwendet man das Partizip Perfekt Passiv als dominantes Partizip.\footnote{Vgl. NM, S. 717 f., § 502.}\\

\subsection{Version mit Präpositionalphrase}
\subsubsection{Vorüberlegungen zur Umsetzung in der LFG}
%Der Lexikoneintrag zum Partizip der Konstruktion unterscheidet sich nicht wesentlich von den vorherigen.
%\textbf{ (ich glaub wir brauchen hier echt nich nochmal nen Lexikoneintrag...)} \\
Wir gehen von folgendem Lexikoneintrag von \textit{condita} aus:
\begin{singlespace}
\begin{tabular}{ l  l  l  l  } 
\textbf{condita}: & $[1]$ \:  ($\uparrow$PRED) & = & `condor$\langle$SUBJ$\rangle$'\\
%$\qquad$ & $[2]$ \:  ($\uparrow$SUBJ) & = & ((ADJ$\uparrow$)GF)\\
$\qquad$ & $[5]$ \:  \{(($\uparrow$SUBJ GEN) & = & f \\ 
$\qquad$ & $[5.1]$ \:  ($\uparrow$SUBJ NUM) & = & sg \\
$\qquad$ & $[5.2]$ \:  ($\uparrow$SUBJ CASE) & = & \{nom $\mid$ abl\} ) $\mid$\\
$\qquad$ & $[5.2]$ \: (($\uparrow$SUBJ GEN) & = & n \\
$\qquad$ & $[6.3]$ \:  ($\uparrow$SUBJ NUM) & = & pl \\
$\qquad$ & $[6.4]$ \:  ($\uparrow$SUBJ CASE) & = & \{nom $\mid$ acc\} ) \}\\
$\qquad$ & $[2]$ \:  ($\uparrow$MOOD) & = & part\\
$\qquad$ & $[3]$ \:  ($\uparrow$PASSIVE) & = & + \\
$\qquad$ & $[4]$ \:  ($\uparrow$RELTENSE) & = & past \\
$\qquad$ & $[5]$ \:  \{(($\uparrow$GEN) & = & f \\ 
$\qquad$ & $[5.1]$ \:  ($\uparrow$NUM) & = & sg \\
$\qquad$ & $[5.2]$ \:  ($\uparrow$CASE) & = & \{nom $\mid$ abl\} ) $\mid$\\
$\qquad$ & $[5.2]$ \: (($\uparrow$GEN) & = & n \\
$\qquad$ & $[6.3]$ \:  ($\uparrow$NUM) & = & pl \\
$\qquad$ & $[6.4]$ \:  ($\uparrow$CASE) & = & \{nom $\mid$ acc\} ) \}\\
\end{tabular}
\newline
\newline
\end{singlespace}

Das dominante Partizip soll zunächst am Beispielsatz \textit{ab urbe condita Roma viguit} betrachtet werden. Da der Restsatz \textit{Roma viguit} auch ohne die Partizipialkonstruktion Sinn ergibt, muss letztere wie beim Abl. abs. ein ADJ zum finiten Satz sein. Als nächstes ergibt sich aufgrund der Präposition \textit{ab} eine Präpositionalphrase, von der wiederum Partizip und Bezugswort abhängen.

%\subsubsection{Variante 1 -- Partizip als attributives XADJ}
Nun sieht das dominante Partizip \textit{condita} rein formal zunächst aus wie ein attributives Partizip zum Bezugswort \textit{urbe} \textbf{(?) +vgl NM}, weswegen man eine NP mit \textit{urbe} als Kopf konstruieren könnte (siehe Variante 1). Das Partizip wäre somit seinem Bezugswort untergeordnet. Da das Subjekt des Partizips aus der übergeordneten Struktur -- in diesem Fall von der NP mit Kopf \textit{urbe} -- bezieht, müsste das Partizip eine X-Rolle erhalten; da ein XCOMP zum Bezugswort -- in diesem Fall \textit{urbe} -- nicht zu rechtfertigen wäre \textbf{(??? weil es dann von urbe gefordert werden müsste? oder wieso eig?)}, bliebe für das Partizip -- hier \textit{condita} -- nur die Rolle des XADJ. Die zugehörigen c- und f-Strukturen sähen demnach wie folgt aus:

\textbf{c-Struktur}
\begin{singlespace}
\Tree [.S 
		[.PP{\textsubscript{$\downarrow$ $\in$ ($\uparrow$ADJ)}}
			[.P'\textsubscript{$\uparrow$=$\downarrow$} 
				[.P\textsubscript{$\uparrow$=$\downarrow$} ab ] 
				[.NP\textsubscript{($\uparrow$OBJ)=$\downarrow$}
					[.N'\textsubscript{$\uparrow$=$\downarrow$} 
						[.N\textsubscript{$\uparrow$=$\downarrow$} urbe ]
						[\qroof{condita}.VP\textsubscript{$\downarrow$ $\in$ ($\uparrow$XADJ)} ]
					] 
				]
			]				
		] 	
		[\qroof{Roma}.NP\textsubscript{($\uparrow$SUBJ)=$\downarrow$} ]
		[.V\textsubscript{$\uparrow$=$\downarrow$} viguit ]	
	]\\
\newline
\end{singlespace}

\textbf{f-Struktur}
\begin{singlespace}
\begin{avm}
\[ PRED &  \rm ‘vigeo \q<SUBJ\q>’\\
SUBJ & ``Roma'' \\
ADJ & \[ PRED &  \rm ‘ab \q<OBJ\q>’\\
OBJ & \[ PRED & `urbs' \tikzmark{begin} \\ 
CASE & abl \\
NUM & sg \\
GEN & f  \\
XADJ & \[PRED &  \rm ‘condo \q<SUBJ\q>’\\
MOOD & part \\
PASSIVE & + \\
RELTENSE & past \\
CASE & abl \\
NUM & sg \\ 
GEN & f  \\
SUBJ &  \tikzmark{end} \] &            $\qquad$ \\
\]  \\
\] \]
\end{avm}
\end{singlespace}

\tikz[remember picture,overlay] 
    \draw[<-] (pic cs:begin) to[out=0,in=0,looseness=2.4]  (pic cs:end);
    
\begin{singlespace}    
\begin{avm}
\[ PRED &  \rm ‘vigeo \q<SUBJ\q>’\\
SUBJ & ``Roma'' \\
ADJ & \[ PRED &  \rm ‘ab \q<OBJ\q>’\\
OBJ & \[ PRED & `urbs' \\ 
CASE & abl \\
NUM & sg \\
GEN & f  \\
XADJ & \[PRED &  \rm ‘condo \q<SUBJ\q>’\\
MOOD & part \\
PASSIVE & + \\
RELTENSE & past \\
CASE & abl \\
NUM & sg \\ 
GEN & f  \\
SUBJ &  \tikzmark{Ziel} \] \] \tikzmark{Start} & $\qquad$ & $\qquad$  \\
\] \\
\]
\end{avm}
\newline
\newline
\end{singlespace}

\tikz[remember picture,overlay] 
    \draw[<-] (pic cs:Start) to[out=10,in=0,looseness=2.4]  (pic cs:Ziel);

%\subsubsection{Variante 2 -- Partizip in tatsächlich dominanter Position}
Da Adjunkte jedoch nach Belieben weggelassen werden können, würde dies bedeuten, dass der Satz \textit{ab urbe Roma viguit} korrekt wäre. Das stimmt zwar formal -- ist jedoch semantisch sinnfrei. Eine semantisch sinnvollere Darstellung ergibt sich, wenn das Bezugswort vom Prädikat des Partizips gefordert wird; da das Partizip sein Bezugswort dominiert, sollte ihm in der LFG-Darstellung eine seinem Bezugswort übergeordnete Funktion zukommen. Somit würde die Partizipialkonstruktion von einer VP mit dem Kopf \textit{condita} abhängen; das Bezugsnomen \textit{urbe} wäre dann schlicht das Subjekt der Partizipialkonstruktion.

\textbf{+ widerspricht allg. Lexikoneintrag (?)}

\subsubsection{Lexikoneintrag}

\begin{singlespace}
\begin{tabular}{ l  l  l  l  } 
\textbf{condita}: & V \\
$\qquad$ & $[1]$ \:  ($\uparrow$PRED) & = & `condor$\langle$SUBJ$\rangle$'\\
%$\qquad$ & $[2]$ \:  ($\uparrow$SUBJ) & = & ((ADJ$\uparrow$)GF)\\
$\qquad$ & $[5]$ \:  \{(($\uparrow$SUBJ GEN) & = & f \\ 
$\qquad$ & $[5.1]$ \:  ($\uparrow$SUBJ NUM) & = & sg \\
$\qquad$ & $[5.2]$ \:  ($\uparrow$SUBJ CASE) & = & \{nom $\mid$ abl\} ) $\mid$\\
$\qquad$ & $[5.2]$ \: (($\uparrow$SUBJ GEN) & = & n \\
$\qquad$ & $[6.3]$ \:  ($\uparrow$SUBJ NUM) & = & pl \\
$\qquad$ & $[6.4]$ \:  ($\uparrow$SUBJ CASE) & = & \{nom $\mid$ acc\} ) \}\\
$\qquad$ & $[2]$ \:  ($\uparrow$MOOD) & = & part\\
$\qquad$ & $[4]$ \:  ($\uparrow$FIN) & = & - \\
$\qquad$ & $[3]$ \:  ($\uparrow$PASSIVE) & = & + \\
$\qquad$ & $[4]$ \:  ($\uparrow$RELTENSE) & = & past \\
$\qquad$ & $[5]$ \:  \{(($\uparrow$GEN) & = & f \\ 
$\qquad$ & $[5.1]$ \:  ($\uparrow$NUM) & = & sg \\
$\qquad$ & $[5.2]$ \:  ($\uparrow$CASE) & = & \{nom $\mid$ abl\} ) $\mid$\\
$\qquad$ & $[5.2]$ \: (($\uparrow$GEN) & = & n \\
$\qquad$ & $[6.3]$ \:  ($\uparrow$NUM) & = & pl \\
$\qquad$ & $[6.4]$ \:  ($\uparrow$CASE) & = & \{nom $\mid$ acc\} ) \}\\
\end{tabular}
\newline
\newline
\end{singlespace}

%\subsubsection{Syntaxregeln}
%\begin{singlespace}
%\begin{tabular}{ l  l  c  c  c  c }
 % S & $\rightarrow$ & PP & NP\textsubscript{1} & V\\
  % & $\qquad$ & \textsuperscript{$\downarrow$ $\in$ ($\uparrow$ADJ)} & \textsuperscript{($\uparrow$SUBJ) = $\downarrow$} & \textsuperscript{$\uparrow$ = $\downarrow$} \\
%		    PP & $\rightarrow$ & P' \\
 %  & $\qquad$ & \textsuperscript{$\uparrow$ = $\downarrow$} \\
  %				  P' & $\rightarrow$ & P & VP \\
   %& $\qquad$ & \textsuperscript{$\uparrow$ = $\downarrow$} & \textsuperscript{($\uparrow$OBJ) = $\downarrow$} \\
	%				    VP & $\rightarrow$ & V' \\
%   & $\qquad$ & \textsuperscript{$\uparrow$ = $\downarrow$} \\
%		  				  V' & $\rightarrow$ & V & NP\textsubscript{2} \\
 %  & $\qquad$ & \textsuperscript{$\uparrow$ = $\downarrow$} & \textsuperscript{($\uparrow$SUBJ) = $\downarrow$} \\
	%	   					 NP\textsubscript{2} & $\rightarrow$ & N \\
   %& $\qquad$ & \textsuperscript{$\uparrow$ = $\downarrow$} \\
    %NP\textsubscript{1} & $\rightarrow$ & N \\
   %& $\qquad$ & \textsuperscript{$\uparrow$ = $\downarrow$} \\
%\end{tabular} 
%\newline
%\end{singlespace}

\subsubsection{c-Struktur}
\begin{singlespace}
\Tree [.S 
		[.PP{\textsubscript{$\downarrow$ $\in$ ($\uparrow$ADJ)}}
			[.P\textsubscript{$\uparrow$=$\downarrow$} ab ] 
			[.VP\textsubscript{($\uparrow$OBJ)=$\downarrow$}
				[.V\textsubscript{$\uparrow$=$\downarrow$} condita ]
				[\qroof{urbe}.NP\textsubscript{($\uparrow$SUBJ) = $\downarrow$} ]
			]
			]				
		[\qroof{Roma}.NP\textsubscript{($\uparrow$SUBJ)=$\downarrow$} ]
		[.V\textsubscript{$\uparrow$=$\downarrow$} viguit ]	
	]\\
\newline
\end{singlespace}

\subsubsection{f-Struktur}
\begin{singlespace}
\begin{avm}
\[ PRED &  \rm ‘vigeo \q<SUBJ\q>’\\
SUBJ & ``Roma'' \\
ADJ & \{ \[ PRED &  \rm ‘ab \q<OBJ\q>’\\
OBJ & \[ PRED &  \rm ‘condo \q<SUBJ\q>’\\
MOOD & part \\
PASSIVE & + \\
RELTENSE & past \\
CASE & abl \\
NUM & sg \\
GEN & f \\
SUBJ & \[PRED & `urbs' \\
CASE & abl \\
NUM & sg \\
GEN  & f \] \] \] \} \]
\end{avm}\\
\end{singlespace}

\subsection{Version ohne Präpositionalphrase}
\subsubsection{Vorüberlegungen zur Umsetzung in der LFG}
Nun war zu klassischen Zeiten jedoch die präpositionslose Variante des dominanten Partizips vorherrschend,\footnote{Vgl. LHS \textbf{§ ???}} weswegen auch hierzu ein Beispielsatz betrachtet werden soll: \textit{libertate amissa doleo.} Formal ist diese Konstruktion im Ablativ kaum vom Abl. abs. zu unterscheiden; der Satz könnte schließlich auch bedeuten: ``Ich trauere wegen der verlorenen Freiheit''. Korrekter, da näher an der lateinischen Bedeutung, wäre jedoch die Übersetzung: ``Ich trauere wegen des Verlusts der Freiheit.'' Um diesem -- wenn hier auch semantisch geringen -- Unterschied gerecht zu werden, sollte auch hier in der LFG-Darstellung die Dominanz des Partizips über sein Bezugswort deutlich werden. \textbf{(diese Erklärung evt. schon früher (?) )} Auch hier ist daher die gesamte Partizipialkonstruktion ein ADJ zum finiten Prädikat und das Bezugsnomen darin seinem Partizip unterstellt. Der Unterschied zu Variante 2 oben ergibt sich lediglich aus dem Fehlen der Präposition.
\textbf{neu}
Da lateinische Partizipialkonstruktionen jedoch stets als VP definiert werden, haben sie ohnehin V als Kopf; daher kann die besondere Dominanz des Partizips nach unserer Darstellungsweise nicht gesondert hervorgehoben werden. Dies wäre ein Argument, die Partizipialkonstruktionen als gesonderte Partizipialphrasen darzustellen. \textbf{(auch wegen der nominalen Eigenschaften der Partizipien) in Schlussfolgerung + Diese Darstellungsweise wurde in dieser Arbeit bereits in Anfängen versucht, nämlich bei der c-Struktur-Darstellung des Abl. abs. als S\textsubscript{part}.}

%\subsubsection{Syntaxregeln}
%\begin{singlespace}
%\begin{tabular}{ l  l  c  c  c  c }
%  S & $\rightarrow$ & VP & V\\
 %  & $\qquad$ & \textsuperscript{$\downarrow$ $\in$ ($\uparrow$ADJ)} & \textsuperscript{($\uparrow$SUBJ) = $\downarrow$} & \textsuperscript{$\uparrow$ = $\downarrow$} \\
%	    VP & $\rightarrow$ & V' \\
 %  & $\qquad$ & \textsuperscript{$\uparrow$ = $\downarrow$} \\
	%		  V' & $\rightarrow$ & NP& V \\
  % & $\qquad$ & \textsuperscript{($\uparrow$SUBJ) = $\downarrow$} &\textsuperscript{$\uparrow$ = $\downarrow$} \\
	%	   					 NP & $\rightarrow$ & N \\
   %& $\qquad$ & \textsuperscript{$\uparrow$ = $\downarrow$} \\
%\end{tabular} 
%\newline
%\end{singlespace}

\subsubsection{c-Struktur}
%Der Lexikoneintrag von \textit{condita} bleibt selbstverständlich unverändert. 
Es ergeben sich folgende c- und f-Strukturen:

\begin{singlespace}
\Tree [.S 
		[.VP{\textsubscript{$\downarrow$ $\in$ ($\uparrow$ADJ)}}
				[\qroof{libertate}.NP\textsubscript{($\uparrow$SUBJ) = $\downarrow$} ]
				[.V\textsubscript{$\uparrow$=$\downarrow$} amissa ]
		]				 	
			[.V\textsubscript{$\uparrow$=$\downarrow$} doleo ]		
	]\\
\newline
\end{singlespace}

\subsubsection{f-Struktur}
\begin{singlespace}
\begin{avm}
\[ PRED &  \rm ‘doleo \q<SUBJ\q>’\\
SUBJ & \[PRED & `pro' \\
PRON-Type & mis\] \\
ADJ & \{ \[ PRED &  \rm ‘amitto \q<SUBJ\q>’\\
MOOD & part \\
PASSIVE & + \\
RELTENSE & past \\
CASE & abl \\
NUM & sg \\
GEN & f \\
SUBJ & \[PRED & `libertas' \\
CASE & abl \\
NUM & sg \\
GEN  & f \] \] \} \]
\end{avm}\\
\end{singlespace}


\section{Abl. abs.}
Wie beim PC vertritt auch die Partizipialkonstruktion des Ablativus absolutus einen Adverbialsatz, wobei das Bezugswort dem Subjekt, das Partizip dem Prädikat entspricht \textbf{(das kann eig weg wenn wir das in der Einführung lassen)}. Dabei wird das Bezugswort nicht vom Prädikat des finiten Satzes gefordert, und besitzt demnach keine eigene Satzgliedfunktion. Der Abl. abs. ist somit vom Rest des Satzes losgelöst, weswegen seine Satzgliedfunktion stets die der freien Angabe ist. \textbf{Partizip und Bezugswort stehen immer im Ablativ. (doppelt - bei Neugliederung beachten} Da dem Abl. abs. ein Adverbialsatz zugrunde liegt, ist sein Partizip prädikativ verwendet; dass es nicht als Attribut zu einem Nomen steht, wird zudem daran deutlich, dass der Satz bei Wegfall des Partizips grammatikalisch nicht mehr korrekt wäre. Der Ablativ ist im Lateinischen für diese Konstruktion gewählt, da dieser Kasus bereits ohne Partizip adverbiale Verhältnisse, beispielsweise der Zeit, bezeichnet.\footnote{Vgl. KSt, S. 766, § 138,1 u. S. 771, § 138,5b; Vgl. NM, S. 718 f., § 503. Anstelle eines Partizips können auch bestimmte Nomina in den Ablativus absolutus treten. Auf dies kann im Rahmen des Umfangs dieser Arbeit, die sich auf Partizipialkonstruktionen konzentriert, nicht näher eingegangen werden. Vgl. NM, S. 720, § 504.} \\

\subsection{Vorüberlegungen zur Umsetzung in der LFG}
Beispielsatz: \\
\textit{barbaris in Gallia victis Caesar gaudet.} \\

* wichtig: trotz S-part-Bezeichnung kein Nebensatz im eigentlichen Sinne!

\subsection{Einschränkungen}
%Da der Restsatz (S\textsubscript{fin}) auch ohne den Abl. abs. noch Sinn ergeben würde,  steht er in der Funktion eines ADJ: \\
%($\uparrow$ADJ) = $\downarrow$ \\
%Auch beim Abl. abs. muss das Partizip in Kasus, Numerus und Genus mit seinem Bezugswort übereinstimmen:\footnote{Vgl. KSt S. 771, § 138,5a.}\\
%($\uparrow$SUBJ KNG) = ($\uparrow$KNG)\\
%Sowohl Partizip als auch Bezugswort stehen stets im Ablativ:\footnote{Vgl. KSt S. 771, § 138,5b.} \\
%($\uparrow$CASE) = abl \\
%($\uparrow$SUBJ CASE) = abl \\
%Da das Bezugswort des Partizips im Abl. abs. keine Rolle im übergeordneten Satz spielen darf, ist es keine grammatikalische Funktion der dem XADJ übergeordneten Struktur. Der Abl. abs. ist daher vom finiten Satz losgelöst:\footnote{Vgl. KSt S. 771, § 138,5b.} \\
%$\neg$ ($\uparrow$SUBJ) = ((ADJ$\uparrow$)GF) \\
%Da sich diese Arbeit ausschließlich auf das klassische Latein Caesars und Ciceros bezieht, gilt für die folgenden Betrachtungen die Annahme, dass im Abl. abs. kein Partizip Futur Aktiv (PFA) verwendet wird.\footnote{Vgl. KSt. S. 760, § 136,4c oder NM S. 771, § 469.}\\
%$\neg$ ($\uparrow$RELTENSE) = future \\

%($\uparrow$RELTENSE (ADJ)) $\neq$ future \\
%$\neg$ ($\downarrow$PRED) = ($\uparrow$GF PRED) \\

\subsection{Lexikoneintrag}
Obige, für den Abl. abs. gültige Einschränkungen können jedoch nicht im Lexikoneintrag der Partizipien festgehalten werden, da Partizipien im Ablativ auch in anderen Partizipialkonstruktionen vorkommen; ist ein Partizip wie \textit{victis} beispielsweise teil eines PC, ist sein Subjekt eine grammatikalische Funktion der der Partizipialkonstruktion übergeordneten Struktur.
\begin{singlespace}
\begin{tabular}{ l  l  l  l  } 
\textbf{victis}: & V \\
$\qquad$ & $[1]$ \:  ($\uparrow$PRED) & = & `vincor$\langle$SUBJ$\rangle$'\\
$\qquad$ & $[2]$ \: $\neg$ ($\uparrow$SUBJ) & = & \{((XADJ$\uparrow$)GF) $\mid$ ((ADJ$\uparrow$)GF)\} \\
$\qquad$ & $[5]$ \: ($\uparrow$SUBJ CASE) & = & \{dat $\mid$ abl\} \\
$\qquad$ & $[6]$ \:  ($\uparrow$SUBJ NUM) & = & pl \\
$\qquad$ & $[7]$ \: ($\uparrow$SUBJ GEN) & = & \{m $\mid$ f $\mid$ n\} \\
$\qquad$ & $[2]$ \:  ($\uparrow$MOOD) & = & part\\
$\qquad$ & $[4]$ \:  ($\uparrow$FIN) & = & - \\
$\qquad$ & $[3]$ \: ($\uparrow$PASSIVE) & = & + \\
$\qquad$ & $[4]$ \: ($\uparrow$RELTENSE) & = & past \\
$\qquad$ & $[5]$ \: ($\uparrow$CASE) & = & \{dat $\mid$ abl\} \\
$\qquad$ & $[6]$ \:  ($\uparrow$NUM) & = & pl \\
$\qquad$ & $[7]$ \: ($\uparrow$GEN) & = & \{m $\mid$ f $\mid$ n\} \\
\end{tabular}
\end{singlespace}

\subsection{Syntaxregeln}
Somit muss die Losgelöstheit der Ablativus-absolutus-Konstruktion in den Syntaxregeln festgehalten werden. Dies geschieht, indem der Abl. abs. in einem gesonderten Satz, hier bezeichnet als S\textsubscript{part}, dargestellt wird und die Funktion eines Adjunkts erhält. Wir haben uns für diese Variante entschieden, da durch die bloße Bezeichnung als VP nicht zur Geltung kommen würde, dass der Abl. abs. nur durch einen adverbialen Gliedsatz ersetzt werden kann und sowohl sein Subjekt als auch sein Prädikat innerhalb desselben Knotens enthalten sind (und nicht wie beispielsweise beim PC das Subjekt aus der übergeordneten Struktur bezogen werden muss).\footnote{Vergleiche auch die diesbezüglichen Anmerkungen in der Schlussfolgerung dieser Arbeit.} Die oben genannten Syntaxregeln müssen daher erweitert werden: \\
\begin{singlespace}
\begin{tabular}{ l  l  c  c  c  c }
   S\textsubscript & $\rightarrow$ & S\textsubscript{part} & NP & V\\
   & $\qquad$ & \textsuperscript{ $\downarrow$ $\in$ ($\uparrow$ADJ)} & \textsuperscript{($\uparrow$SUBJ) = $\downarrow$} & \textsuperscript{$\uparrow$ = $\downarrow$} \\
   S\textsubscript{part} & $\rightarrow$ & NP & PP & V & \\
   & $\qquad$ &  \textsuperscript{($\uparrow$SUBJ) = $\downarrow$} &\textsuperscript{$\downarrow$ $\in$ ($\uparrow$ADJ)} & \textsuperscript{$\uparrow$ = $\downarrow$} \\
\end{tabular} 
\end{singlespace}

\subsection{c-Struktur}
\begin{singlespace}
\Tree [.S\textsubscript{fin} 
		[.S{\textsubscript{part} \textsubscript{$\downarrow$ $\in$ ($\uparrow$ADJ)}}
			[\qroof{barbaris}.NP{\textsubscript{($\uparrow$SUBJ)=$\downarrow$}}	]
			[\qroof{in Gallia}.PP\textsubscript{($\downarrow$ $\in$ $\uparrow$ADJ)} ]
			[.V\textsubscript{$\uparrow$=$\downarrow$} victis ]
		]							
		[\qroof{Caesar}.{NP\textsubscript{($\uparrow$SUBJ)=$\downarrow$}} ] 
		[.V{\textsubscript{$\uparrow$=$\downarrow$}} gaudet ]
	]
\end{singlespace}

\subsection{f-Struktur}
\begin{singlespace}
\begin{avm}
\[ PRED &  \rm ‘gaudeo \q<SUBJ\q>’\\
SUBJ & \["Caesar" \]\\
ADJ & \{ \[PRED &  \rm ‘vincor \q<SUBJ\q>’\\
MOOD & part \\
PASSIVE & + \\
RELTENSE & past \\
CASE & abl \\
NUM & pl \\
GEN & m \\
SUBJ & \[PRED & `barbarus' \\
CASE & abl \\
NUM & pl \\
GEN & m \\ \] \\
ADJ & \{\[``in Gallia''\] \} \] \\
\}
\]
\end{avm}
\end{singlespace}


\section{der Accusativus cum Participio}
Bei den Verben der unmittelbaren sinnlichen Wahrnehmung, oft bei \textit{videre} und \textit{audire}, sowie bei den Verben des Darstellens und Einführens, besonders bei \textit{facere} und \textit{inducere} -- im Sinne von ,in einem Werk, in einem Drama darstellen, (auftreten) lassen' -- steht die satzwertige Ergänzung oft in Verbindung mit einem Objekt und dem PPA im Akkusativ. Die Verben \textit{habere} und \textit{tenere} stehen mit einem PPP. Das Objekt ist dabei optional. Man nennt diese Verbindung Accusativus cum Participio (AcP). Das Partizip wird dabei in prädikativem Sinn verwendet.\footnote{Vgl. KSt, S. 763, § 137,2 \textbf{(ich glaub es is ja jetzt 2 a und b, oder?)}; NM, S. 714, § 499.}\\


%Der AcP ist von einem Verb der unmittelbaren sinnlichen Wahrnehmung oder von \textit{facere} bzw. \textit{inducere} im Sinne von ‚in einem Werk, in einem Drama darstellen, (auftreten) lassen‘ abhängig.\footnote{Vgl. NM S. 714, § 499. Vgl. auch KSt S. 763, § 137,2a.} \\ 

%\textbf{beim KsT heißt es: Zweitens wird das Partizip in prädikativem Sinne zur Ergänzung eines Verbalbegriffes gebraucht. Dieser Fall tritt ein: a) "Beshreibung des 'normalen AcPs' b) bei den Verben habeo und teneo in Verbindung mit dem PPP, entweder allein oder mit einem Objekte, um einen aus einer vollendeten Handlung hervorgegangenen Zusatnd oder bleibenden Besitz zu bezeichnen}

%\textbf{NM: Das PPP findet sich zur Umschreibung des Perfekts zusammen mit finiten Verbformen von \textit{habere} bzw. \textit{tenere} und bezeichnet dann einen dauernden Zustand oder bleibenden Besitz.}

%\footnote{Vgl. KSt S. 763, § 137,2a. Vgl. auch NM S. 763, § 137. \textbf{Vgl. auch LHS S. 387-88 § 207 c; auch KSt S. 763, § 137,2b? tenere + habere mit PPP?}} \\


\subsection{Vorüberlegungen zur Umsetzung in der LFG}
* Beispielsatz: \\
\textit{militem in campo iacentem vidit.} \\
* da prädikativ: direkt von S abhängig, nicht z.B. von der NP \\
* muss auf jeden Fall entweder XADJ oder XCOMP sein, da das Subjekt zum Prädikat der Struktur vom Prädikat der darüberliegenden Struktur (d.h. vom finiten Verb, "vidit") gefordert wird // da das Prädikat der AcP-Konstruktion, d.h. das Partizip, sein Subjekt aus der übergeordneten Struktur bezieht. \\
* XCOMP oder XADJ?
	
	\textbf{*für XADJ spricht:} Restsatz ergibt auch so Sinn; analog zum PC;
	
	\textbf{*für XCOMP spricht:} semantisch großer Unterschied (andere Bedeutung als PC wegen Verben der Wahrnehmung, würde dem Sinn der Konstruktion sonst nicht gerecht werden); facere / inducere; analog zu AcI -> also haben wir uns dafür entschieden + wird also vom PRED gefordert\\
	
	--> AcI als XCOMP: siehe Skript S. 53!
	
Verben der Wahrnehmung, wie sie im AcI und AcP vorkommen, erfordern im Lateinischen eine Ergänzung, die sowohl durch ein bloßes Nomen, als auch durch eine Partizipialkonstruktion ausgedrückt werden kann. Da das Partizip einer AcI- oder auch AcP-Konstrukion eine Ergänzung des Verbalbegriffs ist, nimmt es die Funktion des XCOMP an. Formal sind AcI und AcP kaum auseinanderzuhalten; der Unterschied liegt in der Semantik: Während beim AcI der Inhalt der Verbalhandlung betont wird, liegt beim AcP der Nachdruck auf der sinnlichen Rezeption der Handlung oder eines Zustandes.\footnote:{Vgl. LHS S. 387, §207.} Diese Bedeutungsdifferenz kann jedoch nicht in der f-Struktur ausgedrückt werden.

Letztere verleiht dem Satzgefüge eine enorme Bedeutungserweiterung, da sie satzwertig ist. --> XCOMP und nicht XADJ

\subsection{Einschränkungen}
%Da das Partizip im AcP eine Ergänzung des Verbalbegriffs ist, erfüllt es im Satz stets die Funktion eines XCOMP: \\
%($\uparrow$XCOMP) = $\downarrow$ \\
%Das Partizip und sein Bezugswort stehen auch beim AcP im selben Kasus, Numerus und Genus:\footnote{Vgl. KSt S. 771, § 138,5a.}\\
%($\uparrow$SUBJ KNG) = ($\uparrow$KNG)\\
%Wie auch hier der Name der Konstruktion vermuten lässt, stehen beim AcP Partizip und Bezugswort im Akkusativ:\footnote{Vgl. KSt S. 763, § 137,2a.} \\
%($\uparrow$CASE) = acc \\
%($\uparrow$SUBJ CASE) = acc \\
%Das Bezugswort des Partizips ist das Objekt der dem XCOMP übergeordneten Struktur: (kann ja nur OBJ sein wegen Akk) \\
%	($\uparrow$SUBJ) = ((XCOMP$\uparrow$)OBJ) \\
%($\uparrow$XCOMP SUBJ) = ($\uparrow$OBJ) (?) \\

%Das Partizip ist beim AcP meist ein PPA, selten ein PPP. \\
%($\uparrow$XCOMP RELTENSE) = present   \\
%$\neg$ ($\uparrow$RELTENSE) = future \\
	

\subsection{Lexikoneintrag}

\begin{singlespace}
\begin{tabular}{ l  l  l  l  } 
\textbf{iacentem}: & V \\ 
$\qquad$ & $[1]$ \: ($\uparrow$PRED) & = & `iaceo$\langle$SUBJ, OBL\textsubscript{LOC}$\rangle$'\\
$\qquad$ & $[2]$ \:  ($\uparrow$SUBJ) & = & \{((XADJ$\uparrow$)GF) $\mid$ ((XCOMP$\uparrow$)GF)\} \\
$\qquad$ & $[5]$ \: ($\uparrow$SUBJ CASE) & = & acc \\
$\qquad$ & $[6]$ \: ($\uparrow$SUBJ NUM) & = & sg \\
$\qquad$ & $[7]$ \: ($\uparrow$SUBJ GEN) & = & \{m $\mid$ f\} \\
$\qquad$ & $[2]$ \: ($\uparrow$MOOD) & = & part\\
$\qquad$ & $[4]$ \:  ($\uparrow$FIN) & = & - \\
$\qquad$ & $[3]$ \: ($\uparrow$PASSIVE) & = & - \\
$\qquad$ & $[4]$ \: ($\uparrow$RELTENSE) & = & present \\
$\qquad$ & $[5]$ \: ($\uparrow$CASE) & = & acc \\
$\qquad$ & $[6]$ \: ($\uparrow$NUM) & = & sg \\
$\qquad$ & $[7]$ \: ($\uparrow$GEN) & = & \{m $\mid$ f\} \\

\end{tabular}\\
\newline
\end{singlespace}
Im Lexikoneintrag des Prädikats der dem AcP-XCOMP übergeordneten Struktur -- hier vidit -- müsste, wie oben erwähnt, zunächst spezifiziert sein, dass es ein XCOMP zu sich nehmen kann, und im Folgenden die Bedingungen, die dieses XCOMP erfüllen muss:\footnote{Vgl. auch Skript, s. 56.} \\
\begin{singlespace}
\begin{tabular}{ l  l  l  l  } 
\textbf{vidit}: & V \\
$\qquad$ & $[1]$ \: ($\uparrow$PRED) & = & `video$\langle$SUBJ, OBJ, XCOMP$\rangle$'\\
$\qquad$ & $\qquad$ & . \\
$\qquad$ & $\qquad$ & . \\
$\qquad$ & $\qquad$ & . \\
$\qquad$ & $[4]$ \:  ($\uparrow$FIN) & = & + \\
$\qquad$ & $[2]$ \:  ($\uparrow$SUBJ XCOMP) & = & ($\uparrow$OBJ)\\
$\qquad$ & $[2.1]$ \: ($\uparrow$XCOMP CASE) & = & acc \\
$\qquad$ & $[3]$ \: ($\uparrow$OBJ CASE) & = & acc \\
\end{tabular}\\
\newline
\end{singlespace}

%Alternative:
%video: $\langle$SUBJ, OBJ, COMP$\rangle$\\
%($\uparrow$COMP SUBJ) = `pro'\\
%($\uparrow$COMP SUBJ KNG) = ($\uparrow$OBJ KNG)\\


%\subsection{Syntaxregeln}

%\begin{singlespace}
%\renewcommand{\arraystretch}{1}  
%\begin{tabular}{ l  l  c  c  c }
 % S & $\rightarrow$ & NP\textsubscript{1} & VP & V\\
  % & $\qquad$ & \textsuperscript{($\uparrow$OBJ) = $\downarrow$} & \textsuperscript{($\uparrow$XCOMP) = $\downarrow$} & \textsuperscript{$\uparrow$ = $\downarrow$} \\
   % NP\textsubscript{1} & $\rightarrow$ & N \\
   %& $\qquad$ & \textsuperscript{$\uparrow$ = $\downarrow$} \\
    %VP & $\rightarrow$ & V' \\
   %& $\qquad$ & \textsuperscript{$\uparrow$ = $\downarrow$} \\
    %V' & $\rightarrow$ & PP & V & \\
   %& $\qquad$ & \textsuperscript{($\uparrow$OBL\textsubscript{LOC}) = $\downarrow$ } & \textsuperscript{$\uparrow$ = $\downarrow$} \\
    %PP & $\rightarrow$ & P' \\
	%& $\qquad$   & \textsuperscript{$\uparrow$ = $\downarrow$} \\
    %P' & $\rightarrow$ & P & NP\textsubscript{2} \\
   %& $\qquad$ & \textsuperscript{$\uparrow$ = $\downarrow$} & \textsuperscript{($\uparrow$OBJ) = $\downarrow$} \\
   % NP\textsubscript{2} & $\rightarrow$ & N \\
   %& $\qquad$ & \textsuperscript{$\uparrow$ = $\downarrow$} \\
%\end{tabular} 
%\end{singlespace}
 
\subsection{c-Struktur}
\begin{singlespace}
\Tree [.S 
		[\qroof{militem}.{NP\textsubscript{($\uparrow$OBJ)=$\downarrow$}} ] 
		[.VP{\textsubscript{($\uparrow$XCOMP)=$\downarrow$}}
			[\qroof{in campo}.PP\textsubscript{($\uparrow$OBL\textsubscript{LOC})=$\downarrow$} ]
				[.V\textsubscript{$\uparrow$=$\downarrow$} iacentem ]
		] 
		[.V\textsubscript{$\uparrow$=$\downarrow$} vidit ]	
	]
\end{singlespace}

\subsection{f-Struktur}
\begin{singlespace}
\begin{avm}
\[ PRED &  \rm ‘video \q<SUBJ, OBJ, XCOMP\q>’\\
SUBJ & \[ PRED & `pro' \\
		PRON-TYPE & mis	\]\\
OBJ & \[ PRED & `miles' \\
CASE & acc \\
NUM & sg \\
GEN & m \]\tikzmark{topic} \\
XCOMP & \[PRED &  \rm ‘iaceo \q<SUBJ, OBL\textsubscript{LOC}\q>’\\
MOOD & part \\
PASSIVE & - \\
RELTENSE & present \\
CASE & acc \\
NUM & sg \\
GEN & m \\
SUBJ &  \tikzmark{object} \\
OBL\textsubscript{LOC} & \[``in campo''\] \]  &            $\qquad$\\
\]
\end{avm}
\end{singlespace}

\tikz[remember picture,overlay] 
    \draw[<-] (pic cs:topic) to[out=0,in=0,looseness=3]  (pic cs:object);

\section{Zusammenfassung und Ausblick}
... \\
Auch aus psycholinguistischer Sicht bietet die LFG aufgrund der Prozeduralität und Lokalität hinsichtlich des Aufbaus ihrer Strukturen einen interessanten Ansatz.\footnote{Vgl. \cite[?]{Rohrer}.} --> * prozeduraler Charakter der LFG, Repräsentation mentaler Prozesse usw (vgl. Rohrer) --> Prozess wird beschrieben! Führt über... 

*--> LFG bietet evt. auch hinsichtlich des Spracherwerbs Vorteile: ``In diesem Grammatikmodell werden je nach den Bestandteilen und Einheiten innerhalb eines Satzes Verbindungen zwischen den Äußerungsteilen hergestellt, die den Erwerbsstufen zugeordnet werden. Die Abfolge ergibt sich bei dieser Grundlegung daraus, dass sich der Informationsaustausch auf immer größere sprachliche Einheiten von zunächst isolierten Lexemen über Wortgruppen, z.B. Nominalphrasen, bis hin zu Sätzen erstreckt. Dieser Ansatz bringt morphologische Mittel wie die Mehrfachkodierung von Kasussuffixen innerhalb und zwischen nominalen Wortgruppen und der Verbalmorphologie mit syntaktischen Rollen in Verbindung. Die Verbindung ist eher mechanisch und auf Beziehungen innerhalb von Sätzen begrenzt.''
Grießhaber, Wilhelm: Linguistische Grundlagen und Lernermerkmale bei der Profilanalyse, in: Martina Rost-Roth (Hrsg.): DAZ. Spracherwerb und Sprachförderung. Deutsch als Zweitsprache, S. 21. 
%\end{singlespace}
%\bibliographystyle{plain}
\pagebreak
\section*{Literaturverzeichnis}
\addcontentsline{toc}{section}{Literaturverzeichnis}
\printbibliography
\end{document}
