\documentclass[12pt,a4paper]{article}
%%% PACKAGES
%\usepackage{times} % Schriftart Times verwenden
\usepackage{graphicx} % support the \includegraphics command and options
\usepackage{booktabs} % for much better looking tables
\usepackage{array} % for better arrays (eg matrices) in maths
\usepackage{paralist} % very flexible & customisable lists (eg. enumerate/itemize, etc.)
\usepackage{verbatim} % adds environment for commenting out blocks of text & for better verbatim
\usepackage{subfig} % make it possible to include more than one captioned figure/table in a single float
\usepackage{colortbl} % enables to shade tables
\usepackage[style=authoryear]{biblatex}
\bibliography{quellen}
\usepackage[utf8]{inputenc}
\usepackage{url}
\usepackage{qtree}
\usepackage{amsmath}
\usepackage{amsfonts}
\usepackage{amssymb}
\usepackage[T1]{fontenc}  % stellt sicher, dass im PDF auch Umlaute gefunden werden
\usepackage{tgtermes}
\usepackage{pdfpages}
\usepackage{listings}
\usepackage{fixltx2e}
\usepackage[ngerman]{babel} % deutsche Begriffe (z.B. Inhaltsverzeichnis statt Contents)
\usepackage[german=quotes]{csquotes}
%\renewcommand{\baselinestretch}{1.47} % Zeilenabstand
%\usepackage[onehalfspacing]{setspace} %Zeilenabstand
\usepackage{setspace}
%Seitenränder
\usepackage{geometry}
%\geometry{a4paper, top=2cm, left=2cm, right=2cm, bottom=2cm}
%Linenumbers
\usepackage[modulo]{lineno}
\usepackage{tabularx}
\usepackage{tikz}
\usetikzlibrary{tikzmark,positioning}
\usepackage{avm}
\DeclareBibliographyCategory{primary}
\DeclareBibliographyCategory{secondary}
\DeclareBibliographyCategory{online}
\addtocategory{primary}{lucil1, lucil2, original, seneca66}
\addtocategory{secondary}{hachmann1995, bartsch, becker1893sittlichen, cancik, inwood, edwards, motto, becker1893sittlichen}
\addtocategory{online}{philatinFLU, philatinAUDIS}
\defbibheading{primary}{\subsection*{Textausgaben und Kommentare}}
\defbibheading{secondary}{\subsection*{Sekundärliteratur}}
\defbibheading{online}{\subsection*{Online Ressourcen}}

\begin{document}
%%%%%%%%%%%%%%%%%%%%%%%%%%
% Deckblatt
% The title
\begin{titlepage}

\begin{center}


% Upper part of the page
\begin{minipage}{0.55\textwidth}
\begin{flushleft} \small
Ruprecht-Karls-Universität Heidelberg\\
Seminar für Klassische Philologie\\
Sommersemester 2013\\
Leitung: Dr. Kathrin Winter\\ 
Proseminar: Seneca, \textit{epistulae morales}
\end{flushleft}
\end{minipage}
\begin{minipage}{0.4\textwidth}
\begin{flushright} \large

\end{flushright}
\end{minipage}
\\[3.3cm]
\rule{\textwidth}{0.4pt}\\[0.4cm]

% Title

{\Large Bedeutung, Notwendigkeit und Konsequenzen \\ der Selbstgenügsamkeit} \linebreak {\large -- Eine Betrachtung anhand von Sen. \textit{epist.} 72,7-8}\\[0.2cm]

\rule{\textwidth}{0.4pt}\\[2.4cm]

% Author and supervisor
%\begin{minipage}{0.4\textwidth}
\begin{flushleft} \small
Natalia Bihler\\
Matrikelnummer: 2925340\\
6. Fachsemester (Gymnasiallehramt nach GymPO)\\
Latein und Englisch\\
Dammweg 1, 69123 Heidelberg\\
E-mail: Bihler@stud.uni-heidelberg.de
\end{flushleft}
%\end{minipage}


\vfill

% Bottom of the page
{\large 21. August 2014}

\end{center}

\end{titlepage}
%%%%%%%%%%%%%%%%%%%%%%%%%%
\setcounter{page}{2}
\begingroup
\flushbottom
\tableofcontents
\thispagestyle{empty}
%\newpage
\pagebreak
\endgroup
%\setcounter{page}{1}
% The introduction
\section{Einleitung}
\nocite{lucil1}
\nocite{lucil2} 
\nocite{original}
\nocite{seneca66} 
\nocite{hachmann1995} 
\nocite{bartsch}  
\nocite{philatinFLU} 
\nocite{becker1893sittlichen} 
\nocite{cancik} 
\nocite{inwood}
\nocite{edwards}
\nocite{motto} 
\nocite{becker1893sittlichen}
\nocite{philatinAUDIS}

% AVM Referenz http://nlp.stanford.edu/manning/tex/avm-doc.pdf
%https://github.com/Bhlini/LFG
%https://en.wikibooks.org/wiki/LaTeX/Linguistics#Syntactic_trees
%https://en.wikibooks.org/wiki/LaTeX/Labels_and_Cross-referencing
%http://nlp.stanford.edu/manning/tex/avm-doc.pdf
%http://nlp.stanford.edu/cmanning/tex/
%http://tex.stackexchange.com/questions/157131/problems-using-avm-package-for-lfg-structures
%http://latex-community.org/forum/viewtopic.php?f=12&t=11365
%http://www.essex.ac.uk/linguistics/external/clmt/latex4ling/avms/
%http://web.ift.uib.no/Teori/KURS/WRK/TeX/symALL.html Zeichen

\subsection{Einschränkungen}

\subsubsection{PC}

($\uparrow$SUBJ KNG) = ($\uparrow$KNG)\\
($\uparrow$SUBJ) = ((ADJ$\uparrow$)GF) \\
(SUBJ XADJ) $\in$ ($\uparrow$S)\\

XADJ (KNG) = SUBJ XADJ (KNG) \\

\subsubsection{PC (objektabhängig}

\subsubsection{Abl. abs.}

XADJ (KNG) = SUBJ XADJ (KNG) = abl\\
$\neg$ ($\uparrow$SUBJ) = ((ADJ$\uparrow$)GF) \\
d.h. Subjekt des Abl. abs. darf eigentlich keine Rolle im übergeordneten Satz spielen; nur bei AcI geht das\\
($\uparrow$RELTENSE (ADJ)) $\neq$ future \\
$\neg$ ($\downarrow$PRED) = ($\uparrow$GF PRED)\\

\subsubsection{AcP}

($\uparrow$XCOMP) = $\downarrow$\\
($\downarrow$SUBJ CASE) = acc\\
($\downarrow$CASE) = acc\\
($\uparrow$XCOMP SUBJ) = ($\uparrow$OBJ)\\
($\uparrow$XCOMP MOOD) = part\\
$\neg$ ($\uparrow$XCOMP RELTENSE) = future\\
VERB TYPE = verb of perception $\mid$ `facere' $\mid$ `inducere'

\subsection{Lexikoneinträge}

\subsubsection{PC objektabhängig}

\begin{tabular}{ l  l  l  l  } 
\textbf{missum}: & $[1]$ \:  ($\uparrow$PRED) & = & `mitto$\langle$SUBJ, OBJ, OBL\textsubscript{GOAL} $\rangle$\\
$\qquad$ & $[2]$ \:  ($\uparrow$SUBJ) & = & ((XADJ$\uparrow$)OBJ)\\
$\qquad$ & $[3]$ \:  ($\uparrow$MOOD) & = & part)\\
$\qquad$ & $[4]$ \:  ($\uparrow$PASSIVE) & = & + \\
$\qquad$ & $[5]$ \:  ($\uparrow$RELTENSE) & = & past \\
$\qquad$ & $[6]$ \:  ($\uparrow$NUM) & = & sg \\
$\qquad$ & $[7]$ \:  \{(($\uparrow$GEN) & = & m \\ 
$\qquad$ & $[7.1]$ \:  ($\uparrow$CASE) & = & acc ) $\mid$\\
$\qquad$ & $[7.2]$ \: (($\uparrow$GEN) & = & n \\
$\qquad$ & $[7.3]$ \:  ($\uparrow$CASE) & = & \{nom $\mid$ acc\} ) \}\\
\end{tabular}
\newline
\newline

((XADJ$\uparrow$)OBJ) = ($\uparrow$SUBJ) \\

\subsubsection{PC subjektabhängig}

\begin{tabular}{ l  l  l  l  } 
\textbf{missi}: & $[1]$ \:  ($\uparrow$PRED) & = & `mitto$\langle$SUBJ, OBJ, OBL\textsubscript{GOAL} $\rangle$\\
$\qquad$ & $[2]$ \:  ($\uparrow$SUBJ) & = & ((XADJ$\uparrow$)SUBJ) \textbf{(?)}\\
$\qquad$ & $[3]$ \:  ($\uparrow$MOOD) & = & part)\\
$\qquad$ & $[4]$ \:  ($\uparrow$PASSIVE) & = & + \\
$\qquad$ & $[5]$ \: ($\uparrow$RELTENSE) & = & past \\
$\qquad$ & $[6]$ \:  \{(($\uparrow$NUM) & = & pl \\ 
$\qquad$ & $[6.1]$ \:  ($\uparrow$CASE) & = & nom \\
$\qquad$ & $[6.2]$ \:  ($\uparrow$GEN) & = & m) $\mid$\\
$\qquad$ & $[6.3]$ \:  (($\uparrow$NUM) & = & sg \\ 
$\qquad$ & $[6.4]$ \: ($\uparrow$CASE) & = & gen \\
$\qquad$ & $[6.5]$ \:  ($\uparrow$GEN) & = & \{m $\mid$ n\} ) \} \\
\end{tabular}

\subsubsection{Abl. abs.}

\begin{tabular}{ l  l  l  l  } 
\textbf{victis}: & $[1]$ \:  ($\uparrow$PRED) & = & `vinco$\langle$SUBJ, OBJ, OBL\textsubscript{LOC} $\rangle$\\
$\qquad$ & $[2]$ \:  ($\uparrow$MOOD) & = & part\\
$\qquad$ & $[3]$ \: ($\uparrow$PASSIVE) & = & + \\
$\qquad$ & $[4]$ \: ($\uparrow$RELTENSE) & = & past \\
$\qquad$ & $[5]$ \: ($\uparrow$CASE) & = & \{dat $\mid$ abl\} \\
$\qquad$ & $[6]$ \:  ($\uparrow$NUM) & = & pl \\
$\qquad$ & $[7]$ \: ($\uparrow$GEN) & = & \{m $\mid$ f $\mid$ n\} \\
\end{tabular}


\subsubsection{AcP}

\begin{tabular}{ l  l  l  l  } 
\textbf{iacentem}: & $[1]$ \: ($\uparrow$PRED) & = & `iaceo$\langle$SUBJ, OBL\textsubscript{LOC} $\rangle$\\
$\qquad$ & $[2]$ \: ($\uparrow$MOOD) & = & part\\
$\qquad$ & $[3]$ \: ($\uparrow$PASSIVE) & = & - \\
$\qquad$ & $[4]$ \: ($\uparrow$RELTENSE) & = & present \\
$\qquad$ & $[5]$ \: ($\uparrow$CASE) & = & acc \\
$\qquad$ & $[6]$ \: ($\uparrow$NUM) & = & sg \\
$\qquad$ & $[7]$ \: ($\uparrow$GEN) & = & \{m $\mid$ f\} \\
\end{tabular}\\
\newline
\newline
induco: $\langle$SUBJ, OBJ, COMP$\rangle$\\
($\uparrow$COMP SUBJ) = `pro'\\
($\uparrow$COMP SUBJ KNG) = ($\uparrow$OBJ KNG)\\

ODER\\
induco: $\langle$SUBJ, OBJ, XCOMP$\rangle$\\
($\uparrow$XCOMP SUBJ) = ($\uparrow$OBJ)\\
($\uparrow$OBJ CASE) = acc\\

\subsubsection{PC (substantiviert)}
\textbf{Variante 1: XADJ}:\\
($\downarrow$SUBJ) = ((OBJ$\uparrow$)XADJ)
= das Subjekt der untergeordneten Struktur ist das Objekt der dem XADJ übergeordneten Struktur (welches fehlt).


\subsection{Zeichen}

$\theta$

$\mid$

$\neq$

$\in$

$\ni$

$\vdash$

$\subset$

$\ast$

$\neg$


\subsection{Syntaxregeln}

S $\rightarrow$ NP \, VP \: XP\\


\subsubsection{PC objektabhängig}

S $\rightarrow$ NP \, VP \: V\\

\begin{tabular}{ l  l  l  l  l  l }
  \textbf{S} & $\rightarrow$ & \: \: \textbf{NP} & \: \: \textbf{VP} & \: \: \textbf{NP} & \: \textbf{V}\\
   & $\qquad$ & ($\uparrow$OBJ) = $\downarrow$ & $\downarrow$ $\in$ ($\uparrow$XADJ) & ($\uparrow$SUBJ) = $\downarrow$ & $\uparrow$ = $\downarrow$ \\
   $\qquad$ & $\qquad$ \\
    \textbf{NP} & $\rightarrow$ & \: \textbf{N} \\
   & $\qquad$ & $\uparrow$ = $\downarrow$\\
      $\qquad$ & $\qquad$ \\
    \textbf{VP} & $\rightarrow$ & \: \textbf{V'} \\
   & $\qquad$ & $\uparrow$ = $\downarrow$\\
   $\qquad$ & $\qquad$ \\
    \textbf{V'} & $\rightarrow$ & \: \: \textbf{PP} & \: \: \: \textbf{V} & \\
   & $\qquad$ &($\uparrow$OBL\textsubscript{GOAL}) = $\downarrow$  & \: $\uparrow$ = $\downarrow$\\
   $\qquad$ & $\qquad$ \\
    \textbf{PP} & $\rightarrow$ & \: \: \textbf{P'} \\
	& $\qquad$   & $\uparrow$ = $\downarrow$\\
   $\qquad$ & $\qquad$ \\
    \textbf{P'} & $\rightarrow$ & \: \: \textbf{P} & \: \: \textbf{NP} \\
   & $\qquad$ & $\uparrow$ = $\downarrow$ & ($\uparrow$OBJ) = $\downarrow$ \\
   $\qquad$ & $\qquad$ \\
    \textbf{NP} & $\rightarrow$ & \: \textbf{N} \\
   & $\qquad$ & $\uparrow$ = $\downarrow$\\
      $\qquad$ & $\qquad$ \\     
\end{tabular}\\

\newpage
\textbf{ALTERNATIVE FORMATIERUNG 1}\\
\begin{tabular}{ l  l  l  l  l  l }
  \textbf{S} & $\rightarrow$ & \: \: \textbf{NP} & \: \: \textbf{VP} & \: \: \textbf{NP} & \: \textbf{V}\\
   & $\qquad$ & ($\uparrow$OBJ) = $\downarrow$ & $\downarrow$ $\in$ ($\uparrow$XADJ) & ($\uparrow$SUBJ) = $\downarrow$ & $\uparrow$ = $\downarrow$ \\
    \textbf{NP} & $\rightarrow$ & \: \textbf{N} \\
   & $\qquad$ & $\uparrow$ = $\downarrow$\\
    \textbf{VP} & $\rightarrow$ & \: \textbf{V'} \\
   & $\qquad$ & $\uparrow$ = $\downarrow$\\
    \textbf{V'} & $\rightarrow$ & \: \: \textbf{PP} & \: \: \: \textbf{V} & \\
   & $\qquad$ &($\uparrow$OBL\textsubscript{GOAL}) = $\downarrow$  & \: $\uparrow$ = $\downarrow$\\
    \textbf{PP} & $\rightarrow$ & \: \: \textbf{P'} \\
	& $\qquad$   & $\uparrow$ = $\downarrow$\\
    \textbf{P'} & $\rightarrow$ & \: \: \textbf{P} & \: \: \textbf{NP} \\
   & $\qquad$ & $\uparrow$ = $\downarrow$ & ($\uparrow$OBJ) = $\downarrow$ \\
    \textbf{NP} & $\rightarrow$ & \: \textbf{N} \\
   & $\qquad$ & $\uparrow$ = $\downarrow$\\
\end{tabular}
\newline
\\
\\
\textbf{ALTERNATIVE FORMATIERUNG 2}\\ \\
\begin{tabular}{ l  l  l  l  l  l }
  \textbf{S} & $\rightarrow$ & \: \: \textbf{NP} & \: \: \textbf{VP} & \: \: \textbf{NP} & \: \textbf{V}\\
   & $\qquad$ & \textsuperscript{($\uparrow$OBJ) = $\downarrow$} & \textsuperscript{$\downarrow$ $\in$ ($\uparrow$XADJ)} & ($\uparrow$SUBJ) = $\downarrow$ & $\uparrow$ = $\downarrow$ \\
    \textbf{NP} & $\rightarrow$ & \: \textbf{N} \\
   & $\qquad$ & $\uparrow$ = $\downarrow$\\
    \textbf{VP} & $\rightarrow$ & \: \textbf{V'} \\
   & $\qquad$ & $\uparrow$ = $\downarrow$\\
    \textbf{V'} & $\rightarrow$ & \: \: \textbf{PP} & \: \: \: \textbf{V} & \\
   & $\qquad$ &($\uparrow$OBL\textsubscript{GOAL}) = $\downarrow$  & \: $\uparrow$ = $\downarrow$\\
    \textbf{PP} & $\rightarrow$ & \: \: \textbf{P'} \\
	& $\qquad$   & $\uparrow$ = $\downarrow$\\
    \textbf{P'} & $\rightarrow$ & \: \: \textbf{P} & \: \: \textbf{NP} \\
   & $\qquad$ & $\uparrow$ = $\downarrow$ & ($\uparrow$OBJ) = $\downarrow$ \\
    \textbf{NP} & $\rightarrow$ & \: \textbf{N} \\
   & $\qquad$ & $\uparrow$ = $\downarrow$\\
\end{tabular}

\subsubsection{PC attributiv}

S $\rightarrow$ NP \, VP \: V\\

\begin{tabular}{ l  l  l  l  l  l }
  \textbf{S} & $\rightarrow$ & \: \: \textbf{NP} & \: \textbf{V}\\
   & $\qquad$ & ($\uparrow$OBJ) = $\downarrow$ & $\uparrow$ = $\downarrow$ \\
   $\qquad$ & $\qquad$ \\
    \textbf{NP} & $\rightarrow$ & \: \textbf{N} \\
   & $\qquad$ & $\uparrow$ = $\downarrow$\\
      $\qquad$ & $\qquad$ \\
    \textbf{VP} & $\rightarrow$ & \: \textbf{V'} \\
   & $\qquad$ & $\uparrow$ = $\downarrow$\\
   $\qquad$ & $\qquad$ \\
    \textbf{V'} & $\rightarrow$ & \: \: \textbf{PP} & \: \: \textbf{V} & \\
   & $\qquad$ &($\uparrow$OBL\textsubscript{LOC}) = $\downarrow$  & \: $\uparrow$ = $\downarrow$\\
   $\qquad$ & $\qquad$ \\
    \textbf{PP} & $\rightarrow$ & \: \textbf{P'} \\
	& $\qquad$   & $\uparrow$ = $\downarrow$\\
   $\qquad$ & $\qquad$ \\
    \textbf{P'} & $\rightarrow$ & \: \textbf{P} & \: \: \textbf{NP} \\
   & $\qquad$ & $\uparrow$ = $\downarrow$ & ($\uparrow$OBJ) = $\downarrow$ \\
   $\qquad$ & $\qquad$ \\
    \textbf{NP} & $\rightarrow$ & \: \textbf{N} \\
   & $\qquad$ & $\uparrow$ = $\downarrow$\\
      $\qquad$ & $\qquad$ \\     
\end{tabular} 

\subsubsection{Abl. abs.}

S\textsubscript{part} $\rightarrow$ NP \: V'\\

S $\rightarrow$ NP \, VP \: V\\

\begin{tabular}{ l  l  l  l  l  l }
  \textbf{S\textsubscript{fin}} & $\rightarrow$ & \: \: \textbf{S\textsubscript{part}} & \: \: \textbf{NP}(1) & \: \textbf{V}\\
   & $\qquad$ &  $\downarrow$ $\in$ ($\uparrow$ADJ) & ($\uparrow$OBJ) = $\downarrow$ & $\uparrow$ = $\downarrow$ \\
   $\qquad$ & $\qquad$ \\
   \textbf{S\textsubscript{part}} & $\rightarrow$ & \: \: \textbf{NP}(2) & \: \textbf{V'}\\
   & $\qquad$ & ($\uparrow$SUBJ) = $\downarrow$ & $\uparrow$ = $\downarrow$ \\
   $\qquad$ & $\qquad$ \\
    \textbf{NP}(2) & $\rightarrow$ & \: \textbf{N} \\
   & $\qquad$ & $\uparrow$ = $\downarrow$\\
      $\qquad$ & $\qquad$ \\
    \textbf{V'} & $\rightarrow$ & \: \: \textbf{PP} & \: \: \textbf{V} & \\
   & $\qquad$ &($\uparrow$OBL\textsubscript{LOC}) = $\downarrow$  & \: $\uparrow$ = $\downarrow$\\
   $\qquad$ & $\qquad$ \\
    \textbf{PP} & $\rightarrow$ & \: \textbf{P'} \\
	& $\qquad$   & $\uparrow$ = $\downarrow$\\
   $\qquad$ & $\qquad$ \\
    \textbf{P'} & $\rightarrow$ & \: \textbf{P} & \: \: \textbf{NP}(3) \\
   & $\qquad$ & $\uparrow$ = $\downarrow$ & ($\uparrow$OBJ) = $\downarrow$ \\
   $\qquad$ & $\qquad$ \\
    \textbf{NP}(3) & $\rightarrow$ & \: \textbf{N} \\
   & $\qquad$ & $\uparrow$ = $\downarrow$\\
      $\qquad$ & $\qquad$ \\     
          \textbf{NP}(1) & $\rightarrow$ & \: \textbf{N} \\
   & $\qquad$ & $\uparrow$ = $\downarrow$\\
      $\qquad$ & $\qquad$ \\     
\end{tabular} 


\subsubsection{AcP}

S $\rightarrow$ NP \, VP \: V\\

($\uparrow$OBJ) = $\downarrow$\\

\renewcommand{\arraystretch}{1}  
\begin{tabular}{ l  l  l  l  l }
  \textbf{S} & $\rightarrow$ & \: \: \textbf{NP} & \: \: \: \textbf{VP} & \: \textbf{V}\\
   & $\qquad$ &($\uparrow$OBJ) = $\downarrow$ & ($\uparrow$COMP) = $\downarrow$ & $\uparrow$ = $\downarrow$\\
   $\qquad$ & $\qquad$ \\
    \textbf{NP} & $\rightarrow$ & \: \textbf{N} \\
   & $\qquad$ & $\uparrow$ = $\downarrow$\\
      $\qquad$ & $\qquad$ \\
    \textbf{VP} & $\rightarrow$ & \: \textbf{V'} \\
   & $\qquad$ & $\uparrow$ = $\downarrow$\\
   $\qquad$ & $\qquad$ \\
    \textbf{V'} & $\rightarrow$ & \: \: \textbf{PP} & \: \: \: \textbf{V} & \\
   & $\qquad$ &($\uparrow$OBL\textsubscript{LOC}) = $\downarrow$  & $\uparrow$ = $\downarrow$\\
   $\qquad$ & $\qquad$ \\
    \textbf{PP} & $\rightarrow$ & \: \: \textbf{P'} \\
	& $\qquad$   & $\uparrow$ = $\downarrow$\\
   $\qquad$ & $\qquad$ \\
    \textbf{P'} & $\rightarrow$ & \: \: \textbf{P} & \: \: \textbf{NP} \\
   & $\qquad$ & $\uparrow$ = $\downarrow$ & ($\uparrow$OBJ) = $\downarrow$ \\
   $\qquad$ & $\qquad$ \\
    \textbf{NP} & $\rightarrow$ & \: \textbf{N} \\
   & $\qquad$ & $\uparrow$ = $\downarrow$\\
      $\qquad$ & $\qquad$ \\     
\end{tabular} 
 
 

\newpage
\section{PARTICIPIUM CONJUNCTUM (attributiv)}

\Tree [.S 
		[.{NP\textsubscript{($\uparrow$OBJ)=$\downarrow$}} 
		[.NP\textsubscript{$\uparrow$=$\downarrow$} insulam ]		
		[.VP{\textsubscript{$\downarrow$ $\in$ ($\uparrow$XADJ)}}
			[.{V'\textsubscript{$\uparrow$=$\downarrow$}}
				[.V\textsubscript{$\uparrow$=$\downarrow$} obiectam ] 
		[\qroof{portui}.NP\textsubscript{($\uparrow$OBJ\textsubscript{DAT})=$\downarrow$} ]
		] 
		]
			]
		[.V\textsubscript{$\uparrow$=$\downarrow$} tenuit ]	
	]

\subsection{f-Struktur PC (attributiv)}
\begin{avm}
\[ PRED &  \rm ‘teneo \q<SUBJ, OBJ\q>’\\
SUBJ & \[ PRED & 'pro' \\
PRON-TYPE & mis \] \\
OBJ & \[PRED & `insula' \\
CASE & acc \\
NUM & sg \\
GEN & f \]\tikzmark{a} \\
XADJ & \{ \[PRED &  \rm ‘obicio \q<SUBJ, OBJ\textsubscript{DAT}\q>’\\
MOOD & part \\
PASSIVE & + \\
RELTENSE & past \\
CASE & acc \\
NUM & sg \\
GEN & f \\
SUBJ &  \tikzmark{z} \\
OBJ\textsubscript{DAT} & \[PRED & `portus' \\
CASE & dat \\
NUM & sg \\
GEN & m \\
\] \]\\
\} &            $\qquad$ \\
\]
\end{avm}

\tikz[remember picture,overlay] 
    \draw[<-] (pic cs:a) to[out=0,in=0,looseness=3.4]  (pic cs:z);

\newpage
\section{PARTICIPIUM CONJUNCTUM (objektabhängig)}

\Tree [.S 
		[\qroof{legatum}.{NP\textsubscript{($\uparrow$OBJ)=$\downarrow$}} ] 
		[.VP{\textsubscript{$\downarrow$ $\in$ ($\uparrow$XADJ)}}
			[.{V'\textsubscript{$\uparrow$=$\downarrow$}}
				[\qroof{in Galliam}.PP\textsubscript{($\uparrow$OBL\textsubscript{GOAL})=$\downarrow$} ]
				[.V\textsubscript{$\uparrow$=$\downarrow$} missum ]						
			] 
		] 
		[\qroof{Caesar}.NP\textsubscript{($\uparrow$SUBJ)=$\downarrow$} ]
		[.V\textsubscript{$\uparrow$=$\downarrow$} revocat ]	
	]

\subsection{f-Struktur PC (objektabhängig)}
\begin{avm}

\[ PRED &  \rm ‘revoco \q<SUBJ, OBJ\q>’\\
SUBJ & \[PRED & `Caesar' \\
CASE & nom \\
NUM & sg \\
GEN & m \]\\
OBJ & \[ PRED & 'legatus' \\
CASE & acc \\
NUM & sg \\
GEN & m \]\tikzmark{aim} \\
XADJ & \{ \[PRED &  \rm ‘mitto \q<SUBJ, OBL\textsubscript{GOAL}\q>’\\
MOOD & part \\
PASSIVE & + \\
RELTENSE & past \\
CASE & acc \\
NUM & sg \\
GEN & m \\
SUBJ &  \tikzmark{start} \\
OBL\textsubscript{GOAL} & \[``in scholam''\] \]\\
\} &            $\qquad$ \\
TENSE & present \\
NUM & sg \\
PERS & 3 \\
PASSIVE & - \\
MODE & ind \\
\]
\end{avm}

\tikz[remember picture,overlay] 
    \draw[<-] (pic cs:aim) to[out=0,in=0,looseness=3.5]  (pic cs:start);

\newpage
\section{PARTICIPIUM CONJUNCTUM (subjektabhängig)}

\Tree [.S 
		[\qroof{milites}.{NP\textsubscript{($\uparrow$SUBJ)=$\downarrow$}} ] 
		[.VP{\textsubscript{$\downarrow$ $\in$ ($\uparrow$XADJ)}}
			[.{V'\textsubscript{$\uparrow$=$\downarrow$}}
				[\qroof{in Galliam}.PP\textsubscript{($\uparrow$OBL\textsubscript{GOAL})=$\downarrow$} ]
				[.V\textsubscript{$\uparrow$=$\downarrow$} missi ]						
			] 
		] 
		[\qroof{hostes}.NP\textsubscript{($\uparrow$OBJ)=$\downarrow$} ]
		[.V\textsubscript{$\uparrow$=$\downarrow$} vicerunt ]	
	]

\subsection{f-Struktur PC (subjektabhängig)}
\begin{avm}
\[ PRED &  \rm ‘vinco \q<SUBJ, OBJ\q>’\\
SUBJ & \[ PRED & 'miles' \\
CASE & nom \\
NUM & pl \\
GEN & m \]\tikzmark{meow} \\
XADJ & \{ \[PRED &  \rm ‘mitto \q<SUBJ, OBL\textsubscript{GOAL}\q>’\\
MOOD & part \\
PASSIVE & + \\
RELTENSE & past \\
CASE & nom \\
NUM & pl \\
GEN & m \\
SUBJ &  \tikzmark{objectmeow} \\
OBL\textsubscript{GOAL} & \[``in Galliam''\] \]\\
\} &            $\qquad$ \\
OBJ & \[``hostes'' \]\\
\]
\end{avm}

\tikz[remember picture,overlay] 
    \draw[<-] (pic cs:meow) to[out=0,in=0,looseness=3.5]  (pic cs:objectmeow);

\newpage
\section{Abl. abs.}

\Tree [.S\textsubscript{fin} 
		[.S{\textsubscript{part} \textsubscript{($\downarrow$ $\in$ $\uparrow$ADJ)}}
			[\qroof{barbaris}.NP{\textsubscript{($\uparrow$SUBJ)=$\downarrow$}}			
			 ]
			[.{V'\textsubscript{$\uparrow$=$\downarrow$}}
				[\qroof{in Gallia}.PP\textsubscript{($\uparrow$OBL\textsubscript{LOC})=$\downarrow$} ]
					[.V\textsubscript{$\uparrow$=$\downarrow$} victis ]
			 ]
		]							
		[\qroof{Caesar}.{NP\textsubscript{($\uparrow$SUBJ)=$\downarrow$}} ] 
		[.V{\textsubscript{$\uparrow$=$\downarrow$}} gaudet ]
	]

\subsection{f-Struktur Abl. abs.}
\begin{avm}
\[ PRED &  \rm ‘gaudeo \q<SUBJ\q>’\\
SUBJ & \["Caesar" \]\\
ADJ & \{ \[PRED &  \rm ‘vinco \q<SUBJ, OBL\textsubscript{LOC}\q>’\\
MOOD & part \\
PASSIVE & + \\
RELTENSE & past \\
CASE & abl \\
NUM & pl \\
GEN & m \\
SUBJ & \[PRED & `barbarus' \\
CASE & abl \\
NUM & pl \\
GEN & m \\ \] \\
OBL\textsubscript{LOC} & \[``in Gallia''\] \]\\
\}
\]
\end{avm}

\newpage
\section{AcP - Accusativus cum Participio}
\Tree [.S 
		[\qroof{militem}.{NP\textsubscript{($\uparrow$OBJ)=$\downarrow$}} ] 
		[.VP{\textsubscript{($\uparrow$XCOMP)=$\downarrow$}}
			[.{V'\textsubscript{$\uparrow$=$\downarrow$}}
				[\qroof{in campo}.PP\textsubscript{($\uparrow$OBL\textsubscript{LOC})=$\downarrow$} ]
				[.V\textsubscript{$\uparrow$=$\downarrow$} iacentem ]					
			]	
		] 
		[.V\textsubscript{$\uparrow$=$\downarrow$} vidit ]	
	]


\subsection{f-Struktur AcP}
\begin{avm}
\[ PRED &  \rm ‘video \q<SUBJ, OBJ, XCOMP\q>’\\
SUBJ & \[ PRED & `pro' \\
		PRON-TYPE & mis	\]\\
OBJ & \[ PRED & `miles' \\
CASE & acc \\
NUM & sg \\
GEN & m \]\tikzmark{topic} \\
XCOMP & \[PRED &  \rm ‘iaceo \q<SUBJ, OBL\textsubscript{LOC}\q>’\\
MOOD & part \\
PASSIVE & - \\
RELTENSE & present \\
CASE & acc \\
NUM & sg \\
GEN & m \\
SUBJ &  \tikzmark{object} \\
OBL\textsubscript{LOC} & \[``in campo''\] \]  &            $\qquad$\\
\]
\end{avm}

\tikz[remember picture,overlay] 
    \draw[<-] (pic cs:topic) to[out=0,in=0,looseness=3]  (pic cs:object);

\section{PARTICIPIUM CONJUNCTUM (substantiviert)}
\subsection{Variante 1: XADJ}

\Tree [.S 
		[.VP{\textsubscript{$\downarrow$ $\in$ ($\uparrow$XADJ)}}
			[.{V'\textsubscript{$\uparrow$=$\downarrow$}}
					[\qroof{auxilium}.NP\textsubscript{($\uparrow$OBJ)=$\downarrow$} ]
					[.V\textsubscript{$\uparrow$=$\downarrow$} petentibus ] 
		]
			]
		[\qroof{Caesar}.NP\textsubscript{($\uparrow$SUBJ)=$\downarrow$} ]
		[.V\textsubscript{$\uparrow$=$\downarrow$} parcit ]	
	]

\subsubsection{f-Struktur PC (substantiviert)}
\begin{avm}
\[ PRED &  \rm ‘parco \q<SUBJ, OBJ\textsubscript{REC}\q>’\\
SUBJ & \[``Caesar'' \] \\
OBJ\textsubscript{REC} & \[PRED & `pro' \\
PRON-TYPE & mis \\
CASE & dat \\
NUM & pl \\
GEN & m \]\tikzmark{alpha} \\
XADJ & \{ \[PRED &  \rm ‘peto \q<SUBJ, OBJ\q>’\\
MOOD & part \\
PASSIVE & - \\
RELTENSE & present \\
CASE & dat \\
NUM & pl \\
GEN & m \\
SUBJ &  \tikzmark{omega} \\
OBJ & \[PRED & `auxilium' \\
CASE & acc \\
NUM & sg \\
GEN & n \\
\] \]\\
\} &            $\qquad$ \\
\]
\end{avm}

\tikz[remember picture,overlay] 
    \draw[<-] (pic cs:alpha) to[out=0,in=0,looseness=2.5]  (pic cs:omega);

\newpage
\section{PARTICIPIUM CONJUNCTUM (substantiviert)}
\subsection{Variante 2: OBJ}

\Tree [.S 
		[.VP{\textsubscript{($\uparrow$OBJ) = $\downarrow$}}
			[.{V'\textsubscript{$\uparrow$=$\downarrow$}}
					[\qroof{auxilium}.NP\textsubscript{($\uparrow$OBJ)=$\downarrow$} ]
					[.V\textsubscript{$\uparrow$=$\downarrow$} petentibus ] 
		]
			]
		[\qroof{Caesar}.NP\textsubscript{($\uparrow$SUBJ)=$\downarrow$} ]
		[.V\textsubscript{$\uparrow$=$\downarrow$} parcit ]	
	]
	
\subsubsection{f-Struktur PC (substantiviert)}

\begin{avm}
\[ PRED &  \rm ‘parco \q<SUBJ, OBJ\textsubscript{REC}\q>’\\
SUBJ & \[``Caesar'' \] \\
OBJ\textsubscript{REC} & \[PRED &  \rm ‘peto \q<SUBJ, OBJ\q>’\\
MOOD & part \\
PASSIVE & - \\
RELTENSE & present \\
CASE & dat \\
NUM & pl \\
GEN & m \\
SUBJ & \[PRED & `pro' \\
PRON-TYPE  & mis \\
CASE & dat \\
NUM & pl \\
GEN  & m \] \\
OBJ & \[PRED & `auxilium' \\
CASE & acc \\
NUM & sg \\
GEN & n \] \\
\] \]
\end{avm}

\newpage
\section{dominantes Partizip}
\subsection{dom Part - Geigers Variante 1}

\Tree [.S 
		[.PP{\textsubscript{$\downarrow$ $\in$ ($\uparrow$ADJ)}}
			[.P'\textsubscript{$\uparrow$=$\downarrow$} 
				[.P\textsubscript{$\uparrow$=$\downarrow$} ab ] 
				[.NP\textsubscript{($\uparrow$OBJ)=$\downarrow$}
					[.N'\textsubscript{$\uparrow$=$\downarrow$} 
						[.N\textsubscript{$\uparrow$=$\downarrow$} urbe ]
						[\qroof{condita}.VP\textsubscript{$\downarrow$ $\in$ ($\uparrow$ADJ)} ]
					] 
				]
				]				
			] 	
		[\qroof{Roma}.NP\textsubscript{($\uparrow$SUBJ)=$\downarrow$} ]
		[.V\textsubscript{$\uparrow$=$\downarrow$} viguit ]	
		]
	]\\
\newline
\\
\begin{avm}
\[ PRED &  \rm ‘vigeo \q<SUBJ\q>’\\
SUBJ & ``Roma'' \\
ADJ & \[ PRED &  \rm ‘ab \q<OBJ\q>’\\
OBJ & \[ PRED & `urbs' \tikzmark{begin} \\ 
CASE & abl \\
NUM & sg \\
GEN & f  \\
ADJ & \[PRED &  \rm ‘condo \q<SUBJ\q>’\\
MOOD & part \\
PASSIVE & + \\
RELTENSE & past \\
CASE & abl \\
NUM & sg \\ 
GEN & f  \\
SUBJ &  \tikzmark{end} \] &            $\qquad$ \\
\]  \\
\] \]
\end{avm}

\tikz[remember picture,overlay] 
    \draw[<-] (pic cs:begin) to[out=0,in=0,looseness=2.4]  (pic cs:end);
    
    \begin{avm}
\[ PRED &  \rm ‘vigeo \q<SUBJ\q>’\\
SUBJ & ``Roma'' \\
ADJ & \[ PRED &  \rm ‘ab \q<OBJ\q>’\\
OBJ & \[ PRED & `urbs' \\ 
CASE & abl \\
NUM & sg \\
GEN & f  \\
ADJ & \[PRED &  \rm ‘condo \q<SUBJ\q>’\\
MOOD & part \\
PASSIVE & + \\
RELTENSE & past \\
CASE & abl \\
NUM & sg \\ 
GEN & f  \\
SUBJ &  \tikzmark{Ziel} \] &            $\qquad$ &            $\qquad$  \] \\
\] \tikzmark{Start} \\
\]
\end{avm}

\tikz[remember picture,overlay] 
    \draw[<-] (pic cs:Start) to[out=10,in=0,looseness=2.4]  (pic cs:Ziel);

ich kapier nicht, wo genau der Pfeil hingehen soll... entweder auf urbs, aber dann wär er mitten in der box, oder auf die ganze OBJ-Box... ?\\

\subsection{dom Part - Geigers Variante 2 (findet er besser)}

\Tree [.S 
		[.PP{\textsubscript{$\downarrow$ $\in$ ($\uparrow$ADJ)}}
			[.P'\textsubscript{$\uparrow$=$\downarrow$} 
				[.P\textsubscript{$\uparrow$=$\downarrow$} ab ] 
				[.VP\textsubscript{($\uparrow$OBJ)=$\downarrow$}
					[.V'\textsubscript{$\uparrow$=$\downarrow$} 
						[.V\textsubscript{$\uparrow$=$\downarrow$} condita ]
						[\qroof{urbe}.VP\textsubscript{($\uparrow$SUBJ) = $\downarrow$} ]
					] 
				]
				]				
			] 	
		[\qroof{Roma}.NP\textsubscript{($\uparrow$SUBJ)=$\downarrow$} ]
		[.V\textsubscript{$\uparrow$=$\downarrow$} viguit ]	
		]
	]\\
\newline
\\	
\begin{avm}
\[ PRED &  \rm ‘vigeo \q<SUBJ\q>’\\
SUBJ & ``Roma'' \\
ADJ & \[ PRED &  \rm ‘ab \q<OBJ\q>’\\
OBJ & \[ PRED &  \rm ‘condor \q<SUBJ\q>’\\
MOOD & part \\
PASSIVE & + \\
RELTENSE & past \\
CASE & abl \\
NUM & sg \\
GEN & f \\
SUBJ & \[PRED & `urbs' \\
CASE & abl \\
NUM & sg \\
GEN  & f \] \] \] \]
\end{avm}\\


\subsection{dom Part - meine Variante}

\Tree [.S 
		[.VP{\textsubscript{$\uparrow$=$\downarrow$}}
			[.V'\textsubscript{($\uparrow$COMP) = $\downarrow$}
				[\qroof{illum}.NP\textsubscript{($\uparrow$SUBJ) = $\downarrow$} ]
				[.V\textsubscript{$\uparrow$=$\downarrow$} interfectum ]
			] 
				]				 	
			[.V'\textsubscript{$\uparrow$=$\downarrow$}
				[\qroof{aegre}.NP\textsubscript{($\uparrow$ADJ)=$\downarrow$} ]
				[.V\textsubscript{$\uparrow$=$\downarrow$} tuli ]	
			]	
			]
		]
	]\\
\newline
\\
\begin{avm}
\[ PRED &  \rm ‘fero \q<SUBJ, (OBJ), COMP\q>’\\
SUBJ & \[PRED & `pro' \\
PRON-Type & mis\] \\
ADJ & \{\[ PRED &  \rm ‘aeger \q<OBJ\q>’\\
TYPE & adverb \\
CASE & indecl\\
NUM & indecl\\
GEN &  indecl\] \} \\
COMP & \[ PRED &  \rm ‘interficior \q<SUBJ, (OBJ)\q>’\\
MOOD & part \\
PASSIVE & + \\
RELTENSE & past \\
CASE & acc \\
NUM & sg \\
GEN & m \\
SUBJ & \[PRED & `ille' \\
CASE & acc \\
NUM & sg \\
GEN  & m \] \] \]
\end{avm}\\


\newpage
\section{Die Textstelle Sen. \textit{epist.} 72.7-8 und deren Übersetzung}
%\begin{singlespace}
\renewcommand\linenumberfont{\normalfont\small}
\begin{linenumbers}
\begin{quotation}
\fontfamily{ybv}\selectfont
Dicam quomodo intellegas sanum: si se ipse contentus est, si confidit sibi, si scit omnia vota mortalium, omnia beneficia quae dantur petunturque, nullum in beata vita habere momentum. Nam cui aliquid accedere potest, id inperfectum est; cui aliquid abscedere potest, id inperpetuum est: cuius perpetua futura laetitia est, is suo gaudeat. Omnia autem quibus vulgus inhiat ultro citroque fluunt: nihil dat fortuna mancipio. Sed haec quoque fortuita tunc delectant cum illa ratio temperavit ac miscuit: haec est quae etiam externa commendet, quorum avidis usus ingratus est. Solebat Attalus hac imagine uti: 'vidisti aliquando canem missa a domino frusta panis aut carnis aperto ore captantem? quidquid excepit protinus integrum devorat et semper ad spem venturi hiat. Idem evenit nobis: quid\-quid expectantibus fortuna proiecit, id sine ulla voluptate demittimus statim, ad rapinam alterius erecti et attoniti.' Hoc sapienti non evenit: plenus est; etiam si quid obvenit, secure excipit ac reponit; laetitia fruitur maxima, continua, sua.\footnote{Die Textstelle sowie der textkritische Apparat wurden entnommen aus Reynolds (1965, S. 219-20), die Zeilenangaben wurden jedoch der Einfachheit halber geändert. Auch alle übrigen verwendeten lateinischen Zitate aus den \textit{epistulae morales} entstammen Reynolds (1965).}
\end{quotation}
\end{linenumbers}
\vspace{0.5cm}
\fontfamily{ybv}\selectfont

Referenz auf Abbildung \ref{MyTree}!
%\end{singlespace}
%\bibliographystyle{plain}
\pagebreak
\section*{Literaturverzeichnis}
\bibbycategory
\addcontentsline{toc}{section}{Literaturverzeichnis}
\end{document}
