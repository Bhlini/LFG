\documentclass[12pt,a4paper]{article}
%%% PACKAGES
%\usepackage{times} % Schriftart Times verwenden
\usepackage{graphicx} % support the \includegraphics command and options
\usepackage{booktabs} % for much better looking tables
\usepackage{array} % for better arrays (eg matrices) in maths
\usepackage{paralist} % very flexible & customisable lists (eg. enumerate/itemize, etc.)
\usepackage{verbatim} % adds environment for commenting out blocks of text & for better verbatim
\usepackage{subfig} % make it possible to include more than one captioned figure/table in a single float
\usepackage{colortbl} % enables to shade tables
\usepackage[style=authoryear]{biblatex}
\bibliography{quellen}
\usepackage[utf8]{inputenc}
\usepackage{url}
\usepackage{qtree}
\usepackage{amsmath}
\usepackage{amsfonts}
\usepackage{amssymb}
\usepackage[T1]{fontenc}  % stellt sicher, dass im PDF auch Umlaute gefunden werden
\usepackage{tgtermes}
\usepackage{pdfpages}
\usepackage{listings}
\usepackage{fixltx2e}
\usepackage[ngerman]{babel} % deutsche Begriffe (z.B. Inhaltsverzeichnis statt Contents)
\usepackage[german=quotes]{csquotes}
%\renewcommand{\baselinestretch}{1.47} % Zeilenabstand
%\usepackage[onehalfspacing]{setspace} %Zeilenabstand
\usepackage{setspace}
%Seitenränder
\usepackage{geometry}
%\geometry{a4paper, top=2cm, left=2cm, right=2cm, bottom=2cm}
%Linenumbers
\usepackage[modulo]{lineno}
\usepackage{tikz}
\usetikzlibrary{tikzmark,positioning}
\usepackage{avm}
\DeclareBibliographyCategory{primary}
\DeclareBibliographyCategory{secondary}
\DeclareBibliographyCategory{online}
\addtocategory{primary}{lucil1, lucil2, original, seneca66}
\addtocategory{secondary}{hachmann1995, bartsch, becker1893sittlichen, cancik, inwood, edwards, motto, becker1893sittlichen}
\addtocategory{online}{philatinFLU, philatinAUDIS}
\defbibheading{primary}{\subsection*{Textausgaben und Kommentare}}
\defbibheading{secondary}{\subsection*{Sekundärliteratur}}
\defbibheading{online}{\subsection*{Online Ressourcen}}

\begin{document}
%%%%%%%%%%%%%%%%%%%%%%%%%%
% Deckblatt
% The title
\begin{titlepage}

\begin{center}


% Upper part of the page
\begin{minipage}{0.55\textwidth}
\begin{flushleft} \small
Ruprecht-Karls-Universität Heidelberg\\
Seminar für Klassische Philologie\\
Sommersemester 2013\\
Leitung: Dr. Kathrin Winter\\ 
Proseminar: Seneca, \textit{epistulae morales}
\end{flushleft}
\end{minipage}
\begin{minipage}{0.4\textwidth}
\begin{flushright} \large

\end{flushright}
\end{minipage}
\\[3.3cm]
\rule{\textwidth}{0.4pt}\\[0.4cm]

% Title

{\Large Bedeutung, Notwendigkeit und Konsequenzen \\ der Selbstgenügsamkeit} \linebreak {\large -- Eine Betrachtung anhand von Sen. \textit{epist.} 72,7-8}\\[0.2cm]

\rule{\textwidth}{0.4pt}\\[2.4cm]

% Author and supervisor
%\begin{minipage}{0.4\textwidth}
\begin{flushleft} \small
Natalia Bihler\\
Matrikelnummer: 2925340\\
6. Fachsemester (Gymnasiallehramt nach GymPO)\\
Latein und Englisch\\
Dammweg 1, 69123 Heidelberg\\
E-mail: Bihler@stud.uni-heidelberg.de
\end{flushleft}
%\end{minipage}


\vfill

% Bottom of the page
{\large 21. August 2014}

\end{center}

\end{titlepage}
%%%%%%%%%%%%%%%%%%%%%%%%%%
\setcounter{page}{2}
\begingroup
\flushbottom
\tableofcontents
\thispagestyle{empty}
%\newpage
\pagebreak
\endgroup
%\setcounter{page}{1}
% The introduction
\section{Einleitung}
\nocite{lucil1}
\nocite{lucil2} 
\nocite{original}
\nocite{seneca66} 
\nocite{hachmann1995} 
\nocite{bartsch}  
\nocite{philatinFLU} 
\nocite{becker1893sittlichen} 
\nocite{cancik} 
\nocite{inwood}
\nocite{edwards}
\nocite{motto} 
\nocite{becker1893sittlichen}
\nocite{philatinAUDIS}

% AVM Referenz http://nlp.stanford.edu/manning/tex/avm-doc.pdf
%https://github.com/Bhlini/LFG
%https://en.wikibooks.org/wiki/LaTeX/Linguistics#Syntactic_trees
%https://en.wikibooks.org/wiki/LaTeX/Labels_and_Cross-referencing
%http://nlp.stanford.edu/manning/tex/avm-doc.pdf
%http://nlp.stanford.edu/cmanning/tex/
%http://tex.stackexchange.com/questions/157131/problems-using-avm-package-for-lfg-structures
%http://latex-community.org/forum/viewtopic.php?f=12&t=11365
%http://www.essex.ac.uk/linguistics/external/clmt/latex4ling/avms/

\textbf{PC (abhängig von Objekt)}

\Tree [.S 
		[\qroof{puellam}.{NP\textsubscript{($\uparrow$OBJ)=$\downarrow$}} ] 
		[.VP{\textsubscript{$\downarrow$=($\uparrow$XADJ)}}
			[.{V'\textsubscript{$\uparrow$=$\downarrow$}}
				[\qroof{in Galliam}.PP\textsubscript{($\uparrow$OBL\textsubscript{GOAL})=$\downarrow$} ]
				[.V\textsubscript{$\uparrow$=$\downarrow$} missum ]						
			] 
		] 
		[\qroof{Caesar}.NP\textsubscript{($\uparrow$SUBJ)=$\downarrow$} ]
		[.V\textsubscript{$\uparrow$=$\downarrow$} revocat ]	
	]
\\
\\
\\
\\
\textbf{ALTERNATIVE 1}

	\Tree [.S
		[.{NP\textsubscript{($\uparrow$OBJ)=$\downarrow$}} legatum ]
		[.VP{\textsubscript{$\downarrow$=($\uparrow$XADJ)}}
			[.{V'\textsubscript{$\uparrow$=$\downarrow$}}
				[\qroof{in Galliam}.PP\textsubscript{($\uparrow$OBL\textsubscript{GOAL})=$\downarrow$} ]
				[.V\textsubscript{$\uparrow$=$\downarrow$} missum ]						
			] 
		] 
		[.NP\textsubscript{($\uparrow$SUBJ)=$\downarrow$} Caesar ]
		[.V\textsubscript{$\uparrow$=$\downarrow$} revocat ]	
	]
\\
\\
\textbf{ALTERNATIVE 2}

\Tree [.S [\qroof{puellam}.{\textbf{NP} ($\uparrow$OBJ)=$\downarrow$} ] [.VP [.V is ] [.NP fun ] ] ]


\Tree [.S [.NP puellam ] [.VP [.V is ] [.NP fun ] ] ]

\newpage
\Tree [.S [\qroof{LaTeX is fun}.NP ] [.VP [.V is ] [.NP fun ] ] ]



\begin{figure}[!ht]

\Tree[.IP [.NP [.Det \textit{the} ]
               [.N\1 [.N \textit{package} ]]]
          [.I\1 [.I \textsc{3sg.Pres} ]
                [.VP [.V\1 [.V \textit{is} ]
                           [.AP [.Deg \textit{really} ]
                                [.A\1 [.A \textit{simple} ]
                                      \qroof{\textit{to use}}.CP ]]]]]]

  \caption{Look Ma, a tree!}
  \label{MyTree}
\end{figure}

\newpage
Due to a limitation on LaTeX, large avms can cause problems (you might get errors referring to semantic nest size. To get round this, you have to build complex avms by nesting smaller ones (it is also easier to avoid mistakes if you do things like this). The best way to do this is by defining and using `boxes':

\begin{avm}
\[ cat\|subcat & \<NP$_{it}$, NP$_{\@2}$, S[comp]:\@3\> \\
content & \[ relation & \bf bother\\
bothered & \@2 \\
soa-arg & \@3 \] \]
\end{avm}
\\
\\

\begin{avm}
\[ subj & \[ pers & 3 \\
num & sg \\
gend & masc\\
pred & \rm ‘pro’ \]\\
pred & \rm ‘eat\q<SUBJ, OBJ\q>’\\
obj & \[ pers & 3 \\
num & pl \\
gend & fem \\
pred & \rm ‘pro’ \]
\]
\end{avm}
\\
\\

\newpage
\begin{avm}
\[qstore & \{ \[det & \rm the \\
restpar & \[para & \@1 \\
restr & \{ \[reln & \rm poss \\
possessor & \@3 \\
possessed & \@1\]\}\ \bf union \q\{\@2\q\}
\] \] \} \]
\end{avm}
\\
\\
\newpage
\textbf{PARTICIPIUM CONJUNCTUM!!!}
\\
\\

\begin{avm}
\[ PRED &  \rm ‘voco \q<SUBJ, OBJ\q>’\\
SUBJ & \[``mater'' \]\\
OBJ & \[ PRED & 'legatus' \\
CASE & acc \\
NUM & sg \\
GEN & m \]\tikzmark{topic} \\
XADJ & \{ \[PRED &  \rm ‘mitto \q<SUBJ, OBL\q>’\\
PASSIVE & + \\
RELTENSE & past \\
MOOD & part \\
CASE & acc \\
NUM & sg \\
GEN & m \\
SUBJ &  \tikzmark{object} \\
OBL\textsubscript{GOAL} & \[``in scholam''\] \]\\
\} &            $\qquad$
\]
\end{avm}

\tikz[remember picture,overlay] 
    \draw[<-] (pic cs:topic) to[out=0,in=0,looseness=3]  (pic cs:object);

\newpage
\textbf{PARTICIPIUM CONJUNCTUM - NOCH VON SUBJEKT ABHÄNGIG MACHEN!}
\\
\\
\begin{avm}
\[ PRED &  \rm ‘voco \q<SUBJ, OBJ\q>’\\
SUBJ & \["mater" \]\\
OBJ & \[ PRED & 'legatus' \\
CASE & acc \\
NUM & sg \\
GEN & m \]\\
XADJ & \{ \[PRED &  \rm ‘mitto \q<SUBJ, OBL\q>’\\
PASSIVE & + \\
RELTENSE & past \\
MOOD & part \\
CASE & acc \\
NUM & sg \\
GEN & m \\
SUBJ & \bf{weird arrow!!!} \\
OLBgoal & \[\"in scholam"\] \]\\
\}
\]
\end{avm}

\newpage
\textbf{ABL ABS!!!}
\\
\\
\begin{avm}
\[ PRED &  \rm ‘voco \q<SUBJ, OBJ\q>’\\
SUBJ & \["mater" \]\\
OBJ & \[ PRED & 'legatus' \\
CASE & acc \\
NUM & sg \\
GEN & m \]\\
XADJ & \{ \[PRED &  \rm ‘mitto \q<SUBJ, OBL\q>’\\
PASSIVE & + \\
RELTENSE & past \\
MOOD & part \\
CASE & acc \\
NUM & sg \\
GEN & m \\
SUBJ & \bf{weird arrow!!!} \\
OLBgoal & \[\"in scholam"\] \]\\
\}
\]
\end{avm}

\newpage
\begin{avm}
\{ \[ pred & \rm ‘see\q<girl, Mary\q>’ \\
subj & \rm ‘girl’ \\
obj & \rm ‘Mary’ \] \\
5
\[ pred & \rm ‘heard\q<girl, Bill\q>’ \\
subj & \rm ‘girl’ \\
obj & \rm ‘Bill’ \] \}
\end{avm}

\newpage

\newbox\matrixsynsem \newbox\headdtr 
\newbox\compdtrone   \newbox\compdtrtwo

{\scriptsize
\avmoptions{center}
\setbox\matrixsynsem=\hbox{\begin{avm}
\osort{synsem}{\[local & \[cat & \[ \] \\
                    content & \[ \]\]\]}
\end{avm}}

\setbox\headdtr=\hbox{\begin{avm}
\osort{word}{\[ phon\;\<\rm likes\>\\
            synsem\;\[local\[ cat\;\[ head  \; \@{5}\\
                                      arg-s\;\< \@{1},\@{2}\>\] \\
                              content\;\@{6} \]\]\]} \end{avm}}
                              
                              \setbox\compdtrtwo=\hbox{\begin{avm}
\sort{phrase}{\[ phon\;\<\rm bones\>\\
            synsem\;\@{2}\;\[local\[ cat\;\[ head\;noun\\
                                      arg-s\;\< \>\] \\
                              content\;\@{4} \]\]\]} \end{avm}}

\setbox\compdtrone=\hbox{\begin{avm}
\sort{phrase}{\[ phon\;\<\rm fido\>\\
            synsem\;\@{1}\;\[local\[ cat\;\[ head\;noun\\
                                      arg-s\;\< \>\] \\
                              content\;\@{3} \]\]\]} \end{avm}}



\avmoptions{active,sorted}
\begin{avm} \hspace{-1in}
[{phrase} phon\;\<\rm fido likes bones\> \\
       synsem\;[{synsem} local [{} cat & [{} head & @{5}\\
                                             arg-s & \< \; \>] \\
                                   content & @{6}[{psoa} reln\;like \\
                                                     arg1\;@{3} \\
                                                     arg2\;@{4} ]]]\\
       dtrs\;\sort{head-struc}{[{} head-dtr\;\box\headdtr \\
                  comp-dtrs\;< \box\compdtrone , \\ \hspace{.5in}
                                            \box\compdtrtwo>]}]
\end{avm}}

\begin{avm}
\[ subj & \[ pers & 3 \\
num & sg \\
gend & masc\\
pred & \rm ‘pro’ \]\\
pred & \rm ‘eat\q<SUBJ, OBJ\q>’\\
obj & \[ pers & 3 \\
num & pl \\
gend & fem \\
pred & \rm ‘pro’ \]
\]
\end{avm}




\section{Die Textstelle Sen. \textit{epist.} 72.7-8 und deren Übersetzung}
%\begin{singlespace}
\renewcommand\linenumberfont{\normalfont\small}
\begin{linenumbers}
\begin{quotation}
\fontfamily{ybv}\selectfont
Dicam quomodo intellegas sanum: si se ipse contentus est, si confidit sibi, si scit omnia vota mortalium, omnia beneficia quae dantur petunturque, nullum in beata vita habere momentum. Nam cui aliquid accedere potest, id inperfectum est; cui aliquid abscedere potest, id inperpetuum est: cuius perpetua futura laetitia est, is suo gaudeat. Omnia autem quibus vulgus inhiat ultro citroque fluunt: nihil dat fortuna mancipio. Sed haec quoque fortuita tunc delectant cum illa ratio temperavit ac miscuit: haec est quae etiam externa commendet, quorum avidis usus ingratus est. Solebat Attalus hac imagine uti: 'vidisti aliquando canem missa a domino frusta panis aut carnis aperto ore captantem? quidquid excepit protinus integrum devorat et semper ad spem venturi hiat. Idem evenit nobis: quid\-quid expectantibus fortuna proiecit, id sine ulla voluptate demittimus statim, ad rapinam alterius erecti et attoniti.' Hoc sapienti non evenit: plenus est; etiam si quid obvenit, secure excipit ac reponit; laetitia fruitur maxima, continua, sua.\footnote{Die Textstelle sowie der textkritische Apparat wurden entnommen aus Reynolds (1965, S. 219-20), die Zeilenangaben wurden jedoch der Einfachheit halber geändert. Auch alle übrigen verwendeten lateinischen Zitate aus den \textit{epistulae morales} entstammen Reynolds (1965).}
\end{quotation}
\end{linenumbers}
\vspace{0.5cm}
\fontfamily{ybv}\selectfont

Referenz auf Abbildung \ref{MyTree}!
%\end{singlespace}
%\bibliographystyle{plain}
\pagebreak
\section*{Literaturverzeichnis}
\bibbycategory
\addcontentsline{toc}{section}{Literaturverzeichnis}
\end{document}
