\documentclass[12pt,a4paper]{article}
%%% PACKAGES
%\usepackage{times} % Schriftart Times verwenden
\usepackage{graphicx} % support the \includegraphics command and options
\usepackage{booktabs} % for much better looking tables
\usepackage{array} % for better arrays (eg matrices) in maths
\usepackage{paralist} % very flexible & customisable lists (eg. enumerate/itemize, etc.)
\usepackage{verbatim} % adds environment for commenting out blocks of text & for better verbatim
\usepackage{subfig} % make it possible to include more than one captioned figure/table in a single float
\usepackage{colortbl} % enables to shade tables
\usepackage[style=authoryear]{biblatex}
\bibliography{quellen}
\usepackage[utf8]{inputenc}
\usepackage{url}
\usepackage{qtree}
\usepackage{amsmath}
\usepackage{amsfonts}
\usepackage{amssymb}
\usepackage[T1]{fontenc}  % stellt sicher, dass im PDF auch Umlaute gefunden werden
\usepackage{tgtermes}
\usepackage{pdfpages}
\usepackage{listings}
\usepackage{fixltx2e}
\usepackage[ngerman]{babel} % deutsche Begriffe (z.B. Inhaltsverzeichnis statt Contents)
\usepackage[german=quotes]{csquotes}
\renewcommand{\baselinestretch}{1.5} % Zeilenabstand
%\usepackage[onehalfspacing]{setspace} %Zeilenabstand
\usepackage{setspace}
%Seitenränder
\usepackage{geometry}
%\geometry{a4paper, top=2cm, left=3cm, right=3cm, bottom=2cm}
%Linenumbers
\usepackage[modulo]{lineno}
\usepackage{tabularx}
\usepackage{tablefootnote}
\usepackage{tikz}
\usetikzlibrary{tikzmark,positioning}
\usepackage{avm}
\DeclareBibliographyCategory{primary}
\DeclareBibliographyCategory{secondary}
\DeclareBibliographyCategory{online}
\addtocategory{primary}{lucil1, lucil2, original, seneca66}
\addtocategory{secondary}{hachmann1995, bartsch, becker1893sittlichen, cancik, inwood, edwards, motto, becker1893sittlichen}
\addtocategory{online}{philatinFLU, philatinAUDIS}
\defbibheading{primary}{\subsection*{Textausgaben und Kommentare}}
\defbibheading{secondary}{\subsection*{Sekundärliteratur}}
\defbibheading{online}{\subsection*{Online Ressourcen}}

\begin{document}
%%%%%%%%%%%%%%%%%%%%%%%%%%
% Deckblatt
% The title
\begin{titlepage}

\begin{center}


% Upper part of the page
\begin{minipage}{0.55\textwidth}
\begin{flushleft} \small
Ruprecht-Karls-Universität Heidelberg\\
Seminar für Klassische Philologie\\
Sommersemester 2013\\
Leitung: Dr. Kathrin Winter\\ 
Proseminar: Seneca, \textit{epistulae morales}
\end{flushleft}
\end{minipage}
\begin{minipage}{0.4\textwidth}
\begin{flushright} \large

\end{flushright}
\end{minipage}
\\[3.3cm]
\rule{\textwidth}{0.4pt}\\[0.4cm]

% Title

{\Large Bedeutung, Notwendigkeit und Konsequenzen \\ der Selbstgenügsamkeit} \linebreak {\large -- Eine Betrachtung anhand von Sen. \textit{epist.} 72,7-8}\\[0.2cm]

\rule{\textwidth}{0.4pt}\\[2.4cm]

% Author and supervisor
%\begin{minipage}{0.4\textwidth}
\begin{flushleft} \small
Natalia Bihler\\
Matrikelnummer: 2925340\\
6. Fachsemester (Gymnasiallehramt nach GymPO)\\
Latein und Englisch\\
Dammweg 1, 69123 Heidelberg\\
E-mail: Bihler@stud.uni-heidelberg.de
\end{flushleft}
%\end{minipage}


\vfill

% Bottom of the page
{\large 21. August 2014}

\end{center}

\end{titlepage}
%%%%%%%%%%%%%%%%%%%%%%%%%%
\setcounter{page}{2}
\begingroup
\flushbottom
\tableofcontents
\thispagestyle{empty}
%\newpage
\pagebreak
\endgroup
%\setcounter{page}{1}
% The introduction

\nocite{lucil1}
\nocite{lucil2} 
\nocite{original}
\nocite{seneca66} 
\nocite{hachmann1995} 
\nocite{bartsch}  
\nocite{philatinFLU} 
\nocite{becker1893sittlichen} 
\nocite{cancik} 
\nocite{inwood}
\nocite{edwards}
\nocite{motto} 
\nocite{becker1893sittlichen}
\nocite{philatinAUDIS}

% AVM Referenz http://nlp.stanford.edu/manning/tex/avm-doc.pdf
%https://github.com/Bhlini/LFG
%https://en.wikibooks.org/wiki/LaTeX/Linguistics#Syntactic_trees
%https://en.wikibooks.org/wiki/LaTeX/Labels_and_Cross-referencing
%http://nlp.stanford.edu/manning/tex/avm-doc.pdf
%http://nlp.stanford.edu/cmanning/tex/
%http://tex.stackexchange.com/questions/157131/problems-using-avm-package-for-lfg-structures
%http://latex-community.org/forum/viewtopic.php?f=12&t=11365
%http://www.essex.ac.uk/linguistics/external/clmt/latex4ling/avms/
%http://web.ift.uib.no/Teori/KURS/WRK/TeX/symALL.html Zeichen

\section{Einleitung}
LFG im Lateinischen anhang der Partizipialkonstruktionen...
\section{Einführung in Thematik und Terminologie}
\subsection{Partizipien}
Die Partizipien nehmen, wie bereits der Name impliziert, teil an den Eigenschaften des Nomens und des Verbums. Die Kongruenz mit dem Bezugswort in Kasus, Numerus und Genus und die Möglichkeit der Steigerung und Substantivierung spiegeln die nominalen, die Teilnahme an Aktionsart, Genus und Rektion des Verbums die verbalen Eigenschaften wider.\footnote{Vgl. LHS, S. 383, § 206.}
Im Lateinischen werden drei Partizipien verwendet: das Partizip Präsens Aktiv (PPA), das Partizip Perfekt Passiv (PPP) und das Partizip Futur Aktiv (PFA).
Wie alle Partizipialien bezeichnen die Partizipien jedoch nicht die Zeit an sich, sondern das zeitliche Verhältnis des Partizips zum \textit{verbum finitum}: Dabei kennzeichnet das PPA die Gleichzeitigkeit, das PPP die Vorzeitigkeit und das PPA die Nachzeitigkeit.\footnote{Vgl. KSt, S. 756, §136,3f.}
Des Weiteren haben PPA und PFA aktivische Bedeutung, das PPP passivische. In der Regel sind auch die Partizipien von Deponentien in der Bedeutung aktivisch.\footnote{NM, S. 708, § 496. In dieser Arbeit wird nur auf das klassische Latein Caesars und Ciceros Bezug genommen. Deshalb wird entgegen den üblichen wissenschaftlichen Konventionen auch der NM verwendet, der sich auf den Stil dieser beiden spezialisiert hat.} Daneben gibt es jedoch einige Partizipien Perfekt, die die Bedeutung eines PPA haben, wie beispielsweise \textit{confisus} oder \textit{diffisus}.\footnote{Footnote: Vgl. NM, S. 711, § 497.}

\textbf{Die Partizipialkonstruktionen sind satzwertige Konstruktionen, d.h. sie vertreten vollständige Sätze. Würde man eine Partizipialkonstruktion in einen Gliedsatz umwandeln, würde das Bezugswort des Partizips zu dessen Subjekt, das Partizip zu seiner finiten Verbform. Daher entspricht das Bezugswort des Partizips dem Subjekt des Verbs, das dem Partizip zugrunde liegt, was im Weiteren für die Funktionszuteilung der einzelnen Satzbestandteile wichtig ist.}

Partizipien bilden meist in Verbindung mit Substantiven spezifische Konstruktionen. Im Folgenden sollen das rein attributive Partizip, das substantivierte Partizip, das Participium coniunctum (PC), der Ablativus absolutus (Abl. abs.), der Accusativus cum Participio (AcP) und das dominante Partizip näher betrachtet werden, um sie anschließend in das System der LFG einfügen zu können. Dabei sollen, ausgehend von Lexikoneinträgen und Syntaxregeln, sowohl c- als auch f-Strukturen zu den einzelnen Phänomenen entwickelt werden.
\subsection{Die Lexikalisch-Funktionale Grammatik}
\subsubsection{allgemeine Einschränkungen}
\subsubsection{Allgemeines zu Lexikoneinträgen}
*Alles in der PRED-Klammer beeinflusst die Bildung des Prädikats

\subsubsection{Allgemeines zu Syntaxregeln}
*Part trotz nominaler Eigenschaften als V (beste Lösung \textbf{(wieso?)} )
\subsubsection{Allgemeines zu c-Struktur}
\subsubsection{Allgemeines zu f-Struktur}

\subsection{Einschränkungen}
Um die syntaktischen Korrektheit der ausgegebenen Sätze zu gewährleisten, müssen für die verschiedenen grammatikalischen Konstruktionen zunächst spezifische Bedingungen festgelegt werden.
Da im Lateinischen im Gegensatz zu den modernen Sprachen die Wortstellung innerhalb eines Satzes nicht explizit festgelegt ist \textbf{(stimmt das als Grund?)},\footnote{Die gewöhnliche Wortstellung im Lateinischen ist zwar Subjekt – Objekt – Prädikat, jedoch wird diese, vor allem aus Gründen der Betonung und des Wohlklangs, nur selten streng eingehalten. Vgl. LHS S. 397, § 212.} muss der Großteil dieser Bedingungen nicht wie üblicherweise in den Syntaxregeln, sondern im Lexikoneintrag festgelegt werden. Sie sollen zunächst für das PC, den Abl. abs. und den AcP als allgemeine Einschränkungen für die jeweiligen Partizipien definiert werden.

\subsubsection{PC}
\textbf{vielleicht eher zu Vorüberlegungen:} Die Konstruktion des PC erfüllt im vollständigen, finiten Satz immer die syntaktische Funktion des XADJ: \\ 
($\uparrow$XADJ) = $\downarrow$ \\
Das Partizip muss in Kasus, Numerus und Genus mit seinem Bezugswort kongruent sei:\footnote{Vgl. KSt S. 771, § 138,5a.}\\
($\uparrow$SUBJ KNG) = ($\uparrow$KNG)\\
Dieses Bezugswort des Partizips ist eine grammatikalische Funktion der dem XADJ übergeordneten Struktur, und somit Element des finiten Satzes:\footnote{Vgl. KSt S. 771, § 138,5a.}\\
($\uparrow$SUBJ) = ((XADJ$\uparrow$)GF) \\

\subsubsection{Abl. abs.}
Da der \textit{Ablativus absolutus} vom finiten Satz (s\textsubscript{fin}) losgelöst ist, steht er in der Funktion eines ADJ: \\
($\uparrow$ADJ) = $\downarrow$ \\
Auch beim Abl. abs. muss das Partizip in Kasus, Numerus und Genus mit seinem Bezugswort übereinstimmen:\footnote{Vgl. KSt S. 771, § 138,5a.}\\
($\uparrow$SUBJ KNG) = ($\uparrow$KNG)\\
Sowohl Partizip als auch Bezugswort stehen stets im Ablativ:\footnote{Vgl. KSt S. 771, § 138,5b.} \\
($\uparrow$CASE) = abl \\
($\uparrow$SUBJ CASE) = abl \\
Das Bezugswort des Partizips ist keine grammatikalische Funktion der dem XADJ übergeordneten Struktur, und daher vom finiten Satz losgelöst. Somit darf auch das Subjekt des Abl. abs. keine Rolle im übergeordneten Satz spielen:\footnote{Vgl. KSt S. 771, § 138,5b.} \\
$\neg$ ($\uparrow$SUBJ) = ((ADJ$\uparrow$)GF) \\
Da sich diese Arbeit ausschließlich auf das klassische Latein Caesars und Ciceros bezieht, gilt für die folgenden Betrachtungen die Annahme, dass im Abl. abs. kein Partizip Futur Aktiv (PFA) verwendet werden darf.\footnote{Vgl. KSt. S. 760, § 136,4c oder NM S. 771, § 469.}\\
$\neg$ ($\uparrow$RELTENSE) = future \\

%($\uparrow$RELTENSE (ADJ)) $\neq$ future \\
%$\neg$ ($\downarrow$PRED) = ($\uparrow$GF PRED) \\

\subsubsection{AcP}
Die AcP-Konstruktion nimmt im Satz stets die Funktion des XCOMP an: \\
($\uparrow$XCOMP) = $\downarrow$ \\
Das Partizip und sein Bezugswort stehen auch beim AcP im selben Kasus, Numerus und Genus:\footnote{Vgl. KSt S. 771, § 138,5a.}\\
($\uparrow$SUBJ KNG) = ($\uparrow$KNG)\\
Wie auch hier der Name der Konstruktion vermuten lässt, müssen beim AcP Partizip und Bezugswort im Akkusativ stehen:\footnote{Vgl. KSt S. 763, § 137,2a.} \\
($\uparrow$CASE) = acc \\
($\uparrow$SUBJ CASE) = acc \\
Das Bezugswort des Partizips ist das Objekt der dem XCOMP übergeordneten Struktur: \\
	($\uparrow$SUBJ) = ((XCOMP$\uparrow$)OBJ) \\
%($\uparrow$XCOMP SUBJ) = ($\uparrow$OBJ) (?) \\
Das Partizip ist beim Accusativus cum Patricipio meist ein PPA, selten ein PPP:\footnote{Vgl. KSt S. 763, § 137,2a. \textbf{Vgl. auch LHS S. 387-88 § 207 c; auch KSt S. 763, § 137,2b? tenere + habere mit PPP?}} \\
%($\uparrow$XCOMP RELTENSE) = present   \\
$\neg$ ($\uparrow$XCOMP RELTENSE) = future \\
Der Accusativus cum Participio ist von einem Verb der unmittelbaren sinnlichen Wahrnehmung oder von \textit{facere} bzw. \textit{inducere} im Sinne von ‚in einem Werk, in einem Drama darstellen, (auftreten) lassen‘ abhängig.\footnote{Vgl. NM S. 714, § 499.} Dies kann jedoch nicht im Lexikoneintrag des Partizips direkt, sondern nur in dem der übergeordneten Struktur dargestellt werden. Dieser müsste dann folgende Einschränkung beinhalten:\footnote{Vgl. KSt S. 763, § 137,2a. und NM S. 714, § 499.} \\
($\uparrow$VERB TYPE) = verb of perception $\mid$ `facere' $\mid$ `inducere'
	

\subsection{Lexikoneinträge}
Neben diesen für die Partizipialkonstruktionen im Allgemeinen gültigen Einschränkungen finden sich in den Lexikoneinträgen der konkreten Partizipialformen Angaben zur Bestimmung der Wortform. Diese umfassen bei den Partizipien Kasus, Numerus, Genus, \textbf{Verbform} (``mood", d.h. hier stets Partizip (``part''), Zeitverhältnis (``reltense'', abgekürzt für ``relative tense'') und Genus verbi\footnote{Diese Art der Aktiv-Passiv-Unterscheidung stellt ein Problem bei Deponentien dar, deren Diathese aktiv ist, während ihr Genus verbi, also die rein morphologische Form, im Passiv steht. Um den semantischen Funktionen der Deponentien gerecht zu werden, müssten die Lexikoneinträge zwischen Diathese und Genus verbi differenzieren. Dies würde jedoch den Rahmen dieser Arbeit überschreiten, da es sich hierbei nicht um ein spezifisches Problem der Partizipialkonstruktionen handelt.} (wobei dem Attribut ``passive'' hierbei je nach Vorhandensein entweder den Wert ``+'' oder ``-'' erhält).
%Im Folgenden sollen exemplarische Lexikoneinträge zu den Partizipien \textit{missum} (für das PC in Objektstellung), \textit{missi} (für das PC in Subjektstellung), \textit{victis} (für den Abl. abs.) und \textit{iaecentem} für den AcP aufgeführt werden.

\subsubsection{PC attributiv}

\begin{singlespace}
\begin{tabular}{ l  l  l  l  } 
\textbf{obiectam}: & $[1]$ \:  ($\uparrow$PRED) & = & `obicio$\langle$SUBJ, OBJ\textsubscript{LOC}'$\rangle$\\
$\qquad$ & $[2]$ \:  ($\uparrow$SUBJ) & = & ((XADJ$\uparrow$)OBJ)\\
$\qquad$ & $[3]$ \:  ($\uparrow$MOOD) & = & part\\
$\qquad$ & $[4]$ \:  ($\uparrow$PASSIVE) & = & + \\
$\qquad$ & $[5]$ \:  ($\uparrow$RELTENSE) & = & past \\
$\qquad$ & $[6]$ \:  ($\uparrow$NUM) & = & sg \\
$\qquad$ & $[7.1]$ \: ($\uparrow$CASE) & = & acc \\
$\qquad$ & $[7.2]$ \: ($\uparrow$GEN) & = & f \\
\end{tabular}
\newline
\newline
\end{singlespace}

\subsubsection{PC substantiviert}



\subsubsection{dom Part Geig}

\begin{singlespace}
\begin{tabular}{ l  l  l  l  } 
\textbf{condita}: & $[1]$ \:  ($\uparrow$PRED) & = & `condo$\langle$SUBJ, OBJ, OBL\textsubscript{LOC}$\rangle$'\\
%$\qquad$ & $[2]$ \:  ($\uparrow$SUBJ) & = & ((XADJ$\uparrow$)OBJ)\\
$\qquad$ & $[3]$ \:  ($\uparrow$MOOD) & = & part\\
$\qquad$ & $[4]$ \:  ($\uparrow$PASSIVE) & = & + \\
$\qquad$ & $[5]$ \:  ($\uparrow$RELTENSE) & = & past \\
$\qquad$ & $[6]$ \:  \{(($\uparrow$GEN) & = & f \\ 
$\qquad$ & $[6.1]$ \:  ($\uparrow$NUM) & = & sg \\
$\qquad$ & $[6.2]$ \:  ($\uparrow$CASE) & = & \{nom $\mid$ abl\} ) $\mid$\\
$\qquad$ & $[6.2]$ \: (($\uparrow$GEN) & = & n \\
$\qquad$ & $[6.3]$ \:  ($\uparrow$NUM) & = & pl \\
$\qquad$ & $[6.4]$ \:  ($\uparrow$CASE) & = & \{nom $\mid$ acc\} ) \}\\
\end{tabular}
\newline
\newline
\end{singlespace}

\subsubsection{dom Part wir}

\begin{singlespace}
\begin{tabular}{ l  l  l  l  } 
\textbf{amissa}: & $[1]$ \:  ($\uparrow$PRED) & = & `amitto$\langle$SUBJ, OBJ, OBL\textsubscript{LOC}$\rangle$'\\
%$\qquad$ & $[2]$ \:  ($\uparrow$SUBJ) & = & ((XADJ$\uparrow$)OBJ)\\
$\qquad$ & $[3]$ \:  ($\uparrow$MOOD) & = & part\\
$\qquad$ & $[4]$ \:  ($\uparrow$PASSIVE) & = & + \\
$\qquad$ & $[5]$ \:  ($\uparrow$RELTENSE) & = & past \\
$\qquad$ & $[6]$ \:  \{(($\uparrow$GEN) & = & f \\ 
$\qquad$ & $[6.1]$ \:  ($\uparrow$NUM) & = & sg \\
$\qquad$ & $[6.2]$ \:  ($\uparrow$CASE) & = & \{nom $\mid$ abl\} ) $\mid$\\
$\qquad$ & $[6.2]$ \: (($\uparrow$GEN) & = & n \\
$\qquad$ & $[6.3]$ \:  ($\uparrow$NUM) & = & pl \\
$\qquad$ & $[6.4]$ \:  ($\uparrow$CASE) & = & \{nom $\mid$ acc\} ) \}\\
\end{tabular}
\newline
\newline
\end{singlespace}

\subsubsection{PC objektabhängig}

\begin{singlespace}
\begin{tabular}{ l  l  l  l  } 
\textbf{missum}: & $[1]$ \:  ($\uparrow$PRED) & = & `mitto$\langle$SUBJ, OBJ, OBL\textsubscript{GOAL}$\rangle$'\\
$\qquad$ & $[2]$ \:  ($\uparrow$SUBJ) & = & ((XADJ$\uparrow$)OBJ)\\
$\qquad$ & $[3]$ \:  ($\uparrow$MOOD) & = & part\\
$\qquad$ & $[4]$ \:  ($\uparrow$PASSIVE) & = & + \\
$\qquad$ & $[5]$ \:  ($\uparrow$RELTENSE) & = & past \\
$\qquad$ & $[6]$ \:  ($\uparrow$NUM) & = & sg \\
$\qquad$ & $[7]$ \:  \{(($\uparrow$GEN) & = & m \\ 
$\qquad$ & $[7.1]$ \:  ($\uparrow$CASE) & = & acc ) $\mid$\\
$\qquad$ & $[7.2]$ \: (($\uparrow$GEN) & = & n \\
$\qquad$ & $[7.3]$ \:  ($\uparrow$CASE) & = & \{nom $\mid$ acc\} ) \}\\
\end{tabular}
\newline
\newline
\end{singlespace}

%((XADJ$\uparrow$)OBJ) = ($\uparrow$SUBJ) \\

\subsubsection{PC subjektabhängig}

\begin{singlespace}
\begin{tabular}{ l  l  l  l  } 
\textbf{missi}: & $[1]$ \:  ($\uparrow$PRED) & = & `mitto$\langle$SUBJ, OBJ, OBL\textsubscript{GOAL}$\rangle$'\\
$\qquad$ & $[2]$ \:  ($\uparrow$SUBJ) & = & ((XADJ$\uparrow$)SUBJ) \\
$\qquad$ & $[3]$ \:  ($\uparrow$MOOD) & = & part \\
$\qquad$ & $[4]$ \:  ($\uparrow$PASSIVE) & = & + \\
$\qquad$ & $[5]$ \: ($\uparrow$RELTENSE) & = & past \\
$\qquad$ & $[6]$ \:  \{(($\uparrow$NUM) & = & pl \\ 
$\qquad$ & $[6.1]$ \:  ($\uparrow$CASE) & = & nom \\
$\qquad$ & $[6.2]$ \:  ($\uparrow$GEN) & = & m) $\mid$\\
$\qquad$ & $[6.3]$ \:  (($\uparrow$NUM) & = & sg \\ 
$\qquad$ & $[6.4]$ \: ($\uparrow$CASE) & = & gen \\
$\qquad$ & $[6.5]$ \:  ($\uparrow$GEN) & = & \{m $\mid$ n\} ) \} \\
\end{tabular}
\end{singlespace}

\subsubsection{Abl. abs.}

\begin{singlespace}
\begin{tabular}{ l  l  l  l  } 
\textbf{victis}: & $[1]$ \:  ($\uparrow$PRED) & = & `vinco$\langle$SUBJ, OBJ, OBL\textsubscript{LOC}$\rangle$'\\
$\qquad$ & $[2]$ \:  ($\uparrow$MOOD) & = & part\\
$\qquad$ & $[3]$ \: ($\uparrow$PASSIVE) & = & + \\
$\qquad$ & $[4]$ \: ($\uparrow$RELTENSE) & = & past \\
$\qquad$ & $[5]$ \: ($\uparrow$CASE) & = & \{dat $\mid$ abl\} \\
$\qquad$ & $[6]$ \:  ($\uparrow$NUM) & = & pl \\
$\qquad$ & $[7]$ \: ($\uparrow$GEN) & = & \{m $\mid$ f $\mid$ n\} \\
\end{tabular}
\end{singlespace}

\subsubsection{AcP}

\begin{singlespace}
\begin{tabular}{ l  l  l  l  } 
\textbf{iacentem}: & $[1]$ \: ($\uparrow$PRED) & = & `iaceo$\langle$SUBJ, OBL\textsubscript{LOC}$\rangle$'\\
$\qquad$ & $[2]$ \: ($\uparrow$MOOD) & = & part\\
$\qquad$ & $[3]$ \: ($\uparrow$PASSIVE) & = & - \\
$\qquad$ & $[4]$ \: ($\uparrow$RELTENSE) & = & present \\
$\qquad$ & $[5]$ \: ($\uparrow$CASE) & = & acc \\
$\qquad$ & $[6]$ \: ($\uparrow$NUM) & = & sg \\
$\qquad$ & $[7]$ \: ($\uparrow$GEN) & = & \{m $\mid$ f\} \\
\end{tabular}\\
\newline
Im Lexikoneintrag des Prädikats der dem AcP-XCOMP übergeordneten Struktur müsste, wie oben erwähnt, zunächst spezifiziert sein, dass es ein XCOMP zu sich nehmen kann, und im Folgenden die Bedingungen, die dieses XCOMP erfüllen muss:\\
\textbf{video}: $\langle$SUBJ, OBJ, XCOMP$\rangle$\\
($\uparrow$XCOMP SUBJ) = ($\uparrow$OBJ)\\
($\uparrow$OBJ CASE) = acc\\

Alternative:
video: $\langle$SUBJ, OBJ, COMP$\rangle$\\
($\uparrow$COMP SUBJ) = `pro'\\
($\uparrow$COMP SUBJ KNG) = ($\uparrow$OBJ KNG)\\

\end{singlespace}

\subsubsection{PC (substantiviert)}
\textbf{Variante 1: XADJ}:\\
Das Subjekt der untergeordneten Struktur ist das Objekt der dem XADJ übergeordneten Struktur (welches fehlt): \\
($\downarrow$SUBJ) = ((OBJ$\uparrow$)XADJ)


%\subsection{Zeichen}

%$\theta$

%$\mid$

%$\neq$

%$\in$

%$\ni$

%$\vdash$

%$\subset$

%$\ast$

%$\neg$


\subsection{Syntaxregeln}

%S $\rightarrow$ NP \, VP \: XP\\

\subsubsection{PC objektabhängig}

\begin{singlespace}
\begin{tabular}{ l  l  c  c  c  c }
S & $\rightarrow$ & NP\textsubscript{1} & VP & NP\textsubscript{2} & V\\
   & $\qquad$ & \textsuperscript{($\uparrow$OBJ) = $\downarrow$} & \textsuperscript{$\downarrow$ $\in$ ($\uparrow$XADJ)} & \textsuperscript{($\uparrow$SUBJ) = $\downarrow$} & \textsuperscript{$\uparrow$ = $\downarrow$} \\
    NP\textsubscript{1} & $\rightarrow$ & N \\
   & $\qquad$ & \textsuperscript{$\uparrow$ = $\downarrow$} \\
    VP & $\rightarrow$ & V' \\
   & $\qquad$ & \textsuperscript{$\uparrow$ = $\downarrow$} \\
  	  V' & $\rightarrow$ & PP & V & \\
   & $\qquad$ & \textsuperscript{($\uparrow$OBL\textsubscript{GOAL}) = $\downarrow$ } & \textsuperscript{$\uparrow$ = $\downarrow$} \\
   		 PP & $\rightarrow$ & P' \\
	& $\qquad$   & \textsuperscript{$\uparrow$ = $\downarrow$} \\
    		P' & $\rightarrow$ & P & NP\textsubscript{3} \\
   & $\qquad$ & \textsuperscript{$\uparrow$ = $\downarrow$} & \textsuperscript{($\uparrow$OBJ) = $\downarrow$} \\
 		   NP\textsubscript{3} & $\rightarrow$ & N \\
   & $\qquad$ & \textsuperscript{$\uparrow$ = $\downarrow$} \\
    NP\textsubscript{2} & $\rightarrow$ & N \\
   & $\qquad$ & \textsuperscript{$\uparrow$ = $\downarrow$} \\
\end{tabular}\\
\end{singlespace}

\subsubsection{PC attributiv}

%S $\rightarrow$ NP \, VP \: V\\

\begin{singlespace}
\begin{tabular}{ l  l  c  c  c  c }
  S & $\rightarrow$ & NP\textsubscript{1} & V\\
   & $\qquad$ & \textsuperscript{($\uparrow$OBJ) = $\downarrow$} & \textsuperscript{$\uparrow$ = $\downarrow$} \\
    NP\textsubscript{1} & $\rightarrow$ & N' \\
   & $\qquad$ & \textsuperscript{$\uparrow$ = $\downarrow$} \\
       N' & $\rightarrow$ & N & VP \\
   & $\qquad$ & \textsuperscript{$\uparrow$ = $\downarrow$} & \textsuperscript{$\downarrow$ $\in$ ($\uparrow$XADJ)} \\
		    VP & $\rightarrow$ & V' \\
   & $\qquad$ & \textsuperscript{$\uparrow$ = $\downarrow$} \\
  				  V' & $\rightarrow$ & V & NP\textsubscript{2} \\
   & $\qquad$ & \textsuperscript{$\uparrow$ = $\downarrow$} & \textsuperscript{($\uparrow$OBL\textsubscript{LOC}) = $\downarrow$ }  \\
   					 NP\textsubscript{2} & $\rightarrow$ & N \\
   & $\qquad$ & \textsuperscript{$\uparrow$ = $\downarrow$} \\
\end{tabular} 
\end{singlespace}

\subsubsection{substantiviertes PC}
\textbf{Variante 1: XCOMP}




\textbf{Variante 2: OBJ}

\begin{singlespace}
\begin{tabular}{ l  l  c  c  c  c }
  S & $\rightarrow$ & VP & NP\textsubscript{1} & V\\
   & $\qquad$ & \textsuperscript{($\uparrow$OBJ\textsubscript{REC}) = $\downarrow$} & \textsuperscript{($\uparrow$SUBJ) = $\downarrow$} & \textsuperscript{$\uparrow$ = $\downarrow$} \\
		    VP & $\rightarrow$ & V' \\
   & $\qquad$ & \textsuperscript{$\uparrow$ = $\downarrow$} \\
  				  V' & $\rightarrow$ & NP\textsubscript{2} & V \\
   & $\qquad$ & \textsuperscript{($\uparrow$OBJ) = $\downarrow$} & \textsuperscript{$\uparrow$ = $\downarrow$} \\
   					 NP\textsubscript{2} & $\rightarrow$ & N \\
   & $\qquad$ & \textsuperscript{$\uparrow$ = $\downarrow$} \\
    NP\textsubscript{1} & $\rightarrow$ & N' \\
   & $\qquad$ & \textsuperscript{$\uparrow$ = $\downarrow$} \\
\end{tabular} 
\end{singlespace}


\subsubsection{Abl. abs.}

%S\textsubscript{part} $\rightarrow$ NP \: V'\\

%S $\rightarrow$ NP \, VP \: V\\

\begin{singlespace}
\begin{tabular}{ l  l  c  c  c  c }
   S\textsubscript{fin} & $\rightarrow$ & S\textsubscript{part} & NP\textsubscript{1} & V\\
   & $\qquad$ & \textsuperscript{ $\downarrow$ $\in$ ($\uparrow$ADJ)} & \textsuperscript{($\uparrow$SUBJ) = $\downarrow$} & \textsuperscript{$\uparrow$ = $\downarrow$} \\
   S\textsubscript{part} & $\rightarrow$ & NP\textsubscript{2} & V'\\
   & \textsuperscript{$\qquad$} & \textsuperscript{($\uparrow$SUBJ) = $\downarrow$} & \textsuperscript{$\uparrow$ = $\downarrow$} \\
   NP\textsubscript{2} & $\rightarrow$ & N \\
   & $\qquad$ & \textsuperscript{$\uparrow$ = $\downarrow$} \\
   V' & $\rightarrow$ & PP & V & \\
   & $\qquad$ & \textsuperscript{($\uparrow$OBL\textsubscript{LOC}) = $\downarrow$ } & \textsuperscript{$\uparrow$ = $\downarrow$} \\
   PP & $\rightarrow$ & P' \\
	& $\qquad$   & \textsuperscript{$\uparrow$ = $\downarrow$} \\
    P' & $\rightarrow$ & P & NP\textsubscript{3} \\
   & $\qquad$ & \textsuperscript{$\uparrow$ = $\downarrow$} & \textsuperscript{($\uparrow$OBJ) = $\downarrow$} \\
    NP\textsubscript{3} & $\rightarrow$ & N \\
   & $\qquad$ & \textsuperscript{$\uparrow$ = $\downarrow$} \\
   NP\textsubscript{1} & $\rightarrow$ & N \\
   & $\qquad$ & \textsuperscript{$\uparrow$ = $\downarrow$} \\
\end{tabular} 
\end{singlespace}

\subsubsection{AcP}

%S $\rightarrow$ NP \, VP \: V\\

%($\uparrow$OBJ) = $\downarrow$\\

\begin{singlespace}
\renewcommand{\arraystretch}{1}  
\begin{tabular}{ l  l  c  c  c }
  S & $\rightarrow$ & NP\textsubscript{1} & VP & V\\
   & $\qquad$ & \textsuperscript{($\uparrow$OBJ) = $\downarrow$} & \textsuperscript{($\uparrow$XCOMP) = $\downarrow$} & \textsuperscript{$\uparrow$ = $\downarrow$} \\
    NP\textsubscript{1} & $\rightarrow$ & N \\
   & $\qquad$ & \textsuperscript{$\uparrow$ = $\downarrow$} \\
    VP & $\rightarrow$ & V' \\
   & $\qquad$ & \textsuperscript{$\uparrow$ = $\downarrow$} \\
    V' & $\rightarrow$ & PP & V & \\
   & $\qquad$ & \textsuperscript{($\uparrow$OBL\textsubscript{LOC}) = $\downarrow$ } & \textsuperscript{$\uparrow$ = $\downarrow$} \\
    PP & $\rightarrow$ & P' \\
	& $\qquad$   & \textsuperscript{$\uparrow$ = $\downarrow$} \\
    P' & $\rightarrow$ & P & NP\textsubscript{2} \\
   & $\qquad$ & \textsuperscript{$\uparrow$ = $\downarrow$} & \textsuperscript{($\uparrow$OBJ) = $\downarrow$} \\
    NP\textsubscript{2} & $\rightarrow$ & N \\
   & $\qquad$ & \textsuperscript{$\uparrow$ = $\downarrow$} \\
\end{tabular} 
\end{singlespace}
 

\newpage
\section{Das rein attributive Participium Coniunctum}

Das rein attributive Partizip hat zum \textit{verbum finitum} keinerlei Beziehung, sondern charakterisiert nur sein Bezugswort; es ersetzt somit einen attributiven Gliedsatz.\footnote{Vgl. NM, S. 713, § 498.} \\

\subsection{Vorüberlegungen zur Umsetzung in der LFG}

* da rein attributiv ist es abhängig von NP \\

Beispielsatz:\\
\textit{insulam obiectam portui tenuit.}



\begin{singlespace}
\Tree [.S 
		[.{NP\textsubscript{($\uparrow$OBJ)=$\downarrow$}} 
			[.N'\textsubscript{$\uparrow$=$\downarrow$}
				[.N\textsubscript{$\uparrow$=$\downarrow$} insulam ]		
				[.VP{\textsubscript{$\downarrow$ $\in$ ($\uparrow$XADJ)}}
					[.{V'\textsubscript{$\uparrow$=$\downarrow$}}
						[.V\textsubscript{$\uparrow$=$\downarrow$} obiectam ] 
						[\qroof{portui}.NP\textsubscript{($\uparrow$OBJ\textsubscript{DAT})=$\downarrow$} ]
					] 
				]
				]
			]	
		[.V\textsubscript{$\uparrow$=$\downarrow$} tenuit ]	
	]
\end{singlespace}

\subsection{f-Struktur PC (attributiv)}
\begin{singlespace}
\begin{avm}
\[ PRED &  \rm ‘teneo \q<SUBJ, OBJ\q>’\\
SUBJ & \[ PRED & 'pro' \\
PRON-TYPE & mis \] \\
OBJ & \[PRED & `insula' \\
CASE & acc \\
NUM & sg \\
GEN & f \]\tikzmark{a} \\
XADJ & \{ \[PRED &  \rm ‘obicio \q<SUBJ, OBJ\textsubscript{DAT}\q>’\\
MOOD & part \\
PASSIVE & + \\
RELTENSE & past \\
CASE & acc \\
NUM & sg \\
GEN & f \\
SUBJ &  \tikzmark{z} \\
OBJ\textsubscript{DAT} & \[PRED & `portus' \\
CASE & dat \\
NUM & sg \\
GEN & m \\
\] \]\\
\} &            $\qquad$ \\
\]
\end{avm}
\end{singlespace}

\tikz[remember picture,overlay] 
    \draw[<-] (pic cs:a) to[out=0,in=0,looseness=3.4]  (pic cs:z);

\newpage
\section{PARTICIPIUM CONJUNCTUM}
Partizipien können als Vertreter von Adverbialsätzen aufgefasst werden und stehen dabei für Temporal-, Kausal-, Modal-, Kondizional- und Konzessivsätze. Das Partizip ist hierbei mit seinem Bezugswort verbunden, welches in einem der fünf Kasus Bestandteil des Hauptsatzes und gleichzeitig Subjekt des Nebensatzes ist. Partizip und Bezugswort stimmen daher in Kasus, Numerus und Genus überein. Diese Partizipialkonstruktion bezeichnet man als PC.\footnote{Vgl. KSt, S. 766, § 138,1 u. S. 771, § 138,5a; Vgl. NM, S. 715, § 500.} \\
\textbf{+ prädikativ oder attributiv} \\

\subsection{PARTICIPIUM CONJUNCTUM (objektabhängig)}
\subsubsection{Vorüberlegungen zur Umsetzung in der LFG}
Beispielsatz:\\
\textit{legatum in Galliam missum Caesar revocat.} \\
\begin{singlespace}
\Tree [.S 
		[\qroof{legatum}.{NP\textsubscript{($\uparrow$OBJ)=$\downarrow$}} ] 
		[.VP{\textsubscript{$\downarrow$ $\in$ ($\uparrow$XADJ)}}
			[.{V'\textsubscript{$\uparrow$=$\downarrow$}}
				[\qroof{in Galliam}.PP\textsubscript{($\uparrow$OBL\textsubscript{GOAL})=$\downarrow$} ]
				[.V\textsubscript{$\uparrow$=$\downarrow$} missum ]						
			] 
		] 
		[\qroof{Caesar}.NP\textsubscript{($\uparrow$SUBJ)=$\downarrow$} ]
		[.V\textsubscript{$\uparrow$=$\downarrow$} revocat ]	
	]
\end{singlespace}

\subsubsection{f-Struktur PC (objektabhängig)}
\begin{singlespace}
\begin{avm}

\[ PRED &  \rm ‘revoco \q<SUBJ, OBJ\q>’\\
SUBJ & \[PRED & `Caesar' \\
CASE & nom \\
NUM & sg \\
GEN & m \]\\
OBJ & \[ PRED & 'legatus' \\
CASE & acc \\
NUM & sg \\
GEN & m \]\tikzmark{aim} \\
XADJ & \{ \[PRED &  \rm ‘mitto \q<SUBJ, OBL\textsubscript{GOAL}\q>’\\
MOOD & part \\
PASSIVE & + \\
RELTENSE & past \\
CASE & acc \\
NUM & sg \\
GEN & m \\
SUBJ &  \tikzmark{start} \\
OBL\textsubscript{GOAL} & \[``in scholam''\] \]\\
\} &            $\qquad$ \\
TENSE & present \\
NUM & sg \\
PERS & 3 \\
PASSIVE & - \\
MODE & ind \\
\]
\end{avm}
\end{singlespace}

\tikz[remember picture,overlay] 
    \draw[<-] (pic cs:aim) to[out=0,in=0,looseness=3.5]  (pic cs:start);

\newpage
\subsection{PARTICIPIUM CONJUNCTUM (subjektabhängig)}
\subsubsection{Vorüberlegungen zur Umsetzung in der LFG}
Beispielsatz:\\
\textit{milites in Galliam missi hostes vicerunt.} \\
\begin{singlespace}
\Tree [.S 
		[\qroof{milites}.{NP\textsubscript{($\uparrow$SUBJ)=$\downarrow$}} ] 
		[.VP{\textsubscript{$\downarrow$ $\in$ ($\uparrow$XADJ)}}
			[.{V'\textsubscript{$\uparrow$=$\downarrow$}}
				[\qroof{in Galliam}.PP\textsubscript{($\uparrow$OBL\textsubscript{GOAL})=$\downarrow$} ]
				[.V\textsubscript{$\uparrow$=$\downarrow$} missi ]						
			] 
		] 
		[\qroof{hostes}.NP\textsubscript{($\uparrow$OBJ)=$\downarrow$} ]
		[.V\textsubscript{$\uparrow$=$\downarrow$} vicerunt ]	
	]
\end{singlespace}

\subsubsection{f-Struktur PC (subjektabhängig)}
\begin{singlespace}
\begin{avm}
\[ PRED &  \rm ‘vinco \q<SUBJ, OBJ\q>’\\
SUBJ & \[ PRED & 'miles' \\
CASE & nom \\
NUM & pl \\
GEN & m \]\tikzmark{meow} \\
XADJ & \{ \[PRED &  \rm ‘mitto \q<SUBJ, OBL\textsubscript{GOAL}\q>’\\
MOOD & part \\
PASSIVE & + \\
RELTENSE & past \\
CASE & nom \\
NUM & pl \\
GEN & m \\
SUBJ &  \tikzmark{objectmeow} \\
OBL\textsubscript{GOAL} & \[``in Galliam''\] \]\\
\} &            $\qquad$ \\
OBJ & \[``hostes'' \]\\
\]
\end{avm}
\tikz[remember picture,overlay] 
    \draw[<-] (pic cs:meow) to[out=0,in=0,looseness=3.5]  (pic cs:objectmeow);
\end{singlespace}

\newpage
\section{Abl. abs.}
Wie beim PC vertritt auch die Partizipialkonstruktion des Ablativus absolutus einen Adverbialsatz, wobei das Bezugswort dem Subjekt, das Partizip dem Prädikat entspricht \textbf{(das kann eig weg wenn wir das in der Einführung lassen)}. Dabei wird das Bezugswort nicht vom Prädikat des finiten Satzes gefordert, und besitzt demnach keine eigene Satzgliedfunktion. Der Abl. abs. ist somit vom Rest des Satzes losgelöst, welcher auch ohne ihn noch Sinn ergeben würde, weswegen dem Abl. abs. die Satzgliedfunktion der freien Angabe zukommt. \textbf{Partizip und Bezugswort stehen immer im Ablativ. (doppelt - bei Neugliederung beachten} Aufgrund seiner Entsprechung mit dem Prädikat des zugrunde liegenden Satzes kann sein Partizip nur als prädikativ aufgefasst werden; dass es nicht in attributiver Funktion zu einem Nomen steht, wird auch daran deutlich, dass der Satz bei Wegfall des Partizips nicht mehr grammatikalisch korrekt wäre. Der Ablativ ist im Lateinischen für diese Konstruktion gewählt, da dieser Kasus bereits ohne Partizip adverbiale Verhältnisse, beispielsweise der Zeit, bezeichnet.\footnote{Vgl. KSt, S. 766, § 138,1 u. S. 771, § 138,5b; Vgl. NM, S. 718 f., § 503. Anstelle eines Partizips können auch bestimmte Nomina in den Ablativus absolutus treten. Auf dies kann im Rahmen des Umfangs dieser Arbeit, die sich auf Partizipialkonstruktionen konzentriert, nicht näher eingegangen werden. Vgl. NM, S. 720, § 504.} \\

\subsection{Vorüberlegungen zur Umsetzung in der LFG}
Beispielsatz: \\
\textit{barbaris in Gallia victis Caesar gaudet.} \\

\begin{singlespace}
\Tree [.S\textsubscript{fin} 
		[.S{\textsubscript{part} \textsubscript{($\downarrow$ $\in$ $\uparrow$ADJ)}}
			[\qroof{barbaris}.NP{\textsubscript{($\uparrow$SUBJ)=$\downarrow$}}			
			 ]
			[.{V'\textsubscript{$\uparrow$=$\downarrow$}}
				[\qroof{in Gallia}.PP\textsubscript{($\uparrow$OBL\textsubscript{LOC})=$\downarrow$} ]
					[.V\textsubscript{$\uparrow$=$\downarrow$} victis ]
			 ]
		]							
		[\qroof{Caesar}.{NP\textsubscript{($\uparrow$SUBJ)=$\downarrow$}} ] 
		[.V{\textsubscript{$\uparrow$=$\downarrow$}} gaudet ]
	]
\end{singlespace}

\subsection{f-Struktur Abl. abs.}
\begin{singlespace}
\begin{avm}
\[ PRED &  \rm ‘gaudeo \q<SUBJ\q>’\\
SUBJ & \["Caesar" \]\\
ADJ & \{ \[PRED &  \rm ‘vinco \q<SUBJ, OBL\textsubscript{LOC}\q>’\\
MOOD & part \\
PASSIVE & + \\
RELTENSE & past \\
CASE & abl \\
NUM & pl \\
GEN & m \\
SUBJ & \[PRED & `barbarus' \\
CASE & abl \\
NUM & pl \\
GEN & m \\ \] \\
OBL\textsubscript{LOC} & \[``in Gallia''\] \]\\
\}
\]
\end{avm}
\end{singlespace}

\newpage
\section{AcP - Accusativus cum Participio}
Bei den Verben der unmittelbaren sinnlichen Wahrnehmung, oft bei \textit{videre} und \textit{audire}, sowie bei den Verben des Darstellens und Einführens, besonders bei \textit{facere} und \textit{inducere}, steht die satzwertige Ergänzung oft in Verbindung mit einem Objekt \textbf{und dem Partizip Präsens Aktiv im Akkusativ (Lex-Eintrag, nicht doppeln)}. Man nennt diese Verbindung Accusativus cum Participio (AcP).\footnote{Vgl. KSt, S. 763, § 137,2a; Vgl. NM, S. 714, § 499.}\\
\textbf{+ prädikativ} \\
\textbf{+ ähnlich dem AcI} \\

\subsection{Vorüberlegungen zur Umsetzung in der LFG}
* da prädikativ: direkt von S abhängig, nicht z.B. von der NP \\
* muss auf jeden Fall entweder XADJ oder XCOMP sein, da das Subjekt zum Prädikat der Struktur vom Prädikat der darüberliegenden Struktur (d.h. vom finiten Verb, "vidit") gefordert wird // da das Prädikat der AcP-Konstruktion, d.h. das Partizip, sein Subjekt aus der übergeordneten Struktur bezieht. \\
* XCOMP oder XADJ?
	
	\textbf{*für XADJ spricht:} Restsatz ergibt auch so Sinn; analog zum PC;
	
	\textbf{*für XCOMP spricht:} semantisch großer Unterschied (andere Bedeutung als PC wegen Verben der Wahrnehmung, würde dem Sinn der Konstruktion sonst nicht gerecht werden); facere / inducere; analog zu AcI -> also haben wir uns dafür entschieden \\
Beispielsatz: \\
\textit{militem in campo iacentem vidit.} \\

\begin{singlespace}
\Tree [.S 
		[\qroof{militem}.{NP\textsubscript{($\uparrow$OBJ)=$\downarrow$}} ] 
		[.VP{\textsubscript{($\uparrow$XCOMP)=$\downarrow$}}
			[.{V'\textsubscript{$\uparrow$=$\downarrow$}}
				[\qroof{in campo}.PP\textsubscript{($\uparrow$OBL\textsubscript{LOC})=$\downarrow$} ]
				[.V\textsubscript{$\uparrow$=$\downarrow$} iacentem ]					
			]	
		] 
		[.V\textsubscript{$\uparrow$=$\downarrow$} vidit ]	
	]
\end{singlespace}

\subsection{f-Struktur AcP}
\begin{singlespace}
\begin{avm}
\[ PRED &  \rm ‘video \q<SUBJ, OBJ, XCOMP\q>’\\
SUBJ & \[ PRED & `pro' \\
		PRON-TYPE & mis	\]\\
OBJ & \[ PRED & `miles' \\
CASE & acc \\
NUM & sg \\
GEN & m \]\tikzmark{topic} \\
XCOMP & \[PRED &  \rm ‘iaceo \q<SUBJ, OBL\textsubscript{LOC}\q>’\\
MOOD & part \\
PASSIVE & - \\
RELTENSE & present \\
CASE & acc \\
NUM & sg \\
GEN & m \\
SUBJ &  \tikzmark{object} \\
OBL\textsubscript{LOC} & \[``in campo''\] \]  &            $\qquad$\\
\]
\end{avm}
\end{singlespace}

\tikz[remember picture,overlay] 
    \draw[<-] (pic cs:topic) to[out=0,in=0,looseness=3]  (pic cs:object);

\section{substantiviertes Partizip - Variante 1 (XCOMP)}

Da Partizipien einige Eigenschaften der Adjektive übernehmen, können sie wie diese substantiviert werden und die Rolle eines Substantives übernehmen. Der Neue Menge bezeichnet auch das substantivierte Partizip als rein attributiv. Da das Vorhandensein eines Bezugswortes für die LFG jedoch einen erheblichen Unterschied darstellt, wird das substantivierte Partizip in dieser Arbeit gesondert aufgeführt.\footnote{Vgl. NM, S. 713, § 498.} \\
\textbf{+ klassisch selten / weniger häufig als PC, Abl abs, AcP} \\
\textbf{+ kommt v.a. in bestimmten Kontexten vor, wie... ?}


\subsection{Vorüberlegungen zur Umsetzung in der LFG}
Die Umsetzung des substantivierten Partizips in die LFG-Struktur soll anhand des Beispielsatzes \textit{auxilium petentibus Caesar parcit} veranschaulicht werden.

Der Lexikoneintrag des Partizips lautet wie folgt:
\begin{singlespace}
\begin{tabular}{ l  l  l  l  } 
\textbf{petentibus}: & $[1]$ \:  ($\uparrow$PRED) & = & `peto$\langle$(SUBJ, OBJ, OBL\textsubscript{SOURCE}) $\mid$ \\
$\qquad$ & $\qquad$ & $\qquad$ & $\qquad$ \: (SUBJ, OBJ, OBL\textsubscript{LOC}) $\mid$ \\
$\qquad$ & $\qquad$ & $\qquad$ & $\qquad$ \:  (SUBJ, OBJ, OBL\textsubscript{PURPOSE}) $\rangle$'\tablefootnote{Die Aufgeführten Möglichkeiten entsprechen in obiger Reihenfolge den Beispielsätzen \textit{petere auxilium ab aliquo}; \textit{petere hostes in Gallia}; und \textit{petere ab aliquo ut Donaldum Trumpum occidat} \textbf{(Vgl. RHH §119 und § 234)}.} \\  
%$\qquad$ & $[2]$ \:  ($\uparrow$SUBJ) & = & ((XADJ$\uparrow$)OBJ)\\
$\qquad$ & $[3]$ \:  ($\uparrow$MOOD) & = & part\\
$\qquad$ & $[4]$ \:  ($\uparrow$PASSIVE) & = & - \\
$\qquad$ & $[5]$ \:  ($\uparrow$RELTENSE) & = & present \\ 
$\qquad$ & $[6]$ \:  ($\uparrow$CASE) & = & \{abl $\mid$ dat\} \\
$\qquad$ & $[7]$ \:  ($\uparrow$NUM) & = & pl \\
$\qquad$ & $[8]$ \:  ($\uparrow$GEN) & = & \{m $\mid$ n $\mid$ f\} \\
\end{tabular}
\newline
\newline
\end{singlespace}

\footnotetext[16]{Die Aufgeführten Möglichkeiten entsprechen in obiger Reihenfolge den Beispielsätzen \textit{petere auxilium ab aliquo}; \textit{petere hostes in Gallia}; und \textit{petere ab Donaldo Trumpo ut se munere candidatorium abdicet}.} 

\subsection{Variante 1: XADJ}
Folgt man dem Neuen Menge\footnote{Vgl. NM \textbf{§ ??}} und betrachtet das substantivierte Partizip (\textit{petentibus}) als Attribut zu einem sozusagen fehlenden Bezugswort -- in diesem Fall also etwa \textit{eis} oder \textit{viris} -- so würde die Partizipialkonstruktion in der Rolle eines XADJ zu diesem Bezugswort stehen; das Bezugswort selbst wäre dann das Objekt des Hauptsatzprädikats \textit{parcit}. Dieses fehlende Objekt wird in der c-Struktur unten durch ,,?'' bezeichnet. Da vom substantivierten Partizip petentibus in unserem Beispiel noch ein Nomen in Objektfunktion abhängt, spaltet sich die Partizipial-VP noch einmal in V und NP auf. Es ergeben sich die Syntaxregeln wie in Abbildung (\ref{Syntax subst P Obj}), die c-Struktur wie in Abbildung (\ref{c-Struk subst P Obj}) und die und f-Struktur wie in Abbildung (\ref{f-Struk subst P Obj}).\\

\begin{figure}
\begin{singlespace}
\begin{tabular}{ l  l  c  c  c  c }
  S & $\rightarrow$ & NP\textsubscript{1} & NP\textsubscript{2} & V \\
   & $\qquad$ & \textsuperscript{($\uparrow$OBJ) = $\downarrow$} & \textsuperscript{($\uparrow$SUBJ) = $\downarrow$} & \textsuperscript{$\uparrow$ = $\downarrow$} \\
		NP\textsubscript{1} & $\rightarrow$ & N' \\
   & $\qquad$ & \textsuperscript{$\uparrow$ = $\downarrow$} \\
  		  N' & $\rightarrow$ & N & VP \\
   & $\qquad$ & \textsuperscript{$\uparrow$ = $\downarrow$} & \textsuperscript{($\uparrow$XADJ) = $\downarrow$} \\		    
		    VP & $\rightarrow$ & V' \\
   & $\qquad$ & \textsuperscript{$\uparrow$ = $\downarrow$} \\
  				  V' & $\rightarrow$ & NP\textsubscript{3} & V \\
   & $\qquad$ & \textsuperscript{($\uparrow$OBJ) = $\downarrow$} & \textsuperscript{$\uparrow$ = $\downarrow$} \\
   					 NP\textsubscript{3} & $\rightarrow$ & N \\
   & $\qquad$ & \textsuperscript{$\uparrow$ = $\downarrow$} \\
    NP\textsubscript{2} & $\rightarrow$ & N \\
   & $\qquad$ & \textsuperscript{$\uparrow$ = $\downarrow$} \\
\end{tabular} 
\end{singlespace}
\caption{Syntaxregeln substantiviertes Partizip -- Variante ``Objekt''}
\label{Syntax subst P Obj}
\end{figure}

\begin{figure}
\begin{singlespace}
\Tree [.S 
		[.NP{\textsubscript{$\downarrow$ = ($\uparrow$OBJ)}}
			[.{N'\textsubscript{$\uparrow$=$\downarrow$}}
					[.N\textsubscript{$\uparrow$=$\downarrow$} \textit{?} ]
					[.VP\textsubscript{($\uparrow$XADJ)=$\downarrow$}  
						[.{V'\textsubscript{$\uparrow$=$\downarrow$}}
							[\qroof{auxilium}.NP\textsubscript{($\uparrow$OBJ)=$\downarrow$} ]
							[.V\textsubscript{$\uparrow$=$\downarrow$} petentibus ] 
						]
					]
		]	
			]
		[\qroof{Caesar}.NP\textsubscript{($\uparrow$SUBJ)=$\downarrow$} ]
		[.V\textsubscript{$\uparrow$=$\downarrow$} parcit ]	
	]
\end{singlespace}
\caption{c-Struktur substantiviertes Partizip -- Variante ``Objekt''}
\label{c-Struk subst P Obj}
\end{figure}

\subsubsection{f-Struktur PC (substantiviert)}
\begin{figure}
\begin{singlespace}
\begin{avm}
\[ PRED &  \rm ‘parco \q<SUBJ, OBJ\textsubscript{REC}\q>’\\
SUBJ & \[``Caesar'' \] \\
OBJ\textsubscript{REC} & \[PRED & `pro' \\
PRON-TYPE & mis \\
CASE & dat \\
NUM & pl \\
GEN & m \]\tikzmark{alpha} \\
XCOMP & \[PRED &  \rm ‘peto \q<SUBJ, OBJ\q>’\\
MOOD & part \\
PASSIVE & - \\
RELTENSE & present \\
CASE & dat \\
NUM & pl \\
GEN & m \\
SUBJ &  \tikzmark{omega} \\
OBJ & \[PRED & `auxilium' \\
CASE & acc \\
NUM & sg \\
GEN & n \\
\] \]  &            $\qquad$ \\
\]
\end{avm}
\tikz[remember picture,overlay] 
    \draw[<-] (pic cs:alpha) to[out=0,in=0,looseness=2.5]  (pic cs:omega);
    
\end{singlespace}
\caption{f-Struktur substantiviertes Partizip -- Variante ``Objekt''}
\label{f-Struk subst P Obj}
\end{figure}

\newpage
\section{subst. Partizip - Variante 2 (OBJ)}

* XCOMP oder OBJ\\
	* XCOMP: Neuer Menge Variante! man nimmt an, dass das eigentlich Objekt fehlt, zu dem dem das -- dann attributiv aufzufassende -- Partizip kongruent ist (dann als missing object und so); somit würde man sich immer ein nur ausgelassenes Bezugswort des Partizips hinzudenken \\
	* OBJ haben uns für OBJ entschieden, weil das Partizip ja eben substantiviert ist, und damit nicht einem "fehlenden" Bezugswort untergeordnet. Andere Variante bietet aus Sicht der LFG keinen Mehrwert, würde aber durch hinzugedachte Wörter die Sache unnötig verkomplizieren

\begin{singlespace}
\Tree [.S 
		[.VP{\textsubscript{($\uparrow$OBJ\textsubscript{REC}) = $\downarrow$}}
			[.{V'\textsubscript{$\uparrow$=$\downarrow$}}
					[\qroof{auxilium}.NP\textsubscript{($\uparrow$OBJ)=$\downarrow$} ]
					[.V\textsubscript{$\uparrow$=$\downarrow$} petentibus ] 
		]
			]
		[\qroof{Caesar}.NP\textsubscript{($\uparrow$SUBJ)=$\downarrow$} ]
		[.V\textsubscript{$\uparrow$=$\downarrow$} parcit ]	
	]
\end{singlespace}

\subsubsection{f-Struktur PC (substantiviert)}

\begin{singlespace}
\begin{avm}
\[ PRED &  \rm ‘parco \q<SUBJ, OBJ\textsubscript{REC}\q>’\\
SUBJ & \[``Caesar'' \] \\
OBJ\textsubscript{REC} & \[PRED &  \rm ‘peto \q<SUBJ, OBJ\q>’\\
MOOD & part \\
PASSIVE & - \\
RELTENSE & present \\
CASE & dat \\
NUM & pl \\
GEN & m \\
SUBJ & \[PRED & `pro' \\
PRON-TYPE  & mis \] \\
OBJ & \[PRED & `auxilium' \\
CASE & acc \\
NUM & sg \\
GEN & n \] \\
\] \]
\end{avm}
\end{singlespace}

\newpage
\section{dominantes Partizip}
Beim sogenannten dominanten Partizip trägt nicht das Substantiv, sondern das in Kasus, Numerus und Genus übereinstimmenden Partizip die Hauptbedeutung; das Partizip ,dominiert` daher sozusagen sein Bezugswort. Aus diesem Grund wird das dominante Partizip im Deutschen in der Regel mit einem Verbalsubstantiv wiedergegeben, von dem das im Lateinischen regierende Substantiv als Genetiv abhängt. Meistens verwendet man das Partizip Perfekt Passiv als dominantes Partizip.\footnote{Vgl. NM, S. 717 f., § 502.}\\

\subsection{Vorüberlegungen zur Umsetzung in der LFG}
Der Lexikoneintrag zum Partizip der Konstruktion unterscheidet sich nicht wesentlich von den vorherigen.
\textbf{ (ich glaub wir brauchen hier echt nich nochmal nen Lexikoneintrag...)} \\
Das dominante Partizip soll zunächst am Beispielsatz \textit{ab urbe condita Roma viguit} betrachtet werden. Da der Restsatz \textit{Roma viguit} auch ohne die Partizipialkonstruktion Sinn ergibt, muss letztere wie beim Abl. abs. ein ADJ zum finiten Satz sein. Als nächstes ergibt sich aufgrund der Präposition \textit{ab} eine Präpositionalphrase, von der wiederum Partizip und Bezugswort abhängen.

\subsection{dom Part - Geigers Variante 1}
Nun sieht das dominante Partizip \textit{condita} rein formal zunächst aus wie ein attributives Partizip zum Bezugswort \textit{urbe} \textbf{(?) +vgl NM}, weswegen man eine NP mit \textit{urbe} als Kopf konstruieren könnte (siehe Variante 1). Das Partizip wäre somit seinem Bezugswort untergeordnet. Da das Subjekt des Partizips aus der übergeordneten Struktur -- in diesem Fall von der NP mit Kopf \textit{urbe} -- bezieht, müsste das Partizip eine X-Rolle erhalten; da ein XCOMP zum Bezugswort -- in diesem Fall \textit{urbe} -- nicht zu rechtfertigen wäre \textbf{(??? weil es dann von urbe gefordert werden müsste? oder wieso eig?)}, bliebe für das Partizip -- hier \textit{condita} -- nur die Rolle des XADJ. Die zugehörigen c- und f-Strukturen sähen demnach wie folgt aus:

\begin{singlespace}
\Tree [.S 
		[.PP{\textsubscript{$\downarrow$ $\in$ ($\uparrow$XADJ)}}
			[.P'\textsubscript{$\uparrow$=$\downarrow$} 
				[.P\textsubscript{$\uparrow$=$\downarrow$} ab ] 
				[.NP\textsubscript{($\uparrow$OBJ)=$\downarrow$}
					[.N'\textsubscript{$\uparrow$=$\downarrow$} 
						[.N\textsubscript{$\uparrow$=$\downarrow$} urbe ]
						[\qroof{condita}.VP\textsubscript{$\downarrow$ $\in$ ($\uparrow$ADJ)} ]
					] 
				]
			]				
		] 	
		[\qroof{Roma}.NP\textsubscript{($\uparrow$SUBJ)=$\downarrow$} ]
		[.V\textsubscript{$\uparrow$=$\downarrow$} viguit ]	
	]\\
\newline
\end{singlespace}

\begin{singlespace}
\begin{avm}
\[ PRED &  \rm ‘vigeo \q<SUBJ\q>’\\
SUBJ & ``Roma'' \\
ADJ & \[ PRED &  \rm ‘ab \q<OBJ\q>’\\
OBJ & \[ PRED & `urbs' \tikzmark{begin} \\ 
CASE & abl \\
NUM & sg \\
GEN & f  \\
XADJ & \[PRED &  \rm ‘condo \q<SUBJ\q>’\\
MOOD & part \\
PASSIVE & + \\
RELTENSE & past \\
CASE & abl \\
NUM & sg \\ 
GEN & f  \\
SUBJ &  \tikzmark{end} \] &            $\qquad$ \\
\]  \\
\] \]
\end{avm}
\end{singlespace}

\tikz[remember picture,overlay] 
    \draw[<-] (pic cs:begin) to[out=0,in=0,looseness=2.4]  (pic cs:end);
    
\begin{singlespace}    
\begin{avm}
\[ PRED &  \rm ‘vigeo \q<SUBJ\q>’\\
SUBJ & ``Roma'' \\
ADJ & \[ PRED &  \rm ‘ab \q<OBJ\q>’\\
OBJ & \[ PRED & `urbs' \\ 
CASE & abl \\
NUM & sg \\
GEN & f  \\
XADJ & \[PRED &  \rm ‘condo \q<SUBJ\q>’\\
MOOD & part \\
PASSIVE & + \\
RELTENSE & past \\
CASE & abl \\
NUM & sg \\ 
GEN & f  \\
SUBJ &  \tikzmark{Ziel} \] \] \tikzmark{Start} & $\qquad$ & $\qquad$  \\
\] \\
\]
\end{avm}
\newline
\newline
\end{singlespace}

\tikz[remember picture,overlay] 
    \draw[<-] (pic cs:Start) to[out=10,in=0,looseness=2.4]  (pic cs:Ziel);

\subsection{dom Part - Geigers Variante 2 (findet er besser)}
Da Adjunkte jedoch nach Belieben weggelassen werden können, würde dies bedeuten, dass der Satz \textit{ab urbe Roma viguit} korrekt wäre. Das stimmt zwar formal -- ist jedoch semantisch sinnfrei. Eine semantisch sinnvollere Darstellung ergibt sich, wenn das Bezugswort vom Prädikat des Partizips gefordert wird; da das Partizip sein Bezugswort dominiert, sollte ihm in der LFG-Darstellung eine seinem Bezugswort übergeordnete Funktion zukommen. Somit würde die Partizipialkonstruktion von einer VP mit dem Kopf \textit{condita} abhängen; das Bezugsnomen \textit{urbe} wäre dann schlicht das Subjekt der Partizipialkonstruktion.
\\ Es ergeben sich demnach folgende Syntaxregeln:
\subsubsection{Syntaxregeln}
\begin{singlespace}
\begin{tabular}{ l  l  c  c  c  c }
  S & $\rightarrow$ & PP & NP\textsubscript{1} & V\\
   & $\qquad$ & \textsuperscript{$\downarrow$ $\in$ ($\uparrow$ADJ)} & \textsuperscript{($\uparrow$SUBJ) = $\downarrow$} & \textsuperscript{$\uparrow$ = $\downarrow$} \\
		    PP & $\rightarrow$ & P' \\
   & $\qquad$ & \textsuperscript{$\uparrow$ = $\downarrow$} \\
  				  P' & $\rightarrow$ & P & VP \\
   & $\qquad$ & \textsuperscript{$\uparrow$ = $\downarrow$} & \textsuperscript{($\uparrow$OBJ) = $\downarrow$} \\
					    VP & $\rightarrow$ & V' \\
   & $\qquad$ & \textsuperscript{$\uparrow$ = $\downarrow$} \\
		  				  V' & $\rightarrow$ & V & NP\textsubscript{2} \\
   & $\qquad$ & \textsuperscript{$\uparrow$ = $\downarrow$} & \textsuperscript{($\uparrow$SUBJ) = $\downarrow$} \\
		   					 NP\textsubscript{2} & $\rightarrow$ & N \\
   & $\qquad$ & \textsuperscript{$\uparrow$ = $\downarrow$} \\
    NP\textsubscript{1} & $\rightarrow$ & N \\
   & $\qquad$ & \textsuperscript{$\uparrow$ = $\downarrow$} \\
\end{tabular} 
\newline
\end{singlespace}

\subsubsection{c- und f-Strukturen}
Daraus, sowie aus dem hier nicht extra aufgeführten Lexikoneintrag, gehen gemäß der obigen Überlegungen folgende c- und f-Strukturen hervor:

\begin{singlespace}
\Tree [.S 
		[.PP{\textsubscript{$\downarrow$ $\in$ ($\uparrow$ADJ)}}
			[.P'\textsubscript{$\uparrow$=$\downarrow$} 
				[.P\textsubscript{$\uparrow$=$\downarrow$} ab ] 
				[.VP\textsubscript{($\uparrow$OBJ)=$\downarrow$}
					[.V'\textsubscript{$\uparrow$=$\downarrow$} 
						[.V\textsubscript{$\uparrow$=$\downarrow$} condita ]
						[\qroof{urbe}.NP\textsubscript{($\uparrow$SUBJ) = $\downarrow$} ]
					] 
				]
				]				
			] 	
		[\qroof{Roma}.NP\textsubscript{($\uparrow$SUBJ)=$\downarrow$} ]
		[.V\textsubscript{$\uparrow$=$\downarrow$} viguit ]	
	]\\
\newline
\end{singlespace}

\begin{singlespace}
\begin{avm}
\[ PRED &  \rm ‘vigeo \q<SUBJ\q>’\\
SUBJ & ``Roma'' \\
ADJ & \[ PRED &  \rm ‘ab \q<OBJ\q>’\\
OBJ & \[ PRED &  \rm ‘condo \q<SUBJ\q>’\\
MOOD & part \\
PASSIVE & + \\
RELTENSE & past \\
CASE & abl \\
NUM & sg \\
GEN & f \\
SUBJ & \[PRED & `urbs' \\
CASE & abl \\
NUM & sg \\
GEN  & f \] \] \] \]
\end{avm}\\
\end{singlespace}

\subsection{dom Part - meine Variante}
Nun war zu klassischen Zeiten jedoch die präpositionslose Variante des dominanten Partizips vorherrschend,\footnote{Vgl. LHS \textbf{§ ???}} weswegen auch hierzu ein Beispielsatz betrachtet werden soll: \textit{libertate amissa doleo.} Formal ist diese Konstruktion im Ablativ kaum vom Abl. abs. zu unterscheiden; der Satz könnte schließlich auch bedeuten: "Ich trauere wegen der verlorenen Freiheit". Korrekter, da näher an der lateinischen Bedeutung, wäre jedoch die Übersetzung: "Ich trauere wegen des Verlusts der Freiheit." Um diesem -- wenn hier auch semantisch geringen -- Unterschied gerecht zu werden, sollte auch hier in der LFG-Darstellung die Dominanz des Partizips über sein Bezugswort deutlich werden. Auch hier ist daher die gesamte Partizipialkonstruktion ein ADJ zum finiten Prädikat und das Bezugsnomen darin seinem Partizip unterstellt. Der Unterschied zu Variante 2 oben ergibt sich lediglich aus dem Fehlen der Präposition.

\subsubsection{Syntaxregeln}
Letzteres wird auch in den Syntaxregeln deutlich:
\begin{singlespace}
\begin{tabular}{ l  l  c  c  c  c }
  S & $\rightarrow$ & VP & V\\
   & $\qquad$ & \textsuperscript{$\downarrow$ $\in$ ($\uparrow$ADJ)} & \textsuperscript{($\uparrow$SUBJ) = $\downarrow$} & \textsuperscript{$\uparrow$ = $\downarrow$} \\
	    VP & $\rightarrow$ & V' \\
   & $\qquad$ & \textsuperscript{$\uparrow$ = $\downarrow$} \\
			  V' & $\rightarrow$ & NP& V \\
   & $\qquad$ & \textsuperscript{($\uparrow$SUBJ) = $\downarrow$} &\textsuperscript{$\uparrow$ = $\downarrow$} \\
		   					 NP & $\rightarrow$ & N \\
   & $\qquad$ & \textsuperscript{$\uparrow$ = $\downarrow$} \\
\end{tabular} 
\newline
\end{singlespace}

\subsubsection{c- und f-Strukturen}
Daraus, sowie aus dem hier nicht extra aufgeführten Lexikoneintrag, gehen gemäß der obigen Überlegungen folgende c- und f-Strukturen hervor:

\begin{singlespace}
\Tree [.S 
		[.VP{\textsubscript{$\downarrow$ $\in$ ($\uparrow$ADJ)}}
			[.V'\textsubscript{$\uparrow$=$\downarrow$}
				[\qroof{libertate}.NP\textsubscript{($\uparrow$SUBJ) = $\downarrow$} ]
				[.V\textsubscript{$\uparrow$=$\downarrow$} amissa ]
			] 
		]				 	
			[.V\textsubscript{$\uparrow$=$\downarrow$} doleo ]		
	]\\
\newline
\end{singlespace}

\begin{singlespace}
\begin{avm}
\[ PRED &  \rm ‘doleo \q<SUBJ\q>’\\
SUBJ & \[PRED & `pro' \\
PRON-Type & mis\] \\
ADJ & \{ \[ PRED &  \rm ‘amitto \q<SUBJ\q>’\\
MOOD & part \\
PASSIVE & + \\
RELTENSE & past \\
CASE & abl \\
NUM & sg \\
GEN & f \\
SUBJ & \[PRED & `libertas' \\
CASE & abl \\
NUM & sg \\
GEN  & f \] \] \} \]
\end{avm}\\
\end{singlespace}

\newpage
\section{Die Textstelle Sen. \textit{epist.} 72.7-8 und deren Übersetzung}
%\begin{singlespace}
\renewcommand\linenumberfont{\normalfont\small}
\begin{linenumbers}
\begin{quotation}
\fontfamily{ybv}\selectfont
Dicam quomodo intellegas sanum: si se ipse contentus est, si confidit sibi, si scit omnia vota mortalium, omnia beneficia quae dantur petunturque, nullum in beata vita habere momentum. Nam cui aliquid accedere potest, id inperfectum est; cui aliquid abscedere potest, id inperpetuum est: cuius perpetua futura laetitia est, is suo gaudeat. Omnia autem quibus vulgus inhiat ultro citroque fluunt: nihil dat fortuna mancipio. Sed haec quoque fortuita tunc delectant cum illa ratio temperavit ac miscuit: haec est quae etiam externa commendet, quorum avidis usus ingratus est. Solebat Attalus hac imagine uti: 'vidisti aliquando canem missa a domino frusta panis aut carnis aperto ore captantem? quidquid excepit protinus integrum devorat et semper ad spem venturi hiat. Idem evenit nobis: quid\-quid expectantibus fortuna proiecit, id sine ulla voluptate demittimus statim, ad rapinam alterius erecti et attoniti.' Hoc sapienti non evenit: plenus est; etiam si quid obvenit, secure excipit ac reponit; laetitia fruitur maxima, continua, sua.\footnote{Die Textstelle sowie der textkritische Apparat wurden entnommen aus Reynolds (1965, S. 219-20), die Zeilenangaben wurden jedoch der Einfachheit halber geändert. Auch alle übrigen verwendeten lateinischen Zitate aus den \textit{epistulae morales} entstammen Reynolds (1965).}
\end{quotation}
\end{linenumbers}
\vspace{0.5cm}
\fontfamily{ybv}\selectfont

Referenz auf Abbildung \ref{MyTree}!
%\end{singlespace}
%\bibliographystyle{plain}
\pagebreak
\section*{Literaturverzeichnis}
\bibbycategory
\addcontentsline{toc}{section}{Literaturverzeichnis}
\end{document}
