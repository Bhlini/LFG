%\begin{singlespace}
\renewcommand\linenumberfont{\normalfont\small}
\begin{linenumbers}
\begin{quotation}
\fontfamily{ybv}\selectfont
Dicam quomodo intellegas sanum: si se ipse contentus est, si confidit sibi, si scit omnia vota mortalium, omnia beneficia quae dantur petunturque, nullum in beata vita habere momentum. Nam cui aliquid accedere potest, id inperfectum est; cui aliquid abscedere potest, id inperpetuum est: cuius perpetua futura laetitia est, is suo gaudeat. Omnia autem quibus vulgus inhiat ultro citroque fluunt: nihil dat fortuna mancipio. Sed haec quoque fortuita tunc delectant cum illa ratio temperavit ac miscuit: haec est quae etiam externa commendet, quorum avidis usus ingratus est. Solebat Attalus hac imagine uti: 'vidisti aliquando canem missa a domino frusta panis aut carnis aperto ore captantem? quidquid excepit protinus integrum devorat et semper ad spem venturi hiat. Idem evenit nobis: quid\-quid expectantibus fortuna proiecit, id sine ulla voluptate demittimus statim, ad rapinam alterius erecti et attoniti.' Hoc sapienti non evenit: plenus est; etiam si quid obvenit, secure excipit ac reponit; laetitia fruitur maxima, continua, sua.\footnote{Die Textstelle sowie der textkritische Apparat wurden entnommen aus Reynolds (1965, S. 219-20), die Zeilenangaben wurden jedoch der Einfachheit halber geändert. Auch alle übrigen verwendeten lateinischen Zitate aus den \textit{epistulae morales} entstammen Reynolds (1965).}
\end{quotation}
\end{linenumbers}
\vspace{0.5cm}
\fontfamily{ybv}\selectfont

Referenz auf Abbildung \ref{MyTree}!
%\end{singlespace}
%\bibliographystyle{plain}
\pagebreak
\section*{Literaturverzeichnis}
\bibbycategory
\addcontentsline{toc}{section}{Literaturverzeichnis}

FUCK YOU!!!!!